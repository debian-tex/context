\startcomponent correspondence-letter-values

\environment correspondence-environment

\chapter{Values and Labels}

\section{Values}

When you write a letter you have to set some values for the reference line like the date
and other ones like the name and address of the addressee for the address block, these can
be done with the two commands \type{\setlettervalue} and \type{\setupletter}.

\setup[setlettervalue]\flushatnextpar{\index{\tex{setlettervalue}}}

\setup[setupletter:value]\flushatnextpar{\index{\tex{setupletter}}}

The first command \type{\setlettervalue} takes two commands like \type{\setvalue} or
\type{\setvariable} and could be used like:

\starttyping
\setlettervalue{firstname} {Mike}
\setlettervalue{familyname}{Johnson}
\stoptyping

There is also a optional argument between the name and the content of the value
which is explained in the next section.

With the second command \type{\setupletter} you can set multiple values separated
by commas, it's similar to \type{\setvariables}, the above setting will look then:

\starttyping
\setupletter
  [firstname=Mike,
   familyname=Johnson]
\stoptyping

As you can seen in the following command overview for \type{\setupletter} the command has
two arguments while we used in the last example only one, the two argument form is used
to change the layout for the values in the reference (and others) line like:

\starttyping
\setupletter
  [date,name]
  [titlestyle=\tfx,
   titlecolor=gray]
\stoptyping

The complete list of argument are:

\setup[setupletter:setup]\flushatnextpar{\index{\tex{setupletter}}}

\starttyping
\setupletter[..,..=..,..]
\setupletter[...,...][..,..=..,..]

\setlettervalue{...}{...}
\stoptyping

\section{Labels}

When you take a look at the examples in the reference section you can see all of them
have a label above or on the left or the content but this did only happen because the
module provides preset texts for them.

When you try to use a non defined label like e.g. {\em skype} you the content of the value
as seen below but no label above.

The following code

\starttyping
\setlettervalue{date} {\currentdate}
\setlettervalue{skype}{corres.context}

\setupletterstyle[reference][list={skype,date}]
\stoptyping

result in this reference line:

\start

\setlettervalue{date} {\currentdate}
\setlettervalue{skype}{corres.context}

\setupletterstyle[reference][list={skype,date}]

\startelement
\letterelement[reference][a]
\stopelement

\stop

The module use ConTeXt’s labeltext mechanism to define text for various languages
which can be found in the file \filename{default.nle}. To define your own text use
the \type{\setuplabeltext} command, to prevent problems with other macros the names
of the labels are prefixed with {\em letter:}.\footnote{The {\em memo} style makes
a exception to this concept and use {\em memo:} as prefix for the label names.}

\setup[setuplabeltext]\flushatnextpar{\index{\tex{setuplabeltext}}}

To add now a label for our currently used {\em skype} value put the following
line in your document and change the language tag to the mainlanguage you use
in your document.

\starttyping
\setuplabeltext[en][letter:skype=Skype]
\stoptyping

The reference looks now like:

\start

\setuplabeltext[en][letter:skype=Skype]

\setlettervalue{date} {\currentdate}
\setlettervalue{skype}{corres.context}

\setupletterstyle[reference][list={skype,date}]

\startelement
\letterelement[reference][a]
\stopelement

\stop

When you use \type{\setlettervalue} to set the content of the values the optional
argument can be used instead of \type{\setuplabeltext} to define a text for the label
in the current mainlanguage, a empty argument results in a empty labeltext.

The two step setting for the value {\em skype} can the be done with this setting.

\starttyping
\setlettervalue{skype}[Skype]{corres.context}
\stoptyping

\stopcomponent
