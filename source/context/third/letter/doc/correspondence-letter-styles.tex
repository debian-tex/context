\startcomponent correspondence-letter-styles

\environment correspondence-environment

\chapter{Styles}

The already comes with a few predefined styles, to use them write the name in the second
column either as argument to {\em style} when you load the module or later in your document
with \type{\useletterstyle}.

\starttabulate[|l|l|l|]
\NC \bf Description        \NC \bf Name      \NC \bf Page                       \NC\NR
\NC German style DIN 676 A \NC dina          \NC \at[letter:example:dina]       \NC\NR
\NC German style DIN 676 B \NC dinb          \NC \at[letter:example:dinb]       \NC\NR
\NC Dutch letter style     \NC dutch         \NC \at[letter:example:dutch]      \NC\NR
\NC French letter style    \NC french        \NC \at[letter:example:french]     \NC\NR
%\NC English letter style   \NC english       \NC \at[letter:example:english]    \NC\NR
\NC Full-block             \NC fullblock     \NC \at[letter:example:fullblock]  \NC\NR
\NC Semiblock              \NC semiblock     \NC \at[letter:example:semiblock]  \NC\NR
\NC Modified block         \NC modified      \NC \at[letter:example:modified]   \NC\NR
\NC Hanging intended       \NC hanging       \NC \at[letter:example:hanging]    \NC\NR
\NC Memo style             \NC memo          \NC \at[letter:example:memo]       \NC\NR
\NC Simplified style       \NC simplified    \NC \at[letter:example:simplified] \NC\NR
\NC Swiss style            \NC swiss         \NC \at[letter:example:swiss]      \NC\NR
\NC Swiss left style       \NC swissleft     \NC \at[letter:example:swissleft]  \NC\NR
\stoptabulate

The result of each page can be found on the pages shown in the third column.

\stopcomponent
