% texformat=luatex-plain

%D \module
%D   [       file=luatex-test,
%D        version=2009.12.01,
%D          title=\LUATEX\ Support Macros,
%D       subtitle=Simple Test File,
%D         author=Hans Hagen,
%D           date=\currentdate,
%D      copyright={PRAGMA ADE \& \CONTEXT\ Development Team}]

%D See \type {luatex-plain.tex} (or on my machine \type {luatex.tex}
%D for how to make a format.

% You can generate a font database with:
%
%   mtxrun --script fonts --reload --save
%
% The file luatex-fonts-names.lua has to be moved to a place
% where kpse can find it.

\pdfoutput=1

\font\testa=file:lmroman10-regular                  at 12pt \testa \input tufte \par
\font\testb=file:lmroman12-regular:+liga;           at 24pt \testb effe flink fietsen \par
\font\testc=file:lmroman12-regular:mode=node;+liga; at 24pt \testc effe flink fietsen \par
\font\testd=name:lmroman10bold                      at 12pt \testd a bit bold \par

\font\oeps=cmr10

\font\oeps=[lmroman12-regular]:+liga at 30pt \oeps crap
\font\oeps=[lmroman12-regular]       at 40pt \oeps more crap

\font\cidtest=adobesongstd-light

\font\mathtest=cambria(math) {\mathtest 123}

% \font\gothic=msgothic(ms-gothic) {\gothic whatever} % no longer in windows 10

\bgroup

    \ifdefined\pdfprotrudechars \pdfprotrudechars \else \protrudechars \fi 2 \relax
    \ifdefined\pdfadjustspacing \pdfadjustspacing \else \adjustspacing \fi 2 \relax

    \font\testb=file:lmroman12-regular:+liga;extend=1.5         at 12pt \testb \input tufte \par
    \font\testb=file:lmroman12-regular:+liga;slant=0.8          at 12pt \testb \input tufte \par
    \font\testb=file:lmroman12-regular:+liga;protrusion=default at 12pt \testb \input tufte \par

\egroup

\setmplibformat{plain}

\directlua {
    function MpTest()
        metapost.print("fullcircle scaled 3cm")
    end
}

\mplibcode
    beginfig(1) ;
        draw fullcircle
            scaled 10cm
            withcolor red
            withpen pencircle xscaled 4mm yscaled 2mm rotated 30 ;
        draw "test" infont defaultfont scaled 4 ;
        verbatimtex \sl etex;
        draw btex some more test etex scaled 2 ;
        currentpicture := currentpicture shifted (0,1cm) ;
        verbatimtex \bf etex;
        draw btex another test etex scaled 2 ;
        currentpicture := currentpicture shifted (0,1cm) ;
        draw btex another test etex scaled 2 ;
        draw
            runscript("MpTest()")
            withcolor green
            withpen pencircle xscaled 2mm yscaled 1mm rotated 20 ;
    endfig ;
\endmplibcode

\font\mine=file:luatex-fonts-demo-vf-1.lua at 12pt

\mine \input tufte \par

% \font\mine=file:luatex-fonts-demo-vf-2.lua at 12pt \mine [abab] \par
% \font\mine=file:luatex-fonts-demo-vf-3.lua at 12pt \mine [abab] \par

\font\test=dejavuserif:+kern at 10pt  \test


\bgroup \hsize 1mm \noindent Циолковский \par \egroup

\loadpatterns{ru}

\bgroup \hsize 1mm \noindent Циолковский \par \egroup

a bit of math

$\it e=mc^2 \bf  e=mc^2  \Uchar"1D49D$

$$\left( { {1} \over { {1} \over {x} } } \right) $$

$$\sqrt {2} { { {1} \over { {1} \over {x} } } } $$

\font\cows=file:koeieletters.afm at 50pt

\cows Hello World!

% math test

\latinmodern

\def\sqrt{\Uroot "0 "221A{}}

\def\root#1\of{\Uroot "0 "221A{#1}}

Inline $\sqrt{x}{1.2}$ math. % same for $\root n of x$

$\root3\of x$

$\sin{x}$

\lucidabright

\def\sqrt{\Uroot "0 "221A{}}

\def\root#1\of{\Uroot "0 "221A{#1}}

Inline $\sqrt{x}{1.2}$ math. % same for $\root n of x$

$\root3\of x$

$\sin{x}$

\bgroup

    % drawback: no features (so use basemode)

    \font\crapa=lmroman12-regular:mode=base;liga=yes;kern=yes; at 12pt
    \font\crapb=lmsans12-regular:mode=base;liga=yes;kern=yes;  at 30pt
   %\font\mine=file:luatex-fonts-demo-vf-4.lua:1=lmroman12-regular;2=lmsans12-regular{0x41-0x5A+0x30-0x39+0x21}; at 12pt
    \font\mine=file:luatex-fonts-demo-vf-4.lua:1=\fontid\crapa;2=\fontid\crapb{0x41-0x5A+0x30-0x39+0x21}; at 12pt

    \crapa Test\par
    \crapb Test\par

    \mine Zomaar een eindje fietsen! En dan weer terug.

\egroup

\bgroup

    \font\boldera=lmroman10-regular:mode=node;liga=yes;kern=yes;
    \font\bolderb=lmroman10-regular:mode=node;liga=yes;kern=yes;effect=0.1;
    \font\bolderc=lmroman10-regular:mode=node;liga=yes;kern=yes;effect={width=0.1,auto=yes};

    \boldera Just a line. \par
    \bolderb Just a line. \par
    \bolderc Just a line. \par

\egroup

% \font\amiri=file:amiri-regular.ttf:%
%     mode=node;analyze=yes;language=dflt;script=arab;ccmp=yes;%
%     init=yes;medi=yes;fina=yes;isol=yes;%
%     mark=yes;mkmk=yes;kern=yes;curs=yes;%
%     liga=yes;dlig=yes;rlig=yes;clig=yes;calt=yes %
%     at 32pt

% \bgroup
%     \textdir TRT\amiri بِسْمِ اللَّـهِ الرَّ‌حْمَـٰنِ الرَّ‌حِيمِ
% \egroup

% assumes csr10.tfm csr10.pfb csr.enc to be present

% \font\foo=file:luatex-plain-tfm.lua:tfm=csr10;enc=csr;pfb=csr10 at 12pt
%
% \foo áäčďěíĺľňóôŕřšťúýž ff ffi

% \font\foo=file:csr10.tfm:reencode=csr
% \font\foo=file:csr10.tfm:reencode=csr;bitmap=yes % use map file
% \font\foo=file:csr10.tfm:reencode=auto
%
% \foo áäčďěíĺľňóôŕřšťúýž ff ffi \input tufte\par

\end
