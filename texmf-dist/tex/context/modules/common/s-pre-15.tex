%D \module
%D   [      file=s-pre-15,
%D        version=1999.09.01, 
%D          title=\CONTEXT\ Style File,
%D       subtitle=Presentation Environment 15,
%D         author=Hans Hagen,
%D           date=\currentdate,
%D      copyright={PRAGMA ADE \& \CONTEXT\ Development Team}]
%C
%C This module is part of the \CONTEXT\ macro||package and is
%C therefore copyrighted by \PRAGMA. See mreadme.pdf for
%C details.

%D This is one of the styles I made for the presentation on 
%D the \NTS\ project at Euro\TeX\ 1998. You need balls to let
%D \TEX\ typeset graphics, but this style demonstrates that it 
%D can be done. 
%D 
%D This presentation is meant for presentations that build up
%D an idea stepwise. 
%D 
%D \starttyping
%D \TitlePage{Do you know \TEX ?}
%D 
%D \StartIdea
%D   \StartItem We use \TEX\ for typesetting       \unknown \StopItem
%D   \StartItem mathematical text                  \unknown \StopItem
%D   \StartItem but also for text that has no math \unknown \StopItem
%D   \StartItem or presentations like this         \unknown \StopItem
%D   \StartItem and whatever you can come up with!          \StopItem
%D \StopIdea
%D \stoptyping 

%D The basic layout is rather simple and used as much of the 
%D screen as possible. 

\setuppapersize
  [S6][S6]

\setuplayout
  [backspace=25pt,
   topspace=25pt,
   width=middle,
   height=middle,
   header=0pt,
   footer=0pt]

\setupinteraction
  [state=start,
   display=new,
   color=LineColor, 
   contrastcolor=LineColor,
   click=no]
  
\setupinteractionscreen
  [option=max]

\startmode[asintended] \setupbodyfont[lbr] \stopmode

\setupbodyfont[14.4pt]

%D We use a lot of color. You can remap them if you want 
%D differend ones. The ideas circulate over the colors. 

\setupcolors
  [state=start]

\definecolor[TextColor][s=.8]
\definecolor[PageColor][s=.6]
\definecolor[LineColor][s=.4]

\definecolor[red]  [r=.4]        \definecolor[cyan]   [g=.4,b=.4]
\definecolor[green][g=.4]        \definecolor[magenta][r=.4,b=.4]
\definecolor[blue] [b=.4]        \definecolor[yellow] [r=.4,g=.4]

\definecolor[linecolor 1][red]   \definecolor[linecolor 5][cyan]    
\definecolor[linecolor 2][green] \definecolor[linecolor 6][magenta] 
\definecolor[linecolor 3][blue]  \definecolor[linecolor 4][yellow]  

%D We use variables to make sure that the graphics are reused 
%D but unique.  

\setupMPvariables[pageframe][pagecolor=PageColor,linecolor=LineColor]
\setupMPvariables[textframe][textcolor=TextColor,linecolor=LineColor]

\setupbackgrounds
  [page]
  [background={pageframe,nextpage}]

\defineoverlay [pageframe] [\uniqueMPgraphic{pageframe}]
\defineoverlay [textframe] [\uniqueMPgraphic{textframe}]
\defineoverlay [nextpage]  [\overlaybutton{forward}]

\startuniqueMPgraphic{pageframe}{pagecolor,linecolor}
  path p ; p := fullsquare xyscaled (\overlaywidth,\overlayheight) ;
  pickup pencircle scaled 10pt ; 
  fill p withcolor \MPvar{pagecolor} ; 
  draw p withcolor \MPvar{linecolor} ; 
\stopuniqueMPgraphic

\startuniqueMPgraphic{textframe}{textcolor,linecolor}
  path p ; p := fullcircle xyscaled (\overlaywidth,\overlayheight) ;
  pickup pencircle scaled 10pt ; 
  fill p withcolor \MPvar{textcolor} ; 
  draw p withcolor \MPvar{linecolor} ; 
\stopuniqueMPgraphic

%D The rest of the file implements the nasty part: typesetting
%D text embedded in a graphic. The text is collected in a box
%D so that we can reuse it. 

\newbox\CollectedIdeas
\newcounter\CurrentTopic

\def\StartItem% 
  {\setbox\CollectedIdeas=\hbox\bgroup
     \ifdim\wd\CollectedIdeas>0pt \unhbox\CollectedIdeas\hskip25pt \fi
     \setbox\scratchbox=\hbox\bgroup
       \framed      
         [width=160pt,height=160pt,align=middle,frame=off,
          background=textframe,offset=15pt,top=\vfill,bottom=\vfill]
        \bgroup}

\def\StopItem% 
  {\egroup
   \egroup
   \setbox\scratchbox=\hbox{\lower.5\ht\scratchbox\box\scratchbox}%
   \ht\scratchbox=.5\ht\scratchbox
   \dp\scratchbox=  \ht\scratchbox
   \box\scratchbox
   \egroup
   \startstandardmakeup
     \dontcomplain
     \leftskip       0pt plus 50pt
     \rightskip      0pt plus 50pt
     \parfillskip    0pt
     \baselineskip 100pt 
     \unhcopy\CollectedIdeas
   \stopstandardmakeup}

\def\StartIdea%
  {\ifnum\CurrentTopic=6 \doglobal\newcounter\CurrentTopic \fi 
   \doglobal\increment\CurrentTopic
   \definecolor[LineColor][linecolor \CurrentTopic]
   \setbox\CollectedIdeas=\null}

\def\StopIdea%
  {}

\def\StartTitlePage%
  {\startstandardmakeup
   \setupalign[middle]
   \def\\{\vfil\bfb\setupinterlinespace}
   \bfd\setupinterlinespace
   \vfil}

\def\StopTitlePage%
  {\vfil\vfil
   \stopstandardmakeup}

\def\TitlePage#1%
  {\StartTitlePage#1\StopTitlePage}

\doifnotmode{demo}{\endinput}

%D A simple test on functionality. 

\setupoutput[pdftex]

\starttext 

\TitlePage{Do you know \TEX ?}

\startbuffer 
\StartIdea
  \StartItem We use \TEX\ for typesetting       \unknown \StopItem
  \StartItem mathematical text                  \unknown \StopItem
  \StartItem but also for text that has no math \unknown \StopItem
  \StartItem or presentations like this         \unknown \StopItem
  \StartItem and whatever you can come up with!          \StopItem
\StopIdea
\stopbuffer

\dorecurse{6}{\getbuffer}

\stoptext
