% language=uk

\usemodule[art-01,abr-01]

\starttext

\subject {On the agenda}

\subsubject{\LUATEX}

\startitemize
    \startitem
        head||tail cleanup in disc nodes (get rid of temp i.e.\ delay till linebreak)
    \stopitem
    \startitem
        optimize some callback resolution (more direct)
    \stopitem
    \startitem
        add \type {--output-filename} for \PDF\ filename
    \stopitem
    \startitem
        more consistent \type {lang_variables} and \type {tex_language} in \type
        {texlang.w} and also store the \type {*mins}
    \stopitem
    \startitem
        get rid of \type {temp} node in hyphenator i.e. postpone to when needed
    \stopitem
\stopitemize

\subsubject{\CONTEXT}

\startitemize
    \startitem
        play with par callback and properties
    \stopitem
    \startitem
        get rid of components
    \stopitem
    \startitem
        play with box attributes
    \stopitem
    \startitem
        check consistency between footnotes and running text (main color,
        styles, properties)
    \stopitem
    \startitem
        redo some of the spacing (adapt to improvements in engine)
    \stopitem
    \startitem
        use \type {\matheqnogapstep}, \type {\Ustack}, \type {\mathscriptsmode}, \
        \type {\mathdisplayskipmode} and other new math primitives
    \stopitem
\stopitemize

\vfill {\em Feel free to suggest additions.}

\stoptext

% also

check components and pre|post|replace in math-tag

% new:

<cd:command name="showgrid" file="page-grd.mkiv">
    ...
            <cd:constant type="columns" default="yes"/>
    ...
</cd:command>

<cd:command name="itemtag" variant="itemgroup" file="strc-itm.mkvi">
    <cd:arguments>
        <cd:resolve name="keyword-reference-list-optional"/>
    </cd:arguments>
</cd:command>
