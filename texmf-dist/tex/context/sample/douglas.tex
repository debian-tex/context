Donald Knuth has spent the past several years working on a
system allowing him to control many aspects of the design
of his forthcoming books, from the typesetting and layout
down to the very shapes of the letters! Seldom has an
author had anything remotely like this power to control the
final appearance of his or her work. Knuth's \TEX\
typesetting system has become well|-|known and available in
many countries around the world. By contrast, his
\METAFONT\ system for designing families of typefaces has
not become as well known or available.

In his article \quotation {The Concept of a Meta|-|Font},
Knuth sets forth for the first time the underlying
philosophy of \METAFONT, as well as some of its products.
Not only is the concept exiting and clearly well executed,
but in my opinion the article is charmingly written as well.
However, despite my overall enthusiasm for Knuth's idea and
article, there are some points in it that I feel might be
taken wrongly by many readers, and since they are points
that touch close to my deepest interests in artificial
intelligence and esthetic theory, I felt compelled to make
some comments to clarify certain important issues raised by
\quotation {The Concept of a Meta|-|Font}.
