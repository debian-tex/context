Der L\"owe und die M\"ucke

Eine M\"ucke forderte mit den \"uberm\"utigsten Worten
einen L\"owen zum Zweikampf heraus: \quotation {Ich
f\"urchte dich nicht, du gro\SS es Ungeheuer}, rief sie ihm
zu, \quotation {weil du gar keine Vorz\"uge vor mir hast;
oder nenne sie mir, wenn du solche zu haben glaubst; etwa
die, da\SS\ du deinen Raub mit Krallen zerrei\SS est und
mit Z\"ahnen zermalmest? Jedes andere feige Tier, wenn es
mit einem Tapfern k\"ampft, tut dasselbe, es bei\SS t und
kratzt. Du sollst aber empfinden, da\SS\ ich st\"arker bin
als du!} Mit diesen Worten flog sie in eines seiner
Nasenl\"ocher und stach ihn so sehr, da\SS\ er sich vor
Schmerz selbst zerfleischte und sich f\"ur \"uberwunden
erkl\"arte.

Stolz auf diesen Sieg flog die M\"ucke davon, um ihn aller
Welt auszuposaunen, \"ubersah aber das Gewebe einer Spinne
und verfing sich in demselben. Gierig umarmte die Spinne
sie und sog ihr das Heldenblut aus. Sterbend empfand die
M\"ucke ihre Nichtigkeit, indem sie, die Besiegerin des
L\"owen, einem so ver\"achtlichen Tiere, einer Spinne,
erliegen mu\SS te.
