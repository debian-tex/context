Čeho jsem si zvlášť všímal při prohlížení knih, je typografické jejich
zhotovení, technické provedení knihtiskařem, které po mém soudu je základem
dobré knižní úpravy. A tu nelze upřít, že řemeslná stránka je dosud stránkou
slabou většiny našich knih. Často až překvapí rozpor mezi péčí, kterou
věnoval umělec výzdobě knihy, a ledabylostí, s jakou ji tiskárna zhotovila.
Bývá to podivná harmonie: výzdoba umělecká a sazba i tisk jako denních
novin. Nejsou to malé závody, jichž výrobkům lze vytknouti tuto řemeslnou
chybu, i přední veliké tiskárny zhotovují takto \quotation{krásné} knihy. 
Stačí prohlédnouti si podrobněji reprodukce v tomto díle, abychom se
přesvědčili, že to tvrzení není přehnané.

Přece však lze pozorovati potěšitelný obrat k lepšímu: některé české
knihtiskárny se chlubí už správně vysazenými a dobře vytištěnými knihami.
Sice dosud jest málo impressí, které takto jeví úctu a vážnost k svému
černému umění, obracejíce knihtisk k jeho staré lásce, knize, a věnujíce
její vkusné úpravě čas, píči a píli; na prstech jedné ruky by je spočetl.
Nepochybuji však, že jich přibude, vždyť ničím nemůže se knihtiskárna
lépe doporučiti než svojí firmou na dokonale, krásně upravené knize, která
je rozkoší oku a potěšením duši, a ještě po letech a desetiletích, kdy
dávno upadnou v zapomnění drobné akcidence, zůstane v knihovnách jako svědek
odborné vyspělosti knihtiskárny, schopnosti jejích pracovníků
a prozíravosti i vkusu principála nebo řiditele závodu.
