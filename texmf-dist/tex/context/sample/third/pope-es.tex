\startlanguage[es]
\startlines
Feliz quien goza en ocuparse en calma
De unas hectáreas del sol paterno,
Feliz quien puede respirar, gozoso,
Su aire nativo.

Cuyo hato bríndale espumosa leche,
Pan sus trigales, sus ovejas lana,
Sombra en verano sus frondosos árboles,
Fuego en invierno.

Feliz de aquel que indiferente observa
Cómo las horas se deslizan mansas,
Sano de cuerpo y con tranquilo espíritu,
Día por día.

Quien duerme, plácido, y el estudio alterna
Con el reposo, y ameniza el tiempo,
Y une a su pura sencillez dulcísimas
Meditaciones.

Dejad que viva en dulce paz oculto,
Dejad que muera sin lamentos múltiples,
Que me hurte al mundo y ni una losa diga
Donde reposo.
\stoplines
\stoplanguage
