% \enablemode[paper]

\usemodule[present-stepwise,present-wobbling,abr-02]
% \usemodule[present-wobbling,abr-02]

\setuppapersize[S6][S6] \setupbodyfont[10pt] \def\METAPOST{MetaPost}

% \StartText{...}{...}

\startdocument
  [title={\TEX\ and Reality\crlf Clashing Mindsets?},
   topic={Bacho\TEX, May 1, 2010}]

\StartItems{Some reasons to use \TEX}
    \StartItem
        There can be several reasons for using \TEX. Some are subjective.
    \StopItem
    \StartItem
        You like the way it works: you edit a document using a simple editor,
        add a couple of directives and delay rendering. It's the content and
        structure that matter.
    \StopItem
    \StartItem
        You need it for instance because you have to typeset math and you believe that
        no other tool can do a better job on that.
    \StopItem
    \StartItem
        You found out that it can save time because it is programmable and after all,
        programming is a nice distraction from writing.
    \StopItem
    \StartItem
        You don't want to change a 20 year old habit and why quit using something that
        you know well by now.
    \StopItem
    \StartItem
        You like an occasional fight with a batch oriented system and updating (sometimes to
        the extend of compiling) can be done while watching a movie.
    \StopItem
    \StartItem
        You dislike learning a new program every 5 years. Of course it would be different
        if we'd live for 500 years.
    \StopItem
\StopItems

\StartItems{My reasons to use \TEX}
    \StartItem
        I've always used \TEX\ and can do what I need to do with it. I like to
        focus on what can be done instead of what can't.
    \StopItem
    \StartItem
        I don't like disposable tools and am quite lucky that \TEX\ still
        can adapt to my needs.
    \StopItem
    \StartItem
        I like my job but only when using the current tools and cooking up
        reuseable solutions.
    \StopItem
    \StartItem
        I need it for rendering (often educational) content and also use it
        for fun.
    \StopItem
    \StartItem
        In the process I need to implement styles based on designs provided by
        designers, most probably only know click and point tools but some of them
        can think outside that box.
    \StopItem
\StopItems


\StartItems{Using \TEX\ in projects}
    \StartItem
        Each project has at least a few challenges, the input,
        the design, graphics, the boundary conditions, interfaces, etc.
    \StopItem
    \StartItem
        In quite some cases a printed product is an afterthought and coding is
        driven by viewing on the web.
    \StopItem
    \StartItem
        Most time goes into mapping structure. Coding is done in \XML\ because
        we can then manipulate content and publishers can reuse it.
    \StopItem
    \StartItem
        Publishers often use a preselected designer and ask him/her to come
        up with a design.
    \StopItem
    \StartItem
        Chapter openings and title pages take some effort as well, especially
        if the implementation has to be exact. For some reason design comes before
        content so the designer has to guess.
    \StopItem
    \StartItem
        Although one can try to catch bordercases it hardly pays off as the eventual
        solutions are not that logic. Simplification is preferred over heuristics.
    \StopItem
    \StartItem
        Unfortunately designers never use the fact that we can program variations and
        and flexible solutions. On the other hand in a later stage we can quite conveniently
        provide solutions for problems resulting in the editorial workflow.
    \StopItem
\StopItems

\StartItems{Struggling with structure}
    \StartItem
        Structure in regular \TEX\ documents assumes a proper nesting of chapters,
        sections, subsections etc.
    \StopItem
    \StartItem
        In \CONTEXT\ we can clone heads and configure them independently. Often we end up with
        tens of variants.
    \StopItem
    \StartItem
        In practice numberings can intermix, for instance subsections can be numbered
        per chapter instead of per subsection.
    \StopItem
    \StartItem
        Numbers seldom run like 1 \unknown\ 1.1 \unknown\ 1.1.1 and individual components can be omitted and can
        have different properties (font, color). This quickly becomes messy as more (unexpected)
        structure is added.
    \StopItem
    \StartItem
        It's for this reason that we now have a more complex model of resetting and synchronization
        of states in \CONTEXT. Actually we keep adding more structure support.
    \StopItem
    \StartItem
        Additional information that is used in a chapter sometimes is also used elsewhere, as
        in tables of contents (for instance icons). Therefore in \CONTEXT\ \MKIV\ we now have
        the possibility to let userdata travel around.
    \StopItem
\StopItems

\StartItems{Bringing system in color}
    \StartItem
        When making a product line it helps if there is some systematic
        approach in defining colors but it does not work out that way.
    \StopItem
    \StartItem
        Unfortunately we can never use the colorpalet and colorgroup features
        that have been present in \CONTEXT\ from the start.
    \StopItem
    \StartItem
        Spotcolors are nice as they enforce a more systematic approach than
        process colors. In such cases there is often some system.
    \StopItem
    \StartItem
        With processcolors we often have to fight the \quotation {on my screen 0.01
        \letterpercent\ makes a big difference} dilemma.
    \StopItem
    \StartItem
        Automatically converting graphics to such color spaces can save a lot of time and
        money.
    \StopItem
\StopItems

\StartItems{Relations between fonts}
    \StartItem
        Although there is some fashion in using fonts most designs use at least
        a few different ones.
    \StopItem
    \StartItem
        Not all fonts are equally well equipped and one cannot rely too much on
        features without testing them first. Although \OPENTYPE\ makes things
        easier it also introduces problems due to incomplete features.
    \StopItem
    \StartItem
        A macro package assumes some logic in sizes and relations but this is of no
        use in practice. Most if the font mechanism is simply not used.
    \StopItem
    \StartItem
        The same is true for interline spacing. Often some standard latin quote and
        title is used to determine the spec. Not seldom most spacing is inconsistent.
    \StopItem
    \StartItem
        It looks like justification is not wanted that much, let alone advanced features
        like protrusion and expansion. Inter|-|character spacing is sometimes requested.
    \StopItem
\StopItems

\StartItems{Why I still use \TEX}
    \StartItem
        We started making \CONTEXT\ for our own use, especially complex and demanding
        educational documents.
    \StopItem
    \StartItem
        Nowadays we stick to typesetting and as we specialize in automated processing
        we have to operate within strict bounds.
    \StopItem
    \StartItem
        We use not that many handy features as there is hardly any structure in the designs
        we have to implement.
    \StopItem
    \StartItem
        But we use quite some of the manipulative power of \CONTEXT. Also, we are able to
        fulfil even the most extreme demands.
    \StopItem
    \StartItem
        It's user demand that is the driving force behind most new features. Users typically
        use \CONTEXT\ in a different way than we do.
    \StopItem
    \StartItem
        And \unknown\ some things can probably only be done with \TEX, especially in automated
        workflows.
    \StopItem
\StopItems

\StartItems{Suggestions for designers}
    \StartItem
        Talk to those implementing the design, let them show you what can be done. Stick to
        general designs and don't go into much detail. It's the look and feel that matters.
    \StopItem
    \StartItem
        Think in systematic solutions. Lack of freedom in interactive placement of graphics can
        be compensated by other variations.
    \StopItem
    \StartItem
        Think outside the box. Use the fact that the system is programmable and can adapt. And it
        probably goes beyond what you can think of.
    \StopItem
    \StartItem
        Try to make a design extensible. There will always be more structure. Some components
        will have less text that expected. Titles can be very short or quite long. Keep in mind
        that you cannot tweak.
    \StopItem
    \StartItem
        Try to see a pattern in structure and provide escapes for strange cases. Give the implementor
        some freedom.
    \StopItem
\StopItems

\stopdocument

% \StopText
