% language=us

\usemodule[present-boring,abbreviations-logos]

\startdocument
  [title={SVG GRAPHICS},
   banner={some demos and discussion},
   location={context\enspace {\bf 2020}\enspace meeting}]

\starttitle[title=Wrapup]

\startitemize

\startitem
    It expands to for Scalable Vector Graphics.
\stopitem
\startitem
    It is an example of application \XML\ turned standard.
\stopitem
\startitem
    It started out simple, kind of expanded \POSTSCRIPT\ in \XML\ format.
\stopitem
\startitem
    It took a while to be picked up as output format.
\stopitem

\blank[2*line]

\startitem
    In practice you get the same messy build-up as in other vector formats.
\stopitem

\startitem
    This is a side effect of often unstructured editing. \footnote {Afterwards
    Hraban gave a demonstration of editing in InkScape and there was some
    discussion about this aspect}.
\stopitem

\stopitemize

\stoptitle

\starttitle[title=Properties]

\startitemize

\startitem
    Properties can be set as attributes to an element (key/values).
\stopitem
\startitem
    Properties can be set in the \type {style} attribute (semicolon separated key/values).
\stopitem
\startitem
    Properties can be set via one or more \type {class} assignments.
\stopitem
\startitem
    Properties can be bound to a specific element
\stopitem
\startitem
    Properties can be inherited from an ancestor (somewhat vague).
\stopitem
\startitem
    Properties can be redundant (nested), overloaded (parent, style), editors can
    add their own. etc.\ \unknown\ it's kind of a mess.
\stopitem

\stopitemize

\stoptitle

\starttitle[title=Side effects]

\startbuffer
\usemodule[gnuplot]

\externalfigure
  [context-2020-gpdemo.gp]
  [conversion=svg,width=4cm,
   background=color,backgroundcolor=white]

\externalfigure
  [context-2020-gpdemo.gp]
  [conversion=svg,width=6cm,
   background=color,backgroundcolor=white]

\scale
  [height=4cm]
  {\framed
     [background=color,backgroundcolor=white]
     {\includegnuplotsvgfile[context-2020-sin.svg]}}

\stopbuffer

\typebuffer

\getbuffer

\stoptitle

\starttitle[title=Simple examples]

Some examples were shown (they can be found in manuals):

\starttyping
svg-lmtx-context.lua
svg-lmtx-microsoft.lua
svg-lmtx-mozilla.lua
svg-lmtx-xahlee.lua
\stoptyping

Also some examples were shown from the Math4All project.

\stoptitle

% \starttitle[title=Complex examples]
%
% \starttyping
% temporary/svg/test*
% \stoptyping
%
% \stoptitle

\stopdocument
