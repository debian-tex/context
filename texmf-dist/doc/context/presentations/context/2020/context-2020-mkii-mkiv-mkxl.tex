% language=us

\usemodule[present-boring,abbreviations-logos]

\startdocument
  [title={MKII MKIV LMTX},
   banner={where does it end},
   location={context\enspace {\bf 2020}\enspace meeting}]

\starttitle[title=Welcome]

This meeting was kind of special because of the Covid situations. It forces us to
adapt and think about how to deal with this kind of situations. But, we had a
very nice meeting as usual. The first talk was a summary of where we started and
where we are now. The other talks are more specialized.

All presentations use the same simple style. No interaction, no fancy \PDF\
features, also because we had to stream them.

\stoptitle

\starttitle[title=MKII]

\startitemize
\startitem
    In the 80's I bought the \TEX book but it all stayed pretty abstract.
\stopitem
\startitem
    In the beginning of the 90's we had to get some math on paper we bought (!) a
    copy of \LATEX.
\stopitem
\startitem
    Right from the start we had to make in look a bit better than out of the box.
\stopitem
\startitem
    So a shell around if evolved but soon we started from scratch.
\stopitem
\startitem
    We did so first on top \LAMSTEX, then we switched to \INRSTEX.
\stopitem
\startitem
    Soon we only used a few components of that: we learned from trial and error.
\stopitem
\startitem
    We joined the \NTG, met Taco and friends, and slowly got some presence.
\stopitem
\startitem
    And it all went on till we had what we later called \MKII.
\stopitem
\startitem
    But, we always had ideas about what more we wanted.
\stopitem
\startitem
    We went from \TEX\ to \ETEX\ to \PDFTEX\ to \PDFETEX.
\stopitem
\startitem
    We played with the idea of \type {eetex}, different backends etc.\ (show old
    \MAPS\ article).
\stopitem
\stopitemize

\stoptitle

\starttitle[title=MK{\thinspace\periods[2]}]

\startitemize
\startitem
    \CONTEXT\ has been keyword driven and class based from the start.
\stopitem
\startitem
    This came with a performance hit so the reputation was that it was slow:
    inheritance, flexibility, user control \unknown\ it all comes at a price.
\stopitem
\startitem
    \CONTEXT\ always had an abstract driver model (\DVIPS, \DVIPSONE, \DVIWINDO,
    \ACROBAT, \PDFTEX, etc).
\stopitem
\startitem
    It also had an adaptive the front end so we could support successive engines:
    \TEX, \ETEX, \PDFTEX, \ALEPH, \XETEX.
\stopitem
\startitem
    There had to be color and graphics support from the beginning.
\stopitem
\startitem
    The interfaces permitted extension without breaking compatibility. The user
    interface was multilingual: we started with Dutch and German (users).
\stopitem
\startitem
    It came with management tools (like \TEXEXEC, \TEXUTIL, \TEXFONT, \TEXMFSTART) etc.\
    for job control, dealing with (user) fonts, image manipulations etc.
\stopitem
\startitem
    And of course \METAPOST, \XML, combining font setups, mixing encodings, \UTF\
    patterns evolved with the system.
\stopitem
\startitem
    Educational usage was often the reason for new features.
\stopitem
\stopitemize

\stoptitle

\starttitle[title=MKIV]

\startitemize
\startitem
    At some point we started playing with \LUA\ (in \SCITE).
\stopitem
\startitem
    And then with Hartmut started adding some basic \LUA\ support to a clone of
    \PDFTEX\ that soon became \LUATEX.
\stopitem
\startitem
    Next the Oriental \TEX\ project provided means for Taco to transition to
    \CCODE .
\stopitem
\startitem
    And for years we slowly built up the system. A \LUAJITTEX\ version showed up
    and Luigi took over integration in \TEXLIVE\ (like compilation within the
    infrastructure and updating libraries).
\stopitem
\startitem
    In parallel we tested features and explored what we needed with \CONTEXT:
    \MKIV\ evolved.
\stopitem
\startitem
    And \unknown\ soon, all further development happened in \MKIV\ only: \MKII\
    became frozen.
\stopitem
\startitem
    The interface subsystem was upgraded and Wolfgang checked and completed all
    setups while we did. Obsolete (font, language, input) mechanisms were
    removed.
\stopitem
\startitem
    A lot happened: some more \TEX, lots of \LUA, better \METAPOST\ integration,
    more advanced \XML.
\stopitem
\startitem
    To some extend a project like that became to late because the glory days of
    \TEX\ were already past (publishing changed) but just as with \PDFTEX\ a
    conceptual upgrade like was felt needed.
\stopitem
\stopitemize

\stoptitle

\starttitle[title=MKXL (aka LMTX)]

\startitemize
\startitem
    When \LUATEX\ had to be frozen a follow up took place in \LUAMETATEX. The name
    reflects the importance of each core component.
\stopitem
\startitem
    The idea is to have an lean and mean engine, one that will become very stable
    and does not depend on the issues of the day.
\stopitem
\startitem
    It's for all those dedicated users who like quality and playing around but
    also want guarantees that the tools keeps working years from now: it's about
    independence.
\stopitem
\startitem
    Of course we tested and explored with \CONTEXT\ and this time \LMTX\ evolves.
    Here the \type {X} reflects that we consider \XML\ to be part of the picture.
\stopitem
\startitem
    Although there will be (and already is) new functionality the change is less
    dramatic because this we don't have the change in fonts, encoding and regime
    subsystems (which made some \MKII\ commands go away).
\stopitem
\startitem
    Hopefully some of the more tricky (hard to do in good old \TEX) mechanisms
    can be improved now.
\stopitem
\startitem
    And at some point we will freeze \MKIV\ and development will happen in \LMTX\
    only.
\stopitem
\stopitemize

\stoptitle

\starttitle[title=This meeting]

\startitemize
\startitem
    My talks in this meeting are mostly about \LUAMETATEX\ and the \CONTEXT\
    version \LMTX\ that targets it: how it is done, which concepts show up, where
    we want to go.
\stopitem
\startitem
    Unless you kept a close eye on last years development, you will encounter of
    plenty of new features that relate to \LUAMETATEX. So, there is more to tell,
    but most of that is already known from previous meetings.
\stopitem
\startitem
    And, as usual, a \CONTEXT\ meeting is not only a deadline, but also a
    starting point. It's you who keep it all going. And, even more than that,
    it is about us meeting.
\stopitem
\stopitemize

\stoptitle

\stopdocument
