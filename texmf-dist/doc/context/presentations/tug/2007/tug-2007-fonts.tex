\usemodule[present-slanted]

\usemodule[abr-01]

\definefontfeature
  [default]
  [method=node,
   script=latn,
   onum=yes]

\definefont[TitleFont]   [palatinosanscom-bold*default    at 48pt] % is used when defined
\definefont[MainTextFont][palatinosanscom-regular*default at 12pt]

\setupinterlinespace
  [line=15pt]

\def\XETEX    {\dontleavehmode{\morecolor xetex}}
\def\PDFTEX   {\dontleavehmode{\morecolor pdftex}}
\def\LUATEX   {\dontleavehmode{\morecolor luatex}}
\def\LUA      {\dontleavehmode{\morecolor lua}}
\def\FONTFORGE{\dontleavehmode{\morecolor fontforge}}
\def\OPENTYPE {\dontleavehmode{\morecolor opentype}}
\def\TFM      {\dontleavehmode{\morecolor tfm}}
\def\TEX      {\dontleavehmode{\morecolor tex}}
\def\MKIV     {\dontleavehmode{\morecolor mkiv}}

\def|#1|{-}

\startdocument
  [location={tug 2007 san diego},
   author={hans hagen},
   title={{zapfino, a},{torture test},{for luatex}}]

\StartTopic{loading fonts}

\StartSteps

\StartItem the \OPENTYPE\ font reader is borrowed from \FONTFORGE\  \FlushStep \StopItem
\StartItem once it was ready, we could look into such a font  \FlushStep \StopItem
\StartItem it tooks while to figure out the format due to rather fuzzy specs  \FlushStep \StopItem
\StartItem it took us even more time to find out that the loader was flawed  \FlushStep \StopItem
\StartItem one reason was that fonts themselves may have bugs or be incomplete \FlushStep \StopItem
\StartItem then we changed to \FONTFORGE\ version 2 \FlushStep \StopItem
\StartItem this made the missing pieces surface in more complex feature handling \FlushStep \StopItem
\StartItem while implementing features the new table format was cleaned up \FlushStep \StopItem

\StopSteps

\StopTopic

\StartTopic{making it work}

\StartSteps

\StartItem the \FONTFORGE\ library loads the fonts into memory \FlushStep \StopItem
\StartItem the internal data structures are mapped into a \LUA\ table  \FlushStep \StopItem
\StartItem we enhance this table a bit (extra hashes as well as rearranging glyphs)  \FlushStep \StopItem
\StartItem we save the table in a cache and compile it to bytecode \FlushStep \StopItem
\StartItem after that reloading is fast and efficient \FlushStep \StopItem
\StartItem when defining a font instance we share data as much as possible \FlushStep \StopItem
\StartItem the open type specific table is translated into a \TFM\ one \FlushStep \StopItem
\StartItem but for handling features we keep the original table available \FlushStep \StopItem

\StopSteps

\StopTopic

\StartTopic{user control}

\StartSteps

\StartItem in context \MKIV\ currently we provide several interfaces \FlushStep \StopItem
\StartItem the \XETEX\ syntax is (more or less) supported \FlushStep \StopItem
\StartItem there is a higher level interface to associating features with fonts  \FlushStep \StopItem
\StartItem there will be a font instance independent feature switch mechanism \FlushStep \StopItem
\StartItem there can be generic treatments unrelated to the specific font \FlushStep \StopItem
\StartItem fonts can be accessed by various names \FlushStep \StopItem
\StartItem but \unknown\ using features demands that users know what they're doing \FlushStep \StopItem

\StopSteps

\StopTopic

\StartTopic{zapfino}

\StartSteps

\StartItem we needed this exercise as a prelude to handling arab (which we cannot really read) \FlushStep \StopItem
\StartItem this font provides a good testbed because has many features defined \FlushStep \StopItem
\StartItem the associated \LUATEX\ tables are rather large \FlushStep \StopItem
\StartItem there can be hundreds of lookups per character with lookahead and backtracking \FlushStep \StopItem
\StartItem when writing support for such features, bugs in such fonts complicate matters \FlushStep \StopItem
\StartItem visually checking the output is complicated by the fact that wrong alternatives may look good \FlushStep \StopItem
\StartItem we must make sure that \TEX's other mechanisms don't spoil the game \FlushStep \StopItem
\StartItem special care is needed for lookups that involve spaces \FlushStep \StopItem

\StopSteps

\page \start

\definefontfeature[zapfino-a][mode=node,script=latn,language=dflt,liga=no]
\definefontfeature[zapfino-b][mode=node,script=latn,language=dflt,liga=yes]
\definefontfeature[zapfino-c][mode=node,script=latn,language=dflt,liga=no,clig=yes]
\definefontfeature[zapfino-d][mode=node,script=latn,language=dflt,liga=no,calt=yes]
\definefontfeature[zapfino-e][mode=node,script=latn,language=dflt,liga=yes,clig=yes,calt=yes]

\font\testa=ZapfinoExtraLTPro*zapfino-a at 24pt
\font\testb=ZapfinoExtraLTPro*zapfino-b at 24pt
\font\testc=ZapfinoExtraLTPro*zapfino-c at 24pt
\font\testd=ZapfinoExtraLTPro*zapfino-d at 24pt
\font\teste=ZapfinoExtraLTPro*zapfino-e at 24pt

\unexpanded\def\ShowSample #1 #2 %
  {\StartTopic{examples : #2}
   \start
   \noligs1\nokerns1\language0
   \setupinterlinespace[line=35pt]
   \StartItem
   \getvalue{test#1}%
   \input tufte \wordright{\morecolor Mr. E.R.~Tufte}
   \StopItem
   \stop
   \StopTopic}

\ShowSample a nothing
\ShowSample b ligatures
\ShowSample c {cont ligs}
\ShowSample d {cont alts}
\ShowSample e all of these

\page \stop

\StartTopic{arabtype}

\StartSteps

\StartItem this font has many advanced arab related features \FlushStep \StopItem
\StartItem there is quite some kerning and positioning going on \FlushStep \StopItem
\StartItem vowels travel with the base characters (glyphs) and ligatures \FlushStep \StopItem
\StartItem vowels need to be positioned relative to each other as well \FlushStep \StopItem

\StopSteps

\page

\definefontfeature
  [arab-none]
  [mode=node,language=dflt,script=arab,kern=no,liga=no,dlig=no,rlig=no]

\definefontfeature
  [arab-replace]
  [mode=node,language=dflt,script=arab,kern=no,liga=no,dlig=no,rlig=no,
   init=yes,medi=yes,fina=yes,isol=yes]

\definefontfeature
  [arab-mark]
  [mode=node,language=dflt,script=arab,kern=no,liga=no,dlig=no,rlig=no,
   init=yes,medi=yes,fina=yes,isol=yes,mark=yes]

\definefontfeature
  [arab-mkmk]
  [mode=node,language=dflt,script=arab,kern=no,liga=no,dlig=no,rlig=no,
   init=yes,medi=yes,fina=yes,isol=yes,mkmk=yes,mark=yes]

\definefontfeature
  [arab-curs]
  [mode=node,language=dflt,script=arab,kern=no,liga=no,dlig=no,rlig=no,
   init=yes,medi=yes,fina=yes,isol=yes,mkmk=yes,mark=yes,curs=yes]

\definefontfeature
  [arab-kern]
  [mode=node,language=dflt,script=arab,kern=no,liga=no,dlig=no,rlig=no,
   init=yes,medi=yes,fina=yes,isol=yes,mkmk=yes,mark=yes,kern=yes]

\definefontfeature
  [arab-context]
  [mode=node,language=dflt,script=arab,kern=no,liga=no,dlig=no,rlig=no,
   init=yes,medi=yes,fina=yes,isol=yes,mkmk=yes,mark=yes,kern=yes,calt=yes]

\definefontfeature
  [arab-ligs]
  [mode=node,language=dflt,script=arab,kern=no,liga=yes,dlig=yes,rlig=yes,
   init=yes,medi=yes,fina=yes,isol=yes,mkmk=yes,mark=yes,curs=yes]

\definefontfeature
  [arab-ligkern]
  [mode=node,language=dflt,script=arab,kern=no,liga=yes,dlig=yes,rlig=yes,
   init=yes,medi=yes,fina=yes,isol=yes,mkmk=yes,mark=yes,kern=yes,curs=yes]

\startbuffer
اَلْحَمْدُ لِلّٰهِ حَمْدَ مُعْتَرِفٍ بِحَمْدِهٖ، مُغْتَرِفٌ مِنْ بِحَارِ مَجْدِهٖ، بِلِسَانِ
الثَّنَاۤءِ شَاكِرًا، وَلِحُسْنِ اٰلاۤئِهٖ نَاشِرًا؛ اَلَّذِيْ خَلَقَ الْمَوْتَ وَالْحَيٰوةَ، وَالْخَيْرَ
وَالشَّرَّ، وَالنَّفْعَ وَالضَّرَّ، وَالسُّكُوْنَ وَالْحَرَكَةَ، وَالْأَرْوَاحَ
وَالْأَجْسَامَ، وَالذِّكْرَ وَالنِّسْيَانَ.
\stopbuffer

\unexpanded\def\ShowSample #1 #2 #3 %
  {\StartTopic{arab : #1}
   \start \setuptolerance[verytolerant,stretch]
   \noligs1\nokerns1\language0
   \setupinterlinespace[line=35pt]
   \font\Test=arabtype*#3 at 35pt \Test
   \setupinterlinespace
   \textdir TRT
   \pardir TRT
   \ifnum#2=1
     \enabletrackers[otf.analyzing]
   \fi
   \slantedshapedelta5cm
   \StartItem \getbuffer \par \StopItem
   \slantedshapedelta\zeropoint
   \ifnum#2=1
     \disabletrackers[otf.analyzing]
   \fi
   \stop
   \StopTopic}

\ShowSample analyze 1 arab-none
\ShowSample replace 0 arab-replace
\ShowSample mark    0 arab-mark
\ShowSample mkmk    0 arab-mkmk
\ShowSample curs    0 arab-curs
\ShowSample kern    0 arab-kern
\ShowSample context 0 arab-context
\ShowSample ligs    0 arab-ligs
\ShowSample ligkern 0 arab-ligkern

\StartTopic{Remark}

\StartItem
    the samples on the previous pages were generated with an experimental version
    of \LUATEX\ and \MKIV\ code and therefore may contain errors
\StopItem

\StartItem
    when this document is generated with a post-beta version of \LUATEX\ and \CONTEXT\ the results
    can be different
\StopItem

\StopTopic

\stopdocument
