\startcomponent co-en-11

\environment contextref-env
\product contextref

\chapter[descriptions]{Descriptions}

\section{Introduction}

In a document we can find text elements that bring structure
to a document. We have already seen the numbered chapter and
section titles, but there are more elements with a
recognizable layout. We can think of numbered and
non||numbered definitions, itemizations and citations. One
of the advantages of \TEX\ and therefore of \CONTEXT\ is
that coding these elements enables us to guarantee a
consistent design in our document, which in turn allows us
to concentrate on the content of our writing.

In this chapter we will discuss some of the elements that
will bring structure to your text. We advise you to
experiment with the commands and their setups. When applied
correctly you will notice that layout commands in your text
are seldom necessary.

\section[definitions]{Definitions}
\index{definitions}
\index{theses}
\macro{\tex{definedescription}}
\macro{\tex{setupdescriptions}}
\macro[name]{\tex{<<name>>}}
\macro[description]{\tex{<<description>>}}
\macro[startdescription]{\tex{start<<description>>}}

Definitions of concepts and|/|or ideas, that are to be
typeset in a distinctive way, can be defined by \type
{\definedescription}.

\showsetup{definedescription}

The first argument of this command contains the name. After
the definition a new command is available.

\showsetup{<<description>>}

\startbuffer[examp-1]
\definedescription[definition][location=top,headstyle=bold]

\definition{icon}

An icon is a representation of an action or the name of a computer
program. Icons are frequently used in operating systems on several
computer platforms. \par
\stopbuffer

\startbuffer[examp-2]
\definedescription[definition][location=right,headstyle=bold]
\definition{icon}

Some users of those computer platforms are using these icons with an
almost religious fanaticism. This brings the word icon almost back
to its original meaning. \par
\stopbuffer

\startbuffer[examp-3]
\definedescription[definition][location=left,headstyle=bold]
\definition{icon}

An icon should be recognizable for every user but they are designed
within a cultural and historical setting. In this fast and ever changing
era the recognizability of icons is relative. \par
\stopbuffer

\startbuffer[examp-4]
\definedescription[definition][location=inmargin,headstyle=bold]
\definition{icon}

The 8||bit principle of computers was the reason that non||Latin
scriptures were hardly supported by the operating systems. Not long
ago this changed. \par
\stopbuffer

\startbuffer[examp-5]
\definedescription
  [definition]
  [location=serried,headstyle=bold,width=broad,sample={icon}]

\definition{icon}

What for some languages looked like a handicap has now become a feature.
Thousands of words and concepts are already layed down in characters.
These characters therefore can be considered icons. \par
\stopbuffer

\startbuffer[examp-6]
\definedescription[definition][location=serried,headstyle=bold]
\definition{icon}

It is to be expected that people with expressive languages overtake
us in computer usage because they are used to thinking in concepts. \par
\stopbuffer

\startbuffer[examp-7]
\definedescription[definition][location=left,headstyle=bold,hang=broad]
\definition{icon}

The not||so||young generation remembers the trashcan in the earlier
operating systems used to delete files. We in Holland were lucky that the
text beneath it said: trashcan. A specific character for the trashcan
would have been less sensitive misinterpretation, than the rather
American||looking garbage receptacle unknown to many young people.\par
\stopbuffer

An example of the definition is:

\startexample
\typebuffer[examp-1]
\stopexample

Several alternatives are displayed below:

\startreality \getbuffer[examp-1] \stopreality
\startreality \getbuffer[examp-2] \stopreality
\startreality \getbuffer[examp-3] \stopreality
\startreality \getbuffer[examp-4] \stopreality
\startreality \getbuffer[examp-5] \stopreality
\startreality \getbuffer[examp-6] \stopreality
\startreality \getbuffer[examp-7] \stopreality

In the fifth example the definition is placed \type
{serried} and defined as:

\startexample
\typebuffer[examp-5]
\stopexample

In the seventh example we have set \type {hang} at \type
{broad}. This parameter makes only sense when we set the
label at the right or left. When we set \type {width} at
\type {fit} or \type {broad} instead of a number, the width
of the sample is used. With \type {fit}, no space is added,
with \type {broad}, a space of \type {distance} is inserted.
When no sample is given the with of the defined word is
used. The parameter \type {align} specifies in what way the
text is aligned. When the definition is placed in the margin
or typeset in a serried format, the parameter \type {margin}
is of importance. When set to \type {standard} or \type {ja},
the marging follows the document setting. Alternatively you
can pass a dimension.

Some characteristics of the description can be specified with:

\showsetup{setupdescriptions}

The setup of a description can be changed with the command
below. This has the same construct as \type
{\definedescription}:

\startexample
\starttyping
\setupdescriptions[<<name>>][<<setups>>]
\stoptyping
\stopexample

When a description consists of more than one paragraph, use:

\showsetup{start<<description>>}

\startexample
\starttyping
\startdefinition{icon}

An icon is a painting of Jesus Christ, Mother Mary or other holy figures.
These paintings may have a special meaning for some religious people.

For one reason or the other the description icon found its way to the
computer world where it leads its own life.

\stopdefinition
\stoptyping
\stopexample

These commands will handle empty lines adequately.

\section[enumeration]{Enumeration}
\index{descriptions}
\index{itemization}
\index{enumeration+texts}
\macro{\tex{enumeration}}
\macro{\tex{setupenumerations}}
\macro[name]{\tex{<<name>>}}
\macro[subname]{\tex{sub<<name>>}}
\macro[subsubname]{\tex{subsub<<name>>}}
\macro[subsubsubname]{\tex{subsubsub<<name>>}}
\macro[resetname]{\tex{reset<<name>>}}
\macro[nextname]{\tex{next<<name>>}}
\macro[nextsubname]{\tex{nextsub<<name>>}}
\macro[nextsubsubname]{\tex{nextsubsub<<name>>}}
\macro[enumeration]{\tex{<<enumeration>>}}
\macro[startenumeration]{\tex{start<<enumeration>>}}

Sometimes you will encounter text elements you would like to
number, but they do not fit into the category of figures,
tables, etc. Therefore \CONTEXT\ has a numbering mechanism
that we use for numbering text elements like questions,
remarks, examples, etc. Such a text element is defined with:

\showsetup{defineenumeration}

After such a definition, the following commands are
available:

\startexample
\starttyping
\<<name>>
\sub<<name>>
\subsub<<name>>
\subsubsub<<name>>
\stoptyping
\stopexample

Where {\sl \type{name}} stands for any chosen name.

\showsetup{<<enumeration>>}

The numbering can take place at four levels. Conversion is
related to the last level. If you specify a text, then this
will be a label that preceeds every generated number. A
number can be set and reset with the command:

\startexample
\starttyping
\set<<enumeration>>{value}
\reset<<enumeration>>
\stoptyping
\stopexample

You can use the \type {start} parameter in the setup command
to explictly state a startnumber. Keep in mind that the
enumeration commands increase the number, so to start at~4,
one must set the number at~3. Numbers and subnumbers and be
explictly increased with the commands:

\startexample
\starttyping
\next<<enumeration>>
\nextsub<<enumeration>>
\nextsubsub<<enumeration>>
\stoptyping
\stopexample

The example below illustrates the use of \type
{\enumeration}. After the shown commands the content of a
remark can be typed after \type{\remark}.

\startexample
\starttyping
\defineenumeration
  [remark]
  [location=top,
   text=Remark,
   between=\blank,
   before=\blank,
   after=\blank]
\stoptyping
\stopexample

\defineenumeration
  [remark]
  [location=top,
   text=Remark,
   before=\blank,
   between=\blank,
   after=\blank]

Some examples of remarks are:

\remark After definition the \quote {remark} is available at
four levels: \type {\remark}, \type {\subremark}, \type
{\subsubremark} and \type {\subsubsubremark}.

\remark This command looks much like the command \type
{\definedescription}.

The characteristics of numbering are specified with \type
{\setupenumerations}. Many parameters are like that of the
descriptions because numbering is a special case of
descriptions.

\startexample
\starttyping
\setupenumerations[<<name>>][<<setups>>]
\stoptyping
\stopexample

\showsetup{setupenumerations}

The characteristics of sub and subsub enumerations can be
set too. For example:

\startexample
\starttyping
\setupenumerations[example][headstyle=bold]
\setupenumerations[subexample][headstyle=slanted]
\stoptyping
\stopexample

Just like the description command there is a \type
{\start}||\type {\stop} construction for multi paragraph
typesetting.

\showsetup{start<<enumeration>>}

Sometimes the number is obsolete. For example when we number
per chapter and we have only {\em one} example in a specific
chapter. In that case you can indicate with a \type{[-]}
that you want no number to be displayed.

\remark[-] Because this remark was recalled by \type
{\remark[-]} there is {\em no} number. Just as with other
commands, we can also pass a reference label between
\setchars. Also, we can setup the enumeration to stop
numbering by setting \type {number} to \type {no}.

The numbering command can be combined usefully with the
feature to move textblocks. An example is given in
\in{section}[textblocks]. In that example we also
demonstrate how to couple one numbered text to another.
These couplings only have a meaning in interactive documents
where cross references (hyperlinks) can be useful.

The numbering of text elements can appear in different
forms. In that case we can let one numbered text element
inherit its characteristic from another. We illustrate this
in an example.

\startbuffer
\defineenumeration[first]

\first  The numbering \type {first} is unique. We see that one
argument is sufficient. By default label and number are placed at the left
hand side.

\defineenumeration[second][first][location=right]

\second  The \type {second} inherits its counters from \type {first},
but is placed at the right hand side. In case of three arguments the first
one is the copy and the second the original.

\defineenumeration[third,fourth][location=inright]

\third The numbered elements \type {third} and \type {fourth} are both
unique and are placed in right margin.

\fourth  Both are defined in one command but they do have own
counters that are in no way coupled.

\defineenumeration[fifth,sixth][first]

\fifth The elements \type {fifth} and \type {sixth} inherit the properties
and counters of \type {first}.

\sixth  Note: inheriting of \type{second} is not allowed because \type
{second} is not an original! \par
\stopbuffer

\startexample
\typebuffer
\stopexample

It may seem very complex but the text below may shed some
light on this issue:

\getbuffer

It is possible to couple a numbered text element to another.
For example we may couple questions and answers. In an
interactive document we can click on a question which will
result in a jump to the answer. And vice versa. The counters
must be synchronised. Be aware of the fact that the counters
need some resetting now and then. For example at the
beginning of each new chapter. This can be automated by
setting the parameter \type {way} to \type {bychapter}.

\starttyping
\definedescription [question] [coupling=answer]
\definedescription [answer]   [coupling=question]
\stoptyping

\section[indenting]{Indenting}
\index{tabulate}
\index{indenting}
\macro{\tex{indentation}}
\macro{\tex{setupindentations}}
\macro[name]{\tex{<<name>>}}
\macro[indentation]{\tex{<<indentation>>}}

Indented itemizations, like dialogues, can be typeset with
the command defined by

\showsetup{defineindenting}

After this command \type{\<<name>>}, \type{\sub<<name>>} and
\type{\subsub<<name>>} are available.

\showsetup{<<indentation>>}

The parameters can be set up with the command:

\showsetup{setupindentations}

It is possible to change the setup of \type{\indentation}
with the command:

\startexample
\starttyping
\setupindentations[<<name>>][<<setups>>]
\stoptyping
\stopexample

An example of how you can use the indentation mechanism is
given below:

\startbuffer
\setupindentations
  [sample={rime m},
   separator={:},
   distance=.5em]

\defineindenting[ra][text=rime a]
\defineindenting[rb][text=rime b]
\defineindenting[rc][text=rime c]

\startpacked
\ra  pretty litte girl \par
\ra  pretty litte girl in a blue dress \par
\rb  pretty little girl in a blue dress \par
\rc  playing in the sand \par
\rb  make my day \par
\rc  smile for me \par
\stoppacked
\stopbuffer

\startexample
\typebuffer
\stopexample

This results in:

\startreality
\getbuffer
\stopreality

A series of indenting commands can be enclosed with the
commands:

\startexample
\starttyping
\startindentation
\stopindentation
\stoptyping
\stopexample

\section[label]{Numbered labels}
\index{label}
\index{numbering+label}
\macro{\tex{label}}
\macro[resetname]{\tex{reset<<name>>}}
\macro[incrementname]{\tex{increment<<name>>}}
\macro[nextname]{\tex{next<<name>>}}
\macro[currentname]{\tex{current<<name>>}}
\macro[label]{\tex{<<label>>}}

There is another numbering mechanism that is used for
numbering specific text labels that also enables you to
refer to these labels. For example, when you want to refer
in your text to a number of transparencies that you use in
presentations the next command can be used:

\showsetup{definelabel}

Where the parameter \type {location} is set at \type
{intext} and \type {inmargin}. After this definition the
following commands are available:

\startexample
\starttyping
\reset<<name>>
\increment<<name>>
\next<<name>>
\current<<name>>[reference]
\stoptyping
\stopexample

\startbuffer
\definelabel[video][text=video,location=inmargin]
\stopbuffer

\getbuffer

The \type {[reference]} after \type {currentname} is optional.
After

\typebuffer

This defines \video\type {\video}, that results in a
numbered label {\em video} in the margin. The command \type
{\currentvideo} would have resulted in the number~0. The
label can also be recalled with:

\showsetup{<<labeling>>}

In our case, saying \type {\video} results in the marginal
note concerning a video. The values of \type {before} and
\type {after} are executed around the label (which only makes
sense for in||text labels.

\section[itimize]{Itemize}
\index{itemize}
\index{numbering+itemize}
\index{items}
\index{questionnaire}
\index{forms}
\macro{\tex{startitemize}}
\macro{\tex{setupitemize}}
\macro{\tex{mar}}
\macro{\tex{nop}}
\macro{\tex{item}}
\macro{\tex{head}}
\macro{\tex{sub}}
\macro{\tex{its}}
\macro{\tex{sym}}
\macro{\tex{ran}}
\macro{\tex{but}}

Items in an itemization are automatically preceded by
symbols or by enumerated numbers or characters. The symbols
and the enumeration can be set up (see \in {table}
[tab:item]). The layout can also be influenced. Itemization
has a maximum of four levels.

\placetable
  [here][tab:item]
  {Item separator identifications in itemizations.}
\startcombination[2]
  {\starttable[|c|c|]
   \HL
   \VL \bf setup      \VL \bf result                             \VL\SR
   \HL
   \VL \type{n}       \VL 1, 2, 3, 4                             \VL\FR
   \VL \type{a}       \VL a, b, c, d                             \VL\MR
   \VL \type{A}       \VL A, B, C, D                             \VL\MR
   \VL \type{KA}      \VL \kap{A}, \kap{B}, \kap{C}, \kap{D}     \VL\MR
   \VL \type{r}       \VL i, ii, iii, iv                         \VL\MR
   \VL \type{R}       \VL I, II, III, IV                         \VL\MR
   \VL \type{KR}      \VL \kap{I}, \kap{II}, \kap{III}, \kap{IV} \VL\MR
   \VL \type{m}       \VL {\os 1}, {\os 2}, {\os 3}, {\os 4}     \VL\MR
   \VL \type{g}       \VL $\alpha$, $\beta$, $\gamma$            \VL\MR
   \VL \type{G}       \VL A, B, $\Gamma$                         \VL\LR
   \HL
   \stoptable} {}
  {\starttable[|c|c|]
   \HL
   \VL \bf setup      \VL \bf result                  \VL\SR
   \HL
   \VL \type{1}       \VL dot (\symbol[1])            \VL\FR
   \VL \type{2}       \VL dash (\symbol[2])           \VL\MR
   \VL \type{3}       \VL star (\symbol[3])           \VL\MR
   \VL \type{4}       \VL triangle (\symbol[4])       \VL\MR
   \VL \type{5}       \VL circle (\symbol[5])         \VL\MR
   \VL \type{6}       \VL big circle (\symbol[6])     \VL\MR
   \VL \type{7}       \VL bigger circle (\symbol[7])  \VL\MR
   \VL \type{8}       \VL square (\symbol[8])         \VL\MR
   \VL                \VL                             \VL\MR
   \VL                \VL                             \VL\LR
   \HL
   \stoptable} {}
\stopcombination

The command to itemize is:

\startexample
\starttyping
\startitemize[<<setups>>]
\item ........
\item ........
\stopitemize
\stoptyping
\stopexample

So you can do things like this:

\startbuffer
Which of these theses are true?

\startitemize[A]
\item The difference between a village and a city is the existence of
      a townhall.
\item The difference between a village and a city is the existence of
      a courthouse.
\stopitemize
\stopbuffer

\startexample
\typebuffer
\stopexample

This will lead to:

\startreality
\getbuffer
\stopreality

The symbols used under \type{1} to~\type{8} can be defined
with the command \type {\definesymbol} (see \in {section}
[symbols]) and the conversion of the numbering with \type
{\defineconversion} (see \in {section} [convert]). For
example:

\startbuffer
Do the following propositions hold some truth?

\definesymbol[1][$\diamond$]

\startitemize[1]
\item The city of Amsterdam is built on wooden poles.
\item The city of Rome was built in one day.
\stopitemize
\stopbuffer

\startexample
\typebuffer
\stopexample

results in:

\startreality
\getbuffer
\stopreality

The keys \type{n}, \type{a}, etc. are related to the
conversions. This means that all conversions are accepted.
Take for example:

\startitemize[g,packed]
\item a \type{g} for Greek characters
\item a \type{G} for Greek capitals
\stopitemize

When the setup and the \setchars\ are left out then the
default symbol is typeset.

The indentation and horizontal whitespace is set up locally
or globally with:

\showsetup{setupitemize}

These arguments may appear in different combinations, like:

\startbuffer
What proposition is true?

\startitemize[a,packed][stopper=:]
\item 2000 is a leap-year
\item 2001 is a leap-year
\item 2002 is a leap-year
\item 2003 is a leap-year
\stopitemize
\stopbuffer

\startexample
\typebuffer
\stopexample

this will become:

\startreality
\getbuffer
\stopreality

Both argument are optional. The key \type {packed} is one
of the most commonly used:

\startbuffer
What proposition is true?

\startitemize[n,packed,inmargin]
\item[ok] 2000 is a leap-year
\item 2001 is a leap-year
\item 2002 is a leap-year
\item 2003 is a leap-year
\stopitemize
\stopbuffer

\startexample
\typebuffer
\stopexample

will result in:

\startreality
\getbuffer
\stopreality

It happens very often that an itemization is preceded by a
sentence like \quotation {\em\unknown\ can be seen below:}.
In that case we add the key \type {intro} and the
introduction sentence will be \quote {connected} to the
itemization. After this setup a pagebreak between sentence
and itemization is discouraged.

\startexample
\starttyping
\startitemize[n,packed,inmargin,intro]
\stoptyping
\stopexample

The setup of the itemization commands are presented in
\in{table}[tab:setupitemize].

\placetable
  [here][tab:setupitemize]
  {Setup of \tex{setupitemize}.}
\starttable[|l|l|]
\HL
\VL \bf setup            \VL \bf result                                  \VL\SR
\HL
\VL \type{standard}      \VL default setup                               \VL\FR
\VL \type{packed}        \VL no white space between items                \VL\MR
\VL \type{joinedup}      \VL no white space before and after itemization \VL\MR
\VL \type{paragraph}     \VL no white space before an itemization        \VL\MR
\VL \type{<<n>>*serried} \VL little horizontal white space after symbol  \VL\MR
\VL \type{<<n>>*broad}   \VL extra horizontal white space after symbol   \VL\MR
\VL \type{inmargin}      \VL item separator in margin                    \VL\MR
\VL \type{atmargin}      \VL item separator at the margin                \VL\MR
\VL \type{stopper}       \VL punctuation after item separator            \VL\MR
\VL \type{intro}         \VL no pagebreak                                \VL\MR
\VL \type{columns}       \VL two columns                                 \VL\LR
\HL
\stoptable

In the last example we saw a reference point behind the
command \type{\item} for future cross referencing. In this
case we could make a cross reference to \in{answer}[ok] with
the command \type{\in[ok]}.

The enumeration may be continued  by adding the key
\type{continue}, for example:

\startbuffer
\startitemize[continue]
\item 2005 is a leap-year
\stopitemize
\stopbuffer

\startexample
\typebuffer
\stopexample

This would result in a rather useless addition:

\startreality
\getbuffer
\stopreality

Another example illustrates that \type {continue}
even works at other levels of itemizations:

\startbuffer
\startitemize[1,packed]
\head  supported image formats in \PDFTEX \par
      \startitemize[a]
      \item png \item eps \item pdf
      \stopitemize
\head  non supported image formats in \PDFTEX \par
      \startitemize[continue]
      \item jpg \item gif \item tif
      \stopitemize
\stopitemize
\stopbuffer

\startreality
\getbuffer
\stopreality

This was typed as (in this document we have set \type
{headstyle=bold}):

\startexample
\typebuffer
\stopexample

When we use the key \type {columns} the items are typeset
in two columns. The number of columns can be set by the keys
\type{one}, \type{two} (default), \type{three} or
\type{four}.

\startbuffer
\startitemize[n,columns,four]
\item png \item tif \item jpg \item eps \item pdf
\item gif \item pic \item bmp \item bsd \item jpe
\stopitemize
\stopbuffer

\startexample
\typebuffer
\stopexample

We can see that we can type the items at our own preference.

\startreality
\getbuffer
\stopreality

In such a long enumerated list the horizontal space between
itemseparator and text may be too small. In that case we use
the key \type {broad}, here \type {2*broad}:

\startbuffer
\startitemize[R,columns,four,2*broad]
\item png \item tif \item jpg \item eps \item pdf
\item gif \item pic \item bmp \item bsd \item jpe
\stopitemize
\stopbuffer

\startreality
\getbuffer
\stopreality

The counterpart of \type {broad} is \type{serried}. We can
also add a factor. Here we used \type {2*serried}.

\startbuffer
\startitemize[2*serried,packed]
\item What format is this?
\stopitemize
\stopbuffer

\startreality
\getbuffer
\stopreality

We can abuse the key \type {broad} for very simple tables.
It takes some guessing to reach the right spacing.

\startbuffer
\startitemize[4*broad,packed]
\sym {yes} this is a nice format
\sym {no}  this is very ugly
\stopitemize
\stopbuffer

This results in a rather strange example:

\startexample
\typebuffer
\stopexample

\startreality
\getbuffer
\stopreality

The parameter \type {stopper} expects a character of your own
choice. By default it is set at a period. When no level is
specified and the \setchars\ are empty the actual level is
activated. In \in {section} [symbols] we will discuss this
in more detail. Stoppers only apply to ordered (numbered)
list.

There are itemizations where a one line head is followed by
a text block. In that case you use \type{\head} instead of
\type {\item}. You can specify the layout of \type {\head}
with the command \type {\setupitemize}. For example:

\startbuffer
\setupitemize[each][headstyle=bold]

\startitemize[n]

\head A title head in an itemization

      After the command \type{\head} an empty line is mandatory. If you
      leave that out you will get a very long header.

\stopitemize
\stopbuffer

\startexample
\typebuffer
\stopexample

This becomes:

\startreality
\getbuffer
\stopreality

If we would have used \type{\item} the head would have been
typeset in a normal font. Furthermore a pagebreak could
have been introduced between head and textblock. This is not
permitted when you use \type {\head}.

\showsetup{head}

\startmode[mkii]
When an itemization consists of only one item we can leave
out the commands \type {\startitemize} and \type
{\stopitemize} and the level~1 symbol is used.

\startbuffer
\item The itemization commands force the user into a consistent layout
      of the itemizations. \par
\stopbuffer

\startexample
\typebuffer
\stopexample

Instead of the \type{\par} you could have used an empty line.
In each case, we get the following output:

\startreality
\getbuffer
\stopreality

Only the text directly following the command and ended by an
empty line or \type{\par} is indented.
\stopmode

When you want to re-use the last number instead of
increasing the next item you can use \type{\sub}.
This feature is used in discussion documents where earlier
versions should not be altered too much for reference
purposes.

\startreality
\startitemize[n,packed]
\item This itemization is preceded by \type
      {\startitemize[n,packed]}.
\sub  This item is preceded by \type{\sub}, the other items
      by \type{\item}.
\item The itemization is ended by \type
      {\stopitemize}.
\stopitemize
\stopreality

The most important commands are:

\showsetup{startitemize}

\showsetup{item}

\showsetup{sub}

In addition to \type{\item} there is \type{\sym}. This
command enables us to type an indented text with our own
symbol.

\showsetup{sym}

Another alternative to \type{\item} is \type{\mar}.
The specified argument is set in the margin (by default a
typeletter) and enables us to comment on an item.

\showsetup{mar}

Some at first sight rather strange alternatives are:

\showsetup{its}

\showsetup{ran}

These acronyms are placeholders for \type {items} and \type
{range}. We illustrate most of these commands with an
example that stems from a \NTG\ questionnaire:

\startbuffer
\startitemize[5,packed][width=8em,distance=2em,items=5]

\ran {no\hss yes}

\its I can not do without \TeX.
\its I will use \TeX\ forever.
\its I expect an alternative to \TeX\ in the next few years.
\its I use \TeX\ and other packages.
\its I hardly use \TeX.
\its I am looking for another system.

\stopitemize
\stopbuffer

\startreality
\getbuffer
\stopreality

The source is typed below. Look at the setup, it is local.

% Dit ziet er in broncode als volgt uit. Let op de
% instellingen. Het verdient de voorkeur dergelijke
% instellingen los te koppelen van de vragen en slechts
% eenmalig te doen.

\startexample
\typebuffer
\stopexample

For the interactive version there is:

\showsetup{but}

This command resembles \type {\item} but produces an
interactive symbol that executes the reference sequence
specified.

The example below shows a combination of the mentioned
commands. We also see the alternative \type {\nop}.

\startbuffer
\startitemize
\head  he got a head ache

      \startitemize[n,packed]
      \item     of all the items
      \nop      he had to learn at school
      \mar{++}  because the marginal explanation
      \sub      of the substantial content
      \sym{\#}  turned out to be mostly symbolic
      \stopitemize
\stopitemize
\stopbuffer

\startreality
\getbuffer
\stopreality

This list was typed like this:

\startexample
\typebuffer
\stopexample

With the no||operation command:

\showsetup{nop}

During the processing of itemizations the number of items is
counted. This is the case with all versions. The next pass
this information is used to determine the optimal location
to start a new page. So do not despair when at the first
parse your itemizations do not look the way you expected.
When using \TEXEXEC\ this is all taken care of.

We have two last pieces of advises. When items consist of
two or more paragraphs always use \type {\head} instead of
\type {\item}, especially when the first paragraph consists
only one line. The command \type{\head} takes care of
adequate pagebreaking between two paragraphs. Also, always
use the key \type {[intro]} when a one line sentence
preceeds the itemization. This can be automated by:

\starttyping
\setupitemize[each][autointro]
\stoptyping

\section[items]{Items}
\index{items}
\index{questionnaire}
\index{itemize}
\index{forms}
\index{lists}
\macro{\tex{items}}
\macro{\tex{setupitems}}

A rarely used variant of producing lists is the command
\type {\items}. It is used to produce simple, one level,
vertical or horizontal lists. The command in its simplest
form looks like this:

\startexample
\starttyping
\items{<<alternative 1>>,<<alternative 2>>,...,<<alternative N>>}
\stoptyping
\stopexample

Instead of an alternative you can also type \type{-}. In
that case space is reserved but the item is not set. The
layout of such a list is set with the command:

\showsetup{setupitems}

The number (\type{n}) as well as the width are calculated
automatically. When you want to do this yourself you can use
the previous command or you pass the options directly. We
show some examples.

\startbuffer
\items[location=left]{png,eps,pdf}
\stopbuffer

\startexample
\typebuffer
\stopexample

\getbuffer

\startbuffer
\items[location=bottom]{png,eps,pdf}
\stopbuffer

\startexample
\typebuffer
\stopexample

\getbuffer

\startbuffer
\items[location=right,width=2cm]{png,eps,pdf}
\stopbuffer

\startexample
\typebuffer
\stopexample

\getbuffer

\startbuffer
\items[location=top,width=6cm,align=left]{png,eps,pdf}
\stopbuffer

\startexample
\typebuffer
\stopexample

\getbuffer

\startbuffer
\items[location=inmargin]{png,eps,pdf}
\stopbuffer

\startexample
\typebuffer
\stopexample

\getbuffer

\startbuffer
\items[location=left,n=2,symbol=5]{jpg,tif}
\stopbuffer

\startexample
\typebuffer
\stopexample

\getbuffer

\startbuffer
\items[symbol=3,n=6,width=\hsize,location=top]{png,eps,pdf,jpg,tif}
\stopbuffer

\startexample
\typebuffer
\stopexample

\getbuffer

The setup just after \type{\items} have the same effect as
those of \type{\setupitems}:

\showsetup{items}

\section[citations]{Citations}
\index{citation}
\index{quotation}
\index{language+quotes}
\macro{\tex{quote}}
\macro{\tex{quotation}}
\macro{\tex{setupquotation}}
\macro{\tex{startquotation}}

The use of quotes depends on the language of a country:
{\mainlanguage [nl]\quote {Nederlands},
\mainlanguage [en]\quote {English},
\mainlanguage [de]\quote {Deutsch},
\mainlanguage [fr]\quote {Fran\cc ais}}.
The consistent use of single and double quotes is supported
by a number of commands. A citation in the running text is
typeset by:

\showsetup{startquotation}

This command can be compared with \type {\startnarrower} and
has the same setup parameters. The quotes are placed around
the text and they fall outside the textblock:

\startbuffer
\startquotation
In commercial advertising \quote {experts} are quoted. Not too
long ago I saw a commercial where a washing powder was recommended
by the Dutch Society of Housewives. The remarkable thing was that
there was a spokesman and not a spokeswoman. He was introduced as
the \quotation {director}. It can't be true that the director of the
Society of Housewives is a man. Can it?
\stopquotation
\stopbuffer

\startreality
\getbuffer
\stopreality

In this example we see two other commands:

\startexample
\typebuffer
\stopexample

The command \type {\quotation} produces double quotes and
\type {\quote} single quotes.

\showsetup{quote}

\showsetup{quotation}

These commands adapt to the language. In Dutch, English,
German and French texts other quotes are activated.
The body font is set with:

\showsetup{setupquote}

The location of a period, inside or outside a citation is
somewhat arbitrary. The opinions on this issue differ
considerately.

\startbuffer
He said: \quotation {That is a bike} to which she replied:
\quotation {Take a hike}.
\stopbuffer

\getbuffer

The quotes are language dependent. Therefore it is of some
importance that language switching is done correctly.

\startbuffer
\quotation {He answered: \fr \quotation {Je ne parle pas fran\c cais}.}
\quotation {He answered: \quotation {\fr Je ne parle pas fran\c cais}.}
\quotation {\fr Il r\'epondait: \quotation{Je ne parle pas fran\c cais}.}
\fr \quotation {Il r\'epondait: \quotation{Je ne parle pas fran\c cais}.}
\stopbuffer

\typebuffer

Watch the subtle difference.

\startlines
\getbuffer
\stoplines

When we want different quotes, we can change them. This is a
language related setting.

\starttyping
\setuplanguage
  [en]
  [leftquote=\upperleftsinglesixquote,
   leftquotation=\upperleftdoublesixquote]
\stoptyping

Fo rconsistency, such a setting can best be put into the
local system file \type {cont-sys.tex}, together with other
local settings. The following quotes are available:

\def\ToonQuotes#1#2%
  {\NC \tttf \string#1 \NC {#1\showstruts\strut}
   \NC \tttf \string#2 \NC {\showstruts\strut#2} \NC \NR}

\starttabulate[|l|c|l|c|]
\ToonQuotes \lowerleftsingleninequote \lowerrightsingleninequote
\ToonQuotes \lowerleftdoubleninequote \lowerrightdoubleninequote
\ToonQuotes \upperleftsingleninequote \upperrightsingleninequote
\ToonQuotes \upperleftdoubleninequote \upperrightdoubleninequote
\ToonQuotes \upperleftsinglesixquote  \upperrightsinglesixquote
\ToonQuotes \upperleftdoublesixquote  \upperrightdoublesixquote
\stoptabulate

\stopcomponent
