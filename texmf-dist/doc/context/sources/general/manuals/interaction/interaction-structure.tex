% language=uk

\environment interaction-style

\startcomponent interaction-structure

\startchapter[title=Structure]

There is a lot of structure in \CONTEXT:

\startitemize[packed]
\startitem document structure: projects, products, components, environments \stopitem
\startitem sectioning, with or without numbers (visible), support for lists and userdata \stopitem
\startitem lists, most often related to sections, but there are more \stopitem
\startitem registers \stopitem
\startitem itemized lists \stopitem
\startitem images, \METAPOST\ graphics, different types of tables \stopitem
\startitem typographical objects: constructions, descriptions and enumerations \stopitem
\startitem notes, like footnotes, endnotes, linenotes \stopitem
\startitem marginal notes \stopitem
\startitem formulas (and subformulas) \stopitem
\startitem text areas, layers, overlays \stopitem
\startitem graphic placement with captions and references \stopitem
\startitem cross references to most structural components \stopitem
\startitem bibliographic databases and citations \stopitem
\blank
\startitem \unknown\ and more \unknown \stopitem
\stopitemize

Most of them in some way carry information about their location in the document
and on the page, and sometimes their exact position. This also means that we can
use that information for annotations. But most users will use the standard
functionality.

\starttyping
\startsection[title=Whatever]
    ...
\stopsection
\stoptyping

In addition to typesetting this will add the title to a list. In order to do that
some anchor has to be placed in the text, because we need to register the exact
location in order to get the right pagenumber after \TEX\ has broken the flow into
pages.

\starttyping
\placelist[section][criterium=text]
\stoptyping

This will place a list of all sections. If you want the whole entry to be a
clickable areas, you can say:

\starttyping
\placelist[section][interaction=all]
\stoptyping

Otherwise only clicking on the title will bring you to the spot. If you also say:

\starttyping
\setuphead[interaction=list]
\stoptyping

Clicking on the head will bring you back to the table of contents. There are
special list rendering alternatives for interactive documents (\typ
{alternative=e} onwards). You can use the \type {list} and \type {bookmark}
parameters to a section head to deviate from the given \type {title}.

Many commands accept a \type {reference} as optional argument and when you use
an alternative with key|/|values a \type {reference} key will do the job.

{\em What should go into this chapter.}

\stopchapter

\stopcomponent

