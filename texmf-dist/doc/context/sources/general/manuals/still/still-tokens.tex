% language=us

\environment still-environment

\starttext

\startchapter[title=Scanning input]

\startsection[title=Introduction]

Tokens are the building blocks of the input for \TEX\ and they drive the process
of expansion which in turn results in typesetting. If you want to manipulate the
input, intercepting tokens is one approach. Other solutions are preprocessing or
writing macros that do something with their picked|-|up arguments. In \CONTEXT\
\MKIV\ we often forget about manipulating the input but manipulate the
intermediate typesetting results instead. The advantage is that only at that
moment do you know what you're truly dealing with, but a disadvantage is that
parsing the so-called node lists is not always efficient and it can even be
rather complex, for instance in math. It remains a fact that until \LUATEX\
version 0.80 \CONTEXT\ hardly used the token interface.

In version 0.80 a new scanner interface was introduced, demonstrated by Taco
Hoekwater at the \CONTEXT\ conference 2014. Luigi Scarso and I integrated that
code and I added a few more functions. Eventually the team will kick out the old
token library and overhaul the input|-|related code in \LUATEX, because no
callback is needed any more (and also because the current code still has traces
of multiple \LUA\ instances). This will happen stepwise to give users who use the
old mechanism an opportunity to adapt.

Here I will show a bit of the new token scanners and explain how they can be used
in \CONTEXT. Some of the additional scanners written on top of the built|-|in ones
will probably end up in the generic \LUATEX\ code that ships with \CONTEXT.

\stopsection

\startsection[title=The \TEX\ scanner]

The new token scanner library of \LUATEX\ provides a way to hook \LUA\ into \TEX\
in a rather natural way. I have to admit that I never had any real demand for
such a feature but now that we have it, it is worth exploring.

The \TEX\ scanner roughly provides the following sub-scanners that are used to
implement primitives: keyword, token, token list, dimension, glue and integer.
Deep down there are specific variants for scanning, for instance, font dimensions
and special numbers.

A token is a unit of input, and one or more characters are turned into a token.
How a character is interpreted is determined by its current catcode. For instance
a backslash is normally tagged as `escape character' which means that it starts a
control sequence: a macro name or primitive. This means that once it is scanned a
macro name travels as one token through the system. Take this:

\starttyping
\def\foo#1{\scratchcounter=123#1\relax}
\stoptyping

Here \TEX\ scans \type {\def} and turns it into a token. This particular token
triggers a specific branch in the scanner. First a name is scanned with
optionally an argument specification. Then the body is scanned and the macro is
stored in memory. Because \type {\scratchcounter}, \type
{\relax} and \type {#1} are
turned into tokens, this body has 7~tokens.

When the macro \type {\foo} is referenced the body gets expanded which here means
that the scanner will scan for an argument first and uses that in the
replacement. So, the scanner switches between different states. Sometimes tokens
are just collected and stored, in other cases they get expanded immediately into
some action.

\stopsection

\startsection[title=Scanning from \LUA]

The basic building blocks of the scanner are available at the \LUA\ end, for
instance:

\starttyping
\directlua{print(token.scan_int())} 123
\stoptyping

This will print \type {123} to the console. Or, you can store the number and
use it later:

\starttyping
\directlua{SavedNumber = token.scan_int())} 123

We saved: \directlua{tex.print(SavedNumber)}
\stoptyping

The number of scanner functions is (on purpose) limited but you can use them to
write additional ones as you can just grab tokens, interpret them and act
accordingly.

The \type {scan_int} function picks up a number. This can also be a counter, a
named (math) character or a numeric expression. In \TEX, numbers are integers;
floating|-|point is not supported naturally. With \type {scan_dimen} a dimension
is grabbed, where a dimen is either a number (float) followed by a unit, a dimen
register or a dimen expression (internally, all become integers). Of course
internal quantities are also okay. There are two optional arguments, the first
indicating that we accept a filler as unit, while the second indicates that math
units are expected. When an integer or dimension is scanned, tokens are expanded
till the input is a valid number or dimension. The \type {scan_glue} function
takes one optional argument: a boolean indicating if the units are math.

The \type {scan_toks} function picks up a (normally) brace|-|delimited sequence of
tokens and (\LUATEX\ 0.80) returns them as a table of tokens. The function \type
{get_token} returns one (unexpanded) token while \type {scan_token} returns
an expanded one.

Because strings are natural to \LUA\ we also have \type {scan_string}. This one
converts a following brace|-|delimited sequence of tokens into a proper string.

The function \type {scan_keyword} looks for the given keyword and when found skips
over it and returns \type {true}. Here is an example of usage: \footnote {In
\LUATEX\ 0.80 you should use \type {newtoken} instead of \type {token}.}

\starttyping
function ScanPair()
  local one = 0
  local two = ""
  while true do
    if token.scan_keyword("one") then
      one = token.scan_int()
    elseif token.scan_keyword("two") then
      two = token.scan_string()
    else
      break
    end
  end
  tex.print("one: ",one,"\\par")
  tex.print("two: ",two,"\\par")
end
\stoptyping

This can be used as:

\starttyping
\directlua{ScanPair()}
\stoptyping

You can scan for an explicit character (class) with \type {scan_code}. This
function takes a positive number as argument and returns a character or \type
{nil}.

\starttabulate[|r|r|l|]
\NC \cldcontext{tokens.bits.escape     } \NC  0 \NC \type{escape}      \NC \NR
\NC \cldcontext{tokens.bits.begingroup } \NC  1 \NC \type{begingroup}  \NC \NR
\NC \cldcontext{tokens.bits.endgroup   } \NC  2 \NC \type{endgroup}    \NC \NR
\NC \cldcontext{tokens.bits.mathshift  } \NC  3 \NC \type{mathshift}   \NC \NR
\NC \cldcontext{tokens.bits.alignment  } \NC  4 \NC \type{alignment}   \NC \NR
\NC \cldcontext{tokens.bits.endofline  } \NC  5 \NC \type{endofline}   \NC \NR
\NC \cldcontext{tokens.bits.parameter  } \NC  6 \NC \type{parameter}   \NC \NR
\NC \cldcontext{tokens.bits.superscript} \NC  7 \NC \type{superscript} \NC \NR
\NC \cldcontext{tokens.bits.subscript  } \NC  8 \NC \type{subscript}   \NC \NR
\NC \cldcontext{tokens.bits.ignore     } \NC  9 \NC \type{ignore}      \NC \NR
\NC \cldcontext{tokens.bits.space      } \NC 10 \NC \type{space}       \NC \NR
\NC \cldcontext{tokens.bits.letter     } \NC 11 \NC \type{letter}      \NC \NR
\NC \cldcontext{tokens.bits.other      } \NC 12 \NC \type{other}       \NC \NR
\NC \cldcontext{tokens.bits.active     } \NC 13 \NC \type{active}      \NC \NR
\NC \cldcontext{tokens.bits.comment    } \NC 14 \NC \type{comment}     \NC \NR
\NC \cldcontext{tokens.bits.invalid    } \NC 15 \NC \type{invalid}     \NC \NR
\stoptabulate

So, if you want to grab the character you can say:

\starttyping
local c = token.scan_code(2^10 + 2^11 + 2^12)
\stoptyping

In \CONTEXT\ you can say:

\starttyping
local c = tokens.scanners.code(
  tokens.bits.space +
  tokens.bits.letter +
  tokens.bits.other
)
\stoptyping

When no argument is given, the next character with catcode letter or other is
returned (if found).

In \CONTEXT\ we use the \type {tokens} namespace which has additional scanners
available. That way we can remain compatible. I can add more scanners when
needed, although it is not expected that users will use this mechanism directly.

\starttabulate[||||]
\NC \type {(new)token}   \NC \type {tokens}             \NC arguments \NC \NR
\HL
\NC                      \NC \type {scanners.boolean}   \NC \NC \NR
\NC \type {scan_code}    \NC \type {scanners.code}      \NC \type {(bits)}      \NC \NR
\NC \type {scan_dimen}   \NC \type {scanners.dimension} \NC \type {(fill,math)} \NC \NR
\NC \type {scan_glue}    \NC \type {scanners.glue}      \NC \type {(math)}      \NC \NR
\NC \type {scan_int}     \NC \type {scanners.integer}   \NC \NC \NR
\NC \type {scan_keyword} \NC \type {scanners.keyword}   \NC \NC \NR
\NC                      \NC \type {scanners.number}    \NC \NC \NR
\NC \type {scan_token}   \NC \type {scanners.token}     \NC \NC \NR
\NC \type {scan_tokens}  \NC \type {scanners.tokens}    \NC \NC \NR
\NC \type {scan_string}  \NC \type {scanners.string}    \NC \NC \NR
\NC \type {scan_word}    \NC \type {scanners.word}      \NC \NC \NR
\NC \type {get_token}    \NC \type {getters.token}      \NC \NC \NR
\NC \type {set_macro}    \NC \type {setters.macro}      \NC \type {(catcodes,cs,str,global)} \NC \NR
\stoptabulate

All except \type {get_token} (or its alias \type {getters.token}) expand tokens
in order to satisfy the demands.

Here are some examples of how we can use the scanners. When we would call
\type {Foo} with regular arguments we do this:

\starttyping
\def\foo#1{%
  \directlua {
    Foo("whatever","#1",{n = 1})
  }
}
\stoptyping

but when \type {Foo} uses the scanners it becomes:

\starttyping
\def\foo#1{%
  \directlua{Foo()} {whatever} {#1} n {1}\relax
}
\stoptyping

In the first case we have a function \type {Foo} like this:

\starttyping
function Foo(what,str,n)
  --
  -- do something with these three parameters
  --
end
\stoptyping

and in the second variant we have (using the \type {tokens} namespace):

\starttyping
function Foo()
  local what = tokens.scanners.string()
  local str  = tokens.scanners.string()
  local n    = tokens.scanners.keyword("n") and
               tokens.scanners.integer() or 0
  --
  -- do something with these three parameters
  --
end
\stoptyping

The string scanned is kind of special as the result depends ok what is seen.
Given the following definition:

\startbuffer
           \def\bar  {bar}
\unexpanded\def\ubar {ubar} % \protected in plain etc
           \def\foo  {foo-\bar-\ubar}
           \def\wrap {{foo-\bar}}
           \def\uwrap{{foo-\ubar}}
\stopbuffer

\typebuffer

\getbuffer

We get:

\def\TokTest{\ctxlua{
    local s = tokens.scanners.string()
    context("\\bgroup\\red\\tt")
    context.verbatim(s)
    context("\\egroup")
}}

\starttabulate[|l|Tl|]
\NC \type{{foo}}       \NC \TokTest {foo}       \NC \NR
\NC \type{{foo-\bar}}  \NC \TokTest {foo-\bar}  \NC \NR
\NC \type{{foo-\ubar}} \NC \TokTest {foo-\ubar} \NC \NR
\NC \type{foo-\bar}    \NC \TokTest foo-\bar    \NC \NR
\NC \type{foo-\ubar}   \NC \TokTest foo-\ubar   \NC \NR
\NC \type{foo$bar$}    \NC \TokTest foo$bar$    \NC \NR
\NC \type{\foo}        \NC \TokTest \foo        \NC \NR
\NC \type{\wrap}       \NC \TokTest \wrap       \NC \NR
\NC \type{\uwrap}      \NC \TokTest \uwrap      \NC \NR
\stoptabulate

Because scanners look ahead the following happens: when an open brace is seen (or
any character marked as left brace) the scanner picks up tokens and expands them
unless they are protected; so, effectively, it scans as if the body of an \type
{\edef} is scanned. However, when the next token is a control sequence it will be
expanded first to see if there is a left brace, so there we get the full
expansion. In practice this is convenient behaviour because the braced variant
permits us to pick up meanings honouring protection. Of course this is all a side
effect of how \TEX\ scans.\footnote {This lookahead expansion can sometimes give
unexpected side effects because often \TEX\ pushes back a token when a condition
is not met. For instance when it scans a number, scanning stops when no digits
are seen but the scanner has to look at the next (expanded) token in order to
come to that conclusion. In the process it will, for instance, expand
conditionals. This means that intermediate catcode changes will not be effective
(or applied) to already-seen tokens that were pushed back into the input. This
also happens with, for instance, \cs {futurelet}.}

With the braced variant one can of course use primitives like \type {\detokenize}
and \type {\unexpanded} (in \CONTEXT: \type {\normalunexpanded}, as we already
had this mechanism before it was added to the engine).

\stopsection

\startsection[title=Considerations]

Performance|-|wise there is not much difference between these methods. With some
effort you can make the second approach faster than the first but in practice you
will not notice much gain. So, the main motivation for using the scanner is that
it provides a more \TEX|-|ified interface. When playing with the initial version
of the scanners I did some tests with performance|-|sensitive \CONTEXT\ calls and
the difference was measurable (positive) but deciding if and when to use the
scanner approach was not easy. Sometimes embedded \LUA\ code looks better, and
sometimes \TEX\ code. Eventually we will end up with a mix. Here are some
considerations:

\startitemize
\startitem
    In both cases there is the overhead of a \LUA\ call.
\stopitem
\startitem
    In the pure \LUA\ case the whole argument is tokenized by \TEX\ and then
    converted to a string that gets compiled by \LUA\ and executed.
\stopitem
\startitem
    When the scan happens in \LUA\ there are extra calls to functions but
    scanning still happens in \TEX; some token to string conversion is avoided
    and compilation can be more efficient.
\stopitem
\startitem
    When data comes from external files, parsing with \LUA\ is in most cases more
    efficient than parsing by \TEX .
\stopitem
\startitem
    A macro package like \CONTEXT\ wraps functionality in macros and is
    controlled by key|/|value specifications. There is often no benefit in terms
    of performance when delegating to the mentioned scanners.
\stopitem
\stopitemize

Another consideration is that when using macros, parameters are often passed
between \type {{}}:

\starttyping
\def\foo#1#2#3%
  {...}
\foo {a}{123}{b}
\stoptyping

and suddenly changing that to

\starttyping
\def\foo{\directlua{Foo()}}
\stoptyping

and using that as:

\starttyping
\foo {a} {b} n 123
\stoptyping

means that \type {{123}} will fail. So, eventually you will end up with something:

\starttyping
\def\myfakeprimitive{\directlua{Foo()}}
\def\foo#1#2#3{\myfakeprimitive {#1} {#2} n #3 }
\stoptyping

and:

\starttyping
\foo {a} {b} {123}
\stoptyping

So in the end you don't gain much here apart from the fact that the fake
primitive can be made more clever and accept optional arguments. But such new
features are often hidden for the user who uses more high|-|level wrappers.

When you code in pure \TEX\ and want to grab a number directly you need to test
for the braced case; when you use the \LUA\ scanner method you still need to test
for braces. The scanners are consistent with the way \TEX\ works. Of course you
can write helpers that do some checking for braces in \LUA, so there are no real
limitations, but it adds some overhead (and maybe also confusion).

One way to speed up the call is to use the \type {\luafunction} primitive in
combinations with predefined functions and although both mechanisms can benefit
from this, the scanner approach gets more out of that as this method cannot be
used with regular function calls that get arguments. In (rather low level) \LUA\
it looks like this:

\starttyping
luafunctions[1] = function()
  local a token.scan_string()
  local n token.scan_int()
  local b token.scan_string()
  -- whatever --
end
\stoptyping

And in \TEX:

\starttyping
\luafunction1 {a} 123 {b}
\stoptyping

This can of course be wrapped as:

\starttyping
\def\myprimitive{\luafunction1 }
\stoptyping

\stopsection

\startsection[title=Applications]

The question now pops up: where can this be used? Can you really make new
primitives? The answer is yes. You can write code that exclusively stays on the
\LUA\ side but you can also do some magic and then print back something to \TEX.
Here we use the basic token interface, not \CONTEXT:

\startbuffer
\directlua {
local token = newtoken or token
function ColoredRule()
  local w, h, d, c, t
  while true do
    if token.scan_keyword("width") then
      w = token.scan_dimen()
    elseif token.scan_keyword("height") then
      h = token.scan_dimen()
    elseif token.scan_keyword("depth") then
      d = token.scan_dimen()
    elseif token.scan_keyword("color") then
      c = token.scan_string()
    elseif token.scan_keyword("type") then
      t = token.scan_string()
    else
      break
    end
  end
  if c then
    tex.sprint("\\color[",c,"]{")
  end
  if t == "vertical" then
    tex.sprint("\\vrule")
  else
    tex.sprint("\\hrule")
  end
  if w then
    tex.sprint("width ",w,"sp")
  end
  if h then
    tex.sprint("height ",h,"sp")
  end
  if d then
    tex.sprint("depth ",d,"sp")
  end
  if c then
    tex.sprint("\\relax}")
  end
end
}
\stopbuffer

\typebuffer \getbuffer

This can be given a \TeX\ interface like:

\startbuffer
\def\myhrule{\directlua{ColoredRule()} type {horizontal} }
\def\myvrule{\directlua{ColoredRule()} type {vertical} }
\stopbuffer

\typebuffer \getbuffer

And used as:

\startbuffer
\myhrule width \hsize height 1cm color {darkred}
\stopbuffer

\typebuffer

giving:

% when no newtokens:
%
% \startbuffer
% \blackrule[width=\hsize,height=1cm,color=darkred]
% \stopbuffer

\startlinecorrection \getbuffer \stoplinecorrection

Of course \CONTEXT\ users can use the following commands to color an
otherwise-black rule (likewise):

\startbuffer
\blackrule[width=\hsize,height=1cm,color=darkgreen]
\stopbuffer

\typebuffer \startlinecorrection \getbuffer \stoplinecorrection

The official \CONTEXT\ way to define such a new command is the following. The
conversion back to verbose dimensions is needed because we pass back to \TEX.

\startbuffer
\startluacode
local myrule = tokens.compile {
  {
    { "width",  "dimension", "todimen" },
    { "height", "dimension", "todimen" },
    { "depth",  "dimension", "todimen" },
    { "color",  "string" },
    { "type",   "string" },
  }
}

interfaces.scanners.ColoredRule = function()
  local t = myrule()
  context.blackrule {
    color  = t.color,
    width  = t.width,
    height = t.height,
    depth  = t.depth,
  }
end
\stopluacode
\stopbuffer

\typebuffer \getbuffer

With:

\startbuffer
\unprotect \let\myrule\clf_ColoredRule \protect
\stopbuffer

\typebuffer \getbuffer

and

\startbuffer
\myrule width \textwidth height 1cm color {maincolor} \relax
\stopbuffer

\typebuffer

we get:

% when no newtokens:
%
% \startbuffer
% \blackrule[width=\hsize,height=1cm,color=maincolor]
% \stopbuffer

\startlinecorrection \getbuffer \stoplinecorrection

There are many ways to use the scanners and each has its charm. We will look at
some alternatives from the perspective of performance. The timings are more meant
as relative measures than absolute ones. After all it depends on the hardware. We
assume the following shortcuts:

\starttyping
local scannumber  = tokens.scanners.number
local scankeyword = tokens.scanners.keyword
local scanword    = tokens.scanners.word
\stoptyping

We will scan for four different keys and values. The number is scanned using a
helper \type {scannumber} that scans for a number that is acceptable for \LUA.
Thus, \type {1.23} is valid, as are \type {0x1234} and \type {12.12E4}.

% interfaces.scanners.test_scaling_a

\starttyping
function getmatrix()
  local sx, sy = 1, 1
  local rx, ry = 0, 0
  while true do
    if scankeyword("sx") then
      sx = scannumber()
    elseif scankeyword("sy") then
      sy = scannumber()
    elseif scankeyword("rx") then
      rx = scannumber()
    elseif scankeyword("ry") then
      ry = scannumber()
    else
      break
    end
  end
  -- action --
end
\stoptyping

Scanning the following specification  100000 times takes 1.00 seconds:

\starttyping
sx 1.23 sy 4.5 rx 1.23 ry 4.5
\stoptyping

The \quote {tight} case takes 0.94 seconds:

\starttyping
sx1.23 sy4.5 rx1.23 ry4.5
\stoptyping

% interfaces.scanners.test_scaling_b

We can compare this to scanning without keywords. In that case there have to be
exactly four arguments. These have to be given in the right order which is no big
deal as often such helpers are encapsulated in a user|-|friendly macro.

\starttyping
function getmatrix()
  local sx, sy = scannumber(), scannumber()
  local rx, ry = scannumber(), scannumber()
  -- action --
end
\stoptyping

As expected, this is more efficient than the previous examples. It takes 0.80
seconds to scan this 100000 times:

\starttyping
1.23 4.5 1.23 4.5
\stoptyping

A third alternative is the following:

\starttyping
function getmatrix()
  local sx, sy = 1, 1
  local rx, ry = 0, 0
  while true do
    local kw = scanword()
    if kw == "sx" then
      sx = scannumber()
    elseif kw == "sy" then
      sy = scannumber()
    elseif kw == "rx" then
      rx = scannumber()
    elseif kw == "ry" then
      ry = scannumber()
    else
      break
    end
  end
  -- action --
end
\stoptyping

Here we scan for a keyword and assign a number to the right variable. This one
call happens to be less efficient than calling \type {scan_keyword} 10 times
($4+3+2+1$) for the explicit scan. This run takes 1.11 seconds for the next line.
The spaces are really needed as words can be anything that has no space.
\footnote {Hard|-|coding the word scan in a \CCODE\ helper makes little sense, as
different macro packages can have different assumptions about what a word is. And
we don't extend \LUATEX\ for specific macro packages.}

\starttyping
sx 1.23 sy 4.5 rx 1.23 ry 4.5
\stoptyping

Of course these numbers need to be compared to a baseline of no scanning (i.e.\
the overhead of a \LUA\ call which here amounts to 0.10 seconds. This brings
us to the following table.

\starttabulate[|l|l|]
\NC keyword checks \NC 0.9 sec\NC \NR
\NC no keywords    \NC 0.7 sec\NC \NR
\NC word checks    \NC 1.0 sec\NC \NR
\stoptabulate

The differences are not that impressive given the number of calls. Even in a
complex document the overhead of scanning can be negligible compared to the
actions involved in typesetting the document. In fact, there will always be some
kind of scanning for such macros so we're talking about even less impact. So you
can just use the method you like most. In practice, the extra overhead of using
keywords in combination with explicit checks (the first case) is rather
convenient.

If you don't want to have many tests you can do something like this:

\starttyping
local keys = {
  sx = scannumber, sy = scannumber,
  rx = scannumber, ry = scannumber,
}

function getmatrix()
  local values = { }
  while true do
    for key, scan in next, keys do
      if scankeyword(key) then
        values[key] = scan()
      else
        break
      end
    end
  end
  -- action --
end
\stoptyping

This is still quite fast although one now has to access the values in a table.
Working with specifications like this is clean anyway so in \CONTEXT\ we have a
way to abstract the previous definition.

\starttyping
local specification = tokens.compile {
  {
    { "sx", "number" }, { "sy", "number" },
    { "rx", "number" }, { "ry", "number" },
  },
}

function getmatrix()
  local values = specification()
  -- action using values.sx etc --
end
\stoptyping

Although one can make complex definitions this way, the question remains if it
is a better approach than passing \LUA\ tables. The standard \CONTEXT\ way for
controlling features is:

\starttyping
\getmatrix[sx=1.2,sy=3.4]
\stoptyping

So it doesn't matter much if deep down we see:

\starttyping
\def\getmatrix[#1]%
  {\getparameters[@@matrix][sx=1,sy=1,rx=1,ry=1,#1]%
   \domatrix
     \@@matrixsx
     \@@matrixsy
     \@@matrixrx
     \@@matrixry
   \relax}
\stoptyping

or:

\starttyping
\def\getmatrix[#1]%
  {\getparameters[@@matrix][sx=1,sy=1,rx=1,ry=1,#1]%
   \domatrix
     sx \@@matrixsx
     sy \@@matrixsy
     rx \@@matrixrx
     ry \@@matrixry
   \relax}
\stoptyping

In the second variant (with keywords) can be a scanner like we defined before:

\starttyping
\def\domatrix#1#2#3#4%
  {\directlua{getmatrix()}}
\stoptyping

but also:

\starttyping
\def\domatrix#1#2#3#4%
  {\directlua{getmatrix(#1,#2,#3,#4)}}
\stoptyping

given:

\starttyping
function getmatrix(sx,sy,rx,ry)
    -- action using sx etc --
end
\stoptyping

or maybe nicer:

\starttyping
\def\domatrix#1#2#3#4%
  {\directlua{domatrix{
     sx = #1,
     sy = #2,
     rx = #3,
     ry = #4
   }}}
\stoptyping

assuming:

\starttyping
function getmatrix(values)
    -- action using values.sx etc --
end
\stoptyping

If you go for speed the scanner variant without keywords is the most efficient
one. For readability the scanner variant with keywords or the last shown example
where a table is passed is better. For flexibility the table variant is best as
it makes no assumptions about the scanner \emdash\ the token scanner can quit on
unknown keys, unless that is intercepted of course. But as mentioned before, even
the advantage of the fast one should not be overestimated. When you trace usage
it can be that the (in this case matrix) macro is called only a few thousand
times and that doesn't really add up. Of course many different sped-up calls can
make a difference but then one really needs to optimize consistently the whole
code base and that can conflict with readability. The token library presents us
with a nice chicken||egg problem but nevertheless is fun to play with.

\stopsection

\startsection[title=Assigning meanings]

The token library also provides a way to create tokens and access properties but
that interface can change with upcoming versions when the old library is replaced
by the new one and the input handling is cleaned up. One experimental function is
worth mentioning:

\starttyping
token.set_macro("foo","the meaning of bar")
\stoptyping

This will turn the given string into tokens that get assigned to \type {\foo}.
Here are some alternative calls:

\starttabulate
\NC \type {set_macro("foo")}                    \NC \type { \def \foo {}}        \NC \NR
\NC \type {set_macro("foo","meaning")}          \NC \type { \def \foo {meaning}} \NC \NR
\NC \type {set_macro("foo","meaning","global")} \NC \type {\gdef \foo {meaning}} \NC \NR
\stoptabulate

The conversion to tokens happens under the current catcode regime. You can
enforce a different regime by passing a number of an allocated catcode table as
the first argument, as with \type {tex.print}. As we mentioned performance
before: setting at the \LUA\ end like this:

\starttyping
token.set_macro("foo","meaning")
\stoptyping

is about two times as fast as:

\starttyping
tex.sprint("\\def\\foo{meaning}")
\stoptyping

or (with slightly more overhead) in \CONTEXT\ terms:

\starttyping
context("\\def\\foo{meaning}")
\stoptyping

The next variant is actually slower (even when we alias \type {setvalue}):

\starttyping
context.setvalue("foo","meaning")
\stoptyping

but although 0.4 versus 0.8 seconds looks like a lot on a \TEX\ run I need a
million calls to see such a difference, and a million macro definitions during a
run is a lot. The different assignments involved in, for instance, 3000 entries
in a bibliography (with an average of 5 assignments per entry) can hardly be
measured as we're talking about milliseconds. So again, it's mostly a matter of
convenience when using this function, not a necessity.

\stopsection

\startsection[title=Conclusion]

For sure we will see usage of the new scanner code in \CONTEXT, but to what
extent remains to be seen. The performance gain is not impressive enough to
justify many changes to the code but as the low|-|level interfacing can sometimes
become a bit cleaner it will be used in specific places, even if we sacrifice
some speed (which then probably will be compensated for by a little gain
elsewhere).

The scanners will probably never be used by users directly simply because there
are no such low level interfaces in \CONTEXT\ and because manipulating input is
easier in \LUA. Even deep down in the internals of \CONTEXT\ we will use wrappers
and additional helpers around the scanner code. Of course there is the fun-factor
and playing with these scanners is fun indeed. The macro setters have as their
main benefit that using them can be nicer in the \LUA\ source, and of course
setting a macro this way is also conceptually cleaner (just like we can set
registers).

Of course there are some challenges left, like determining if we are scanning
input of already converted tokens (for instance in a macro body or token\-list
expansion). Once we can properly feed back tokens we can also look ahead like
\type {\futurelet} does. But for that to happen we will first clean up the
\LUATEX\ input scanner code and error handler.

\stopsection

\stopchapter

\stoptext

