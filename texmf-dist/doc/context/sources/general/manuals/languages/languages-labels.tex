% language=uk

\startcomponent languages-labels

\environment languages-environment

\startchapter[title=Labels][color=darkcyan]

\startsection[title=Introduction]

When we started using \TEX, I naturally started with plain \TEX. But it didn't
take long before we tried \LATEX. Because our documents were in Dutch one of the
first fights with this package was to get rid of the english labels. Because
rather soon we decided to cook up an alternative package, a decent label
mechanism was one of the first things to show up. And as soon as multiple
language typesetting gets into view, such a mechanism becomes one of those
language dependent features. In this chapter the basics will be covered.

\stopsection

\startsection[title=Defining labels]

Before we define a label we need to define a label class. You probably seldom
need that but this is how it's done:

\startbuffer
\definelabelclass [mylabel]
\stopbuffer

\typebuffer \getbuffer

There are some classes predefined:

\starttabulate[|lB|l|]
    \NC head      \NC (complete) titles like \headtext {chapter} and \headtext {figure} \NC \NR
    \NC label     \NC in||text labels like \labeltext {chapter} and \labeltext {figure} \NC \NR
    \NC mathlabel \NC function names like \mathlabeltext{sin} and \mathlabeltext{cos} \NC \NR
    \NC taglabel  \NC labels used for tagging purposed in the backend \NC \NR
    \NC btxlabel  \NC labels used in typesetting bibliographic items \NC \NR
\stoptabulate

The physical units mechanism also uses labels: unit, operator, prefix and suffix.
All these labels are defined per language with a fall back on english.

Given that we have defined class \type {mylabel}, a label itself is set like
this:

\startbuffer
\setupmylabeltext
  [en]
  [first={<after first},
   second={{before second>},{<after second}}]
\stopbuffer

\typebuffer \getbuffer

The first argument (the language) is optional. In the next section we will see
how these labels are used. A lot of labels are predefined, in \MKIV\ this happens
in the file \type {lang-txt.lua}. There is no need to adapt this file as you can
always add labels run time.

\stopsection

\startsection[title=Using labels]

How a label is called depends on the way it needs to be used. In any case the
main language set determines the language of the label. So, when in an Dutch text
we temporary switch to German, the Dutch labels are used.

\startbuffer
\starttabulate[||||]
    \NC  \bf command                         \NC \ttbf first                    \NC \ttbf {second}                  \NC \NR
    \HL
    \NC \type {\leftmylabeltext {tag}}       \NC \leftmylabeltext {first}       \NC \leftmylabeltext {second}       \NC \NR
    \NC \type {\rightmylabeltext{tag}}       \NC \rightmylabeltext{first}       \NC \rightmylabeltext{second}       \NC \NR
    \NC \type {\mylabeltext     {tag}}       \NC \mylabeltext     {first}       \NC \mylabeltext     {second}       \NC \NR
    \NC \type {\mylabeltexts    {tag}{text}} \NC \mylabeltexts    {first}{text} \NC \mylabeltexts    {second}{text} \NC \NR
\stoptabulate
\stopbuffer

\getbuffer

\stopsection

\startsection[title=Hooks]

Some mechanisms have label support built in, most noticeably sections
heads and numbered items, like figure captions.

\startbuffer
\definehead
  [myhead]
  [subsection]

\setuphead
  [myhead]
  [bodypartlabel=bodypartmyhead]

\setuplabeltext
  [en]
  [bodypartmyhead=My Head: ]

\myhead{Welcome}
\stopbuffer

\typebuffer \getbuffer

The head text label class can be used as follows:

\startbuffer
\setupheadtext
  [SomeHead=Just A Title]

\subsection
  [title=\headtext{SomeHead}]
\stopbuffer

\typebuffer \getbuffer

A label will obey the style settings, as in:

\startbuffer
\definehead
  [MyFancyHead]
  [subsection]
  [style={\bs\setcharactercasing[Words]}]

\setupheadtext
  [SomeHead=just another title]

\MyFancyHead
  [title=\headtext{SomeHead}]
\stopbuffer

\typebuffer \getbuffer

\stopsection

\stopchapter

\stopcomponent
