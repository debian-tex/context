% language=us runpath=texruns:manuals/languages

\startcomponent languages-hyphenation

\environment languages-environment

\startchapter[title=Hyphenation][color=darkmagenta]

\startsection[title=How it works]

Proper hyphenation is one of the strong points of \TEX. Hyphenation in \TEX\ is
done using so called hyphenation patterns. Making these patterns is an art
and most users (including me) happily use whatever is available. Patterns can be
created automatically using \type {patgen} but often manual tweaking is needed
too. A pattern looks as follows:

\starttyping
pat1tern
\stoptyping

This means as much as: you can split the word \type {pattern} in two pieces, with
a hyphen between the two \type {t}'s. Actually it will also split the word \type
{patterns} because the hyphenation mechanism looks at substrings. When no number
between characters in a pattern is given, a zero is assumed. This means as much
as {\em undefined}. An even number inhibits hyphenation, an odd number permits
it. The larger the number (weight), the more influence it has. A more restricted
pattern is:

\starttyping
.pat1tern.
\stoptyping

Here the periods set the word boundaries. The pattern dictionary for us
english has smaller patterns and the next trace shows how these are applied.

\starthyphenation[traditional]
\showhyphenationtrace[en][pattern]
\stophyphenation

The effective hyphenation of a word is determined by several factors:

\startitemize[packed]
\startitem the current language, each language can have different patterns \stopitem
\startitem the characters, as some characters might block hyphenation \stopitem
\startitem the settings of \type {\lefthyphenmin} and \type {\righthyphenmin} \stopitem
\stopitemize

A place where a word can be hyphenated is called a discretionary. When \TEX\
analyzes a stream, it will inject discretionary nodes into that stream.

\starttyping
pat\discretionary{-}{}{}tern.
\stoptyping

In traditional \TEX\ hyphenation, ligature building and kerning are tightly
interwoven which is quite effective. However, there was also a strong
relationship between the current font and hyphenation. This is a side effect of
traditional \TEX\ having at most 256 characters in a font and the fact that the
used character is fact a reference to a slot in a font. There a character in the
input initially ends up as a character node and eventually becomes a glyph node.
For instance two characters \type {fi} can become a ligature glyph representing
this combination.

In \LUATEX\ the hyphenation, ligature building and kerning stages are separated
and can be overloaded. In \CONTEXT\ all three can be replaced by code written in
\LUA. Because normally hyphenation happens before font logic is applied, there is
no relationship with font encoding. I wrote the first \LUA\ version of the
hyohenator on a rainy weekend and the result was not that bad so it was presented
at the 2014 \CONTEXT\ meeting. After some polishing I decided to add this routine
to the standard \MKIV\ repertoire which then involved some proper interfacing.

You can enable the \LUA\ variant with the following command:

\starttyping
\setuphyphenation[method=traditional]
\stoptyping

We call this method \type {traditional} because in principle we can have
many more methods and this one is (supposed to be) mostly compatible to the
built-in method. This is a global setting. You can switch back with:

\starttyping
\setuphyphenation[method=default]
\stoptyping

In the next sections we will see how we can provide alternatives within the
traditional method. These alternatives can be set local and therefore can operate
over a limited range of characters.

One complication in interfacing is that \TEX\ has grouping (which permits local
settings) and we want to limit some of the above functionality using groups. At
the same time hyphenation is a paragraph related action so we need to enable the
hyphenation related code at a global level (or at least make sure that it gets
exercised by forcing a \type {\par}). That means that the alternative
hyphenator has to be quite compatible so that we could just enable it for a whole
document. This can have an impact on performance but in practice that can be
neglected. In \LUATEX\ the \LUA\ variant is 4~times slower than the built-in one,
in \LUAJITTEX\ it's 3~times slower. But the good news is that the amount of time
spent in the hyphenator is relatively small compared to other manipulations and
macro expansion. The additional time needed for loading and preparing the
patterns into a more \LUA\ specific format can be neglected.

You can check how words get hyphenated using the patterns management script:

\starttyping
>mtxrun --script patterns --hyphenate language

hyphenator      |
hyphenator      | . l a n g u a g e .   . l a n g u a g e .
hyphenator      |    0a2n0               0 0 2 0 0 0 0 0 0
hyphenator      |    2a0n0g0             0 2 2 0 0 0 0 0 0
hyphenator      |      0n1g0u0           0 2 2 1 0 0 0 0 0
hyphenator      |        0g0u4a0         0 2 2 1 0 4 0 0 0
hyphenator      |              2g0e0.0   0 2 2 1 0 4 2 0 0
hyphenator      | .0l2a2n1g0u4a2g0e0.   . l a n-g u a g e .
hyphenator      |
mtx-patterns    | us 3 3 : language : lan-guage
\stoptyping

\stopsection

\startsection[title=The last words]

Mid 2014 we had to upgrade a style for a \PDF\ assembly service: chapters from
(technical) school books are combined into arbitrary new books. There are some
nasty aspects with this flow: for instance, all section numbers in a chapter are
replaced by new numbers and this also involves figure and table prefixes.
It boils down to splitting up books, analyzing the typeset content and
preparing it for replacements. The structure is described in \XML\ files so that
we can generate tables of contents. The reason for not generating from \XML\
sources is that the publisher doesn't have a \XML\ workflow and that books
already were available. Also, books from several series are combined and even
within a series structure (and rendering) differs.

What has this to do with hyphenation? Writing a style for such a flow always
results in a more complex one that estimated and as usual it's in the details.
The original style was written in \MKII\ and used some box juggling to achieve
reasonable results but in \MKIV\ we can do better.

Each chapter has a title and books get titles and subtitles as well. The titles
are typeset each time a new book is composed. This happens within some layout
constraints. Think of constraints like these:

\startitemize[packed]
\startitem the title goes on top of a shape that doesn't permit much overflow \stopitem
\startitem there can be very long words (not uncommon in Dutch or German) \stopitem
\startitem a short word or hyphenated part should not end up on the last line \stopitem
\startitem the left and right hyphenation minima are at least four \stopitem
\stopitemize

The last requirement is a compromise because in most cases publishers seem to
want ragged right not hyphenated rendering (at least in Dutch schoolbooks). The
arguments for this are quite weak and probably originate in fear of bad rendering
given past experiences. It's this kind of situations that drive the development
of the more obscure features that ship with \CONTEXT\ and a (partial) solution
for this specific case will be given later.

If you look at thousands of titles and turn these into (small) paragraphs \TEX\
does a pretty good job. It's the few exceptions that we need to catch. The next
examples demonstrate such an extreme case.

\startbuffer[example]
\dorecurse{5} { % dejavu
    \startlinecorrection[blank]
        \bTABLE
            \bTR
                \bTD[align=middle,width=2em,foregroundstyle=bold]
                    #1
                \eTD
                \bTD[align={verytolerant,flushleft},width=15em,offset=1ex]
                    \hsize \dimexpr11\emwidth-#1\dimexpr.5\emwidth\relax
                    \dontcomplain
                    \lefthyphenmin=4\righthyphenmin=4
                    \blackrule[color=darkyellow,width=\hsize,height=-3pt,depth=5pt]\par
                    \begstrut\getbuffer[long]\endstrut\par
                \eTD
                \bTD[align={verytolerant,flushleft},width=15em,offset=1ex]
                    \sethyphenationfeatures[demo]
                    \hsize \dimexpr11\emwidth-#1\dimexpr.5\emwidth\relax
                    \dontcomplain
                    \blackrule[color=darkyellow,width=\hsize,height=-3pt,depth=5pt]\par
                    \begstrut\getbuffer[long]\endstrut\par
                \eTD
            \eTR
        \eTABLE
    \stoplinecorrection
}
\stopbuffer

\definehyphenationfeatures
  [demo]
  [rightwords=1,
   lefthyphenmin=4,
   righthyphenmin=4]

\startbuffer[long]
a verylongword and then anevenlongerword
\stopbuffer

\starthyphenation[traditional]
    \enabletrackers[hyphenator.visualize]
    \getbuffer[example]\par
    \disabletrackers[hyphenator.visualize]
\stophyphenation

Of course in practice there need to be some reasonable width and when we pose
these limits the longest possible word should fit into the allocated space. In
these examples the rule shows the width. In the right columns we see a red
colored word and that one will not get hyphenated.

\stopsection

\startsection[title=Explicit hyphens]

Another special case that we needed to handle were (compound) words with explicit
hyphens. Because often data comes from \XML\ files we can not really control the
typesetting as in a \TEX\ document where the author sees what gets done. So here
we need a way to turn these hyphens into proper hyphenation directives and at the
same time permit the words to be hyphenated.

\definehyphenationfeatures
  [demo]
  [hyphens=yes,
   lefthyphenmin=4,
   righthyphenmin=4]

\startbuffer[long]
a very-long-word and then an-even-longer-word
\stopbuffer

\starthyphenation[traditional]
    \enabletrackers[hyphenator.visualize]
    \getbuffer[example]\par
    \disabletrackers[hyphenator.visualize]
\stophyphenation

\stopsection

\startsection[title=Extended patterns]

As with more opened up mechanisms, in \MKIV\ we can extend functionality. As an
example I have implemented the extensions discussed in the article by László
Németh in the Proceedings of Euro\TEX\ 2006: {\em Hyphenation in OpenOffice.org}
(TUGboat, Volume 27, 2006). The syntax for these extension is somewhat ugly and
involves optional offsets and ranges. \footnote {I'm not sure if there were ever
patterns released that used this syntax.}

\startbuffer
\registerhyphenationpattern[nl][e1ë/e=e]
\registerhyphenationpattern[nl][a9atje./a=t,1,3]
\registerhyphenationpattern[en][eigh1tee/t=t,5,1]
\registerhyphenationpattern[de][c1k/k=k]
\registerhyphenationpattern[de][schif1f/ff=f,5,2]
\stopbuffer

\typebuffer \getbuffer

These patterns result in the following hyphenations:

\starthyphenation[traditional]
    \switchtobodyfont[big]
    \starttabulate[|||]
        \NC reëel      \NC \language[nl]\hyphenatedcoloredword{reëel}      \NC \NR
        \NC omaatje    \NC \language[nl]\hyphenatedcoloredword{omaatje}    \NC \NR
        \NC eighteen   \NC \language[en]\hyphenatedcoloredword{eighteen}   \NC \NR
        \NC Zucker     \NC \language[de]\hyphenatedcoloredword{Zucker}     \NC \NR
        \NC Schiffahrt \NC \language[de]\hyphenatedcoloredword{Schiffahrt} \NC \NR
    \stoptabulate
\stophyphenation

In a specification, the \type {.} indicates a word boundary and numbers indicate
the weight of a breakpoint. The optional extended specification comes after the
\type {/}. The values separated by a \type {=} are the pre and post sequences:
these end up at the end of the current line and beginning of the next one. The
optional numbers are the start position and length. These default to~1 and~2, so
in the first example they identify \type {eë} (the weights don't count).

There is a pitfall here. When the language already has patterns that for
instance prohibit a hyphen between \type {e} and type {ë}, like \type{e2ë}, we
need to make sure that we give our new one a higher priority, which is why we
used a \type{e9ë}.

This feature is somewhat experimental and can be improved. Here is a more \LUA-ish
way of setting such patterns:

\starttyping
local registerpattern =
    languages.hyphenators.traditional.registerpattern

registerpattern("nl","e1ë", {
    start  = 1,
    length = 2,
    before = "e",
    after  = "e",
} )

registerpattern("nl","a9atje./a=t,1,3")
\stoptyping

Just adding extra patterns to an existing set without much testing is not wise. For
instance we could add these to the dutch dictionary:

\starttyping
\registerhyphenationpattern[nl][e3ë/e=e]
\registerhyphenationpattern[nl][o3ë/o=e]
\registerhyphenationpattern[nl][e3ï/e=i]
\registerhyphenationpattern[nl][i3ë/i=e]
\registerhyphenationpattern[nl][a5atje./a=t,1,3]
\registerhyphenationpattern[nl][toma8at5je]
\stoptyping

That would work oke well for words like

\starttyping
coëfficiënt
geïntroduceerd
copiëren
omaatje
tomaatje
\stoptyping

However, the last word only goes right because we explicitly added a pattern
for it. One reason is that the existing patterns already contain rules to
prevent weird hyphenations. The same is true for the accented characters. So,
consider these examples and coordinate additional patterns with other users
so that errors can be identified.

\stopsection

\startsection[title=Exceptions]

We have a variant on the \TEX\ primitive \type {\hyphenation}, the official way
to register a specific way to hyphenate a word.

\startbuffer
\registerhyphenationexception[aaaaa-bbbbb]
aaaaabbbbb \par
\stopbuffer

\typebuffer

This code is self explaining and results in:

\blank

\starthyphenation[traditional]
\setupindenting[no]\hsize 1mm \lefthyphenmin 1 \righthyphenmin 1 \getbuffer
\stophyphenation

There can be multiple hyphens and even multiple words in such a specification:

\startbuffer
\registerhyphenationexception[aaaaa-bbbbb cc-ccc-ddd-dd]
aaaaabbbbb \par
cccccddddd \par
\stopbuffer

\typebuffer

We get:

\blank

\starthyphenation[traditional]
\setupindenting[no]\hsize 1mm \lefthyphenmin 1 \righthyphenmin 1 \getbuffer
\stophyphenation

Some languages are a bit picky with respect to ligatures and hyphenation so we
have ways to control this.

% \zwj  : no ligatures
% \zwnj : no kerns either

\startbuffer
\startexceptions[de]
begri{ff-}{l}{ffl}(f\zwj fl)ich
xegri{ff-}{l}{ffl}(ff\zwj l)ich
zegri{ff-}{l}{ffl}(ffl)ich
wegri{ff-}{l}{ffl}(f\zwj f\zwj l)ich
\stopexceptions
\stopbuffer

\typebuffer \getbuffer

Here \type {\zwj} prevents a ligature and \type {\zwnj} prevents a ligature as
well as a font kern (in for instance Latin Modern ligatures are a bit more
distinctive).

\startlinecorrection
    \showglyphs \showfontkerns
    \startcombination[2*2]
        {\de\glyphscale\numexpr4*\glyphscale\relax begrifflich} {}
        {\de\glyphscale\numexpr4*\glyphscale\relax xegrifflich} {}
        {\de\glyphscale\numexpr4*\glyphscale\relax zegrifflich} {}
        {\de\glyphscale\numexpr4*\glyphscale\relax wegrifflich} {}
    \stopcombination
\stoplinecorrection

\stopsection

\startsection[title=Boundaries]

A box, rule, math or discretionary will end a word and prohibit hyphenation
of that word. Take this example:

\startbuffer[demo]
whatever \par
whatever\hbox{!} \par
\vl whatever\vl \par
whatever$x$ \par
whatever-whatever \par
\stopbuffer

\typebuffer[demo]

These lines will hyphenate differently and in traditional \TEX\ you need to
insert penalties and|/|or glue to get around it unless you instruct \LUATEX\ to
be more. In the \LUA\ variant we can enable that limitation.

\startbuffer
\definehyphenationfeatures
  [strict]
  [rightedge=tex]
\stopbuffer

\typebuffer \getbuffer

Here we show the three variants: traditional \TEX\ and \LUA\ with and without
strict settings.

\starttabulate[|p|p|p|]
\HL
\NC \ttbf \hbox to 11em{default\hss}
\NC \ttbf \hbox to 11em{traditional\hss}
\NC \ttbf \hbox to 11em{traditional strict\hss}
\NC \NR
\HL
\NC \starthyphenation[default]     \hsize1mm \getbuffer[demo] \stophyphenation
\NC \starthyphenation[traditional] \hsize1mm \getbuffer[demo] \stophyphenation
\NC \starthyphenation[traditional] \sethyphenationfeatures[strict]
                                   \hsize1mm \getbuffer[demo] \stophyphenation
\NC \NR
\HL
\stoptabulate

By default \CONTEXT\ is configured to hyphenate words that start with an
uppercase character. This behaviour is controlled in \TEX\ by the \typ {\uchyph}
variable. A positive value will enable this and a negative one disables it.

\starttabulate[|p|p|p|p|]
\HL
\NC \ttbf \hbox to 8em{default     0\hss}
\NC \ttbf \hbox to 8em{default     1\hss}
\NC \ttbf \hbox to 8em{traditional 0\hss}
\NC \ttbf \hbox to 8em{traditional 1\hss}
\NC \NR
\HL
\NC \starthyphenation[default]     \hsize1mm \uchyph\zerocount TEXified \dontcomplain \stophyphenation
\NC \starthyphenation[traditional] \hsize1mm \uchyph\zerocount TEXified \dontcomplain \stophyphenation
\NC \starthyphenation[default]     \hsize1mm \uchyph\plusone   TEXified \dontcomplain \stophyphenation
\NC \starthyphenation[traditional] \hsize1mm \uchyph\plusone   TEXified \dontcomplain \stophyphenation
\NC \NR
\HL
\stoptabulate

The \LUA\ variants behaves the same as the built-in implementation (that of course
remains the reference).

\stopsection

\startsection[title=Plug-ins]

The default hyphenator is similar to the built-in one, with a couple of
extensions as mentioned. However, you can plug in your own code, given that it
does return a proper hyphenation result. One reason for providing this plug is
that there are users who want to play with hyphenators based  on a different
logic. In \CONTEXT\ we already have some methods to deal with languages that
(for instance) have no spaces but split on words or syllables. A more tight
integration with the hyphenator can have advantages so I will explore these
options when there is demand.

A result table indicates where we can break a word. If we have a four character
word and can break after the second character, the result looks like this:

\starttyping
result = { false, true, false, false }
\stoptyping

Instead of \type {true} we can also have a table that has entries like the
extensions discussed in a previous section. Let's give an example of a
plug-in.

\startbuffer
\startluacode
    local subset = {
        a = true,
        e = true,
        i = true,
        o = true,
        u = true,
        y = true,
    }

    languages.hyphenators.traditional.installmethod("test",
        function(dictionary,word,n)
            local t = { }
            for i=1,#word do
                local w = word[i]
                if subset[w] then
                    t[i] = {
                        before = "<" .. w,
                        after  = w .. ">",
                        left   = false,
                        right  = false,
                    }
                else
                    t[i] = false
                end
            end
            return t
        end
    )
\stopluacode
\stopbuffer

\typebuffer \getbuffer

Here we hyphenate on vowels and surround them by angle brackets when
split over lines. This alternative is installed as follows:

\startbuffer
\definehyphenationfeatures
  [demo]
  [alternative=test]
\stopbuffer

\typebuffer \getbuffer

We can now use it as follows:

\starttyping
\setuphyphenation[method=traditional]
\sethyphenationfeatures[demo]
\stoptyping

When applied to one the tufte example we get:

\startbuffer[demo]
\starthyphenation[traditional]
    \setuptolerance[tolerant]
    \sethyphenationfeatures[demo]
    \dontleavehmode
    \input tufte\relax
\stophyphenation
\stopbuffer

\blank \startnarrower \getbuffer[demo] \stopnarrower \blank

A more realistic (but not perfect) example is the following:

\startbuffer
\startluacode
    local packslashes = false

    local specials = {
        ["!"]  = "before", ["?"]  = "before",
        ['"']  = "before", ["'"]  = "before",
        ["/"]  = "before", ["\\"] = "before",
        ["#"]  = "before",
        ["$"]  = "before",
        ["%"]  = "before",
        ["&"]  = "before",
        ["*"]  = "before",
        ["+"]  = "before", ["-"]  = "before",
        [","]  = "before", ["."]  = "before",
        [":"]  = "before", [";"]  = "before",
        ["<"]  = "before", [">"]  = "before",
        ["="]  = "before",
        ["@"]  = "before",
        ["("]  = "before",
        ["["]  = "before",
        ["{"]  = "before",
        ["^"]  = "before", ["_"]  = "before",
        ["`"]  = "before",
        ["|"]  = "before",
        ["~"]  = "before",
        --
        [")"]  = "after",
        ["]"]  = "after",
        ["}"]  = "after",
    }

    languages.hyphenators.traditional.installmethod("url",
        function(dictionary,word,n)
            local t = { }
            local p = nil
            for i=1,#word do
                local w = word[i]
                local s = specials[w]
                if s == "after" then
                    s = {
                        start  = 1,
                        length = 1,
                        after  = w,
                        left   = false,
                        right  = false,
                    }
                    specials[w] = s
                elseif s == "before" then
                    s = {
                        start  = 1,
                        length = 1,
                        before = w,
                        left   = false,
                        right  = false,
                    }
                    specials[w] = s
                end
                if not s then
                    s = false
                elseif w == p and w == "/" then
                    t[i-1] = false
                end
                t[i] = s
                if packslashes then
                    p = w
                end
            end
            return t
        end
    )
\stopluacode
\stopbuffer

\typebuffer \getbuffer

Again we define a plug:

\startbuffer
\definehyphenationfeatures
  [url]
  [characters=all,
   alternative=url]
\stopbuffer

\typebuffer \getbuffer

So, we only break a line after symbols.

\startlinecorrection[blank]
    \starthyphenation[traditional]
        \tt
        \sethyphenationfeatures[url]
        \scale[width=\hsize]{\hyphenatedcoloredword{http://www.pragma-ade.nl}}
    \stophyphenation
\stoplinecorrection

A quick test can look as follows:

\startbuffer
\starthyphenation[traditional]
    \sethyphenationfeatures[url]
    \tt
    \dontcomplain
    \hsize 1mm
    http://www.pragma-ade.nl
\stophyphenation
\stopbuffer

\typebuffer

Or:

\getbuffer

\stopsection

\startsection[title=Blocking ligatures]

Yet another predefined feature is the ability to block a ligature. In
traditional \TEX\ this can be done by putting a \type {{}} between
the characters, although that effect can get lost when the text is
manipulated. The natural way to do this in a \UNICODE\ environment
is to use the special characters \type {zwj} and \type {zwnj}.

We use the following example lines:

\startbuffer[sample]
supereffective \blank
superef\zwnj fective
\stopbuffer

\typebuffer[sample]

and define two featuresets:

\startbuffer
\definehyphenationfeatures
  [demo-1]
  [characters=\zwnj\zwj,
   joiners=yes]

\definehyphenationfeatures
  [demo-2]
  [joiners=no]
\stopbuffer

\typebuffer \getbuffer

We limit the width to 1mm and get:

\startlinecorrection[blank]
\bTABLE[option=stretch,offset=.5ex]
    \bTR
        \bTD \tx
            \type{method=default}
        \eTD
        \bTD \tx
            \type{method=traditional}
        \eTD
        \bTD \tx
            \type{method=traditional}\par
            \type{featureset=demo-1}
        \eTD
        \bTD \tx
            \type{method=traditional}\par
            \type{featureset=demo-2}
        \eTD
    \eTR
    \bTR
        \bTD
            \hsize 1mm \dontcomplain
            \starthyphenation[default]
                \getbuffer[sample]
            \stophyphenation
        \eTD
        \bTD
            \hsize 1mm \dontcomplain
            \starthyphenation[traditional]
                \getbuffer[sample]
            \stophyphenation
        \eTD
        \bTD
            \hsize 1mm \dontcomplain
            \starthyphenation[traditional]
                \sethyphenationfeatures[demo-1]
                \getbuffer[sample]
            \stophyphenation
        \eTD
        \bTD
            \hsize 1mm \dontcomplain
            \starthyphenation[traditional]
                \sethyphenationfeatures[demo-2]
                \getbuffer[sample]
            \stophyphenation
        \eTD
    \eTR
\eTABLE
\stoplinecorrection

\stopsection

\startsection[title=Special characters]

The \type {characters} example can be used (to some extend) to do the
same as the breakpoints mechanism (compounds).

\startbuffer
\definehyphenationfeatures
  [demo-3]
  [characters={()[]}]
\stopbuffer

\typebuffer \blank \getbuffer \blank

\startbuffer[demo]
\starthyphenation[traditional]
    \sethyphenationfeatures[demo-3]
    \dontcomplain
    \hsize 1mm
    we use (super)special(ized) patterns
\stophyphenation
\stopbuffer

\typebuffer[demo] \blank \getbuffer[demo] \blank

We can make this more clever by adding patterns:

\startbuffer
\registerhyphenationpattern[en][)9]
\registerhyphenationpattern[en][9(]
\stopbuffer

\typebuffer \blank \getbuffer \blank

This gives:

\blank \getbuffer[demo] \blank

A detailed trace shows that these patterns get applied:

\starthyphenation[traditional]
    \ttx
    \showhyphenationtrace[en][(super)special(ized)]
\stophyphenation

\unregisterhyphenationpattern[en][)9]
\unregisterhyphenationpattern[en][9(]

The somewhat weird hyphens at the edges will in practice not show up because
there is always one regular character there.

\stopsection

\startsection[title=Counting]

There is not much you can do about patterns. It's a craft to make them and so
they are shipped with the distribution. In order to hyphenate well, \TEX\ looks
at some character properties. In \CONTEXT\ only the characters used in the
patterns of a language get tagged as valid in a word.

The following example illustrates that there can be corner cases. In fact, this
example might render differently depending on the patterns available. First we
define an extra language, based on French.

\startbuffer
\installlanguage[frf][default=fr,patterns=fr,factor=yes]
\stopbuffer

\typebuffer \getbuffer

Here we set the \type {factor} parameter which tells the loader that it should
look at the characters used in a special way: some count for none, and some count
for more than one when determining the min values used to determine if and where
hyphenation is to be applied.

\startbuffer
\startmixedcolumns[n=3,balance=yes]
    \hsize 1mm \dontcomplain
    \language[fr]  aesop oedipus æsop œdipus \column
    \hsize 1mm \dontcomplain
    \language[frf] aesop oedipus æsop œdipus \column
    \startexceptions æ-sop \stopexceptions
    \hsize 1mm \dontcomplain
    \language[frf] aesop oedipus æsop œdipus
\stopmixedcolumns
\stopbuffer

\typebuffer

We get three (when writing this manual) different columns:

\getbuffer

The trick is in the \type {factor}: when set to \type {yes} an \type {æ} is
counted as two characters. Combining marks count as zero but you will not
find them being used as we already resolve them in an earlier stage.

\startluacode
context.startcolumns { n = 2 }
context.starttabulate { "|Tc|c|c|l|" }
for u, data in table.sortedhash(languages.hjcounts) do
    if data.category ~= "combining" then
        context.NC() context("%05U",u)
        context.NC() context("%c",u)
        context.NC() context(data.count)
        context.NC() context(data.category)
        context.NC() context.NR()
    end
end
context.stoptabulate()
context.stopcolumns()
\stopluacode

It is very unlikely to find an \type {ffi} in the input and even an \type {ij} is
rare. The \type {æ} is marked as character and the \type {œ} a ligatyure in
\UNICODE. Maybe all the characters here are dubious but al least we provide a
way to experiment with them.

\stopsection

\startsection[title=Tracing]

Among the tracing options (low level trackers) there is one for pattern developers:

\startbuffer
\usemodule[s-languages-hyphenation]

\startcomparepatterns[de,nl,en,fr]
    \input zapf \quad (\showcomparepatternslegend)
\stopcomparepatterns
\stopbuffer

\typebuffer

The different hyphenation points are shown with colored bars. Some valid points
might not be shown because the font engine can collapse successive
discretionaries.

\getbuffer

\stopsection

\startsection[title=Neat tricks]

The following two examples are for users to test. The first one shows all hyphenation
points in a paragraph:

\starttyping
\bgroup
    \setupalign[flushright]
    \hyphenpenalty-100000
    \input tufte
    \par % force hyphenation
\egroup
\stoptyping

The second one shows the cases where a hyphenated word ends a page:

\starttyping
\bgroup
    \page
    \interlinepenalty10000
    \brokenpenalty-10000
    \input tufte
    \page
\egroup
\stoptyping

A less space consuming variant of that one is:

\starttyping
\bgroup
    \setbox\scratchboxone\vbox \bgroup
        \interlinepenalty10000
        \brokenpenalty-10000
        \input tufte
    \egroup
    \doloop {
        \ifvoid\scratchboxone
            \hrule
            \exitloop
        \else
            \setbox\scratchboxtwo\vsplit\scratchboxone to 1pt
            \hrule
            \unvbox\scratchboxtwo
        \fi
    }
\egroup
\stoptyping

\stopsection

\stopchapter

\stopcomponent
