% language=us

\startcomponent bidi-lua

\environment bidi-style

\startchapter[title={The \LUA\ interface}]

We assume that you run \CONTEXT\ \MKIV\ in combination with \LUATEX. Direction
support in this engine has been improved over time. Originally the \OMEGA\
(\ALEPH) direction model was used but in the meantime it has been stripped to the
basics, and what used to be so called whatsits (extension nodes) are now first
class nodes. Of the many directions only four are kept in \LUATEX\ and they are
indicated by three letters:

\starttabulate[|l|l|l|]
\NC 0 \NC \type {TLT} \NC left to right \NC \NR
\NC 1 \NC \type {TRT} \NC right to left \NC \NR
\NC 2 \NC \type {LTL} \NC not used in context (obsolete) \NC \NR
\NC 3 \NC \type {RTT} \NC not used in context (obsolete) \NC \NR
\stoptabulate

In \LUAMETATEX, and therefore \CONTEXT\ \LMTX\ we only have the first two.
Therefore in \LMTX\ you normally don't have to worry about checking for them at
the \LUA\ end because they are irrelevant for calculations (the vertical ones
swapped the horizontal and vertical progression). Also, when really needed, we
use the \type {direction} keys with numerical indicators, so zero for \type {l2r}
and one for \type {r2l}. These values are used for local par nodes as well as
direction nodes. In addition a direction node has a subtype:

\starttabulate[|l|l|l|]
\NC 0 \NC \type {normal} \NC comparable to \type {+} \NC \NR
\NC 1 \NC \type {cancel} \NC comparable to \type {-} \NC \NR
\stoptabulate

\stopchapter

\stopcomponent
