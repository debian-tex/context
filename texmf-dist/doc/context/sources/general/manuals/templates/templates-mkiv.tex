% language=us runpath=texruns:manuals/templates

% author    : Hans Hagen
% copyright : ConTeXt Development Team
% license   : Creative Commons Attribution ShareAlike 4.0 International
% reference : pragma-ade.nl | contextgarden.net | texlive (related) distributions
% origin    : the ConTeXt distribution
%
% comment   : Because this manual is distributed with TeX distributions it comes with a rather
%             liberal license. We try to adapt these documents to upgrades in the (sub)systems
%             that they describe. Using parts of the content otherwise can therefore conflict
%             with existing functionality and we cannot be held responsible for that. Many of
%             the manuals contain characteristic graphics and personal notes or examples that
%             make no sense when used out-of-context.
%
% comment   : Some chapters might have been published in TugBoat, the NTG Maps, the ConTeXt
%             Group journal or otherwise. Thanks to the editors for corrections. Also thanks
%             to users for testing, feedback and corrections.

\usemodule[art-01,abr-02]

\definecolor[maincolor] [r=.4]
\definecolor[extracolor][b=.4]

\setupbodyfont
  [10pt]

\usesymbols
  [cc]

\setuptyping
  [color=extracolor]

\setuptype
  [color=extracolor]

\setuphead
  [section]
  [color=maincolor]

\setupinteraction
  [hidden]

\startdocument
  [metadata:author=Hans Hagen,
   metadata:title=LMX Templates,
   author=Hans Hagen,
   affiliation=PRAGMA ADE,
   location=Hasselt NL,
   title=LMX Templates,
   extra-1=LMX,
   extra-2=TEMPLATES,
   extra-3=HANS HAGEN,
   support=www.contextgarden.net,
   website=www.pragma-ade.nl]

\startMPpage

    StartPage;

    numeric n, m ; n := 3 * 4 ; m := 4 * 4 ;
    numeric w, h ; w := PaperWidth/n ; h := PaperHeight/m ;
    numeric max ; max := 20 ;

    for i=1 upto n :
        for j=1 upto m :
            fill
                unitsquare
                xysized (w+1/5,h+1/5) % a little overlap to get rid of viewer inaccuracies
                shifted ((i-1)*w-1/10,(j-1)*h-1/10)
                withcolor (.5[red,blue] randomized(.75,.75,.75))
            ;
        endfor ;
    endfor ;

    path p ; p := Page enlarged -5mm ;

    pair a[] ; % <

    a[1] := .80[lrcorner p,urcorner p] ;
    a[2] := .50[llcorner p,ulcorner p] ;
    a[3] := .20[lrcorner p,urcorner p] ;

    pair b[] ; % \

    b[1] := ulcorner p ;
    b[2] := center   p ;
    b[3] := lrcorner p ;

    path c[] ; % from < (xml) to \ (tex)

    c[1] := a[1] .. b[1] ;
    c[2] := a[2] .. b[2] ;
    c[3] := a[3] .. b[3] ;

    linecap := butt ;

    numeric fraction ;

    for i=1 step 1 until max :
        fraction := i/max ;
        draw
            ((point fraction along c[1]) -- (point fraction along c[2]) -- (point fraction along c[3]))
            withpen pencircle scaled 5mm
            withcolor .75[(max+1-i)*green/n,i*yellow/max]
            withtransparency (1,.5)
            ;
    endfor ;

    draw
        textext.rt("\ssbf{\documentvariable{extra-1}}")
        xsized (7w)
        shifted (.8w,3h)
        withcolor white
    ;

    draw
        textext.rt("\ssbf{\documentvariable{extra-2}}")
        xsized (8w)
        shifted (w,h)
        withcolor white
    ;

    draw
        textext.ulft("\ssbf{\documentvariable{extra-3}}")
        rotated 90
        ysized (5.9h)
        shifted (PaperWidth-1.2w,PaperHeight/2+2.95h)
        withcolor white
    ;

    StopPage;

\stopMPpage

\startsubject[title={Contents}]

\placelist[section][criterium=all,interaction=all]

\stopsubject

\startsection [title={Introduction}]

{\em This manual is not finished yet. The main reason is that what is described
here is an afternoon experiment resulting in a dozen lines of \LUA\ glue code
that builds upon an already existing mechanism. When users like this, I will
extend the basic \LMX\ handler to suit the \TEX\ end better. There will also be
also support for cache based and in||document templates.)}

The acronym \type {lmx} stands for document that are a mix of \LUA\ and \XML\ and
is just the three letters \type {xml} reversed. Such documents showed up pretty
soon in \MKIV\ while I was exploring ways to present debugging and error
information to users using \XML. As a consequence this is one of the older
mechanisms available, although I doubt if users start looking for it when they
start using \CONTEXT.

Anyhow, because we also use \LMX\ for populating web pages, at some point I wondered
if using the same approach for \TEX\ files made sense. I'm still not sure about it
but who knows where this ends up. Currently code is shared but in the future we might
end up with a variant that adds some more flexibility.

\stopsection

\startsection [title={How it works}]

First of all, using this mechanism involves yet another kind of \type {mk}, so
now we have:

\starttabulate[|TB||]
\HL
\NC mkii \NC The old version of \CONTEXT, using \PDFTEX,\XETEX, etc. \NC \NR
\HL
\NC mkiv \NC The new version of \CONTEXT, using \LUATEX. \NC \NR
\NC mkvi \NC Similar to \MKIV\ but with named macro parameters. \NC \NR
\HL
\NC mkix \NC A \MKIV\ file mixed with \LUA\ wrapped in \XML\ processing instructions. \NC \NR
\NC mkxi \NC Similar to \MKIX\ but with named macro parameters. \NC \NR
\HL
\stoptabulate

The nice thing about sticking to wrapping in angle brackets is that it plays nice
with syntax highlighting. Before we show an example of such a mix, we first point
out that loading (and thereby conversion) happens automatically. A \type {mkix} or
\type {mkxi} file is just a regular \CONTEXT\ file. In the test suite there
is a demo file that can be included like this:

\starttyping
\input lmxlike-001.mkxi
\stoptyping

This file is loaded and converted on the fly. No caching takes place, but in due time
we can use the cache built into the \LMX\ handlers if needed. The template itself
can be fed with variables in the \type {document} namespace:

\starttyping
\starttext

  \startluacode
     document.variables.text = "set"
  \stopluacode

  \input lmxlike-001.mkxi

\stoptext
\stoptyping

Instead of a special suffix, you can also force conversion with the \type {macros}
directive:

\starttyping
% macros=mkix
\stoptyping

Part of the mentioned looks as follows:

\starttyping
\bTABLE
  <?lua for i=1,40 do ?>
    \bTR
      <?lua for j=1,5 do ?>
        \bTD
          cell (<?lua inject(i) ?>,<?lua inject(j)?>)
          is <?lua inject(variables.text or "unset") ?>
        \eTD
      <?lua end ?>
    \eTR
  <?lua end ?>
\eTABLE
\stoptyping

The \type {<?lua ... ?>} command is conceptually different from (say) \type
{\ctxlua} in the sense that the later executes some \LUA\ code at that spot,
while in a template a \LUA\ function is constructed out of the whole that gets
executed. In fact, we use \LUA\ to construct an input file.

For the moment we only mention the predefined \type {inject} command. There are
some more but they make more sense for \XML\ and \HTML\ and in due time this will
be decoupled so that we can have dedicated helpers. Even the \XML\ and \HTML\
part is somewhat in flux.

The previous example can of course also be done differently. It's a matter of
taste and usage what method gets used:

\starttyping
\startluacode
  context.bTABLE()
    for i=1,40 do
      context.bTR()
        for j=1,5 do
          context.bTD()
            context("cell (%s,%s) is %s",i,j,document.variables.text or "unset")
          context.eTD()
        end
      context.eTR()
    end
  context.eTABLE()
\stopluacode
\stoptyping

The difference between a \MKIX\ and \MKXI\ file is the same as between a \MKIV\
and \MKVI\ file: the way macros can be defined:

\starttyping
\def\testmacro#one#two{[#one,#two]}

\testmacro{1}{2}
\testmacro{one}{two}
\testmacro{second}{first}
\stoptyping

In practice one will seldom need macro definitions in a template file but the
possibility is provided.

\stopsection

\startsubject[title={Colofon}]

\starttabulate[|B|p|]
\NC author    \NC \documentvariable{author}, \documentvariable{affiliation}, \documentvariable{location} \NC \NR
\NC version   \NC \currentdate \NC \NR
\NC website   \NC \documentvariable{website} \endash\ \documentvariable{support} \NC \NR
\NC copyright \NC \symbol[cc][cc-by-sa] \NC \NR
\stoptabulate

\stopsubject

\stopdocument
