% language=us runpath=texruns:manuals/mathml

% author    : Hans Hagen
% copyright : PRAGMA ADE & ConTeXt Development Team
% license   : Creative Commons Attribution ShareAlike 4.0 International
% reference : pragma-ade.nl | contextgarden.net | texlive (related) distributions
% origin    : the ConTeXt distribution
%
% comment   : Because this manual is distributed with TeX distributions it comes with a rather
%             liberal license. We try to adapt these documents to upgrades in the (sub)systems
%             that they describe. Using parts of the content otherwise can therefore conflict
%             with existing functionality and we cannot be held responsible for that. Many of
%             the manuals contain characteristic graphics and personal notes or examples that
%             make no sense when used out-of-context.
%
% comment   : Some chapters might have been published in TugBoat, the NTG Maps, the ConTeXt
%             Group journal or otherwise. Thanks to the editors for corrections. Also thanks
%             to users for testing, feedback and corrections.

\environment envexamp

\startbuffer[colofon]

This document shows a few formulas coded in \MATHML\ and typeset by \CONTEXT. The
examples are taken from an old copy of \quote {Handbook of Chemistry and Physics}
as well as \quote {Wiskunde voor het HBO (R.~van Asselt et al.)}. We assume no
responsibility for the coding being 100\% all correct.

These examples are typeset using the default settings. There are several ways to
influence the look and feel of a formula. Details on how to process \MATHML\ can
be found in the \XML\ related documentation that comes with \CONTEXT.

You can get more information on \CONTEXT\ at our website, in \TEX\ usergroup
publications and in (the archives of) the \CONTEXT\ mailing list.

\blank[2*big]

\startlines
Hans Hagen
Hasselt, January 2001 / June 2008 / June 2015
\goto{www.pragma-ade.com}[url(http://www.pragma-ade.com)]
\stoplines

\stopbuffer

\startdocument
  [color=darkred,
   columns=2,
   title=MathML in \ConTeXt]

\section{Derivatives}

\showXMLsample {pc-d-001}
\showXMLsample {pc-d-002}
\showXMLsample {pc-d-003}
\showXMLsample {pc-d-004}
\showXMLsample {pc-d-005}
\showXMLsample {pc-d-006}
\showXMLsample {pc-d-007}
\showXMLsample {pc-d-008}
\showXMLsample {pc-d-009}
\showXMLsample {pc-d-010}
\showXMLsample {pc-d-011}
\showXMLsample {pc-d-043}
\showXMLsample {pc-d-051}

\section{Integrals}

\showXMLsample {pc-i-022}
\showXMLsample {pc-i-380}

\section{Series}

\showXMLsample {pc-s-001}
\showXMLsample {pc-s-002}
\showXMLsample {pc-s-003}
\showXMLsample {wh-s-001}
\showXMLsample {wh-s-002}

\section{Logs}

\showXMLsample {wh-l-001}
\showXMLsample {wh-l-002}
\showXMLsample {wh-l-003}
\showXMLsample {wh-l-004}

\section{Goniometrics}

\showXMLsample {wh-g-001}
\showXMLsample {wh-g-002}
\showXMLsample {wh-g-003}
\showXMLsample {wh-g-004}
\showXMLsample {wh-g-005}
\showXMLsample {wh-g-006}
\showXMLsample {wh-g-007}
\showXMLsample {wh-g-008}
\showXMLsample {wh-g-009}
\showXMLsample {wh-g-010}
\showXMLsample {wh-g-011}
\showXMLsample {wh-g-012}
\showXMLsample {wh-g-013}
\showXMLsample {wh-g-014}
\showXMLsample {wh-g-015}
\showXMLsample {wh-g-016}

% \section{Openmath}

% \showXMLsample {openmath-1001}
% \showXMLsample {openmath-1002}
% \showXMLsample {openmath-1003}
% \showXMLsample {openmath-1004}
% \showXMLsample {openmath-1005}
% \showXMLsample {openmath-1006}

% \showXMLsample {openmath-2001}
% \showXMLsample {openmath-2002}
% \showXMLsample {openmath-2003}
% \showXMLsample {openmath-2004}
% \showXMLsample {openmath-2005}
% \showXMLsample {openmath-2006}
% \showXMLsample {openmath-2007}
% \showXMLsample {openmath-2008}

\stopdocument
