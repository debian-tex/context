% language=us runpath=texruns:manuals/rules

% author    : Hans Hagen
% copyright : ConTeXt Development Team
% license   : Creative Commons Attribution ShareAlike 4.0 International
% reference : pragma-ade.nl | contextgarden.net | texlive (related) distributions
% origin    : the ConTeXt distribution
%
% comment   : Because this manual is distributed with TeX distributions it comes with a rather
%             liberal license. We try to adapt these documents to upgrades in the (sub)systems
%             that they describe. Using parts of the content otherwise can therefore conflict
%             with existing functionality and we cannot be held responsible for that. Many of
%             the manuals contain characteristic graphics and personal notes or examples that
%             make no sense when used out-of-context.
%
% comment   : Some chapters might have been published in TugBoat, the NTG Maps, the ConTeXt
%             Group journal or otherwise. Thanks to the editors for corrections. Also thanks
%             to users for testing, feedback and corrections.

\setupbodyfont
  [dejavu,10pt]

\setuphead
  [section]
  [style=\bfb]

\setupwhitespace
  [big]

\setuplayout
  [header=15mm,
   topspace=15mm,
   footer=0mm,
   bottomspace=20mm,
   height=middle,
   backspace=20mm,
   cutspace=20mm,
   width=middle]

\usemodule[x][setups-basics]

\loadsetups[context-en]
\loadsetups[i-linefiller]

\startdocument

\startMPpage

    StartPage ;

        linecap := butt ;

        fill Page withcolor .25[yellow/4,green/2] ;

        path p ; p := (ulcorner Page ..  urcorner Page .. lrcorner Page) ;

        draw image (
            for i=1/200 step 1/200 until 1 :
                draw p scaled i dashed dashpattern (on 4 randomized 2 off 4 randomized 2) ;
            endfor ;
        ) withcolor white ;

        draw anchored.urt(
            textext("\ss RULES") xsized .5PaperWidth,
            urcorner Page shifted(-15mm,-20mm)
        ) withcolor white ;

        draw anchored.urt(
            textext("\ss HANS HAGEN") xsized .5PaperWidth,
            lrcorner Page shifted(-15mm,40mm)
        ) withcolor white ;

        draw anchored.urt(
            textext("\ss A CONTEXT MKIV MANUAL") xsized .5PaperWidth,
            lrcorner Page shifted(-15mm,20mm)
        ) withcolor white ;

        setbounds currentpicture to Page ;

    StopPage ;

\stopMPpage

\startsubject[title=Introduction]

This manual describes just one type of rules: those that somehow magically are
bound to the typeset text. We will mention a few mechanisms that relate to this
in the sense that they share some of the underlaying code and logic. The term
\quotation {rules} should be interpreted liberally as we can kick in some
\METAPOST\ which then can get us away from straight rules.

This manual will not be that extensive, after all these mechanisms are not that
complex to configure.

\stopsubject

\startsubject[title=Underlining and overstriking]

Already in \CONTEXT\ \MKII\ we had underlining available but with some
limitations. We could handle more than one word but at some point you hit the
limits of the engine. The \MKIV\ implementation is more flexible. In fact you can
underline a whole document (which was actually a request from a user). This
feature was also used by a collegue who was experimenting with texts for
dyslectic readers.

This mechanism is generic in the sense that a framework is provided to define
rules that run alongside text. Take this:

\setupbars[foregroundcolor=darkyellow,color=darkred]

\startbuffer
\underbars {drawing \underbar{bars} under words is a typewriter leftover}
\overstrikes {striking words makes them \overstrike {unreadable} but
sometimes even \overbar {top lines} come into view.}
\stopbuffer

\typebuffer

This shows up as:

\getbuffer

We can best explain what happens by looking at how these commands are
defined:

\starttyping
\definebar[overbar]   [method=1,dy=0.4, offset=1.8, continue=yes]
\definebar[underbar]  [method=1,dy=-0.4,offset=-0.3,continue=yes]
\definebar[overstrike][method=0,dy=0.4, offset=0.5, continue=yes]

\definebar
  [understrike]
  [method=0,
   offset=1.375,
   rulethickness=2.5,
   continue=yes,
   order=background]

\definebar[overbars]    [overbar]    [continue=no]
\definebar[underbars]   [underbar]   [continue=no]
\definebar[overstrikes] [overstrike] [continue=no]
\definebar[understrikes][understrike][continue=no]
\stoptyping

The formal definitions of the commands are show in \definition [definebar, setupbar].

\showdefinition{definebar}
\showdefinition{setupbar}

The \type {dy} parameter specifies the shift up or down. The offset defines how
nested bars are shifted. The \type {method} determines centering of the bar: we
set it to zero when we want an overstrike. The \type {continue} parameter is
responsible for drawing over spaces and the \type {order} determines the layering.

The units are either hard coded values like points or relate to the font at the spot
of the bar. Here are some examples:

\startbuffer
\setupbars[unit=mm,rulethickness=1]     bar \underbar{foo} bar\quad
\setupbars[unit=ex,rulethickness=1]     bar \underbar{foo} bar\quad
\setupbars[unit=pt,rulethickness=1]     bar \underbar{foo} bar\quad
\setupbars[unit=pt,rulethickness=10pt]  bar \underbar{foo} bar
\stopbuffer

\typebuffer \blank \start \getbuffer \stop \blank

As if underlining wasn't already bad enough, of course at some point there came a
request for dashed lines.

\startbuffer
test \underrandoms{test me} and \underrandom{test} or \underrandom{grep}
test \underdashes {test me} and \underdash  {test} or \underdash  {grep}
test \underdots   {test me} and \underdot   {test} or \underdot   {grep}
\stopbuffer

\typebuffer

The above variants are predefined and render as:

\startlines
\tfb \getbuffer
\stoplines

A graphic is defined as follows. It boils down to drawing one or more shapes. In
this example we also force a specific boundingbox so that the result gets
positioned right.

\starttyping
\startuseMPgraphic{rules:under:...}
    draw
        ((0,RuleDepth) -- (RuleWidth,RuleDepth))
        shifted (0,RuleFactor*RuleOffset)
        withpen pencircle scaled RuleThickness
        withcolor RuleColor ;
    setbounds currentpicture to unitsquare xysized(RuleWidth,RuleHeight) ;
\stopuseMPgraphic
\stoptyping

The following variables are available:

\starttabulate[|T|||]
\BC variable      \BC type     \BC meaning                                 \NC \NR
\ML
\NC RuleDirection \NC string   \NC the direction of the line               \NC \NR
\NC RuleOption    \NC string   \NC whatever the caller finds useful        \NC \NR
\NC RuleWidth     \NC number   \NC the requested width of the rule         \NC \NR
\NC RuleHeight    \NC number   \NC the requested height of the rule        \NC \NR
\NC RuleDepth     \NC number   \NC the requested depth of the rule         \NC \NR
\NC RuleThickness \NC number   \NC the linewidth                           \NC \NR
\NC RuleFactor    \NC number   \NC the set factor (e.g. an \type {ex})     \NC \NR
\NC RuleOffset    \NC number   \NC an (optional) offset in case of nesting \NC \NR
\NC RuleColor     \NC color    \NC the color                               \NC \NR
\stoptabulate

The \type {RuleFactor} can be used as multiplier for the \type {RuleOffset}.
Later we will see an example of how to use the \type {RuleDirection} and \type
{RuleOption}.

The extra under commands are defined as follows. Watch the \type {mp} parameter:
it refers to a graphic.

\starttyping
\definebar
  [undergraphic]
  [mp=rules:under:dash,
   offset=-.2,
   order=background]

\definebar[underrandom] [undergraphic][mp=rules:under:random]
\definebar[underrandoms][underrandom] [continue=yes]

\definebar[underdash]   [undergraphic][mp=rules:under:dash]
\definebar[underdashes] [underdash]   [continue=yes]

\definebar[underdot]    [undergraphic][mp=rules:under:dots]
\definebar[underdots]   [underdot]    [continue=yes]
\stoptyping

A nasty side effect of the implementation is that because we look mostly at glyphs,
optionally separated by glue or kern some text might get unseen and therefore not
treated.

\startbuffer
\underbars{We see this \high{\tfxx ®} symbol \runninghbox to 1cm{\hss} often.}
\underbar {We see this \high{\tfxx ®} symbol \runninghbox to 1cm{\hss} often.}
\stopbuffer

\typebuffer

This gives:

\startlines
\getbuffer
\stoplines

A running box is seen as text. As you (probably) expect, a nested ornamental
rule is supported as well:

\startbuffer
\underbars
  {We see this \high{\tfxx\underdot{®}} symbol \runninghbox to 1cm{\hss} often.}
\underbar
  {We see this \high{\tfxx\underdot{®}} symbol \runninghbox to 1cm{\hss} often.}
\stopbuffer

\typebuffer

This time we get (you might need a magnifier to see it):

\startlines
\getbuffer
\stoplines

We end this section with an extreme example:

\startbuffer
\definebar
  [xbarone]
  [text=\lower\exheight\hbox{\darkred \infofont +},
   repeat=yes]
\definebar
  [xbartwo]
  [text=\lower\exheight\hbox{\darkblue\infofont +},
   repeat=yes,
   continue=yes]
\stopbuffer

\typebuffer \getbuffer

\startbuffer
Klein    : \xbarone{\samplefile   {klein}\removeunwantedspaces}\par
Sapolsky : \xbartwo{\samplefile{sapolsky}\removeunwantedspaces}\par
\stopbuffer

\typebuffer \getbuffer

As a reminder that one can keep things simple, here are a few more examples that
use defaults (and no colors):

\startbuffer
\underbar {\underbar {\samplefile{ward}}}\blank
\underbar {\underdot {\samplefile{ward}}}\blank
\underbar {\underdot {\samplefile{ward}}}\blank
\underdot {\underbar {\samplefile{ward}}}\blank
\underbars{\underdot {\samplefile{ward}}}\blank
\underbar {\underdots{\samplefile{ward}}}\blank
\underdots{\underdots{\samplefile{ward}}}\blank
\stopbuffer

\typebuffer {\setupbars[foregroundcolor=,color=]\getbuffer}

\stopsubject

\startsubject[title=Shifting]

We mention shifting here because it shares code with the bars. There are two
shifts defined but you can define more:

\starttyping
\defineshift
  [shiftup]
  [method=0,
   dy=-1,
   unit=ex,
   continue=yes,
   style=\txx]

\defineshift
  [shiftdown]
  [method=1,
   dy=.3,
   unit=ex,
   continue=yes,
   style=\txx,
   color=]
\stoptyping

An example of using the commands defined this way is:

\startbuffer
Let's go \shiftup{up} and \shiftdown{down} a bit!
\stopbuffer

\typebuffer

or: \inlinebuffer\ Here we just shift words but you can shift more than that
although I haven't yet seen a useful example of that:

\startbuffer
We can \shiftup {\samplefile{tufte}} whole paragraphs if we really want.
\stopbuffer

\typebuffer

\getbuffer

The formal definitions are given in \definition[defineshift, setupshift,
startshift]. The \type {align} switch is there for directional (and testing)
purposes and is normally not used (or even useful in a line). The \type {dy}
is multiplied by the \type {factor} that itself can depend on the used font.

\showdefinition{defineshift}
\showdefinition{setupshift}
\showdefinition{startshift}

\stopsubject

\startsubject[title=Fillers]

The possibility of line fillers was mentioned by Mojca on the \CONTEXT\ mailing
list and it's actually not that hard to implement them. The only need I ever had
for it was to fill out lines on some legal form and that was actually just some
fun challenge in \MKII\ times. The code got lost and never made it into \CONTEXT.
This time it was added as a side effect of a thread at the tenth \CONTEXT\
meeting.

The ideas is to fill the rest of a line with some kind of (ornamental) rule. I'm
not sure what sense it makes, even in legal documents. If it is to prevent
additions then one should wonder if additions at the end of a (kind of arbitrary)
broken line is what we should be afraid of most. So, for now, let's consider it
an educational feature.

\startbuffer
\definelinefiller
  [filler-1]
  [height=.75\exheight,
   distance=.25\emwidth,
   rulethickness=.25\exheight,
   textcolor=darkyellow,
   before=\blank,
   after=\blank,
   color=darkred]

\startlinefiller[filler-1]
    \samplefile{ward}
\stoplinefiller
\stopbuffer

\typebuffer

Here we define a filler. As you can see, a rule gets added at the end of a
paragraph.

\getbuffer

\startbuffer
\startalign[flushleft,broad]
    \startlinefiller[filler-1]
        \samplefile{ward}
    \stoplinefiller
\stopalign
\stopbuffer

This time we don't justify:

\typebuffer

Now more lines get a rule appended:

\getbuffer

Before we continue it must be noted that the environment creates a paragraph. If
you don't want that you need to use \type {\setlinefiller} instead. Next we show
a \type {middle} alignment:

\startbuffer
\startalign[middle]
    \startlinefiller[filler-1]
        \samplefile{ward}
    \stoplinefiller
\stopalign
\stopbuffer

\getbuffer

\startbuffer
\startalign[middle]
    \startnarrower
        \startlinefiller[filler-1]
            \samplefile{ward}
        \stoplinefiller
    \stopnarrower
\stopalign
\stopbuffer

Let's add another level of complexity, left- and right skips:

\typebuffer

Here we get:

\getbuffer

The lines stay within the narrower boundaries but you can extend them
to the margins if you like:

\startbuffer
\startalign[middle]
    \startnarrower
        \startlinefiller[filler-1][scope=global]
            \samplefile{ward}
        \stoplinefiller
    \stopnarrower
\stopalign
\stopbuffer

\typebuffer

This looks like:

\getbuffer

You can also use a \type {left} or \type {right} scope, as in:

\startbuffer
\startalign[middle]
    \startnarrower
        \startlinefiller[filler-1][scope=right]
            \samplefile{ward}
        \stoplinefiller
    \stopnarrower
\stopalign
\stopbuffer

\typebuffer

Only the right rules extend into the margins.

\getbuffer

\startbuffer
\startalign[middle]
    \startnarrower
        \startlinefiller[filler-1][scope=right,location=right]
            \samplefile{ward}
        \stoplinefiller
    \stopnarrower
\stopalign
\stopbuffer

You can get rid of the left rules:

\typebuffer

So:

\getbuffer

Of course these rules are somewhat boring so let's now kick in some \METAPOST.

\startbuffer[mp]
\setuplinefiller
  [filler-1]
  [mp=rules:filler:demo,
  %threshold=.25\emwidth,
   color=darkred]

\startuseMPgraphic{rules:filler:demo}
    drawarrow
        if RuleDirection == "TRT" : reverse fi
            ((0,RuleHeight) -- (RuleWidth,RuleHeight))
        withpen
            pencircle scaled RuleThickness
        withcolor
            if RuleOption == "left" : complemented fi RuleColor ;
    setbounds currentpicture to
        unitsquare xysized(RuleWidth,RuleHeight) ;
\stopuseMPgraphic
\stopbuffer

\typebuffer[mp] \getbuffer[mp]

The previous example now looks like:

\getbuffer

\startbuffer
\startalign[middle,r2l]
    \startnarrower[4*middle]
        \startlinefiller[filler-1]  [scope=global]
            \samplefile{ward}
        \stoplinefiller
    \stopnarrower
\stopalign
\stopbuffer

This time we also change the direction and we can let the \METAPOST\ graphic
adapt to that by reverting the arrows.

\typebuffer

The direction \type {TRT} is \TEX\ speak for a right|-|to|-|left direction. We
use a latin script example for convenience.

\getbuffer

\startbuffer[mp]
\startuseMPgraphic{rules:filler:demo}
    drawarrow
        if RuleDirection == "TRT" : reverse fi
            if RuleOption == "right" : reverse fi
            ((0,RuleHeight) -- (RuleWidth,RuleHeight))
        withpen
            pencircle scaled RuleThickness
        withcolor
            if RuleOption == "left" : complemented fi RuleColor ;
    setbounds currentpicture to
        unitsquare xysized(RuleWidth,RuleHeight) ;
\stopuseMPgraphic
\stopbuffer

% \startbuffer
% \startnarrower[4*middle]
%     \startlinefiller[filler-1] [scope=global,align=middle]
%         \parindent   100pt
%         \parfillskip 100pt
%         \samplefile{ward}
%     \stoplinefiller
% \stopnarrower
% \stopbuffer

\startbuffer
\startnarrower[4*middle]
    \startlinefiller[filler-1] [scope=global,align={middle,r2l}]
        \parindent   100pt
        \parfillskip 100pt
        \samplefile{ward}
    \stoplinefiller
\stopnarrower
\stopbuffer

The next rendering shows what happens when we set \type {\parindent} and \type
{\parfillskip} to an excessive have a \type {100pt}.

\getbuffer[mp] \getbuffer

Here we have adapted the graphic a bit:

\starttyping
if RuleDirection == "TRT" : reverse fi
    if RuleOption == "right" : reverse fi
    ((0,RuleHeight) -- (RuleWidth,RuleHeight))
\stoptyping

\showdefinition{definelinefiller}
\showdefinition{setuplinefiller}

\stopsubject

% \startsubject[title=Backgrounds]
% \stopsubject

\startsubject[title=User rules]

Characters and rules are the only graphical elements that \TEX\ really knows
about. Even if you see images in a document, you should realize that they are
just blobs with dimensions and that the backend replaces such blobs by real
images.

The primitive operations for rules are \type {\hrule} and \type {\vrule} and the
main difference is to what way they adapt to their situation when no dimensions
are given and the mode change they trigger.

\startbuffer
hrule{\darkred   \hrule width 10cm height 3mm depth 2mm}\par
vrule{\darkyellow\vrule width 10cm height 3mm depth 2mm}\par
hrule{\darkred   \hrule width 10cm                     }\par
vrule{\darkyellow\vrule            height 3mm depth 2mm}\par

hrule{\darkred   \leaders\hrule height 1mm\relax\hfill}hrule\par
\stopbuffer

\typebuffer

When more text is to follow you should end a specification with \type {\relax} to
make sure that the scanner stops looking for more arguments. With \type {\leaders}
you can create flexible rules.

\startlinecorrection
\getbuffer
\stoplinecorrection

In \CONTEXT\ we also have so called frame rules:

\startbuffer
\color[darkred]{\frule
    width 10cm
    height 1cm
    line   1mm
\relax}
\stopbuffer

\typebuffer

This will produce a rectangle:

\startlinecorrection
\getbuffer
\stoplinecorrection

There are a few more keywords. Keep in mind that we actually have a new kind of
primitive here, so we follow the \TEX\ conventions of keywords.

\startbuffer
\ruledhbox\bgroup
    \darkgray
    \frule width 100mm height 10mm depth 8mm radius 2mm line 2pt type fill\relax
    \hskip-100mm
    \darkred
    \frule width 100mm height 10mm depth 8mm radius 2mm line 2pt\relax
    \hskip-100mm
    \hbox to 100mm{\white \bold \hfill some handy word with frames\hfill}%
\egroup
\stopbuffer

\typebuffer

Of course this is a rather low level way of doing frames and such, but when you
like that kind of low level programming you get the possibility here.

\startlinecorrection
\getbuffer
\stoplinecorrection

You can combine this with existing mechanisms. Take the following:

\startbuffer
\defineoverlay[normalframe]
  [\frule
     width \overlaywidth
     height\overlayheight
     line  \overlaylinewidth
  ]

\defineoverlay[ovalframe]
  [\frule
     width  \overlaywidth
     height \overlayheight
     line   \overlaylinewidth
     radius \overlayradius
  ]
\stopbuffer

\typebuffer \getbuffer

\startbuffer
\hbox \bgroup
    \framed                                  {test}\quad
    \framed[frame=off]                       {test}\quad
    \framed[background=normalframe,frame=off]{test}\quad
    \framed[background=normalframe,frame=off]{test}\quad
    \framed[corner=round]                    {test}\quad
    \framed[corner=round]                    {test}\quad
    \framed[background=ovalframe,frame=off]  {test}\quad
    \framed[background=ovalframe,frame=off]  {test}\quad
    \framed[background=ovalframe,frame=on]   {test}\quad
    \framed[background=ovalframe,frame=on]   {test}\quad
\egroup
\stopbuffer

This is a variant on the already available round corners:

\startlinecorrection
\getbuffer
\stoplinecorrection

The above result is accomplished with:

\typebuffer

Given the examples in the previous sections it will be no surprise that we
can also use \METAPOST.

\startbuffer
\startuseMPgraphic{demoshape:back}
    fill
        unitcircle xysized (RuleWidth,RuleHeight+RuleDepth)
        withcolor RuleColor ;
\stopuseMPgraphic

\startuseMPgraphic{demoshape:fore}
    draw
        unitcircle xysized (RuleWidth,RuleHeight+RuleDepth)
        withcolor RuleColor
        withpen pencircle scaled 4RuleThickness ;
\stopuseMPgraphic

\hbox\bgroup
    \darkgray \frule width 100mm height 10mm depth 8mm type mp line 2pt
        data {\includeMPgraphic{demoshape:back}}
    \relax
    \hskip-100mm
    \darkred \frule width 100mm height 10mm depth 8mm type mp line 2pt
        data {\includeMPgraphic{demoshape:fore}}
    \relax
    \hskip-100mm
    \hbox to 100mm{\white \bold \hfill some handy word with frames\hfill}
\egroup
\stopbuffer

\typebuffer

Or rendered:

\startlinecorrection
\getbuffer
\stoplinecorrection

% \frule
%     width  10cm
%     height  2cm
%     depth   1cm
%     line    1pt
%     radius  3mm
% \relax x

The primitive \type {\leaders} can also be used with these \type {\frule}s so
here is an example with and without:

\startbuffer
test \leaders \hrule height 1mm \hfill test \par
test \leaders \frule height 6mm depth 3mm radius 1mm\hfill test \par
\stopbuffer

\typebuffer

As you can see, the leader basically stretches the rule. That operation happens
in the backend code; the frontend is only interested in the height and depth
while the width is glue that can stretch.

\startlinecorrection
\getbuffer
\stoplinecorrection

Here are two more:

\startbuffer
\startuseMPgraphic{demoleader}
    fill
        unitcircle xysized (RuleWidth,RuleHeight+RuleDepth)
        withcolor RuleColor ;
\stopuseMPgraphic

test {\red \leaders \frule
    height 6mm
    depth  3mm
    type    mp
    data    {\includeMPgraphic{demoleader}}
\hfill} test
\stopbuffer

\typebuffer

\startlinecorrection
\getbuffer
\stoplinecorrection

And:

\startbuffer
\startuseMPgraphic{demoleader}
    drawdblarrow (0,RuleHeight) -- (RuleWidth,RuleHeight)
        withpen pencircle scaled  RuleThickness
        withcolor RuleColor ;
\stopuseMPgraphic

test {\red \leaders \frule
    height  1mm % we need at least some dimensions
    type    mp
    line    1mm
    data    {\includeMPgraphic{demoleader}}
\hfill} test
\stopbuffer

\typebuffer

\startlinecorrection
\getbuffer
\stoplinecorrection

The combination of \TEX\ and \METAPOST\ driven by \LUA\ (which is hidden from the
user here) is quite powerful and has a pretty good performance too.

The \type {\blackrule} command is the more high level way to inject a rule.

\startbuffer
\blackrule
  [width=10cm,
   height=1cm,
   depth=1cm,
   color=darkred]
\stopbuffer

\typebuffer

This produces a boring rule:

\startlinecorrection
\getbuffer
\stoplinecorrection

Again we can revert to \METAPOST:

\startbuffer
\blackrule
  [width=10cm,
   height=1cm,
   depth=1cm,
   color=darkred,
   type=mp,
   mp=demoshape:back]
\stopbuffer

\typebuffer

or:

\startlinecorrection
\getbuffer
\stoplinecorrection

The formal definition of this command is shown in \definition [setupblackrules,
blackrule].

\showdefinition{setupblackrules}
\showdefinition{blackrule}

\stopsubject

\startsubject[title=Hiding]

In education a to be filled in text is often represented by a gap in the running text
and the bar drawing mechanism supports this. THere is a predefined \type {\hiddenbar}
command:

\starttyping
\definebar
  [hiddenbar] [underbar]
  [continue=yes,empty=yes,
   left=\zwj,right=\zwj]
\stoptyping

\startbuffer
\samplefile{ward}\hiddenbar               {\color[red]{invisible}}
\samplefile{ward}\hiddenbar          {\quad\color[red]{invisible}\quad}
\samplefile{ward}\hiddenbar{\quad\quad\quad\color[red]{invisible}\quad\quad\quad}
\samplefile{ward}\hiddenbar               {\color[red]{invisible}\quad\quad\quad\quad\quad\quad}
\samplefile{ward}
\stopbuffer

\getbuffer

The previous text is generated with:

\typebuffer

Here is a variant that inserts spacing at the left and right edges. In this case
the spacing is kept at a linebreak:

\startbuffer
\definebar
  [widehiddenbar]
  [hiddenbar]
  [left={\quads[3]},
   right={\quads[3]}]

\widehiddenbar{invisible} \samplefile{weisman}
\widehiddenbar{invisible} \samplefile{weisman}
\widehiddenbar{invisible}
\stopbuffer

\typebuffer

\getbuffer

\stopsubject

\startsubject[title=Tabulate]

The previously discussed mechanism is also available in the tabulate mechanism.
We start with simple backgrounds:

\startbuffer
\starttabulate
    \NL[darkred]    foo \NC bar \NC \NR
    \NL[darkgreen]  foo \NC bar \NC \NR
    \NL[darkblue]   foo \NC \samplefile{tufte} \NC \NR
    \NL[darkgray]   foo \NC bar \NC \NR
    \NL[darkyellow] foo \NC bar \NC \NR
    \LL
\stoptabulate
\stopbuffer

\typebuffer

This comes out as:

{\startcolor[white]\getbuffer\stopcolor}

There are several two character commands that deal with this:

\starttabulate[|c|c|l|]
\BC command    \BC related to \BC effect \NC
\NC \type{\NL} \NC \type{\NC} \NC Normal with Line \NC \NR
\NC \type{\ND} \NC \type{\NC} \NC Normal with Default Line \NC \NR
\NC \type{\LB} \NC \type{\BC} \NC Bold   with Line \NC \NR
\NC \type{\DB} \NC \type{\BC} \NC Bold   with Default Line \NC \NR
\NC \type{\NF} \NC \type{\NC} \NC Normal with Filler \NC \NR
\NC \type{\NP} \NC \type{\NC} \NC Normal with Predefined Filler \NC \NR
\NC \type{\FB} \NC \type{\BC} \NC Bold   with Filler \NC \NR
\NC \type{\NA} \NC \type{\NC} \NC Normal with Auto Toggled Line \NC \NR
\NC \type{\BA} \NC \type{\BC} \NC Bold   with Auto Toggled Line \NC \NR
\stoptabulate

Before we show more, we set up tabulate:

\startbuffer[setup]
\setuptabulate
  [blank={small,samepage},
   headstyle=bold,
   rulecolor=darkred,
   rulethickness=1pt,
   background=foo,
   backgroundcolor=darkred,
   foregroundcolor=white]
\stopbuffer

\typebuffer[setup]

This time we don't set colors in the table itself:

\startbuffer
\starttabulate[|l|l|]
    \DB foos        \BC bars            \BC \NR
    \TB
    \NC foo foo foo \NC bar bar bar bar \NC \NR
    \NC foo foo foo \NC bar bar bar bar \NC \NR
    \NC foo foo foo \NC bar bar bar bar \NC \NR
    \NC foo foo foo \NC bar bar bar bar \NC \NR
    \LL
\stoptabulate
\stopbuffer

\typebuffer {\getbuffer[setup]\getbuffer}

Instead of coming up with a separate mechanism for hooking in \METAPOST\ we
use the linefiller mechanism. We use these graphics:

\startbuffer
\startuseMPgraphic{foo}
  fill unitsquare
    xyscaled (RuleWidth,RuleHeight+RuleDepth) enlarged (ExHeight/4,ExHeight/8)
    shifted  (-ExHeight/8,ExHeight/16)
    withcolor RuleColor ;
\stopuseMPgraphic

\startuseMPgraphic{bar}
  fill unitsquare
    xyscaled (RuleWidth,RuleHeight+RuleDepth) enlarged (ExHeight/4,ExHeight/8)
    shifted  (-ExHeight/8,ExHeight/16)
    randomized ExHeight
    withcolor RuleColor ;
\stopuseMPgraphic
\stopbuffer

\typebuffer \getbuffer

With these fillers:

\startbuffer
\definelinefiller[foo][mp=foo,color=darkgreen]
\definelinefiller[bar][mp=bar,color=darkred]
\stopbuffer

\typebuffer \getbuffer

An example of usage is:

\startbuffer
\linefillerhbox[foo] to 12cm{\hss\strut\white\bf FOO\hss} \blank
\linefillerhbox[bar] to 10cm{\hss\strut\white\bf BAR\hss} \blank
\linefillerhbox[foo] to  9cm{\hss\strut\white\bf FOO\hss} \blank
\linefillerhbox[bar] to 14cm{\hss\strut\white\bf BAR\hss}
\stopbuffer

\typebuffer \blank \getbuffer \blank

We can rely on the default:

\startbuffer
\starttabulate[|||]
    \PB foo         \BC bars            \BC \NR
    \NC foo foo foo \NC bar bar bar bar \NC \NR
    \NC foo foo foo \NC bar bar bar bar \NC \NR
    \NC foo foo foo \NC bar bar bar bar \NC \NR
    \NC foo foo foo \NC bar bar bar bar \NC \NR
\stoptabulate
\stopbuffer

\typebuffer \start\getbuffer[setup]\getbuffer\stop

or be explicit:

\startbuffer
\starttabulate[|||]
    \FB[bar] foos        \BC bars            \BC \NR
    \NC      foo foo foo \NC bar bar bar bar \NC \NR
    \NC      foo foo foo \NC bar bar bar bar \NC \NR
    \NC      foo foo foo \NC bar bar bar bar \NC \NR
    \NC      foo foo foo \NC bar bar bar bar \NC \NR
\stoptabulate
\stopbuffer

\typebuffer \start\getbuffer[setup]\getbuffer\stop

The auto variants will switch between colors:

\startbuffer
\setuptabulate[backgroundcolor:1=darkred]
\setuptabulate[backgroundcolor:2=darkgreen]
\setuptabulate[backgroundcolor:3=darkblue]

\starttabulate[|||]
    \BA foo foo foo \BC bar bar bar bar \NC \NR
    \BA foo foo foo \BC bar bar bar bar \NC \NR
    \BA foo foo foo \BC bar bar bar bar \NC \NR
    \BA foo foo foo \BC bar bar bar bar \NC \NR
    \BA foo foo foo \BC bar bar bar bar \NC \NR
    \BA foo foo foo \BC bar bar bar bar \NC \NR
\stoptabulate
\stopbuffer

\typebuffer \start\getbuffer[setup]\getbuffer\stop

\stopsubject

\startsubject[title=Framing]

The \TEX\ engine only has text and rules and all things other graphic has to come
from elsewhere. However, as \quote {elsewhere} is rather integrated in \CONTEXT\
users won't notice this limitation. It does however means that for historic
reasons we have some interesting low level phenomena. One of the oldest commands
in \CONTEXT\ is \type {\framed} which as the name indicates can draw a frame
around something. This command is demonstrated in many places so here we stick to
some remarks about the rules. Watch the following:

\startbuffer
\definecolor[t-one][r=.6,t=.5,a=1]
\definecolor[t-two][g=.6,t=.5,a=1]
\stopbuffer

\typebuffer \getbuffer

\startbuffer
\startoverlay
  {\framed
    [framecolor=t-one,rulethickness=3mm,offset=3mm,frame=closed]
    {Just a bit of text!}}
  {\framed
    [framecolor=t-two,rulethickness=3mm,offset=8mm,frame=on]
    {Just a bit of text!}}
\stopoverlay
\stopbuffer

\typebuffer

When you look closely you will notice the difference: check out the corners.

\startlinecorrection
\getbuffer
\stoplinecorrection

The normal rule drawing happens with overlaps and the reason for that is that
\TEX\ can only draw vertical and horizontal rules. We can of course avoid overlap
but quite often (and certainly in the past) viewers would show small white
stripes due to rounding errors and rendering artifacts. So, overlaps were a safe
bet. However, as nowadays we have better control over the backend the additional
\type {closed} option will draw the path as one.

\startbuffer
\dontleavehmode
\framed
  [width=4cm,height=15mm,rulethickness=3mm,framecolor=t-one,frame=off,
   rightframe=on,leftframe=on,topframe=on,bottomframe=on]
  {one}\quad
\framed
  [width=4cm,height=15mm,rulethickness=3mm,framecolor=t-two,frame=off,
  rightframe=small,leftframe=small,topframe=small,bottomframe=small]
  {two}\quad
\framed
  [width=4cm,height=15mm,rulethickness=3mm,framecolor=t-two,frame=off,
  rightframe=small,leftframe=small,topframe=small,bottomframe=on]
  {three}
\stopbuffer

\typebuffer

This example shows another variant of frames, probably unknown (and not needed)
to many users:

\startlinecorrection
\getbuffer
\stoplinecorrection

\stopsubject

\startsubject[title=Examples]

There are quite some examples in the test suite, mailing list archive and wiki,
so here only a few are given for you to run:

% example by WS on the list:

\starttyping
\definefiller
   [dots]
   [left=\dontleavehmode,
    right=\hskip\zeropoint\par]

\samplefile{knuth} \dorecurse{5}{\filler[dots]}
\samplefile{knuth} \dorecurse{5}{\hairline}
\samplefile{knuth} \thinrules[n=5]
\stoptyping

All of these produce the text plus some visual cure where to fill in
something.

\stopsubject

\stopdocument
