% language=uk
%
% copyright=pragma-ade readme=readme.pdf licence=cc-by-nc-sa

\startenvironment metafun-environment

\environment metafun-environment-layout

\usemodule[abr-01,syntax]

\startuseMPgraphic{cover page}

    numeric w ; w := PaperWidth -eps ; % or clip
    numeric h ; h := PaperHeight-eps ; % or clip

    for i=0cm step 1cm until w  :
        for j=0cm step 1cm until h :
            fill unitsquare scaled 1cm shifted (i,j) withcolor (.6+uniformdeviate.35)*white ;
        endfor ;
    endfor ;

  % clip currentpicture to unitsquare xyscaled (w,h) ;

    for i=0cm step 1cm until w+.5cm :
        draw (i,0) -- (i,h) withpen pensquare scaled .5mm withcolor .625yellow ;
    endfor ;

    for i=0cm step 1cm until h+.5cm :
        draw (0,i) -- (w,i) withpen pensquare scaled .5mm withcolor .625yellow ;
    endfor ;

\stopuseMPgraphic

\startuseMPgraphic{title page}

    StartPage ;

        \includeMPgraphic{cover page}

        picture p ; p := image (draw rawtextext("\darkred\definedfont[Sans]metafun"   )) ;
        picture q ; q := image (draw rawtextext("\darkred\definedfont[Sans]Hans Hagen")) ;
    %   picture r ; r := image (draw rawtextext("\darkred\definedfont[Sans]\doifnotmode{book}{context mkiv}")) ;
        picture r ; r := image (draw rawtextext("\darkred\definedfont[Sans]context mkiv")) ;

        p := p xsized(PaperHeight - 2cm) ;
        q := q xsized(PaperWidth  - 8cm) ;
        r := r xsized(6cm) ;

        p := p rotated 90 ;
        r := r rotated 90 ;

        draw p shifted (urcorner Page - urcorner p - (1cm,1cm) - (-1mm,0) ) ;
        draw q shifted (1cm, 1cm) ;
        draw r shifted (urcorner Page - urcorner r - (5cm,2cm) ) ;

    StopPage ;

    setbounds currentpicture to Page ;

\stopuseMPgraphic

\startuseMPgraphic{back page}

    \includeMPgraphic{cover page}

\stopuseMPgraphic

\startuseMPgraphic{small grid}

    numeric w ; w := \overlaywidth ;
    numeric h ; h := \overlayheight ;
    numeric d ; d := .25cm ;

    drawoptions(withcolor (.6+uniformdeviate.35)*white) ;

    for i=0cm step d until w  :
        for j=0cm step d until h :
            fill unitsquare scaled d shifted (i,j) ;
        endfor ;
    endfor ;

    drawoptions(withpen pencircle scaled .125mm withcolor .625yellow) ;

    for i=0 step d until w+d : draw (i,0) -- (i,h) ; endfor ;
    for i=0 step d until h+d : draw (0,i) -- (w,i) ; endfor ;

\stopuseMPgraphic

\defineoverlay[title page][\useMPgraphic{title page}]

\startnotmode[proof]
    \defineoverlay[back  page][\useMPgraphic{back  page}]
    \defineoverlay[small grid][\useMPgraphic{small grid}]
\stopnotmode

% could be all in mp

\starttexdefinition unexpanded OnGrid#1
    \hbox to \hsize \bgroup
        \ifodd\realpageno
            \hss
        \fi
        \setbox\scratchbox=\hbox {
            \color[darkred]{#1}
        }
        \scratchoffset.25cm
        \scratchwidth\wd\scratchbox
        \ifdim\scratchwidth>\zeropoint
            \advance \scratchwidth by .5\scratchoffset
            \divide  \scratchwidth by   \scratchoffset
            \multiply\scratchwidth by   \scratchoffset
            \advance \scratchwidth by  2\scratchoffset
        \else
            \scratchwidth8\scratchoffset
        \fi
        \dp\scratchbox  \scratchoffset
        \ht\scratchbox 2\scratchoffset
        \framed [
            background=small grid,
            frame=off,
            offset=overlay
        ] {
            \hbox to \scratchwidth {
                \hss
                \box\scratchbox
                \hss
            }
        }
        \unless \ifodd\realpageno
            \hss
        \fi
    \egroup
\stoptexdefinition

\setupfootertexts
  [margin]
  []
  [\OnGrid{\doifelsetext{\getmarking[chapter]}{\getmarking[chapter]}{\getmarking[title]}}]

\startuseMPgraphic{circled}
    pickup pencircle scaled 1mm ;
    drawoptions(withcolor (.6+uniformdeviate.35)*white) ;
    fill fullcircle xscaled 1.5cm yscaled 1cm ;
    drawoptions(withcolor .625yellow) ;
    draw fullcircle xscaled 1.5cm yscaled 1cm ;
\stopuseMPgraphic

\startnotmode[proof]
    \defineoverlay[circled][\useMPgraphic{circled}]
\stopnotmode

\starttexdefinition unexpanded Circled #1
    \framed [
        background=circled,
        frame=off,
        offset=overlay
    ] {
        \color[darkred]{#1}
    }
\stoptexdefinition

\setuppagenumbering
  [location=]

\setupheadertexts
  [margin]
  [][\hbox to \hsize{\hss\Circled\pagenumber\hss}]

\setupheader
  [style=bold]

\setupfooter
  [style=bold]

\startuniqueMPgraphic{titled}
    path p ; p := unitsquare xscaled OverlayWidth yscaled OverlayHeight ;
    pickup pencircle scaled 1mm ;
    drawoptions(withcolor .625yellow) ;
    draw llcorner p -- lrcorner p ;
    setbounds currentpicture to p ;
\stopuniqueMPgraphic

\defineoverlay[titled][\uniqueMPgraphic{titled}]

\starttexdefinition unexpanded ChapterCommand #1#2
    \ifconditional\headshownumber
        \ifdim\leftmarginwidth<\rightmarginwidth
          \donetrue
        \else
          \donefalse
        \fi
        \hskip-\ifdone\leftmargintotal\else\rightmargintotal\fi
        \framed [
            background=titled,
            frame=off,
            offset=0pt
        ] {
            \hbox to \ifdone\leftmarginwidth\else\rightmarginwidth\fi {
                #1
                \hss
            }
            \hskip\ifdone\leftmargindistance\else\rightmargindistance\fi
            #2
        }
    \else
        \framed [
            background=titled,
            frame=off,
            offset=0pt
        ] {
            #2
        }
    \fi
\stoptexdefinition

\starttexdefinition unexpanded TitleCommand #1#2
    \framed [
        background=titled,
        frame=off,
        offset=0pt
    ] {
        #2
    }
\stoptexdefinition

\starttexdefinition unexpanded IntroTitleCommand #1#2
    \framed [
        background=titled,
        frame=off,
        offset=0pt,
        width=\textwidth
    ] {
        \hfill
        #2
    }
\stoptexdefinition

\starttexdefinition unexpanded ExtroTitleCommand #1#2
    \framed [
        background=titled,
        frame=off,
        offset=0pt,
        width=\textwidth
    ] {
        \hss
        #2
        \hss
    }
\stoptexdefinition

\setuphead
  [chapter,section,subsection,subsubsection]
  [color=darkred]

\setuphead
  [chapter]
  [style=\bfc]

\setuphead
  [section]
  [style=\bfa]

\setuphead
  [subsection]
  [style=\bf]

\setuphead
  [subsubsection]
  [style=\bf]

\setuphead
  [chapter,section,subsection,subsubsection]
  [command=\ChapterCommand]

\setuphead
  [title,subject,subsubject,subsubsubject]
  [command=\TitleCommand]


\definehead [introsubject] [subsubject]
\definehead [extrosubject] [subsubject]

\setuphead [introsubject] [command=\IntroTitleCommand]
\setuphead [extrosubject] [command=\ExtroTitleCommand]

\setuphead
  [subsection,subsubject]
  [before=\blank,
   after=\blank]

% charts

\usemodule[chart]

\setupFLOWcharts
  [offset=0pt,
   width=6\bodyfontsize,
   height=3\bodyfontsize,
   dx=\bodyfontsize,
   dy=\bodyfontsize]

\setupFLOWshapes
  [framecolor=darkred]

\setupFLOWlines
  [color=darkyellow]

% hm, slows down the whole doc

\setupbackgrounds
  [page]
  [background={backgraphics,foreground,foregraphics}]

\defineoverlay [backgraphics] [\positionoverlay{backgraphics}]
\defineoverlay [foregraphics] [\positionoverlay{foregraphics}]

\startbuffer[shape-a]
\startuniqueMPgraphic{meta:hash}{linewidth,linecolor,angle,gap}
  if unknown context_back : input mp-back ; fi ;
  some_hash ( OverlayWidth, OverlayHeight ,
              \MPvar{linewidth}, \MPvar{linecolor} ,
              \MPvar{angle}, \MPvar{gap} ) ;
\stopuniqueMPgraphic
\stopbuffer

\startbuffer[shape-b]
\setupMPvariables
  [meta:hash]
  [gap=.25\bodyfontsize,
   angle=45,
   linewidth=\overlaylinewidth,
   linecolor=\overlaylinecolor]
\stopbuffer

\startbuffer[shape-c]
\def\metahashoverlay#1{\uniqueMPgraphic{meta:hash}{angle=#1}}

\defineoverlay[meta:hash:right]     [\metahashoverlay{ +45}]
\defineoverlay[meta:hash:left]      [\metahashoverlay{ -45}]
\defineoverlay[meta:hash:horizontal][\metahashoverlay{+180}]
\defineoverlay[meta:hash:vertical]  [\metahashoverlay{ -90}]
\stopbuffer

\startbuffer[symb-a]
\startuniqueMPgraphic{meta:button}{type,size,linecolor,fillcolor}
  if unknown context_butt : input mp-butt ; fi ;
  some_button ( \MPvar{type},
                \MPvar{size},
                \MPvar{linecolor},
                \MPvar{fillcolor} ) ;
\stopuniqueMPgraphic
\stopbuffer

\startbuffer[symb-b]
\setupMPvariables
  [meta:button]
  [type=1,
   size=2\bodyfontsize,
   fillcolor=gray,
   linecolor=darkred]
\stopbuffer

\startbuffer[symb-c]
\def\metabuttonsymbol#1{\uniqueMPgraphic{meta:button}{type=#1}}

\definesymbol[menu:left]  [\metabuttonsymbol{101}]
\definesymbol[menu:right] [\metabuttonsymbol{102}]
\definesymbol[menu:list]  [\metabuttonsymbol{103}]
\definesymbol[menu:index] [\metabuttonsymbol{104}]
\definesymbol[menu:person][\metabuttonsymbol{105}]
\definesymbol[menu:stop]  [\metabuttonsymbol{106}]
\definesymbol[menu:info]  [\metabuttonsymbol{107}]
\definesymbol[menu:down]  [\metabuttonsymbol{108}]
\definesymbol[menu:up]    [\metabuttonsymbol{109}]
\definesymbol[menu:print] [\metabuttonsymbol{110}]
\stopbuffer

\hyphenation{tool-kit}

\startbuffer[pagetext]
\subject{Edward  R. Tufte}      \input tufte   \par
\subject{Donald  E. Knuth}      \input knuth   \par
\subject{Douglas R. Hostadter}  \input douglas \page
\stopbuffer

\startbuffer[back-0]
\defineoverlay[page][\useMPgraphic{page}]
\setupbackgrounds[page][background=page]
\stopbuffer

\startbuffer[back-1]
\startuseMPgraphic{page}
  StartPage ;
    path Main ;
    if OnRightPage :
      Main := lrcorner Field[OuterMargin][Text] --
              llcorner Field[Text]       [Text] --
              ulcorner Field[Text]       [Text] --
              urcorner Field[OuterMargin][Text] -- cycle ;
    else :
      Main := llcorner Field[OuterMargin][Text] --
              lrcorner Field[Text]       [Text] --
              urcorner Field[Text]       [Text] --
              ulcorner Field[OuterMargin][Text] -- cycle ;
    fi ;
    Main := Main enlarged 6pt ;
    pickup pencircle scaled 2pt ;
    fill Page withcolor .625white ;
    fill Main withcolor .850white ;
    draw Main withcolor .625red ;
  StopPage ;
\stopuseMPgraphic
\stopbuffer

\startbuffer[back-2]
\startuseMPgraphic{page}
  StartPage ;
    pickup pencircle scaled 2pt ;
    fill Page                     withcolor .625white ;
    fill Field[OuterMargin][Text] withcolor .850white ;
    fill Field[Text]       [Text] withcolor .850white ;
    draw Field[OuterMargin][Text] withcolor .625red ;
    draw Field[Text]       [Text] withcolor .625red ;
  StopPage ;
\stopuseMPgraphic
\stopbuffer

\startbuffer[back-3]
\startuseMPgraphic{page}
  StartPage ;
    pickup pencircle scaled 2pt ;
    fill Page                            withcolor .625white ;
    fill Field[Text][Text] enlarged .5cm withcolor .850white ;
    draw Field[Text][Text] enlarged .5cm withcolor .625red ;
  StopPage ;
\stopuseMPgraphic
\stopbuffer

\startbuffer[back-4]
\startuseMPgraphic{page}
  StartPage ;
    def somewhere =
      (uniformdeviate 1cm,uniformdeviate 1cm)
    enddef ;
    path Main ;
    Main := Field[Text][Text] lrmoved somewhere --
            Field[Text][Text] llmoved somewhere --
            Field[Text][Text] ulmoved somewhere --
            Field[Text][Text] urmoved somewhere -- cycle ;
    pickup pencircle scaled 2pt ;
    fill Page withcolor .625white ;
    fill Main withcolor .850white ;
    draw Main withcolor .625red ;
  StopPage ;
\stopuseMPgraphic
\stopbuffer

\startbuffer[back-4x]
\startuseMPgraphic{page}
  StartPage ;
    path Main ; Main := Field[Text][Text] randomized 1cm ;
    pickup pencircle scaled 2pt ;
    fill Page withcolor .625white ;
    fill Main withcolor .850white ;
    draw Main withcolor .625red ;
  StopPage ;
\stopuseMPgraphic
\stopbuffer

\startbuffer[back-5]
\startuseMPgraphic{page}
  StartPage
    for i=Top,Header,Text,Footer,Bottom :
      for j=LeftEdge,LeftMargin,Text,RightMargin,RightEdge :
        draw Field[i][j] withpen pencircle scaled 2pt withcolor .625red ;
      endfor ;
    endfor ;
  StopPage
\stopuseMPgraphic
\stopbuffer

\starttexdefinition unexpanded Literature #1#2#3
    \blank
    \noindentation
    #1
    \space
    \begingroup
        \bf
        #2
    \endgroup
    \space
    #3
    \blank
\stoptexdefinition

\environment metafun-environment-samples

\startbuffer[handwrit]

\usetypescript[serif][chorus]

\definefont[SomeHandwriting][TeXGyreChorus-MediumItalic*default at 12pt]

\start \SomeHandwriting\setstrut

\startMPpage
  StartPage ;
    numeric l, n ; path p ;
    l := 1.5LineHeight ;
    n := 0 ;
    p := origin shifted (l,0) -- origin shifted (PaperWidth-l,0) ;
    for i=PaperHeight-l step -l until l :
      n := n + 1 ;
      fill         p shifted (0,i+StrutHeight) --
        reverse p shifted (0,i-StrutDepth ) -- cycle
        withcolor .85white ;
      draw p shifted (0,i)
        withpen pencircle scaled .25pt
        withcolor .5white ;
      draw p shifted (0,i+ExHeight)
        withpen pencircle scaled .25pt
        withcolor .5white ;
      draw textext.origin("\strut How are those penalty lines called
        in english? I may not steal candies ..." & decimal n)
          shifted (l,i)
          shifted (0,-StrutDepth) ;
    endfor ;
  StopPage ;
\stopMPpage
\stop
\stopbuffer

\startbuffer[gridpage]
\startMPpage
  StartPage ;
    width  := PaperWidth  ; height := PaperHeight ; unit := cm ;
    drawoptions(withpen pencircle scaled .2pt withcolor .8white) ;
    draw vlingrid(0, width /unit, 1/10, width,  height) ;
    draw hlingrid(0, height/unit, 1/10, height, width ) ;
    drawoptions(withpen pencircle scaled .5pt withcolor .4white) ;
    draw vlingrid(0, width /unit, 1,    width,  height) ;
    draw hlingrid(0, height/unit, 1,    height, width ) ;
  StopPage ;
\stopMPpage
\stopbuffer

% needed to get white backgrounds

\startmode[screen]

\startbuffer[wipe]
picture savedpicture ;
savedpicture := currentpicture ;
currentpicture := nullpicture ;
draw savedpicture withcolor black ;
draw savedpicture ;
\stopbuffer

\stopmode

\startnotmode[screen]

\startbuffer[wipe]
  % nothing to whipe
\stopbuffer

\stopnotmode

\startbuffer[backtext]

    This document introduces you in the world of the graphic programming language
    \MetaPost. Not only the language itself is covered in detail, but also the way to
    interface with the typographic language \TeX. We also present the collection of
    \MetaPost\ extensions that come with the \ConTeXt\ typesetting system. This
    collection goes under the name \MetaFun.

    \blank

    All aspects of the \MetaPost\ language are covered. The first chapters focus on
    the language itself, later chapters cover aspects like color, graphic
    inclusions, adding labels, and stepwise constructing graphics. We finish with a
    graphical overview of commands.

\stopbuffer

\startbuffer[backbanner]

  \WidthSpanningText
    {PRAGMA Advanced Document Engineering, Hasselt NL, \currentdate[year]}
    {\hsize}
    {RegularBold}

\stopbuffer

\stopenvironment
