% language=uk
%
% copyright=pragma-ade readme=readme.pdf licence=cc-by-nc-sa

\startcomponent metafun-macros

\environment metafun-environment

\startchapter[title={\METAFUN\ macros}]

\index{metafun}

\startintro

\CONTEXT\ comes with a series of \METAPOST\ modules. In this chapter we will
summarize the most important \TEX\ and \METAPOST\ macros. More information can be
found in the documentation of the modules.

\stopintro

There are several ways to use the power of \METAFUN, either or not using
\CONTEXT.

\startitemize[n]

\startitem
    You can create an independent \type {mp} file and process it with the
    \METAPOST\ program or \MPTOPDF. Depending on what features you use, the
    succes of a run depends on the proper set up of the programs that take care
    of running \TEX\ for \type {btex}.
\stopitem

\startitem
    You can embed the graphic in a \type {\startMPpage} construct and process it
    with \CONTEXT\ \MKIV. In that case you have the full \METAFUN\ functionality
    available. If for some reason you still want to use \MKII, you need to use
    \TEXEXEC\ as before processing the file, it will do a couple of checks on the
    file. It will also make sure that the temporary files (\type {mpt} for \type
    {textext}'s and \type {mpo} for outline fonts) are taken care of too.
\stopitem

\startitem
    You can integrate the graphic in the document flow, using buffers, \METAPOST\
    code sections, or (reusable) graphic containers. In that case the graphics
    are processed runtime or between runs. This happens automatically.
\stopitem

\stopitemize

Unless you want to write low level \CONTEXT\ code yourself, there is no real
reason to look into the modules that deal with \METAPOST\ support. The
traditional (partly generic) code is collected in:

\starttabulate[|lT|p|]
\NC supp-mps.tex \NC low level inclusion macros and housekeeping \NC\NR
\NC supp-mpe.tex \NC experimental extensions (like specials)     \NC\NR
\NC supp-pdf.tex \NC \METAPOST\ to \PDF\ conversion              \NC\NR
\stoptabulate

Especially the last two can be used in other macro packages. However, in
\CONTEXT\ we have regrouped the code (plus more) in other files:

\starttabulate[|lT|p|]
\NC meta-***.tex \NC definition and managing   \NC\NR
\NC mlib-***.tex \NC processing and conversion \NC\NR
\stoptabulate

The last category will certainly grow. Some of these modules are preloaded,
others can be loaded using the command \type {\useMPlibrary}, like

\starttyping
\useMPlibrary[clp,txt]
\stoptyping

for loading the predefined clipping paths and text tricks.

The \METAPOST\ code is organized in files named \type {mp-****.mp}. The core file
is \type {mp-tool.mp} and this file can comfortably be used in stand||alone
graphics. The file \type {metafun.mp} is used to load the collection of modules
into a format. The collection of \METAPOST\ code files will grow in due time, but
as long as you use the \METAFUN\ format, you don't have to keep track of the
organization of files. Most files relate to subsystems and are loaded
automatically, like the files that implement page layout support and flow charts.

Although this document is the main source of information, occasionally the source
code of \METAFUN, and in many cases the source code of \CONTEXT\ may contain
additional information and examples.

\stopchapter

\stopcomponent
