% language=us runpath=texruns:manuals/metafun

\startcomponent mfun-debugging

\environment metafun-environment

\startchapter[reference=sec:debugging,title={Debugging}]

\index{debugging}

\startintro

Those familiar with \CONTEXT\ will know that it has quite some visual debugging
features build in. So, what can you expect of the \METAPOST\ macros that come
with \CONTEXT ? In this chapter we will introduce a few commands that show some
insight in what \METAPOST\ is doing.

\stopintro

\startsection[title=Showing paths]

Since the outcome of \METAPOST\ code is in many respects more predictable than
that of \TEX\ code, we don't need that advanced visual debugging features.
Nevertheless we provide a few, that are all based on visualizing paths.

\startbuffer
path p ; p := fullcircle scaled 4cm ;
drawpath p ; drawpoints p ; drawpointlabels p ;
\stopbuffer

\startlinecorrection[blank]
\processMPbuffer
\stoplinecorrection

This visualization is achieved by using dedicated drawing commands:

\typebuffer

Since control points play an important role in defining the shape, visualizing
them may shed some insight in what \METAPOST\ is doing.

\startbuffer
path p ; p := fullcircle xscaled 4cm yscaled 3cm ;
drawpath p ; drawcontrollines p ;
drawpoints p ; drawcontrolpoints p ; drawpointlabels p ;
\stopbuffer

\typebuffer

The pre and post control points show up as small dots and are connected to their
parent point with thin lines.

\startlinecorrection[blank]
\processMPbuffer
\stoplinecorrection

You can deduce the direction of a path from the way the points are numbered, but
using an arrow to indicate the direction is more clear.

\startbuffer
path p ; p := fullcircle xscaled 4cm yscaled 3cm ;
drawarrowpath p ; drawcontrollines p ;
drawpoints p ; drawcontrolpoints p ; drawpointlabels p ;
\stopbuffer

\typebuffer

The \type {drawarrowpath} is responsible for the arrow. Especially when you are
in the process of defining macros that have to calculate intersections or take
subpaths, knowing the direction may be of help.

\startlinecorrection[blank]
\processMPbuffer
\stoplinecorrection

The next table summarizes the special drawing commands:

\starttabulate[|lT|l|]
\HL
\NC drawpath           \NC the path                         \NC \NR
\NC drawarrowpath      \NC the direction of the path        \NC \NR
\NC drawcontrollines   \NC the lines to the control points  \NC \NR
\NC drawpoints         \NC the points that make up the path \NC \NR
\NC drawcontrolpoints  \NC the control points of the points \NC \NR
\NC drawpointlabels    \NC the numbers of the points        \NC \NR
\HL
\stoptabulate

You can set the characteristics of these like you set \type {drawoptions}. The
default settings are as follows:

\starttyping
drawpathoptions   (withpen pencircle scaled 5   withcolor .8white) ;
drawpointoptions  (withpen pencircle scaled 4   withcolor   black) ;
drawcontroloptions(withpen pencircle scaled 2.5 withcolor   black) ;
drawlineoptions   (withpen pencircle scaled 1   withcolor .5white) ;
drawlabeloptions  () ;
\stoptyping

Two more options are \type {draworiginoptions} and \type {drawboundoptions} which
are used when visualizing the bounding box and origin.

\startbuffer
swappointlabels  := true ;
path p ; p := fullcircle xscaled 4cm yscaled 3cm ;
drawarrowpath p ; drawcontrollines p ;
drawpoints p ; drawcontrolpoints p ; drawpointlabels p ;
drawboundingbox p ; draworigin ;
\stopbuffer

\typebuffer

In this example we have set \type {swappointlabels} to change the place of the
labels. You can set the variable \type {originlength} to tune the appearance of
the origin.

\startlinecorrection[blank]
\processMPbuffer
\stoplinecorrection

You can pass options directly, like you do with \type {draw} and \type {fill}.
Those options override the defaults.

\startbuffer
path p ; p := fullcircle xscaled 6cm yscaled 3cm rotated 15 ;
drawarrowpath     p ;
drawcontrollines  p withcolor .625red ;
drawpoints        p withcolor .625yellow ;
drawcontrolpoints p withcolor .625yellow ;
drawpointlabels   p withcolor .625yellow ;
drawboundingbox   p ;
draworigin          withcolor .625red ;
\stopbuffer

\startlinecorrection[blank]
\processMPbuffer
\stoplinecorrection

Here we used the options:

\typebuffer

Sometimes it makes sense to draw a simple coordinate system, and for that purpose
we have three more macros. They draw axis and tickmarks.

\startbuffer
drawticks unitsquare xscaled 4cm yscaled 3cm shifted (-1cm,-1cm) ;
\stopbuffer

\typebuffer

The system drawn is based on the bounding box specification of the path passed to
the macro. You can also draw one axis, using \type {drawxticks} or \type
{drawyticks}. Here we show the previous command.

\startlinecorrection[blank]
\processMPbuffer
\stoplinecorrection

By default, the ticks are placed at .5cm distance, but you can change this by
setting \type {tickstep} to a different value.

\startbuffer
tickstep := 1cm ; ticklength := 2mm ;
drawticks fullsquare xscaled 4cm yscaled 3cm ;
tickstep := tickstep/2 ; ticklength := ticklength/2 ;
drawticks fullsquare xscaled 4cm yscaled 3cm ;
\stopbuffer

\typebuffer

The \type {ticklength} variable specifies the length of a tick. Here we
manipulated both the variables to get a more advanced system.

\startlinecorrection[blank]
\processMPbuffer
\stoplinecorrection

If visualizing a path would mean that we would have to key in al those
draw||commands, you could hardly call it a comfortable tool. Therefore, we can
say:

\startbuffer
drawwholepath fullsquare scaled 3cm rotated 30 randomized 5mm ;
\stopbuffer

\typebuffer

The \type {drawwholepath} command shows everything except the axis.

\startlinecorrection[blank]
\processMPbuffer
\stoplinecorrection

If even this is too much labour, you may say:

\starttyping
visualizepaths ;
\stoptyping

This redefines the \type {draw} and \type {fill} command in such a way that they
also show all the information.

\startbuffer
visualizepaths ;
draw fullsquare scaled 3cm rotated 30 randomized 2mm ;
\stopbuffer

\typebuffer

You may compare this feature to the \type {\showmakeup} command available in
\CONTEXT, that redefines the \TEX\ primitives that deal with boxes, glues,
penalties, and alike.

\startlinecorrection[blank]
\processMPbuffer
\stoplinecorrection

Of course you may want to take a look at the \METAPOST\ manual for its built in
(more verbose) tracing options. One command that may prove to be useful is \type
{show}, that you can apply to any variable. This command reports the current
value (if known) to the terminal and log file.

\startlinecorrection[blank]
{\showmakeup\processMPbuffer}
\stoplinecorrection

The previous picture shows what is typeset when we also say \type {\showmakeup}.
This command visualizes \TEX's boxes, skips, kerns and penalties. As you can see,
there are some boxes involved, which is due to the conversion of \METAPOST\
output to \PDF.

\starttyping
\startlinecorrection[blank]
... the graphic ...
\stoplinecorrection
\stoptyping

The small bar is a kern and the small rectangles are penalties. More details on
this debugger can be found in the \CONTEXT\ manuals and the documentation of the
modules involved.

\stopsection

\startsection[title=Comments]

Sometimes, when trouble strikes, you might want to peek in the \PDF\ file to see
what gets written there. Each graphic is marked with a number but when you
have many it might make sense to add a comment to help you locate the code.

\startbuffer
\startMPcode
    comment("test graphic") ;
    message("processing a test graphic") ;
    draw fullsquare scaled 1cm ;
\stopMPcode
\stopbuffer

\typebuffer

This renders as:

\startlinecorrection[blank]
    \getbuffer
\stoplinecorrection

On the console we get these messages:

\starttyping
metapost > message : processing a test graphic
metapost > warning : processing a test graphic
\stoptyping

And in the \PDF\ file we will find:

\starttyping
% mps graphic 1: begin
% mps graphic 1: test graphic
q 0 g 0 G 10 M 1 j 1 J
0.500000000 w
-14.173233032 -14.173233032 m
14.173233032 -14.173233032 l
14.173233032 14.173233032 l
-14.173233032 14.173233032 l
-14.173233032 -14.173233032 l
h S
0 g 0 G Q
% mps graphic 1: end
\stoptyping

Here are some examples of constructed messages:

\starttyping
message "2:                okay           (done)" ;
message "1: " & dq      & "okay" & dq & " (done)" ;
message "3: " & quotation "okay"      & " (done)" ;
message "3: " & quote     "okay"      & " (done)" ;
message "4: " & quotation 123         & " (done)" ;
message "5: " & quotation true        & " (done)" ;
message "6: " & quote     true        & " (done)" ;
message "7: " & tostring  true        & " (done)" ;
message "8: " & tostring  (1,2)       & " (done)" ;
message "9: " & topair    (1,2)       & " (done)" ;
\stoptyping

and this is what you get:

\starttyping
metapost > message : 2: okay (done)
metapost > message : 1: "okay" (done)
metapost > message : 3: "okay" (done)
metapost > message : 3: 'okay' (done)
metapost > message : 4: "123" (done)
metapost > message : 5: "true" (done)
metapost > message : 6: 'true' (done)
metapost > message : 7: true (done)
metapost > message : 8: 1 2 (done)
metapost > message : 9: (1,2) (done)
\stoptyping

\stopsection

\startsection[title=Pens]

A circular pen is applied to a path in a different way than for instance a
square pen. Circular pens are mapped onto \POSTSCRIPT\ pens while for other
pens an outline is calculated that gets filled. Take this code:

\startbuffer[a]
\startMPcode
    draw fullcircle xscaled 6cm yscaled 3cm
        withpen pensquare scaled 5mm rotated 30
        withcolor .625yellow ;
\stopMPcode
\stopbuffer

\startbuffer[b]
\startMPcode
    draw envelope pensquare scaled 5mm rotated 30 of
        (fullcircle xscaled 6cm yscaled 3cm)
        withpen pencircle scaled 1mm
        withcolor .375white ;
\stopMPcode
\stopbuffer

\startbuffer[c]
\enabletrackers[metapost.forcestroke]
\startMPcode
    draw fullcircle xscaled 6cm yscaled 3cm
        withpen pensquare scaled 5mm rotated 30
        withcolor .625red ;
\stopMPcode
\disabletrackers[metapost.forcestroke]
\stopbuffer

\typebuffer[a]

and this:

\typebuffer[b]

and:

\typebuffer[c]

When we overlay these three we get. The envelope only returns the outer curve.

\startlinecorrection[blank]
\startoverlay
    {\getbuffer[a]}
    {\getbuffer[b]}
    {\getbuffer[c]}
\stopoverlay
\stoplinecorrection

We show a few nore examples and let it to the user to come up with applications
for this feature. We start by defining a scaled pen that we apply to a simple
path that has three point.

\startbuffer
pen whateverpen ; whateverpen := makepen(fullcircle xyscaled(1/20,1/40));

path p ; p := (origin .. (1,1/2) .. (1,1/4)) scaled 10 ;
path q ; q := envelope whateverpen of p ;

draw              p withpen pencircle xyscaled(1/2,1/4)
                    withcolor "darkgreen"   withtransparency (1,.5) ;
drawpoints        p withpen pencircle scaled 1
                    withcolor "darkgray"    withtransparency (1,.5) ;
drawcontrolpoints p withpen pencircle scaled 1/2
                    withcolor "darkmagenta" withtransparency (1,.5) ;

currentpicture := currentpicture shifted (-20,0) ;

draw              p withpen pencircle xyscaled(1/2,1/4)
                    withcolor "darkgreen"   withtransparency (1,.5) ;
draw              q withpen pencircle scaled 1/5
                    withcolor "darkblue"    withtransparency (1,.5) ;
drawpoints        q withpen pencircle scaled 1
                    withcolor "darkgray"    withtransparency (1,.5) ;
drawcontrolpoints q withpen pencircle scaled 1/2
                    withcolor "darkyellow"  withtransparency (1,.5) ;
draw              q withpen pencircle scaled 1/50
                    withcolor "white" ;

currentpicture := currentpicture ysized 2cm ;
\stopbuffer

\typebuffer

On the left we see the path drawn with a circular pen. The points and control
points are also shown. The path is repeated on the right but there it gets the
envelope overlayed. We show the points and control points of the envelope: they
nearly overlap but that depends of course on the used pen.

\startlinecorrection[blank]
    \processMPbuffer
\stoplinecorrection

In order to illustrate that the envelope is an outline we blow up a piece of
this image:

\startlinecorrection[blank]
    \scale[height=3cm]{\clip[nx=10,ny=20,x=9,y=7]{\processMPbuffer}}
\stoplinecorrection

Without further explanation we let you ponder the results of the following code.

\startbuffer
drawoptions(withcolor "darkred") ;
draw (envelope pensquare of (up--left)) ;
draw (up--left) shifted (4,0) withpen pensquare ;
draw fullcircle shifted (8,0) ;
draw (envelope pensquare of fullcircle) shifted (12,0) ;
draw (fullcircle) shifted (16,0) withpen pensquare ;
draw (fullcircle) shifted (20,0) withpen pencircle ;
currentpicture := currentpicture xsized TextWidth ;
\stopbuffer

\typebuffer

The efficiency of the output of each draw differs a lot because circles are made
from eight points and because a transformed pen results in two paths. Normally
that is not something you have to loose sleep over.

\startlinecorrection[blank]
    \processMPbuffer
\stoplinecorrection

\stopsection

\startsection[title=Performance]

On the average performance of \METAPOST\ is quite okay. The original program uses
scaled numbers, which are floats packed into an integer. The library also
supports doubles, decimal and binary number models. In \CONTEXT\ we only support
scaled, double and decimal. Because the library has to support multiple models
there is more overhead and therefore it is also slower. There's also more dynamic
memory allocation going on. In the transition from \MKII\ to \MKIV\ some of the
critical code (like the code involved in passing \TEX\ states to \METAPOST) had
to be optimized, although when the \LUA\ interface was added, betters ways became
possible. We have to accept the penalty in performance and often can gain back a
lot because we have the \LUA\ interface.

One of the main bottlenecks in storing quantities. \footnote {Recently, Taco
Hoekwater has done some excellent explanations about the way \METAPOST\ scans the
input and create variables and you can find his presentations at meetings on the
\CONTEXT\ garden.} When we see something \type {a[1]} and \type {a[3]} the \type
{a} is a root variable and the \type {1} and {3} are entries in a linked list
from that root. It's not an array in the sense that there is some upper bound and
that there's also a slot \type {2}. There is order but the list is sparse. When
access is needed, for instance to do some calculations, a linear lookup (from the
head of the list) takes place. This is quite okay performance wise because
normally these list are small. The same is true for a path, which is also a
linked list. If you need point 25, it is looked up by starting at the first knot
of the path. The longer the path, the more time it takes to reach arbitrary
points. In the \LUA\ chapter we give an example of how to get around that
limitation.

Concerning the arrays, here is s trick to get around a performance bottleneck:

\starttyping
numeric foo[];

def set_foo(expr c, s) =
    foo[c] := s ;
enddef ;

def get_foo(expr c) =
    foo[c]
enddef ;
\stoptyping

If you use this as follows:

\starttyping
numeric n ; n = 123 ;

for i=1 upto 20000 :
    set_foo(i,n) ;
endfor ;

for i=1 upto 20000 :
    n := get_foo(i) ;
endfor ;
\stoptyping

the runtime can (for instance) be 3.3 seconds, but when you use the following
variant, it goes down to 0.13 seconds.

\starttyping
numeric foo[][][][]; % 12345 : 1  12  123  44 instead of 12344

def set_foo(expr c, s) =
    foo[c div 10000][c div 1000][c div 100][c] := s ;
enddef ;
def get_foo(expr c) =
    foo[c div 10000][c div 1000][c div 100][c]
enddef ;
\stoptyping

This time the lookup is not split into phases each being relatively fast. So, in
order to reach slot 1234 the engine doesn't have to check and jump over what
comes before that. You basically create a tree here: 0 (hit), 1000 (hit in one),
200 (hit in two), 34 (hit in 34). We could go to a single digit but that doesn't
save much. Before we had ways to store data at the \LUA\ end we used this a few
times in macros that dealt with data (like Alan Braslau's node and graphics
modules). This is typically something one can figure out by looking at the (non
trivial) source code.

Here is another example. In \LUA\ we can easily create a large file, like this:

\starttyping
\startluacode
  local t = { }
  for i=1,10000 do
    t[i] = string.rep(
      "here we have number " ..
      tostring(i) ..
      " out of the 10000 numbers that we will test"
    ,100)
  end
  t = table.concat(t,"\n")
  io.savedata("foo1.tmp",t)
  io.savedata("foo2.tmp",t)
  io.savedata("foo3.tmp",t)
\stopluacode
\stoptyping

We make two copies because we do two experiments and we want to treat them equal with
respect to caching.

\starttyping
\startMPcode
  string f ; f := "foo1.tmp" ;
  string s[] ;
  numeric n ; n := 0 ;
  for i=1 upto 10000 :
    s[i] := readfrom f ;
    exitif s[i] = EOF ;
    n := n + 1 ;
  endfor ;
\stopMPcode
\stoptyping

Say that this runs in 2.2 seconds, how come that the next one runs in 1.7 seconds
instead?

\starttyping
\startMPcode
  string f ; f := "foo2.tmp" ;
  string s[] ;
  string ss ;
  numeric n ; n := 0 ;
  for i=1 upto 10000 :
    ss := readfrom f ;
    exitif ss = EOF ;
    s[i] := ss ;
    n := n + 1 ;
  endfor ;
\stopMPcode
\stoptyping

The main reason is that the first case we have two lookups in the linked list
that determines variable \type {s} and the longer the list, the more time it will
take. In the second case we use an intermediate variable. Although that means
extra memory (de)allocation it still pays of. In practice you don't need to worry
too much about it but of course we can again follow the tree approach:

\startMPcode
  string f ; f := "foo3.tmp" ;
  string s[][][] ;
  string ss ;
  numeric n ; n := 0 ;
  for i=1 upto 10000 :
    ss := readfrom f ;
    exitif ss = EOF ;
    s[i div 1000][i div 100][i] := ss ;
    n := n + 1 ;
  endfor ;
\stopMPcode

This time we go down to 1.5 second. Timings could be a bit different in \MKIV\ and
\LMTX\ because in \LUAMETATEX\ all \METAPOST\ file \IO\ goes through \LUA\ but the
relative performance gains are the same. With \LUATEX\ and \MKIV\ I measures
2.9, 2.5 and 2.1 and with \LUAMETATEX\ and \LMTX\ I got 2.3, 1.7 and 1.5.

\stopsection

\stopchapter

\stopcomponent
