% language=us runpath=texruns:manuals/metafun

\startcomponent metafun-conventions

\environment metafun-environment

\startchapter[title={Conventions}]

\index {running}
\index {processing}

When reading this manual, you may be tempted to test the examples shown. This can
be done in several ways. You can make a file and process that file by \METAPOST.
Such a file looks like:

\starttyping
beginfig(1) ;
  fill fullcircle scaled 5cm withcolor red ; % a graphic
endfig ;

end .
\stoptyping

Don't forget the semi||colons that end the statements. If the file is saved as
\type {yourfile.mp}, then the file can be processed. Before we process this file,
we first need to load some basic \METAPOST\ definitions, because the built in
repertoire of commands is rather limited. Such a set is called a format. The
standard format is called {metapost} but we will use a more extensive set of
macros \type {metafun}. In the past such a set was converted into a \type {mem}
file and running the above file was done with:

\starttyping
mpost --mem=metafun.mem yourfile
\stoptyping

However, this is no longer the case and macros need to be loaded at startup as
follows:

\starttyping
mpost --ini metafun.mpii yourfile.mp
\stoptyping

Watch the suffix \type {mpii}: this refers to the stand alone, the one that
doesn't rely on \LUATEX.

After the run the results are available in \type {yourfile.1} and can be viewed
with \GHOSTSCRIPT. You don't need to close the file so reprocessing is very
convenient.

Because we will go beyond standard \METAPOST, we will use the \type {mpiv} files.
These work with the library which in turn means that we will run from within
\CONTEXT. This has the advantage that we also have advanced font support at our
hands. In that case, a simple file looks like:

\starttyping
\starttext
  \startMPpage
    fill fullcircle scaled 5cm withcolor red ;
  \stopMPpage
  \startMPpage
    fill unitsquare scaled 5cm withcolor red ;
  \stopMPpage
\stoptext
\stoptyping

If the file is saved as \type {yourfile.tex}, then you can produce a \PDF\ file
with: \footnote {In fact, you could also process the \METAPOST\ file directly
because the \type {context} script will recognize it as such and wrap it into
a page.}

\starttyping
context yourfile
\stoptyping

The previous call will use \LUATEX\ and \CONTEXT\ \MKIV\ to produce a file with
two pages using the built in \METAPOST\ library with \METAFUN. When you use this
route you will automatically get the integrated text support shown in this
manual, including \OPENTYPE\ support. If one page is enough, you can also say:

\starttyping
\startMPpage
fill fullcircle scaled 5cm withcolor red ;
\stopMPpage
\stoptyping

So when you have a running \CONTEXT\ on your system you don't need to bother
about installing \METAPOST\ and running \METAFUN.

We will use lots of color. Don't worry if your red is not our red, or your yellow
does not match ours. We've made color definitions to match the overall design of
this document, but you should feel free to use any color of choice in the
upcoming examples.

By default, \CONTEXT\ has turned its color mechanism on. If you don't want your
graphics to have color, you should say:

\starttyping
\setupcolors[state=stop]
\stoptyping

but in todays documents color is so normal that you will probably never do that.
Because \METAFUN\ hooks into the \CONTEXT\ color mechanism, you can also use its
color space and conversion related features.

You need to keep in mind that just like \CONTEXT\ \MKII\ is frozen, the \type
{mpii} macros are also not extended. From now on we assume that you use \CONTEXT\
\MKIV\ which exclusively uses \type {mpiv} macros.

Even if you want to use \METAFUN\ but not \CONTEXT, you can still best use the
mentioned page method as it will give you decent text processing. You need to
know (and use) only a few \CONTEXT\ commands then.

\stopchapter

\stopcomponent
