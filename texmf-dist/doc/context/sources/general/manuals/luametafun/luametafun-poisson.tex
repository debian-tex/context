% language=us runpath=texruns:manuals/luametafun

\environment luametafun-style

\startcomponent luametafun-poisson

\startchapter[title={Poisson}]

When, after a post on the \CONTEXT\ mailing list, Aditya pointed me to an article
on mazes I ended up at poisson distributions which to me looks nicer than what I
normally do, fill a grid and then randomize the resulting positions. With some
hooks this can be used for interesting patterns too. The algorithm is based on
the discussion at:

\starttyping
http://devmag.org.za/2009/05/03/poisson-disk-sampling
\stoptyping

Other websites mention some variants on that but I saw no reason to look into
those in detail. I can imagine more random related variants in this domain so
consider this an appetizer. The user is rather simple because some macro is
assumed to deal with the rendering of the distributed points. We just show some
examples (because the interface might evolve).

\startbuffer
\startMPcode
    draw lmt_poisson [
        width    = 40,
        height   = 40,
        distance =  1,
        count    = 20,
        macro    = "draw"
    ] xsized 4cm ;
\stopMPcode
\stopbuffer

\typebuffer[option=TEX]

\startlinecorrection
    \getbuffer
\stoplinecorrection

\startbuffer
\startMPcode
    vardef tst (expr x, y, i, n) =
        fill fullcircle scaled (10+10*(i/n)) shifted (10x,10y)
            withcolor "darkblue" withtransparency (1,.5) ;
    enddef ;

    draw lmt_poisson [
        width     = 50,
        height    = 50,
        distance  =  1,
        count     = 20,
        macro     = "tst",
        arguments =  4
    ] xsized 6cm ;
\stopMPcode
\stopbuffer

\typebuffer[option=TEX]

\startlinecorrection
    \getbuffer
\stoplinecorrection

\startbuffer
\startMPcode
    vardef tst (expr x, y, i, n) =
        fill fulldiamond scaled (5+5*(i/n)) randomized 2 shifted (10x,10y)
            withcolor "darkgreen" ;
    enddef ;

    draw lmt_poisson [
        width     = 50,
        height    = 50,
        distance  =  1,
        count     = 20,
        macro     = "tst",
        initialx  = 10,
        initialy  = 10,
        arguments =  4
    ] xsized 6cm ;
\stopMPcode
\stopbuffer

\typebuffer[option=TEX]

\startlinecorrection
    \getbuffer
\stoplinecorrection

\startbuffer
\startMPcode{doublefun}
    vardef tst (expr x, y, i, n) =
        fill fulldiamond randomized (.2*i/n) shifted (x,y);
    enddef ;

    draw lmt_poisson [
        width     = 150,
        height    = 150,
        distance  =   1,
        count     =  20,
        macro     = "tst",
        arguments =   4
    ] xsized 6cm withcolor "darkmagenta" ;
\stopMPcode
\stopbuffer

\typebuffer[option=TEX]

\startlinecorrection
    \getbuffer
\stoplinecorrection

\startbuffer
\startMPcode
    vardef tst (expr x, y, i, n) =
        draw externalfigure "cow.pdf" ysized (10+5*i/n) shifted (10x,10y);
    enddef ;
    draw lmt_poisson [
        width     = 20,
        height    = 20,
        distance  =  1,
        count     = 20,
        macro     = "tst"
        arguments = 4,
    ] xsized 6cm ;
\stopMPcode
\stopbuffer

\typebuffer[option=TEX]

\startlinecorrection
    \getbuffer
\stoplinecorrection

The supported parameters are:

\starttabulate[|T|T|T|p|]
\FL
\BC name      \BC type    \BC default \BC comment \NC \NR
\ML
\NC width     \NC numeric \NC 50      \NC \NC \NR
\NC height    \NC numeric \NC 50      \NC \NC \NR
\NC distance  \NC numeric \NC  1      \NC \NC \NR
\NC count     \NC numeric \NC 20      \NC \NC \NR
\NC macro     \NC string  \NC "draw"  \NC \NC \NR
\NC initialx  \NC numeric \NC 10      \NC \NC \NR
\NC initialy  \NC numeric \NC 10      \NC \NC \NR
\NC arguments \NC numeric \NC  4      \NC \NC \NR
\LL
\stoptabulate

\stopchapter

\stopcomponent

