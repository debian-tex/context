% language=us runpath=texruns:manuals/luametafun

\environment luametafun-style

\startcomponent luametafun-placeholder

\startchapter[title={Placeholder}]

Placeholders are an old \CONTEXT\ features and have been around since we started using
\METAPOST. They are used as dummy figure, just in case one is not (yet) present. They
are normally activated by loading a \METAFUN\ library:

\starttyping[option=TEX]
\useMPLibrary[dum]
\stoptyping

Just because it could be done conveniently, placeholders are now defined at the
\METAPOST\ end instead of as useable \METAPOST\ graphic at the \TEX\ end. The
variants and options are demonstrated using side floats.

\startbuffer[1]
\startMPcode
    lmt_placeholder [
        width       = 4cm,
        height      = 3cm,
        color       = "red",
        alternative = "circle".
    ] ;
\stopMPcode
\stopbuffer

\startplacefigure[location=left]
    \getbuffer[1]
\stopplacefigure

\typebuffer[1][option=TEX]

In addition to the traditional random circle we now also provide rectangles
and triangles. Maybe some day more variants will show up.

\startbuffer[2]
\startMPcode
    lmt_placeholder [
        width       = 4cm,
        height      = 3cm,
        color       = "green",
        alternative = "square".
    ] ;
\stopMPcode
\stopbuffer

\startplacefigure[location=left]
    \getbuffer[2]
\stopplacefigure

\typebuffer[2][option=TEX]

Here we set the colors but in the image placeholder mechanism we cycle through
colors automatically. Here we use primary, rather dark, colors.

\startbuffer[3]
\startMPcode
    lmt_placeholder [
        width       = 4cm,
        height      = 3cm,
        color       = "blue",
        alternative = "triangle".
    ] ;
\stopMPcode
\stopbuffer

\startplacefigure[location=left]
    \getbuffer[3]
\stopplacefigure

\typebuffer[3][option=TEX]

\startbuffer[4a]
\startMPcode
    lmt_placeholder [
        width       = 4cm,
        height      = 3cm,
        color       = "yellow",
        alternative = "circle".
        reduction   = 0,
    ] ;
\stopMPcode
\stopbuffer

\startbuffer[4b]
\startMPcode
    lmt_placeholder [
        width       = 4cm,
        height      = 3cm,
        color       = "yellow",
        alternative = "circle".
        reduction   = 0.25,
    ] ;
\stopMPcode
\stopbuffer

\startbuffer[4c]
\startMPcode
    lmt_placeholder [
        width       = 4cm,
        height      = 3cm,
        color       = "yellow",
        alternative = "circle".
        reduction   = 0.50,
    ] ;
\stopMPcode
\stopbuffer

\startbuffer[4d]
\startMPcode
    lmt_placeholder [
        width       = 4cm,
        height      = 3cm,
        color       = "yellow",
        alternative = "circle".
        reduction   = 0.75,
    ] ;
\stopMPcode
\stopbuffer

If you want less dark colors, the \type {reduction} parameter can be used to
interpolate between the given color and white; its value is therefore a value
between zero (default) and 1 (rather pointless as it produces white).

\startplacefigure[location=left]
    \startcombination[2*2]
        {\getbuffer[4a]} {0}
        {\getbuffer[4b]} {0.25}
        {\getbuffer[4c]} {0.50}
        {\getbuffer[4d]} {0.75}
    \stopcombination
\stopplacefigure

We demonstrate this with four variants, all circles. Of course you can also use
lighter colors, but this option was needed for the image placeholders anyway.

\typebuffer[4b][option=TEX]

\flushsidefloats

There are only a few possible parameters. As you can see, proper dimensions need
to be given because the defaults are pretty small.

\starttabulate[|T|T|T|p|]
\FL
\BC name        \BC type    \BC default    \BC comment \NC \NR
\ML
\NC color       \NC string  \NC red    \NC \NC \NR
\NC width       \NC numeric \NC 1      \NC \NC \NR
\NC height      \NC numeric \NC 1      \NC \NC \NR
\NC reduction   \NC numeric \NC 0      \NC \NC \NR
\NC alternative \NC string  \NC circle \NC \NC \NR
\LL
\stoptabulate

\stopchapter

\stopcomponent
