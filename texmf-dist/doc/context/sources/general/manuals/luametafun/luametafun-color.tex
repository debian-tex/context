% language=us

\environment luametafun-style

\startcomponent luametafun-color

\startchapter[title={Color}]

There are by now plenty of examples made by users that use color and \METAFUN\
provides all kind of helpers. So do we need more? When I play around with things
or when users come with questions that then result in a nice looking graphic, the
result might en dup as example of coding. The following is an example of showing
of colors. We have a helper that goes from a so called lab specification to rgb
and it does that via xyz transformations. It makes no real sense to interface
this beyond this converter. We use this opportunity to demonstrate how to make
an interface.

\startbuffer
\startMPdefinitions
  vardef cielabmatrix(expr l, mina, maxa, minb, maxb, stp) =
    image (
      for a = mina step stp until maxa :
        for b = minb step stp until maxb :
          draw (a,b) withcolor labtorgb(l,a,b) ;
        endfor ;
      endfor ;
    )
  enddef ;
\stopMPdefinitions
\stopbuffer

\typebuffer[option=TEX]

\getbuffer

Here we define a macro that makes a color matrix. It can be used as follows

\startbuffer
\startcombination[nx=4,ny=1]
  {\startMPcode draw cielabmatrix(20, -100, 100, -100, 100, 5) ysized 35mm withpen pencircle scaled 2.5 ; \stopMPcode} {\type {l = 20}}
  {\startMPcode draw cielabmatrix(40, -100, 100, -100, 100, 5) ysized 35mm withpen pencircle scaled 2.5 ; \stopMPcode} {\type {l = 40}}
  {\startMPcode draw cielabmatrix(60, -100, 100, -100, 100, 5) ysized 35mm withpen pencircle scaled 2.5 ; \stopMPcode} {\type {l = 60}}
  {\startMPcode draw cielabmatrix(80, -100, 100, -100, 100, 5) ysized 35mm withpen pencircle scaled 2.5 ; \stopMPcode} {\type {l = 80}}
\stopcombination
\stopbuffer

\typebuffer[option=TEX]

\startlinecorrection
\getbuffer
\stoplinecorrection

One can of course mess around a bit:

\startbuffer
\startcombination[nx=4,ny=1]
  {\startMPcode draw cielabmatrix(20, -100, 100, -100, 100, 10) ysized 35mm randomized 1 withpen pensquare scaled 4 ; \stopMPcode} {\type {l = 20}}
  {\startMPcode draw cielabmatrix(40, -100, 100, -100, 100, 10) ysized 35mm randomized 1 withpen pensquare scaled 4 ; \stopMPcode} {\type {l = 40}}
  {\startMPcode draw cielabmatrix(60, -100, 100, -100, 100, 10) ysized 35mm randomized 1 withpen pensquare scaled 4 ; \stopMPcode} {\type {l = 60}}
  {\startMPcode draw cielabmatrix(80, -100, 100, -100, 100, 10) ysized 35mm randomized 1 withpen pensquare scaled 4 ; \stopMPcode} {\type {l = 80}}
\stopcombination
\stopbuffer

\typebuffer[option=TEX]

\startlinecorrection
\getbuffer
\stoplinecorrection

Normally, when you don't go beyond this kind of usage, a simple macro like the
above will do. But when you want to make something that is upward compatible
(which is one of the principles behind the \CONTEXT\ user interface(s), you can
do this:

\startbuffer
\startcombination[nx=4,ny=1]
    {\startMPcode draw lmt_labtorgb [ l = 20, step = 20 ] ysized 35mm withpen pencircle scaled 8 ; \stopMPcode} {\type {l = 20}}
    {\startMPcode draw lmt_labtorgb [ l = 40, step = 20 ] ysized 35mm withpen pencircle scaled 8 ; \stopMPcode} {\type {l = 40}}
    {\startMPcode draw lmt_labtorgb [ l = 60, step = 20 ] ysized 35mm withpen pencircle scaled 8 ; \stopMPcode} {\type {l = 60}}
    {\startMPcode draw lmt_labtorgb [ l = 80, step = 20 ] ysized 35mm withpen pencircle scaled 8 ; \stopMPcode} {\type {l = 80}}
\stopcombination
\stopbuffer

\typebuffer[option=TEX]

\startlinecorrection
\getbuffer
\stoplinecorrection

This is a predefined macro in the reserved \type {lmt_} namespace (don't use that
one yourself, create your own). First we preset the possible parameters:

\starttyping[option=MP]
presetparameters "labtorgb" [
  mina = -100,
  maxa =  100,
  minb = -100,
  maxb =  100,
  step =    5,
  l    =   50,
] ;
\stoptyping

Next we define the main interface macro:

\starttyping[option=MP]
def lmt_labtorgb = applyparameters "labtorgb" "lmt_do_labtorgb" enddef ;
\stoptyping

Last we do the actual implementation, which looks a lot like the one we
started with:

\starttyping[option=MP]
vardef lmt_do_labtorgb =
  image (
    pushparameters "labtorgb" ;
      save l ; l := getparameter "l" ;
      for a = getparameter "mina" step getparameter "step"
            until getparameter "maxa" :
        for b = getparameter "minb" step getparameter "step"
            until getparameter "maxb" :
          draw (a,b) withcolor labtorgb(l,a,b) ;
        endfor ;
      endfor ;
    popparameters ;
  )
enddef ;
\stoptyping

Of course we can now add all kind of extra features but this is what we currently
have. Maybe this doesn't belong in the \METAFUN\ core but it's not that much code
and a nice demo. After all, there is much in there that is seldom used.

\stopchapter

\stopcomponent


