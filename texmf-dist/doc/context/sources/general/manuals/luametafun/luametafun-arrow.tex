% language=us

\environment luametafun-style

\startcomponent luametafun-arrow

\startchapter[title={Arrow}]

Arrows are somewhat complicated because they follow the path, are constructed
using a pen, have a fill and draw, and need to scale. One problem is that the
size depends on the pen but the pen normally is only known afterwards.

\startbuffer[1a]
\startMPcode
draw lmt_arrow [
    path = (fullcircle scaled 3cm),
]
    withpen pencircle scaled 2mm
    withcolor "darkred" ;
\stopMPcode
\stopbuffer

\startbuffer[1b]
\startMPcode
draw lmt_arrow [
    path   = (fullcircle scaled 3cm),
    length = 8,
]
    withpen pencircle scaled 2mm
    withcolor "darkgreen" ;
\stopMPcode
\stopbuffer

\startbuffer[1c]
\startMPcode
draw lmt_arrow [
    path = (fullcircle scaled 3cm rotated 145),
    pen  = (pencircle xscaled 4mm yscaled 2mm rotated 45),
]
    withpen pencircle xscaled 1mm yscaled .5mm rotated 45
    withcolor "darkblue" ;
\stopMPcode
\stopbuffer

\startbuffer[1d]
\startMPcode
pickup pencircle xscaled 2mm yscaled 1mm rotated 45 ;
draw lmt_arrow [
    path = (fullcircle scaled 3cm rotated 45),
    pen  = "auto",
]
    withcolor "darkyellow" ;
\stopMPcode
\stopbuffer

To some extent \METAFUN\ can help you with this issue. In \in {figure} [arrows:1]
we see some variants. The definitions are given below:

\typebuffer[1a][option=TEX]
\typebuffer[1b][option=TEX]
\typebuffer[1c][option=TEX]
\typebuffer[1d][option=TEX]

\startplacefigure[reference=arrows:1]
    \startcombination[4*1]
        {\getbuffer[1a]} {default}
        {\getbuffer[1b]} {length}
        {\getbuffer[1c]} {pen}
        {\getbuffer[1d]} {auto}
    \stopcombination
\stopplacefigure

There are some options that influence the shape of the arrowhead and its
location on the path. You can for instance ask for two arrowheads:

\startbuffer[3]
\startMPcode
    pickup pencircle scaled 1mm ;
    draw lmt_arrow [
        pen      = "auto",
        location = "both"
        path     = fullcircle scaled 3cm rotated 90,
    ] withcolor "darkgreen" ;
\stopMPcode
\stopbuffer

\typebuffer[3][option=TEX]

\startlinecorrection
\getbuffer[3]
\stoplinecorrection

The shape can also be influenced although often this is not that visible:

\startbuffer[4]
\startMPcode
    pickup pencircle scaled 1mm ;
    draw lmt_arrow [
        kind        = "draw",
        pen         = "auto",
        penscale    = 4,
        location    = "middle",
        alternative = "curved",
        path        = fullcircle scaled 3cm,
    ] withcolor "darkblue" ;
\stopMPcode
\stopbuffer

\typebuffer[4][option=TEX]

\startlinecorrection
\getbuffer[4]
\stoplinecorrection

The location can also be given as percentage, as this example demonstrates. Watch
how we draw only arrow heads:

\startbuffer[5]
\startMPcode
    pickup pencircle scaled 1mm ;
    for i = 0 step 5 until 100 :
        draw lmt_arrow [
            alternative = "dimpled",
            pen         = "auto",
            location    = "percentage",
            percentage  = i,
            dimple      = (1/5 + i/200),
            headonly    = (i = 0),
            path        = fullcircle scaled 3cm,
        ] withcolor "darkyellow" ;
    endfor ;
\stopMPcode
\stopbuffer

\typebuffer[5][option=TEX]

\startlinecorrection
\getbuffer[5]
\stoplinecorrection

The supported parameters are:

\starttabulate[|T|T|T|p|]
\FL
\BC name        \BC type    \BC default \BC comment \NC \NR
\ML
\NC path        \NC path    \NC        \NC \NC \NR
\NC pen         \NC path    \NC        \NC \NC \NR
\NC             \NC string  \NC auto   \NC \NC \NR
\NC kind        \NC string  \NC fill   \NC \type {fill} or \type {draw} \NC \NR
\NC dimple      \NC numeric \NC 1/5    \NC \NC \NR
\NC scale       \NC numeric \NC 3/4    \NC \NC \NR
\NC penscale    \NC numeric \NC 3      \NC \NC \NR
\NC length      \NC numeric \NC 4      \NC \NC \NR
\NC angle       \NC numeric \NC 45     \NC \NC \NR
\NC location    \NC string  \NC end    \NC \type {end}, \type {middle} or \type {both} \NC \NR  % middle both
\NC alternative \NC string  \NC normal \NC \type {normal}, \type {dimpled} or \type {curved} \NC \NR
\NC percentage  \NC numeric \NC 50     \NC \NC \NR
\NC headonly    \NC boolean \NC false  \NC \NC \NR
\LL
\stoptabulate

\stopchapter

\stopcomponent

