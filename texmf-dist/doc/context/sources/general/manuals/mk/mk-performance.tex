% language=us

\startcomponent mk-performance

\environment mk-environment

\chapter{How about performance}

\subject{remark}

The previous chapters already spent some words on performance
and memory usage. By the time that Taco and I were implementing,
discussing and testing the callbacks related to node lists, we were
already convinced that in all areas covered so far (file management,
handling input characters, dealing with fonts, conversion to tokens,
string and table manipulation, enz.) the \TEX||\LUA\ pair was up to
the task  And so we were quite confident that processing nodes was
not only an important aspect of \LUATEX\ but also quite feasable
in terms of performance (after all we needed it in order to deal
with advanced typesetting of Arab). When Taco was dealing with the
\TEX\ side of the story, I was experimenting with possible
mechanisms at the \LUA\ end.

At the same time I got the opportunity to speed up the \METAPOST\
to \PDF\ converter and both activities involved some timing. Here
I report some of the observations that we made in this process.

\subject{parsing}

Expressions in \LUA\ are powerful and definitely faster than regular
expressions found in other languages, but they have some limits. Most
noticeably is the lack of alternation. In \RUBY\ one can say:

\starttyping
str = "there is no gamma in here, just an beta"

if str =~ /(alph|bet|delt)a/ then
    print($1)
end
\stoptyping

but in \LUA\ you need a few more lines:

\starttyping
str = "there is no gamma in here, just an beta"

for _, v in pairs({'alpha','beta','delta'}) do
    local s = str:match(v)
    if s then
        print(s)
        break
    end
end
\stoptyping

Interesting is that upto now I didn't really miss alternation but
it may as well be that the lack of it drove me to come up with
different solutions. For \CONTEXT\ \MKIV\ the \METAPOST\ to \PDF\
converter has been rewritten in \LUA. This is a prelude to direct
\LUA\ output from \METAPOST\ but I needed the exercise. It was
among the first \LUA\ code in \MKIV.

Progressive (sequential) parsing of the data is an option, and is
done in \MKII\ using pure \TEX. We collect words and compare them
to \POSTSCRIPT\ directives and act accordingly. The messy parts
are scanning the preamble, which has specials to be dealt with as
well as lots of unpredictable code to skip, and the \type
{fshow} command which adds text to a graphic. But real dirty are
the code fragments that deal with setting the line width and
penshapes so the cleanup of this takes some time.

In \LUA\ a different approach is taken. There is an \type {mp} table
which collects a lot of functions that more or less reflect the
output of \METAPOST. The functions take care of generating the right
\PDF\ code and also handle the transformations needed because of
the differences between \POSTSCRIPT\ and \PDF.

The sequential \POSTSCRIPT\ that comes from \METAPOST\ is
collected in one string and converted using \type {gsub} into a
sequence of \LUA\ function calls. Before this can be done, some
cleanup takes place. The resulting string is then executed as
\LUA\ code.

As an example:

\starttyping
1 0 0 2 0 0 curveto
\stoptyping

becomes

\starttyping
mp.curveto(1,0,0,2,0,0)
\stoptyping

which results in:

\starttyping
\pdfliteral{1 0 0 2 0 0 c}
\stoptyping

In between, the path is stored and transformed which is needed in
the case of penshapes, where some \POSTSCRIPT\ feature is used that
is not available in \PDF.

During the development of \LUATEX\ a new feature was added to
\LUA: \type {lpeg}. With \type {lpeg} you can define text scanners. In
fact, you can build parsers for languages quite conveniently so
without doubt we will see it show up all over \MKIV.

Since I needed an exercise to get accustomed with \type {lpeg}, I
rewrote the mentioned converter. I'm sure that a better implementation
is possible than I did (after all, \POSTSCRIPT\ is a language) but
I went for a speedy solution. The following table shows some timings.

\starttabulate[|c|c|l|]
\HL
\NC \tt gsub \NC \tt lpeg \NC \NC \NR
\HL
\NC  2.5     \NC 0.5      \NC 100 times test graphic \NC \NR
\NC  9.2     \NC 1.9      \NC 100 times big graphic  \NC \NR
\HL
\stoptabulate

The test graphic has about everything that \METAPOST\ can output,
including special tricks that deal with transparency and shading. The
big one is just four copies of the test graphic.

So, the \type {lpeg} based variant is about 5~times faster than the
original variant. I'm not saying that the original implementation is
that brilliant, but a 5~time improvement is rather nice especially
when you consider that \type {lpeg} is still experimental and each
version performs better. The tests were done with \type {lpeg}
version 0.5 which performs slightly faster than its predecessor.

It's worth mentioning that the original \type {gsub} based variant
was already a bit improved compared to its first implementation.
There we collected the \TEX\ (\PDF) code in a table and passed it
in its concatenated form to \TEX. Because the \LUA\ to \TEX\
interface is by now quite efficient we can just pass the
intermediate results directly to \TEX.

\subject{file io}

The repertore of functions that deal with individual characters
in \LUA\ is small. This does not bother us too much because the
individual character is not what \TEX\ is mostly dealing with. A
character or sequence of characters becomes a token (internally
represented by a table) and tokens result in nodes (again
tables, but larger). There are many more tokens involved than
nodes: in \CONTEXT\ a ratio of 200~tokens on 1~node are not
uncommon. A letter like \type {x} become a token, but the control
sequence \type {\command} also ends up as one token. Later on,
this \type {x} may become a character node, possibly surrounded
by some kerning. The input characters \type {width} result in
5~tokens, but may not end up as nodes at all, for instance when
they are part of a key|/|value pair in the argument to a command.

Just as there is no guaranteed one||to||one relationship between
input characters and tokens, there is no straight relation between
tokens and nodes. When dealing with input it is good to keep in mind
that because of these interpretation stages one can never say that
1~megabyte of input characters ends up as 1~million something in
memory. Just think of how many megabytes of macros get stored in a
format file much smaller than the sum of bytes.

We only deal with characters or sequences of bytes when reading
from an input medium. There are many ways to deal with the input.
For instance one can process the input lines as \TEX\ sees them,
in which case \TEX\ takes care of the \UTF\ input. When we're
dealing with other input encodings we can hook code into the file
openers and readers and convert the raw data ourselves. We can for
instance read in a file as a whole, convert it using the normal
expression handlers or the byte(pair) iterators that \LUATEX\
provides, or we can go real low level using native \LUA\ code,
as in:

\starttyping
do
    local function nextbyte(f)
        return f:read(1)
    end

    function io.bytes(f)
        return nextbyte, f
    end
end

f = io.open("somefile.dat")
for b in io.bytes(f) do
    do_something(b)
end
f:close()
\stoptyping

Of course in practice one will need to integrate this into one of the
reader callback, but the principle stays the same. In case you wonder
if calling functions for each byte is fast enough \unknown\ it's more than
fast enough for normal purposes, especially if we keep in mind that other
tasks like reading of, preparing of and dealing with fonts of processing
token lists take way more time. You can be sore that when half a second
runtime is spent on reading a file, processing may take minutes. If one
wants to sqeeze more performance out of this part, it's always an option
to write special libraries for that, but this is beyond standard \LUATEX.
We found out that the speed of loading data from files in \LUA\ is
mostly related to the small size of \LUA's file buffer. Reading data stored
in tables is extremely fast, and even faster when precompiled into bytecode.

\subject{tables}

When Taco and I were experimenting with the callbacks that intercept
tokens and nodes, we wondered what the impact would be on performance.
Although in \MKIV\ we allocate quite some memory due to font handling,
we were pretty sure that handling \TEX's internal lists also could
have their impact. Data related to fonts is not always subjected to
garbage collection, simply because it's to be available permanently.
List processing on the other hand involves a lot of temporary allocated
tables. During a run a real huge amount of tokens passes the machinery.
When digested, they become nodes. For testing we normally use this
document (with the name \type {mk.tex}) and at the time of writing
this, it has some 48 pages.

This document is of moderately complexity, but not as complex as
the documents that I normally process; they have with lots of graphics,
layers, structural elements, maybe a bit of \XML\ parsing, etc.
Nevertheless, we're talking of some 24~million tokens entering the engine
for 50 pages of text. Contrary to this the number of nodes is small:
only 120~thousand but the tables making up the nodes are more complex than
token tables (with three numbers per token). When all tokens are intercepted
and returned unchanged, on my machine the run is progressively slow and
memory usage grows from 75M to 112M. There is room for improvement there,
especially in the garbage collector.

Side note: quite some of these tokens result from macro expansion. Also,
when in the input a \type {\command} is used, the callback passes it as one
token. A command stores in a format is already tokenized, but a command read
from the input is tokenized when read, so behind each token reported there
can be a few more input characters, but their number can be neglected compared
to tokens originating from the macro package.

The token callback is rather slow when used for a whole document. However,
this is typically a callback that will only be used in very special
situations and for a controlled number of tokens. The node callback on the
other hand can be set permanently. Fortunately the number of nodes is
relatively small. The overhead of a simple token handler that just counts
nodes is around 5\% but most common manipulations with token lists don't
take much more time. For instance, experiments with adding kerns around
punctuation (a French speciality) hardly takes time, resolving ligatures is
not really noticeable and applying inter||character spacing to a whole document
is not that slow either. Actually, the last example is kind of special
because it more than doubles the size of the node lists. Inserting or removing
table elements in relatively slow when tables are large but there are some
ways around this.

One of the reasons of whole||document token handling being slow is that
each token is a three||element table and so the garbage collector has to work
rather hard. The efficiency of this process is also platform dependent (or
maybe compiler specific). Manipulating the garbage collector parameters
does not improve performance, unless this forces the collector to be inefficient
at the cost of a lot of memory.

However, when we started dealing with nodes, I gave tuning the collector
another try and on the mentioned test document the following observations
were made when manipulating the step multiplier:

\starttabulate[|c|c|c|]
\HL
\NC \bf step \NC \bf runtime \NC \bf memory \NC \NR
\HL
\NC 200 \NC 24.0 \NC 80.5M \NC \NR
\NC 175 \NC 21.0 \NC 78.2M \NC \NR
\NC 150 \NC 22.0 \NC 74.6M \NC \NR
\NC 160 \NC 22.0 \NC 74.6M \NC \NR
\NC 165 \NC 21.0 \NC 77.6M \NC \NR
\NC 125 \NC 21.5 \NC 89.2M \NC \NR
\NC 100 \NC 21.5 \NC 88.4M \NC \NR
\HL
\stoptabulate

As a result, I decided to set the \type {stepmul} variable to~165.

\starttyping
\ctxlua{collectgarbage("setstepmul", 165)}
\stoptyping

However,  when we were testing thenew \type {lpeg} based \METAPOST\ converter, we ran
into problems. For table intensive operations, temporary disabling the
garbage collector gave a significant boost in speed. While testing
performance we used the following loop:

\starttyping
\dorecurse {2000} {
    \setbox \scratchbox \hbox \bgroup
        \convertMPtoPDF{test-mps-procset.mps}{1}{1}
    \egroup
}
\stoptyping

In such a loop, turning the garbage collector on and off is disasterous. Because
no other \LUA\ calls happen between these calls, the garbage collector is never
invoked at all. As a result, memory growed from the baseline of 45M to 120MB and
processing became incrementally slow. I found out that restarting the collector
before each conversion kept memory usage low and the speed also remained okay.

\starttyping
\ctxlua{collectgarbage("restart")}
\stoptyping

Further experiments learned that it makes sense to restart the collector at
each shipout and before table intense operations. On \type {mk.tex} this
results in a memory usage of 74M (at the end of the run) and a runtime of
21~seconds.

Concerning nodes and speed|/|allocation issues, we need to be aware of
the fact that this was still somewhat experimental and in the final version
of \LUATEX\ callbacks may occur at different places and lists may be subjected
to parsing multiple times at different moments and locations (for instance when
we start dealing with attributes, an upcoming new feature).

Back to tokens. The reason why applying the callback to every token takes a
while has to do with the fact that each token goes through the associated
function. If you want to have an idea of what this means for 24~million
tokens, just run the following \LUA\ code:

\starttyping
for i=1,24 do
    print(i)
    for j=1,1000*1000 do
        local t = { 1, 2, 3 }
    end
end
print(os.clock())
\stoptyping

This takes some 60 seconds on my machine. The following code
runs about three times faster because the table has not to
be allocated each time.

\starttyping
t = { 1, 2, 3 }
for i=1,24 do
    print(i)
    for j=1,1000*1000 do
        t[1]=4 t[2]=5 t[3]=6
    end
end
print(os.clock())
\stoptyping

Imagine this code to be interwoven with other code and \TEX\ doing
things with the tokens it gets back. The memory pool will be
scattered and garbage collecting will become more difficult.

However, in practice one will only apply token handling
to a marked piece of the input data. It is for this reason
that the callback is not:

\starttyping
callback.register('token_filter', function(t)
    return t
end )
\stoptyping

but instead

\starttyping
callback.register('token_filter', function()
    return token.get_next()
end )
\stoptyping

This gives the opportunity to fetch more than one token and
keep fetching till a criterium is met (for instance a sentinel).

Because \type {token.get_next} is not bound to the callback you
can fetch tokens anytime you want and only use the callback to
feed back tokens into \TEX. In \CONTEXT\ \MKIV\ there is some
collect and flush tokens present. Here is a trivial example:

\starttyping
\def\SwapChars{\directlua 0 {
    do
        local t = { token.get_next(), token.get_next() }
        callback.register('token_filter', function()
            callback.register('token_filter', nil)
            return { t[2], t[1] }
        end )
    end
}}

\SwapChars HH \SwapChars TH
\stoptyping

Collecting tokens can take place inside the callback but also
outside. This also gives you the opportunity to collect them
in efficient ways and keep an eye on the memory demands.

Of course using \TEX\ directly takes less code:

\starttyping
\def\SwapChars#1#2{#2#1}
\stoptyping

The example shown here involves so little tokens that running
it takes no noticeable time. Here we show this definition in
tokenized form:

\starttokens[demo]\def\SwapChars#1#2{#2#1}\stoptokens \setups{ShowCollect}

\stopcomponent
