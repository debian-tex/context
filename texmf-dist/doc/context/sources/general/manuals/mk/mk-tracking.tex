% language=uk

% \enabletrackers[otf.loading]
% \enabletrackers[otf.lookups]

\startcomponent mk-track

\environment mk-environment

\startbuffer[latin-default-features]
\definefontfeature
  [latin-default]
  [mode=node,language=dflt,script=latn,
   liga=yes,calt=yes,clig=yes,
   kern=yes]
\stopbuffer

\startbuffer[arabtype-default-features]
\definefontfeature
  [arabtype-default]
  [mode=node,language=dflt,script=arab,
   init=yes,medi=yes,fina=yes,isol=yes,
   ccmp=yes,locl=yes,calt=yes,
   liga=yes,clig=yes,dlig=yes,rlig=yes,
   mark=yes,mkmk=yes,kern=yes,curs=yes]
\stopbuffer

\startbuffer[zapfino-default-features]
\definefontfeature
   [zapfino-default]
   [mode=node,language=dflt,script=latn,
    calt=yes,clig=yes,rlig=yes,tlig=yes,
    kern=yes,curs=yes]
\stopbuffer

\getbuffer[latin-default-features]
\getbuffer[arabtype-default-features]
\getbuffer[zapfino-default-features]

\chapter{Tracking}

We entered 2009 with a partial reimplementation of the \OPENTYPE\
feature handler. One of the reasons was an upgrade of the
\FONTFORGE\ libraries that \LUATEX\ uses.

The specification of \OPENTYPE\ is kind of vague. Apart from a
lack of a proper free specifications there's also the problem that
Microsoft and Adobe may have their own interpretation of how and
in what order to apply features. In general the Microsoft website
has more detailed specifications and is a better reference. There
is also some information in the \FONTFORGE\ help files.

Because there is so much possible, fonts might contain bugs and/or
be made to work with certain renderers. These may evolve over time
which may have the side effect that suddenly fonts behave
differently.

After a lot of experiments (mostly by Taco, me and Idris) we're
now at yet another implementation. Of course all errors are mine
and of course the code can be improved. There are quite some
optimization going on here and processing speed is currently
acceptable. Not all functions are implemented yet, often because I
lack the fonts for testing. Many scripts are not yet supported
either, but I will look into them as soon as \CONTEXT\ users ask
for it.

The data provided by the \FONTFORGE\ library has organized lookups
(which relate to features) in a certain way. A first
implementation of this code was organized featurewise: information
related to features was collected and processing boiled down to a
run over the features. The current implementation honours the order
in the main feature table. Since we can reorder this table as we
want, we can eventually support several models of processing. We
kept the static as well as dynamic feature processing, because it
had proved to be rather useful. The formerly three loop variants
have been discarded but might reappear at some time.

One reason for this change is that the interactive version of
\FONTFORGE\ now provides a more detailed overview of the way
lookups are supposed to be handled. When you consult the
information of a font and in particular a glyph in a font, you now
get quite some information about what features can be applied and
in what order this takes place.

In \CONTEXT\ \MKIV\ we deal with this as follows. Keep in mind
that we start with characters but stepwise these can become more
abstract representation, named glyphs. For instance a letter~a can
be represented by a shape (glyph) that is similar to an uppercase~A.

\startitemize

\item We loop over all lookups. Normally there are only a few
lookups but fonts that deal with scripts that resemble
handwriting, like arabic of Zapfino, might have hundreds of them.
Each lookup has a detailed specification of what language and/or
scripts it applies to.

\item For each lookup we do a run over the list of glyphs. So, if
we have 50 lookups, and a paragraph has 500 glyphs, we do some
25000 loops. Keep in mind that for arab we start with a sequence
of characters and vowels, and during a run, these might be
replaced by for instance ligatures and combined vowels, so the 500
stepwise becomes less.

\item We only process the features that are enabled. Normally the
lookups are organized in such a way that features take place in a
similar way: (de)composition, replacement of initial, medial,
final and isolated forms, specific replacements by one or more
variant, composition of ligatures, mark positioning, cursive
corrections and kerning. The font itself does not contain
information about what features are to be enabled by default. Some
applications have built in presets, others might extend their
repertoire over time.

\item A lookup can be a contextual lookup, which means that
treatment takes place on a match of a sequence of characters
(glyphs), either of not preceded or followed by specific other
characters (glyphs). We we loop over all contexts till we have a
match. Some fonts have lots of contextual lookups, which in turn
might increase the number of loops over the list of characters
(glyphs). If we have a match, we process the associated list of
sublookups. Technically it is possible to replace (say) five
characters by first a ligature (that replaces the first two by
one), then a multiple substitution (resulting in an extra three
glyphs replacing one) and then discarding the other rest (being
two characters). Because by that time characters (say, unicode
points) might have been replaced by glyphs (an index in the font)
a contextual lookup can involve quite some match points.

\stopitemize

In \CONTEXT\ we do this for each font that is used in a list, so
in practice we have quite some nested loops. Each font can have
its own set of features enables of features might be applied
dynamically, independent of font related settings. So, around the
mentioned loops there is another one: a loop over the fonts used
in a list (paragraph).

We process the whole list and then consult the glyph nodes. An
alternative approach is to collect strings of characters using the
same font including spaces (because some lookups involve spaces).
However, we then need to reconstruct the list which is no fun.
Also, we need to carry quite some information, like attributes, so
eventually we don't gain much (if we gain something at all).

Another consideration has been to operate on sublists of font
usage (using a subhead and subtail) but again this would
complicate matters as we then neext to keep track of a changing
subhead and subtail. On the other hand, this might save some
runtime. The number of changes in the code needed to do this is
not that large but it only makes sense when we have many fonts
in a list and don't change fonts to frequently.

This whole treatment is rather extensively optimized and so the
process is reasonable fast (you really don't want to know how much
time was spent on figuring out fast methods, testing and
reimplementing this). While I was implementing the \LUA\ code,
Taco made sure that access to the information in nodes was as fast
as possible and in our usual chat sessions we compared the output
with the one produced by the \FONTFORGE\ preview.

It was for this reason that more and more debugging code was added
but even that made tracking of what really happened cumbersome.
Therefore a more visual method was written, which will be shown
laster on.

You can enable tracing using the designated commands:

\starttyping
\enabletracker[otf.ligatures,otf.singles]
\stoptyping

and disable them for instance with:

\starttyping
\disabletracker[otf.*]
\stoptyping

Or you can pass directives to the command line:

\starttyping
context --track=otf.ligatures myfile.tex
\stoptyping

With regards to \OPENTYPE\ handling we have the following tracker
keys available:

\starttabulate
\NC \type{otf.actions}       \NC show all replacements and positioning \NC \NR
\NC \type{otf.alternatives}  \NC show what glyph is replaced by what alternative \NC \NR
\NC \type{otf.analyzing}     \NC color glyphs according to script specific analysis \NC \NR
\NC \type{otf.applied}       \NC applied features per font instance \NC \NR
\NC \type{otf.bugs}          \NC show diagnostic information \NC \NR
\NC \type{otf.contexts}      \NC show what contextual lookups take place \NC \NR
\NC \type{otf.cursive}       \NC show cursive anchoring when applied \NC \NR
\NC \type{otf.details}       \NC show more details about lookup handling \NC \NR
\NC \type{otf.dynamics}      \NC show dynamic feature definitions \NC \NR
\NC \type{otf.features}      \NC show what features are a applied \NC \NR
\NC \type{otf.kerns}         \NC show kerning between glyphs when applied \NC \NR
\NC \type{otf.ligatures}     \NC show what glyphs are replaced by one other \NC \NR
\NC \type{otf.loading}       \NC show more information when loading (caching) a font \NC \NR
\NC \type{otf.lookups}       \NC keep track of what lookups are consulted \NC \NR
\NC \type{otf.marks}         \NC show mark anchoring when applied \NC \NR
\NC \type{otf.multiples}     \NC show what glyph is replaced by multiple others \NC \NR
%NC \type{otf.normal_chain}  \NC \NC \NR
\NC \type{otf.positions}     \NC show what glyphs are positioned (combines other trackers) \NC \NR
\NC \type{otf.preparing}     \NC show what information is collected for later usage in lookups \NC \NR
\NC \type{otf.replacements}  \NC show what glyphs are replaced (combines other trackers) \NC \NR
\NC \type{otf.sequences}     \NC \NC \NR
\NC \type{otf.singles}       \NC show what glyph is replaced by one other \NC \NR
%NC \type{otf.steps}         \NC \NC \NR
%NC \type{otf.verbose_chain} \NC \NC \NR
\stoptabulate

Some other trackers might also come in handy:

\starttabulate
%NC \type{fonts.collecting} \NC  \NC \NR
\NC \type{fonts.combining}  \NC show what extra characters are added when forcing combined shapes \NC \NR
\NC \type{fonts.defining}   \NC show what fonts are defined \NC \NR
\NC \type{fonts.loading}    \NC show more details when a font is loaded (and cached) for the first time \NC \NR
%NC \type{fonts.names}      \NC \NC \NR
%NC \type{fonts.scaling}    \NC \NC \NR
\stoptabulate

We now show another way to track what happens with your text.
Because this is rather verbose, you should only apply it to words.
The second argument can be \type {-1} (right to left), \type {0}
(default) or \type {1} (left to right). The third argument can
be invisible in the code because the font used for verbatim might
lack the shapes. A font has a different ordering than \UNICODE\
because after all one character can have multiple
representations, one shape can be used for multiple characters,
or shapes might not have a \UNICODE\ point at all. In \MKIV\ we
push all shapes that have no direct relationship with \UNICODE\ to
the private area so that \TEX\ still sees them (hence the large
numbers in the following examples).

The next example uses Latin Modern. Here we apply the following
features:

\typebuffer[latin-default-features]

\startbuffer
\showotfcomposition
  {name:lmroman12regular*latin-default at 24pt}
  {0}
  {flinke fietser}
\stopbuffer

\typebuffer \start \veryraggedright \getbuffer \stop

The next example uses Arabtype. Here we apply the following features:

\typebuffer[arabtype-default-features]

\startbuffer
\showotfcomposition
  {arabtype*arabtype-default at 48pt}
  {-1}
  {الضَّرَّ}
\stopbuffer

\typebuffer \start \veryraggedright \getbuffer \stop

\startbuffer
\showotfcomposition
  {arabtype*arabtype-default at 48pt}
  {-1}
  {لِلّٰهِ}
\stopbuffer

\typebuffer \start \veryraggedright \getbuffer \stop

Another arabic example (after all, fonts that support arabic have
lots of nice features) is the following. First we define a bunch
of feature collections

\startbuffer
\definefontfeature
  [salt-n]
  [analyze=yes,mode=node,
   language=dflt,script=arab,
   init=yes,medi=yes,fina=yes,isol=yes,
   liga=yes,calt=yes,ccmp=yes,
   kern=yes,curs=yes,mark=yes,mkmk=yes]

\definefontfeature[salt-y][salt-n][salt=yes]
\definefontfeature[salt-1][salt-n][salt=1]
\definefontfeature[salt-2][salt-n][salt=2]
\definefontfeature[salt-3][salt-n][salt=3]
\definefontfeature[salt-r][salt-n][salt=random]
\stopbuffer

\typebuffer \getbuffer

Next we show a few traced examples. Watch the reported alternatives.

\startbuffer
\showotfcomposition{scheherazaderegot*salt-n at 36pt}{-1}{\char"6DD}
\showotfcomposition{scheherazaderegot*salt-y at 36pt}{-1}{\char"6DD}
\showotfcomposition{scheherazaderegot*salt-1 at 36pt}{-1}{\char"6DD}
\showotfcomposition{scheherazaderegot*salt-2 at 36pt}{-1}{\char"6DD}
\showotfcomposition{scheherazaderegot*salt-3 at 36pt}{-1}{\char"6DD}
\showotfcomposition{scheherazaderegot*salt-r at 36pt}{-1}{\char"6DD}
\showotfcomposition{scheherazaderegot*salt-r at 36pt}{-1}{\char"6DD}
\showotfcomposition{scheherazaderegot*salt-r at 36pt}{-1}{\char"6DD}
\stopbuffer

\typebuffer \start \veryraggedright \getbuffer \stop

The font that we use here can be downloaded from the website of
Sil International.

For a Zapfino example we use the following feature set:

\typebuffer[zapfino-default-features]

\startbuffer
\showotfcomposition
  {zapfinoextraltpro*zapfino-default at 48pt}
  {0}
  {Prof. Dr. Donald E. Knuth}
\stopbuffer

\typebuffer \start \veryraggedright \getbuffer \stop

When dealing with features, we may run into problems due to
characters that are in the input stream but have no associated
glyph in the font. Although we test for this a user might want to
intercept side effect.

\starttyping
\checkcharactersinfont
\removemissingcharacters
\stoptyping

The first command only checks and reports missing characters,
while the second one also removes them.

\stopcomponent
