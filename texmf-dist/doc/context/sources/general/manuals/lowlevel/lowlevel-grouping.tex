% language=us runpath=texruns:manuals/lowlevel

\environment lowlevel-style

\startdocument
  [title=grouping,
   color=middlecyan]

\startsectionlevel[title=Introduction]

This is a rather short explanation. I decided to write it after presenting the
other topics at the 2019 \CONTEXT\ meeting where there was a question about
grouping.

% \stopsectionlevel

\startsectionlevel[title=\PASCAL]

In a language like \PASCAL, the language that \TEX\ has been written in, or
\MODULA, its successor, there is no concept of grouping like in \TEX. But we can
find keywords that suggests this:

\starttyping
for i := 1 to 10 do begin ... end
\stoptyping

This language probably inspired some of the syntax of \TEX\ and \METAPOST. For
instance an assignment in \METAPOST\ uses \type {:=} too. However, the \type
{begin} and \type {end} don't really group but define a block of statements. You
can have local variables in a procedure or function but the block is just a way
to pack a sequence of statements.

\stopsectionlevel

\startsectionlevel[title=\TEX]

In \TEX\ macros (or source code) the following can occur:

\starttyping[option=TEX]
\begingroup
    ...
\endgroup
\stoptyping

as well as:

\starttyping[option=TEX]
\bgroup
    ...
\egroup
\stoptyping

Here we really group in the sense that assignments to variables inside a group
are forgotten afterwards. All assignments are local to the group unless they are
explicitly done global:

\starttyping[option=TEX]
\scratchcounter=1
\def\foo{foo}
\begingroup
    \scratchcounter=2
    \global\globalscratchcounter=2
    \gdef\foo{FOO}
\endgroup
\stoptyping

Here \type {\scratchcounter} is still one after the group is left but its global
counterpart is now two. The \type {\foo} macro is also changed globally.

Although you can use both sets of commands to group, you cannot mix them, so this
will trigger an error:

\starttyping[option=TEX]
\bgroup
\endgroup
\stoptyping

The bottomline is: if you want a value to persist after the group, you need to
explicitly change its value globally. This makes a lot of sense in the perspective
of \TEX.

\stopsectionlevel

\startsectionlevel[title=\METAPOST]

The \METAPOST\ language also has a concept of grouping but in this case it's more like a
programming language.

\starttyping[option=MP]
begingroup ;
    n := 123 ;
endgroup ;
\stoptyping

In this case the value of \type {n} is 123 after the group is left, unless you do
this (for numerics there is actually no need to declare them):

\starttyping[option=MP]
begingroup ;
    save n ; numeric n ; n := 123 ;
endgroup ;
\stoptyping

Given the use of \METAPOST\ (read: \METAFONT) this makes a lot of sense: often
you use macros to simplify code and you do want variables to change. Grouping in
this language serves other purposes, like hiding what is between these commands
and let the last expression become the result. In a \type {vardef} grouping is
implicit.

So, in \METAPOST\ all assignments are global, unless a variable is explicitly
saved inside a group.

\stopsectionlevel

\startsectionlevel[title=\LUA]

In \LUA\ all assignments are global unless a variable is defines local:

\starttyping[option=LUA]
local x = 1
local y = 1
for i = 1, 10 do
    local x = 2
    y = 2
end
\stoptyping

Here the value of \type {x} after the loop is still one but \type {y} is now two.
As in \LUATEX\ we mix \TEX, \METAPOST\ and \LUA\ you can mix up these concepts.
Another mixup is using \type {:=}, \type {endfor}, \type {fi} in \LUA\ after done
some \METAPOST\ coding or using \type {end} instead of \type {endfor} in
\METAPOST\ which can make the library wait for more without triggering an error.
Proper syntax highlighting in an editor clearly helps.

\stopsectionlevel

\startsectionlevel[title=\CCODE]

The \LUA\ language is a mix between \PASCAL\ (which is one reason why I like it)
and \CCODE.

\starttyping
int x = 1 ;
int y = 1 ;
for (i=1; i<=10;i++) {
    int x = 2 ;
    y = 2 ;
}
\stoptyping

The semicolon is also used in \PASCAL\ but there it is a separator and not a
statement end, while in \METAPOST\ it does end a statement (expression).

\stopsectionlevel

\stopsectionlevel

\startsectionlevel[title=Kinds of grouping]

Explicit grouping is accomplished by the two grouping primitives:

\starttyping[option=TEX]
\begingroup
    \sl render slanted here
\endgroup
\stoptyping

However, often you will find this being used:

\starttyping[option=TEX]
{\sl render slanted here}
\stoptyping

This is not only more compact but also avoids the \type {\endgroup} gobbling
following spaces when used inline. The next code is equivalent but also suffers
from the gobbling:

\starttyping[option=TEX]
\bgroup
    \sl render slanted here
\egroup
\stoptyping

The \type {\bgroup} and \type {\egroup} commands are not primitives but aliases
(made by \type {\let}) to the left and right curly brace. These two characters
have so called category codes that signal that they can be used for grouping. The
{\em can be} here suggest that there are other purposes and indeed there are, for
instance in:

\starttyping[option=TEX]
\toks 0 = {abs}
\hbox {def}
\stoptyping

In the case of a token list assignment the curly braces fence the assignment, so scanning
stops when a matching right brace is found. The following are all valid:

\starttyping[option=TEX]
\toks 0 = {a{b}s}
\toks 0 = \bgroup a{b}s}
\toks 0 = {a{\bgroup b}s}
\toks 0 = {a{\egroup b}s}
\toks 0 = \bgroup a{\bgroup b}s}
\toks 0 = \bgroup a{\egroup b}s}
\stoptyping

They have in common that the final fence has to be a right brace. That the first
one can be a an alias is due to the fact that the scanner searches for a brace
equivalent when it looks for the value. Because the equal is optional, there is
some lookahead involved which involves expansion and possibly push back while
once scanning for the content starts just tokens are collected, with a fast
check for nested and final braces.

In the case of the box, all these specifications are valid:

\starttyping[option=TEX]
\hbox {def}
\hbox \bgroup def\egroup
\hbox \bgroup def}
\hbox \bgroup d{e\egroup f}
\hbox {def\egroup
\stoptyping

This is because now the braces and equivalent act as grouping symbols so as long
as they match we're fine. There is a pitfall here: you cannot mix and match
different grouping, so the next issues an error:

\starttyping[option=TEX]
\bgroup xxx\endgroup   % error
\begingroup xxx\egroup % error
\stoptyping

This can make it somewhat hard to write generic grouping macros without trickery
that is not always obvious to the user. Fortunately it can be hidden in macros
like the helper \typ {\groupedcommand}. In \LUAMETATEX\ we have a clean way out
of this dilemma:

\starttyping[option=TEX]
\beginsimplegroup xxx\endsimplegroup
\beginsimplegroup xxx\endgroup
\beginsimplegroup xxx\egroup
\stoptyping

When you start a group with \typ {\beginsimplegroup} you can end it in the three
ways shows above. This means that the user (or calling macro) doesn't take into
account what kind of grouping was used to start with.

When we are in math mode things are different. First of all, grouping with \typ
{\begingroup} and \typ {\endgroup} in some cases works as expected, but because
the math input is converted in a list that gets processed later some settings can
become persistent, like changes in style or family. You can bet better use \typ
{\beginmathgroup} and \typ {\endmathgroup} as they restore some properties. We
also just mention the \type {\frozen} prefix that can be used to freeze
assignments to some math specific parameters inside a group.

\stopsectionlevel

\startsectionlevel[title=Hooks]

In addition to the original \type {\aftergroup} primitive we have some more
hooks. They can best be demonstrated with an example:

\startbuffer
\begingroup \bf
    %
    \aftergroup   A \aftergroup   1
    \atendofgroup B \atendofgroup 1
    %
    \aftergrouped   {A2}
    \atendofgrouped {B2}
    %
    test
\endgroup
\stopbuffer

\typebuffer[option=TEX]

These collectors are accumulative. Watch how the bold is applied to what we
inject before the group ends.

\getbuffer

\stopsectionlevel

\startsectionlevel[title=Local versus global]

When \TEX\ enters a group and an assignment is made the current value is stored
on the save stack, and at the end of the group the original value is restored. In
\LUAMETATEX\ this mechanism is made a bit more efficient by avoiding redundant
stack entries. This is also why the next feature can give unexpected results when
not used wisely.

Now consider the following example:

\startbuffer
\newdimension\MyDimension

\starttabulate[||||]
    \NC         \MyDimension10pt \the\MyDimension
    \NC \advance\MyDimension10pt \the\MyDimension
    \NC \advance\MyDimension10pt \the\MyDimension \NC \NR
    \NC         \MyDimension10pt \the\MyDimension
    \NC \advance\MyDimension10pt \the\MyDimension
    \NC \advance\MyDimension10pt \the\MyDimension \NC \NR
\stoptabulate
\stopbuffer

\typebuffer[option=TEX]  \getbuffer

The reason why we get the same values is that cells are a group and therefore the
value gets restored as we move on. We can use the \type {\global} prefix to get
around this

\startbuffer
\starttabulate[||||]
    \NC \global        \MyDimension10pt \the\MyDimension
    \NC \global\advance\MyDimension10pt \the\MyDimension
    \NC \global\advance\MyDimension10pt \the\MyDimension \NC \NR
    \NC \global        \MyDimension10pt \the\MyDimension
    \NC \global\advance\MyDimension10pt \the\MyDimension
    \NC \global\advance\MyDimension10pt \the\MyDimension \NC \NR
\stoptabulate
\stopbuffer

\typebuffer[option=TEX]  \getbuffer

Instead of using a global assignment and increment we can also use the following

\startbuffer
\constrained\MyDimension\zeropoint
\starttabulate[||||]
    \NC \retained        \MyDimension10pt \the\MyDimension
    \NC \retained\advance\MyDimension10pt \the\MyDimension
    \NC \retained\advance\MyDimension10pt \the\MyDimension \NC \NR
    \NC \retained        \MyDimension10pt \the\MyDimension
    \NC \retained\advance\MyDimension10pt \the\MyDimension
    \NC \retained\advance\MyDimension10pt \the\MyDimension \NC \NR
\stoptabulate
\stopbuffer

\typebuffer[option=TEX]  \getbuffer

So what is the difference with the global approach? Say we have these two buffers:

\startbuffer
\startbuffer[one]
    \global\MyDimension\zeropoint
    \framed {
        \framed {\global\advance\MyDimension10pt \the\MyDimension}
        \framed {\global\advance\MyDimension10pt \the\MyDimension}
    }
    \framed {
        \framed {\global\advance\MyDimension10pt \the\MyDimension}
        \framed {\global\advance\MyDimension10pt \the\MyDimension}
    }
\stopbuffer

\startbuffer[two]
    \global\MyDimension\zeropoint
    \framed {
        \framed {\global\advance\MyDimension10pt \the\MyDimension}
        \framed {\global\advance\MyDimension10pt \the\MyDimension}
        \getbuffer[one]
    }
    \framed {
        \framed {\global\advance\MyDimension10pt \the\MyDimension}
        \framed {\global\advance\MyDimension10pt \the\MyDimension}
        \getbuffer[one]
    }
\stopbuffer
\stopbuffer

\typebuffer[option=TEX] \getbuffer

Typesetting the second buffer gives us:

\startlinecorrection
\getbuffer[two]
\stoplinecorrection

When we want to have these entities independent and not use different variables,
the global settings bleeding from one into the other entity is messy. Therefore
we can use this:

\startbuffer
\startbuffer[one]
    \constrained\MyDimension\zeropoint
    \framed {
        \framed {\retained        \MyDimension10pt \the\MyDimension}
        \framed {\retained\advance\MyDimension10pt \the\MyDimension}
    }
    \framed {
        \framed {\retained        \MyDimension10pt \the\MyDimension}
        \framed {\retained\advance\MyDimension10pt \the\MyDimension}
    }
\stopbuffer

\startbuffer[two]
    \constrained\MyDimension\zeropoint
    \framed {
        \framed {\retained\advance\MyDimension10pt \the\MyDimension}
        \framed {\retained\advance\MyDimension10pt \the\MyDimension}
        \getbuffer[one]
    }
    \framed {
        \framed {\retained\advance\MyDimension10pt \the\MyDimension}
        \framed {\retained\advance\MyDimension10pt \the\MyDimension}
        \getbuffer[one]
    }
\stopbuffer
\stopbuffer

\typebuffer[option=TEX] \getbuffer

Now we get this:

\startlinecorrection
\getbuffer[two]
\stoplinecorrection

The \type {\constrained} prefix makes sure that we have a stack entry, without
being clever with respect to the current value. Then the \type {\retained} prefix
can do its work reliably and avoid pushing the old value on the stack. Without
the constrain it gets a bit unpredictable because then it all depends on where
further up the chain the value was put on the stack. Of course one can argue that
we should not have the \quotation {save stack redundant entries optimization} but
that's not going to be removed.

\stopsectionlevel

\startsectionlevel[title=Files]

Although it doesn't really fit in this chapter, here are some hooks into processing
files:

\starttyping[option=TEX]
Hello World!\atendoffiled         {\writestatus{FILE}{ATEOF B #1}}\par
Hello World!\atendoffiled         {\writestatus{FILE}{ATEOF A #1}}\par
Hello World!\atendoffiled reverse {\writestatus{FILE}{ATEOF C #1}}\par
Hello World!\begingroup\sl \atendoffiled {\endgroup}\par
\stoptyping

Inside a file you can register tokens that will be expanded when the file ends.
You can also do that beforehand using a variant of the \type {\input} primitive:

\starttyping[option=TEX]
\eofinput {\writestatus{FILE}{DONE}} {thatfile.tex}
\stoptyping

This feature is mostly there for consistency with the hooks into groups and
paragraphs but also because \type {\everyeof} is kind of useless given that one
never knows beforehand if a file loads another file. The hooks mentioned above
are bound to the current file.

\stopsectionlevel

\stopdocument
