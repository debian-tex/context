% language=us runpath=texruns:manuals/pdfmerge

% The font merge feature is also availble in \MKIV\ where it is an experiment and
% some cleanup is also there. In \LMTX\ it's more mature. Thanks to the competent
% Team Ramkumar (including Syabil and Fish) for providing input, testing this
% extensively, and feed back. Massimiliano Farinella did extensive tests with
% complex files originating in InDesign and InkScape.

% \nopdfcompression

\usemodule[abbreviations-logos]

\setupbodyfont
  [bonum,10pt]

\setuplayout
  [topspace=.05ph,
   bottomspace=.05ph,
   backspace=.05ph,
   header=.05 ph,
   footer=0pt,
   width=middle,
   height=middle]

\setupwhitespace
  [big]

\setuphead
  [chapter]
  [style=\bfc,
   headerstate=high,
   interaction=all]

\setuphead
  [section]
  [style=\bfb]

\setuphead
  [subsection]
  [style=\bfa]

\setuphead
  [subsubsection]
  [style=\bf,
   after=]

\setuplist
  [interaction=all]

\setuptyping
  [color=darkyellow]

\enabletrackers[graphics.fonts]
\enabletrackers[graphics.fixes]

\startdocument
  [title=PDFmerge,
   author=Hans Hagen]

\startMPpage
    fill Page withcolor "darkyellow" ;

    picture p[] ;

    p1 := image ( draw textext.ulft("PDF")
        ysized 4cm
        shifted lrcorner Page
        withcolor "white"
    ; );
    p2 := image ( draw textext.ulft("\strut merging, embedding, fixing")
        xsized bbwidth p[1]
        shifted lrcorner Page
        withcolor "lightgray"
    ; );
    p3 := image ( draw textext.ulft("\strut and messing a bit around")
        xsized bbwidth p[1]
        shifted lrcorner Page
        withcolor "lightgray"
    ; );
    p4 := image ( draw textext.ulft("\strut in context lmtx")
        xsized bbwidth p[1]
        shifted lrcorner Page
        withcolor "lightgray"
    ; );
    draw p[4] shifted (-1cm,1cm) ;
    draw p[3] shifted (-1cm,1cm+bbheight(p4)+0cm) ;
    draw p[2] shifted (-1cm,1cm+bbheight(p3)+bbheight(p4)+0cm) ;
    p[1] := p[1] shifted (-1cm,1cm+bbheight(p2)+bbheight(p3)+bbheight(p4)+15mm);
    save dy ; dy := 0 ;
    for i=0 upto 7 :
        p[1] := p[1] yscaled (1-i/80) ;
        draw p[1]
            shifted (0,dy)
            withtransparency (1,.8-i/10) ;
            dy := dy + .6*bbheight(p[1]) ;
    endfor ;
\stopMPpage

\startluacode
 -- backends.registered.pdf.codeinjections.registercompactor {
table.save(
    "compactors-pdfmerge.lua",
    {
        compactors = {
            ["mine"] = {
                identify = {
                    content   = true,
                    resources = true,
                    page      = true,
                },
             -- embed = {
             --     type0    = true,
             --     truetype = true,
             --     type1    = true,
             -- },
                merge = {
                    type0    = true, -- check if a..z A..Z 0..9
                    truetype = true,
                    type1    = true,
                    LMTX     = true,
                },
                strip = {
                 -- group     = true,
                 -- extgstate = true,
                    marked    = true,
                },
             -- reduce = {
             --     color = true,
             --     rgb   = true,
             --     cmyk  = true,
             -- },
             -- convert = {
             --     rgb  = true,
             --     cmyk = true,
             -- },
             -- add = {
             --     cidset, -- when missing or even fix
             -- },
                recolor = {
                    viagray = { 1, 0, 0 },
                 -- viagray = { 0, 1, 0 },
                 -- viagray = { 0, 1, 0, .5 },
                 -- viagray = { .75 },
                },
            }
        }
    }
)
\stopluacode

\starttext

\startsubject[title={Introduction}]

The three graphic formats that make most sense for inclusion in \PDF\ are \PNG,
\JPG, and \PDF. The easiest if these is \JPG\ because basically the binary blob
get transferred to the result file. A \PNG\ graphic might need more work because
what actually is supported is basic \PNG\ inclusion. It means that often the
image data has to be unpacked and split into \PDF\ counterparts that get
embedded. The \PDF\ format is quite convenient because basically we only need to
copy the used objects to the result, so when those object are for instance \PNG\
encoded images, we gain runtime, but when we're talking pages of documents it
might take some more. Nevertheless, in practice it is still quite efficient.

This manual describes how to manipulate \PDF\ files that are not behaving well
or from which pages are to be embedded another \PDF\ file within constraints. We
discuss how to cleanup and|/|or embed fonts, fix colors, get rid of interfering
resources and fix the page stream.

Many thanks to Massimiliano Farinella for teaming up to make this all work better
and conducting extensive test on complex documents.

\startlines
Hans Hagen
Hasselt NL
January 2024\high{++}
\stoplines

\stopsubject

\startsubject[title={Embedding \PDF\ files}]

Here we focus on \PDF\ inclusion where we have several scenarios to deal with:

\startitemize[packed]
    \startitem
        A straightforward inclusion of a single page \PDF\ file.
    \stopitem
    \startitem
        Inclusion of a specific page from a \PDF\ file.
    \stopitem
    \startitem
        Inclusion of several pages from a \PDF\ file.
    \stopitem
    \startitem
        Inclusion of one or more pages from several \PDF\ files.
    \stopitem
\stopitemize

To this we can add:

\startitemize
    \startitem
        Inclusion of one or more pages from \PDF\ files that are generated
        independently (subruns) for instance in the process of writing a manual
        about something \CONTEXT. Think of externally processed buffers.
    \stopitem
\stopitemize

The most natural way to include pages is to use the \typ {\externalfigure}
command but later we will see that there are more ways to manipulate \PDF\ files.

\starttyping
\externalfigure[myfile.pdf][page=4]
\stoptyping

If you have problems with the inclusion that originate in the compact features
discussed here you can say:

\starttyping
\externalfigure[myfile.pdf][page=4,compact=]
\stoptyping

but also make sure to tell us what goes wrong so that we can fix it. We can't
predict what \PDF\ is fed into the machinery.

\stopsubject

\startsubject[title={Embedding multiple \PDF\ files and sharing common content}]

When we include more than one page from a file, we only need to embed shared
objects once. Of course it demands some object management but that has to be done
anyway. We could share objects across files but that demands more memory and
runtime and the saving are likely to be small, with one exception: fonts. It
would be nice if we can embed missing fonts and also merge fonts that are the
same. This can make the result much smaller, especially when we're talking of
including examples of typesetting in a manual that uses the same fonts.

Another aspect of inclusion is the quality of the to be embedded page. Here you
can think of errors in the page stream, color spaces that don't match, missing
properties, invalid metadata, etc. Often there's not much we can do about it, but
sometimes we can. However, it has to happen under user control and the outcome
has to be checked, although often a visual check is good enough.

\stopsubject

\startsubject[title={The \type {compact} parameter and fonts merging}]

The \type {compact} parameter of \type {\externalfigure} controls the embedding
of \PDF\ content. When set to \quote {yes} it will merge fonts but only when the
file is produced by \CONTEXT\ \LMTX. The reason for not checking all fonts by
default comes from the fact that references from the page stream to glyphs in the
font depend on the application that made the \PDF. In some cases the mapping is
using the original glyph index, but one can never be sure. Using the \type
{tounicode} map to go from page stream index to glyph index is also not reliable
because multiple glyphs can have the same \UNICODE\ slot and when font features
are applied (say small caps) you actually don't know that.

The mentioned \type {yes} option is a preset that has been defined like:

\starttyping
\startluacode
graphics.registerpdfcompactor ( "yes", {
    merge = {
        lmtx = true,
    },
} )
\stopluacode
\stoptyping

Another preset is \type {merge}:

\starttyping
\startluacode
graphics.registerpdfcompactor ( "merge", {
    merge = {
        type0    = true,
        truetype = true,
        type1    = true,
        lmtx     = true,
    },
} )
\stopluacode
\stoptyping

Currently we don't support \TYPETHREE\ optimization. It is doable but probably
not worth the effort.

We can also force embedding of fonts that are not included in the document that
we get the page from. This is unlikely unless you have old documents.

\starttyping
embed = {
    type0    = true,
    truetype = true,
    type1    = true,
}
\stoptyping

References to glyphs in the page stream use an eight bit string encoding or an
hexadecimal byte pairs. Depending on the font type we have up to 256 references
(using one character or two hex bytes) or at most 65536 references (using two
characters or 4 hex bytes). We normalize everything to hex encoding. That way we
get rid of the ugly escapes and exceptions in page stream glyph string.

There are two trackers:

\starttyping
\enabletrackers[graphics.fonts]
\enabletrackers[graphics.fixes]
\stoptyping

The first one reports what is done with fonts. When embedding of merging is not
possible you can try to remap the found font onto one on your system. Here are
some examples:

\starttyping
graphics.registerpdffont {
    source = "arial",
    target = "file:arial.ttf",
}
graphics.registerpdffont {
    source = "arial-bold",
    target = "file:arialbd.ttf",
}
graphics.registerpdffont {
    source = "arial,bold",
    target = "file:arialbd.ttf",
}
graphics.registerpdffont {
    source = "helvetica",
    target = "file:arial.ttf",
 -- unicode = true,
}
graphics.registerpdffont {
    source = "helvetica-bold",
    target = "file:arialbd.ttf",
 -- unicode = true,
}
graphics.registerpdffont {
    source = "courier",
    target = "file:cour.ttf",
}
graphics.registerpdffont {
    source  = "ms-pgothic",
    unicode = true, -- via unicode (false for composite)
}
\stoptyping

The \type {unicode} key needed when you get rubbish due to the indices in the
page stream being different from glyph indices in the used font. In that case we
go via the \type {tounicode} vector which works ok for the average simple
document not using special font features. There is some trial and error involved
but that is probably worth the effort when you have to manipulate many documents.

\stopsubject

\startsubject[title={Manipulating properties other than fonts}]

There are two activities when we compact: fonts and content. When content is
handled additional parsing of the page stream has to happen. What gets processed
it determined by the \type {identify} table:

\starttyping
identify = {
    content   = true,
    resources = true, -- needs checking
    page      = true, -- needs checking
}
\stoptyping

although this is equivalent:

\starttyping
identify = "all"
\stoptyping

As a proof of concept we can recolor an included file. Of course this assumes
a rather simple use of color. Here is an example:

\startbuffer[demo-1]
\startluacode
    graphics.registerpdfcompactor ( "preset:demo-1", {
        identify = {
            content   = true,
            resources = true,
            page      = true,
        },
        merge = {
            type0    = true,
            truetype = true,
            type1    = true,
            lmtx     = true,
        },
        recolor = {
            viagray = { 1, 0, 0 },
         -- viagray = { 0, 1, 0 },
         -- viagray = { 0, 1, 0, .5 },
         -- viagray = { .75 },
        }
    } )
\stopluacode
\setupexternalfigures[compact=preset:demo-1]
\startTEXpage
    \startcombination[3*4]
        {\externalfigure[test-000.pdf][frame=on]}        {\LUAMETATEX\ 0}
        {\externalfigure[test-001.pdf][frame=on]}        {\LUATEX\ 1}
        {\externalfigure[test-002.pdf][frame=on]}        {\LUATEX\ 2}
        {\externalfigure[test-003.pdf][frame=on,page=1]} {\LUATEX\ 3.1}
        {\externalfigure[test-003.pdf][frame=on,page=2]} {\LUATEX\ 3.2}
        {\externalfigure[test-003.pdf][frame=on,page=3]} {\LUATEX\ 3.3}
        {\externalfigure[test-004.pdf][frame=on,page=1]} {\PDFTEX\ 4.1}
        {\externalfigure[test-004.pdf][frame=on,page=2]} {\PDFTEX\ 4.2}
        {\externalfigure[test-004.pdf][frame=on,page=3]} {\PDFTEX\ 4.3}
        {\externalfigure[test-005.pdf][frame=on,page=1]} {\PDFTEX\ 4.1}
        {\externalfigure[test-005.pdf][frame=on,page=2]} {\PDFTEX\ 5.2}
        {\externalfigure[test-005.pdf][frame=on,page=3]} {\PDFTEX\ 5.3}
    \stopcombination
\stopTEXpage
\stopbuffer

\typebuffer[demo-1]

In \in {figure} [fig:compact-1] we make a single page document that embeds 12
pages from six files made by several engines. The six files have a total of about
114K but the single page combination is only 19K. The test files are:

\typefile{test-000}

So this one is an \LMTX\ produced file. The next two files:

\typefile{test-001}

and

\typefile{test-002}

are done by \LUATEX\ with \MKIV\ and

\typefile{test-003}

as well as

\typefile{test-004}

and

\typefile{test-005}

are typeset with \PDFTEX\ and \MKII\ so they have the \TYPEONE\ instead of the
\OPENTYPE\ Latin Modern file embedded (in fact, the \MKII\ and \MKIV\ files use
the twelve point variant and \LMTX\ the upscaled ten point), so if those were the
same we would have an even smaller final file.

\startplacefigure[title={An example of content manipulation.},reference=fig:compact-1]
    \typesetbuffer[demo-1][compact=yes]
\stopplacefigure

A useful manipulation is removing tags. The fact that the content is tagged
doesn't mean that tagging has any use, certainly not if it relates to editing
specific for some application. Maybe at some point I'll add a re-tagging option
but for now we just strip:

\starttyping
strip = {
    marked    = true,
 -- group     = true,
 -- extgstate = true,
}
\stoptyping

The other two are sort of special and might be needed too, especially when for
instance the states are just there because the producer wasn't clever enough
to leave them out when not applicable.

It happens that producers use color while actually gray scales are meant. In that
case one can use these:

\starttyping
reduce = {
    color = true, -- both rgb and cmyk
    rgb   = true,
    cmyk  = true,
}
\stoptyping

\type {reduce} converts to gray scale all the \RGB\ colors that have the same
values for \type {r}, \type {g} and \type {b} and \typ {rgb = true} or \typ
{color = true}).

The same goes for every \CMYK\ color where \type {c}, \type {m}, \type {m} are
the same and when \typ {cmyk = true} or \typ {color = true}. In this case the
common component component is added to the \type {k} component. For example, \typ
{.2 .2 .2 .5 K} becomes \typ {.2 + .5 = .7 G}, while \typ {.5 .5 .5 .7 K} becomes
\typ {1 G}, because the sum is limited to 1.

Using a gray scale is more efficient and in the case of \CMYK\ a sloppy \typ {.5
.5 .5 0 K} quite likely is meant to be \typ {0 0 0 0.5 K} or just \typ {.5 G}.

Remapping \RGB\ to \CMYK\ (or gray if applicable) is done with:

\starttyping
convert = {
    rgb  = true,
 -- cmyk = true,
}
\stoptyping

and of course one can also remap \CMYK\ to \RGB.

%     \setupexternalfigures[compact=pdfmerge:mine]

%     test \type {test}

%     \startTEXpage \externalfigure[test-000.pdf]\stopTEXpage % luametatex
%     \startTEXpage \externalfigure[test-001.pdf]\stopTEXpage % luatex
%     \startTEXpage \externalfigure[test-002.pdf]\stopTEXpage % luatex
%     \dorecurse{3}{
%         \startTEXpage \externalfigure[test-003.pdf][page=#1]\stopTEXpage % luatex
%     }
%     \dorecurse{3}{
%         \startTEXpage \externalfigure[test-004.pdf][page=#1]\stopTEXpage % pdftex
%     }
%     \dorecurse{3}{
%         \startTEXpage \externalfigure[test-005.pdf][page=#1]\stopTEXpage % pdftex
%     }

% --     add = {
% --         cidset, -- when missing or even fix
% --     },

I want to stress that manipulating the content stream has some limitations. For
instance because objects are shared including a page a second time will reuse the
already converted page. However, you can try the next trick:

\startbuffer[demo-2]
\startluacode
    graphics.registerpdfcompactor ( "preset:demo-2", {
        identify =  "all",
        merge    = { lmtx = true },
        recolor  = { viagray = { 0, 1, 0 } },
    } )
    graphics.registerpdfcompactor ( "preset:demo-3", {
        identify =  "all",
        merge    = { lmtx = true },
        recolor  = { viagray = { 0, 0, 1 } },
    } )

\stopluacode
\setupexternalfigures[compact=preset:demo-1]
\startTEXpage
    \startcombination[2*1]
        {\externalfigure
            [test-000.pdf]
            [frame=on,compact=preset:demo-2,width=6cm,object=no,arguments=1]}
            {demo-2}
        {\externalfigure
            [test-000.pdf]
            [frame=on,compact=preset:demo-3,width=6cm,object=no,arguments=2]}
            {demo-3}
    \stopcombination
\stopTEXpage
\stopbuffer

\typebuffer[demo-2]

In \in {figure} [fig:compact-2] we see that indeed a different compactor is used.
We need to disable sharing by setting \type {object} to \type {no}. However, this
will still share some but we abuse the arguments key to create a different
sharing hash (normally that key is used to pass arguments to converters).

\startplacefigure[title={An example of manipulation content twice.},reference=fig:compact-2]
    \typesetbuffer[demo-2][compact=yes]
\stopplacefigure

I cases where color conversion is problematic (or critical) you can remap
specific colors. Especially \CMYK\ is sensitive for conversion because there we
have four color components while in \RGB\ we have only three. Also watching on a
display (\RGB) is different from looking at a print (\CMYK) and who knows what
transfer function gets applied in the former. Here is how remapping works:

\starttyping
local cmykmap = {
    { 100, 100, 55, 0, 57, 0, 22, 40.8 }
}
graphics.registerpdfcompactor ( "preset:demo-5", {
    identify =  "all",
    merge    = { lmtx = true },
    convert  = { cmyk = cmykmap },
} )
\stoptyping

Here the entries in a \CMYK\ map are:

\starttyping
    { factor, c, m, y, k, r, g, b }
\stoptyping

In this case values are multiplied by 100 which makes sure that we catch rounding
errors in the \PDF\ definitions. Keep in mind that colors in many applications
have at most 256 values per component. Also, even quality \LCD\ displays can use
less than eight bits per component.

\startbuffer[demo-3]
\startluacode
    local cmykmap = {
        { 100, 100, 55, 0, 57, 0, 22, 40.8 } -- factor, c, m, y, k, r, g, b
    }
    graphics.registerpdfcompactor ( "preset:demo-4", {
        identify =  "all",
        merge    = { lmtx = true },
        convert  = { cmyk = true },
    } )
    graphics.registerpdfcompactor ( "preset:demo-5", {
        identify =  "all",
        merge    = { lmtx = true },
        convert  = { cmyk = cmykmap },
    } )
\stopluacode
\setupexternalfigures[compact=preset:demo-1]
\startTEXpage
    \startcombination[2*1]
        {\externalfigure
            [test-006.pdf]
            [frame=on,compact=preset:demo-4,width=6cm,object=no,arguments=3]}
            {demo-4}
        {\externalfigure
            [test-006.pdf]
            [frame=on,compact=preset:demo-5,width=6cm,object=no,arguments=4]}
            {demo-5}
    \stopcombination
\stopTEXpage
\stopbuffer

In \in {figure} [fig:compact-3] we show an example. The file used looks like:

\typefile{test-006.tex}

\startplacefigure[title={An example of remapping \CMYK\ colors.},reference=fig:compact-3]
    \typesetbuffer[demo-3][compact=yes]
\stopplacefigure

In case we wanted to map single \RGB\ values to \CMYK,
we would define an analogous map:

\starttyping
local rgbmap = {
    { 100, 0, 22, 40.8, 100, 55, 0, 57 } -- factor, r, g, b, c, m, y, k
}
graphics.registerpdfcompactor ( "preset:demo-8", {
    identify =  "all",
    merge    = { lmtx = true },
    convert  = { rgb = rgbmap },
} )
\stoptyping

Here the entries in a \RGB\ map are:

\starttyping
    { factor, r, g, b, c, m, y, k }
\stoptyping

\startbuffer[demo-6]
\startluacode
    local cmykmap = {
        { 100, 0, 22, 40.8, 100, 55, 0, 57 } -- factor, r, g, b, c, m, y, k
    }
    graphics.registerpdfcompactor ( "preset:demo-6", {
        identify =  "all",
        merge    = { lmtx = true },
        convert  = { rgb = true },
    } )
    graphics.registerpdfcompactor ( "preset:demo-7", {
        identify =  "all",
        merge    = { lmtx = true },
        convert  = { rgb = rgbmap },
    } )
\stopluacode

\setupexternalfigures[compact=preset:demo-1]

\startTEXpage
    \startcombination[nx=2,ny=1]
        {\externalfigure
            [test-006.pdf]
            [frame=on,compact=preset:demo-6,width=6cm,object=no,arguments=5]}
        {demo-6}
        {\externalfigure
            [test-006.pdf]
            [frame=on,compact=preset:demo-7,width=6cm,object=no,arguments=6]}
        {demo-7}
    \stopcombination
\stopTEXpage
\stopbuffer

In \in {figure} [fig:compact-6] we show an example.

\startplacefigure[title={An example of remapping \RGB\ colors.},reference=fig:compact-6]
    \typesetbuffer[demo-6][compact=yes]
\stopplacefigure

\stopsubject

\startsubject[title={The \type {fixpdf} script}]

I want to stress that you need to check the outcome. Often a visual check is
enough. Extending the compactor beyond what \MKIV\ provided was to a large extend
facilitated by a cooperation with Tan, Syabil M. and Ser, Zheng Y. of \quote
{Team Ramkumar} who did extensive testing and gave enjoyable feedback. In the
process a test script was made that can help with experiments. We assume that
\type {qpdf}, \type {mutool} and \type {graphicmagic} abd \type {verapdf} are
installed. Massimiliano Farinella applied these mechanism to large complex
files from InDesign and InkScape that needed fixing and in the process the code
got extended and improved. \footnote {Feel free to send us files that give
problems so that we can look into it.}

\starttyping
mtxrun --script fixpdf --uncompress                         foo
mtxrun --script fixpdf --convert    --compactor=preset:test foo
mtxrun --script fixpdf --validate                           foo
mtxrun --script fixpdf --check                              foo
mtxrun --script fixpdf --compare    --resolution=300        foo
\stoptyping

Here we produce an uncompressed version (so that we can see what we deal with),
convert the original into a new one, validate (and check the outcome) and create
a version for visual comparison. It's just an example of usage and here the focus
was on fixing existing documents (six digit numbers so the workflow needs to be
carefully checked) and not so much on single page inclusion.

However, this script and setup is somewhat complex so we also provide an
alternative in the \quote {extras} namespace:

\starttyping
context --extra=fixpdf --compactor=mine:test --extrastyle=foo somefile.pdf
\stoptyping

Additional options are \type {--notracing} and \type {--nocompression}. A
compactor can be defined in a file with the name \typ {compactors-mine.lua} that
looks like this. Check out \typ {compactors-preset.lua} for examples.

\starttyping
local fonts = {
    { source = "arial",                    target = "file:arial.ttf"   },
    { source = "arial-bold",               target = "file:arialbd.ttf" },
    { source = "arial,bold",               target = "file:arialbd.ttf" },
    { source = "helvetica",                target = "file:arial.ttf"   },
    { source = "helvetica-bold",           target = "file:arialbd.ttf" },
    { source = "courier",                  target = "file:cour.ttf"    },
    { source = "wingdings",                target = "wingding"         },
    { source = "times-roman",              target = "file:times.ttf"   },
    { source = "timesnewromanpsmt",        target = "file:times.ttf"   },
    { source = "timesnewromanpsitalicmt",  target = "file:timesi.ttf"  },
    { source = "timesnewromanps-italicmt", target = "file:timesi.ttf"  },
    { source = "timesnewroman,italic",     target = "file:timesi.ttf"  },
    { source = "timesitalic",              target = "file:timesi.ttf"  },
    { source = "times-italic",             target = "file:timesi.ttf"  },
    { source = "timesnewromanpsboldmt",    target = "file:timesbd.ttf" },
    { source = "timesnewromanps-boldmt",   target = "file:timesbd.ttf" },
    { source = "timesnewroman,bold",       target = "file:timesbd.ttf" },
    { source = "timesbold",                target = "file:timesbd.ttf" },
    { source = "times-bold",               target = "file:timesbd.ttf" },
}

return {
    name = "compactors-preset",
    version = "1.00",
    comment = "Definitions that complement pdf embedding.",
    author = "Hans Hagen",
    copyright = "ConTeXt development team",
    compactors = {
        ["test"] = {
            fonts = fonts,
            embed = {
                type0    = true,
                truetype = true,
                type1    = true,
            },
            merge = {
                type0    = true,
                truetype = true,
                type1    = true,
                LMTX     = true,
            },
            strip = {
               marked    = true,
            },
            cleanup = {
               pieceinfo = true,
               procset   = true,
               cidset    = true,
            },
        }
    },
}
\stoptyping

A file like this is easier than registering in a \LUA\ snippet. It's also more
future proof. The somewhat weird font list is normally build up as we test and is
often rather specific for a specific set of files.

\stopsubject

\stopdocument
