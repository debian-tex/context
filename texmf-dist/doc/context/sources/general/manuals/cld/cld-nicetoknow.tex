% language=us runpath=texruns:manuals/cld

\startcomponent cld-nicetoknow

\environment cld-environment

\startchapter[title=Nice to know]

\startsection[title=Introduction]

As we like to abstract interfaces it is no surprise that \CONTEXT\ and
therefore it's \LUA\ libraries come with all kind of helpers. In this
chapter I will explain a few of them. Feel free to remind of adding more
here.

\stopsection

\startsection[title=Templates]

{\em Eventually we will move this to the utilities section.}

When dealing with data from tables or when order matters it can be handy
to abstract the actual data from the way it is dealt with. For this we
provide a template mechanism. The following example demonstrate its use.

\startbuffer
require("util-ran") -- needed for this example

local preamble = [[|l|l|c|]]
local template = [[\NC %initials% \NC %surname% \NC %length% \NC \NR]]

context.starttabulate { preamble }
    for i=1,10 do
        local row = utilities.templates.replace(template, {
            surname  = utilities.randomizers.surname(5,10),
            initials = utilities.randomizers.initials(1,3),
            length   = string.format("%0.2f",math.random(140,195)),
        })
        context(row)
    end
context.stoptabulate()
\stopbuffer

\typebuffer

This renders a table with random entries:

\ctxluabuffer

The nice thing is that when we change the order of the columns, we don't need to
change the table builder.

\starttyping
local preamble = [[|c|l|l|]]
local template = [[\NC %length% \NC %initials% \NC %surname% \NC \NR]]
\stoptyping

The \type {replace} function takes a few more arguments. There are also a some
more replacement options.

\starttyping
replace("test '%[x]%' test",{ x = [[a 'x'  a]] }))
replace("test '%[x]%' test",{ x = true }))
replace("test '%[x]%' test",{ x = [[a 'x'  a]], y = "oeps" },'sql'))
replace("test '%[x]%' test",{ x = [[a '%y%'  a]], y = "oeps" },'sql',true))
replace([[test %[x]% test]],{ x = [[a "x"  a]]}))
replace([[test %(x)% test]],{ x = [[a "x"  a]]}))
\stoptyping

The first argument is the template and the second one a table with mappings from
keys to values. The third argument can be used to inform the replace mechanism
what string escaping has to happen. The last argument triggers recursive
replacement. The above calls result in the following strings:

\starttyping
test 'a 'x' \127 a' test
test 'true' test
test 'a ''x''  a' test
test 'a ''oeps''  a' test
test a \"x\" \127 a test
test "a \"x\" \127 a" test
\stoptyping

These examples demonstrate that by adding a pair of square brackets we get
escaped strings. When using parenthesis the quotes get added automatically. This
is somewhat faster in case when \LUA\ is the target, but in practice it is not
that noticeable.

% replace(str,mapping,how,recurse)

\stopsection

\startsection [title=Extending]

Instead of extending tex endlessly we can also define our own extensions. Here
is an example. When you want to manipulate a box at the \LUA\ end you have the
problem that the following will not always work out well:

\starttyping
local b = tex.getbox(0)
-- mess around with b
tex.setbox(0,b)
\stoptyping

So we end up with:

\starttyping
local b = node.copy_list(tex.getbox(0))
-- mess around with b
tex.setbox(0,b)
\stoptyping

The reason is that at the \TEX\ end grouping plays a role which means that values
are saved and restored. However, there is a save way out by defining a function
that cheats a bit:

\starttyping
function tex.takebox(id)
    local box = tex.getbox(id)
    if box then
        local copy = node.copy(box)
        local list = box.list
        copy.list = list
        box.list = nil
        tex.setbox(id,nil)
        return copy
    end
end
\stoptyping

Now we can say:

\starttyping
local b = tex.takebox(0)
-- mess around with b
tex.setbox(b)
\stoptyping

In this case we first get the box content and then let \TEX\ do some housekeeping.
But, because we only keep the list node (which we copied) in the register the
overhead of copying a whole list is gone.

\stopsection

% require("util-sto") require("char-def") require("char-ini")

% local myformatter = utilities.strings.formatters.new()

% string.addformatter("upper",  [[upper (%s)]],[[local upper  = characters.upper ]]) -- maybe handy
% string.addformatter("lower",  [[lower (%s)]],[[local lower  = characters.lower ]]) -- maybe handy
% string.addformatter("shaped", [[shaped(%s)]],[[local shaped = characters.shaped]]) -- maybe handy

% utilities.strings.formatters.add("upper",  [[lpegmatch(p_toupper,%s)]],[[local p_toupper = lpeg.patterns.toupper]]) -- maybe handy
% utilities.strings.formatters.add("lower",  [[lpegmatch(p_tolower,%s)]],[[local p_tolower = lpeg.patterns.tolower]]) -- maybe handy
% utilities.strings.formatters.add("shaped", [[lpegmatch(p_toshape,%s)]],[[local p_toshape = lpeg.patterns.toshape]]) -- maybe handy

% print("\n>>>",string.formatters["Is this %s handy or not?"](characters.upper("ÀÁÂÃÄÅàáâãäå")))
% print("\n>>>",string.formatters["Is this %!upper! handy or not?"]("ÀÁÂÃÄÅàáâãäå"))
% print("\n>>>",string.formatters["Is this %!shaped! handy or not?"]("ÀÁÂÃÄÅàáâãäå"))

\stopchapter

\stopcomponent
