% language=us runpath=texruns:manuals/cld

\startcomponent cld-ctxfunctions

\environment cld-environment

\startchapter[title={The \LUA\ interface code}]

\startsection[title={Introduction}]

There is a lot of \LUA\ code in \MKIV. Much is not exposed and a lot of what is
exposed is not meant to be used directly at the \LUA\ end. But there is also
functionality and data that can be accessed without side effects.

In the following sections a subset of the built in functionality is discussed.
There are often more functions alongside those presented but they might change or
disappear. So, if you use undocumented features, be sure to tag them somehow in
your source code so that you can check them out when there is an update. Best
would be to have more functionality defined local so that it is sort of hidden
but that would be unpractical as for instance functions are often used in other
modules and or have to be available at the \TEX\ end.

It might be tempting to add your own functions to namespaces created by \CONTEXT\
or maybe overload some existing ones. Don't do this. First of all, there is no
guarantee that your code will not interfere, nor that it overloads future
functionality. Just use your own namespace. Also, future versions of \CONTEXT\
might have a couple of protection mechanisms built in. Without doubt the
following sections will be extended as soon as interfaces become more stable.

\stopsection

\startsection[title={Characters}]

% not discussed:
%
% characters.filters.utf.addgrapheme()
% characters.filters.utf.collapse()
% characters.getrange()
% characters.bidi[]
% tex.uprint()
% utf.string()
% characters.flush()

There are quite some data tables defined but the largest is the character
database. You can consult this table any time you want but you're not supposed to
add or change its content if only because changes will be overwritten when you
update \CONTEXT. Future versions may carry more information. The table can be
accessed using an unicode number. A relative simple entry looks as follows:

\ShowLuaExampleTableHex{characters.data[0x00C1]}

Much of this is rather common information but some of it is specific for use with
\CONTEXT. Some characters have even more information, for instance those that
deal with mathematics:

\ShowLuaExampleTableHex{characters.data[0x2190]}

Not all characters have a real entry. For instance most \CJK\ characters are
virtual and share the same data:

\ShowLuaExampleTableHex{characters.data[0x3456]}

You can also access the table using \UTF\ characters:

\ShowLuaExampleTable{characters.data["ä"]}

A more verbose string access is also supported:

\ShowLuaExampleTableHex{characters.data["U+0070"]}

Another (less usefull) table contains information about ranges in this character
table. You can access this table using rather verbose names, or you can use
collapsed lowercase variants.

\ShowLuaExampleTableHex{characters.blocks["CJK Compatibility Ideographs"]}

\ShowLuaExampleTableHex{characters.blocks["hebrew"]}

\ShowLuaExampleTableHex{characters.blocks["combiningdiacriticalmarks"]}

Some fields can be accessed using functions. This can be handy when you need that
information for tracing purposes or overviews. There is some overhead in the
function call, but you get some extra testing for free. You can use characters as
well as numbers as index.

\ShowLuaExampleString{characters.contextname("ä")}
\ShowLuaExampleString{characters.adobename(228)}
\ShowLuaExampleString{characters.description("ä")}

The category is normally a two character tag, but you can also ask for a more
verbose variant:

\ShowLuaExampleString{characters.category(228)}
\ShowLuaExampleString{characters.category(228,true)}

The more verbose category tags are available in a table:

\ShowLuaExampleString{characters.categorytags["lu"]}

There are several fields in a character entry that help us to remap a character.
The \type {lccode} indicates the lowercase code point and the \type {uccode} to
the uppercase code point. The \type {shcode} refers to one or more characters
that have a similar shape.

\ShowLuaExampleString{characters.shape ("ä")}
\ShowLuaExampleString{characters.uccode("ä")}
\ShowLuaExampleString{characters.lccode("ä")}

\ShowLuaExampleString{characters.shape (100)}
\ShowLuaExampleString{characters.uccode(100)}
\ShowLuaExampleString{characters.lccode(100)}

You can use these function or access these fields directly in an
entry, but we also provide a few virtual tables that avoid
accessing the whole entry. This method is rather efficient.

\ShowLuaExampleString{characters.lccodes["ä"]}
\ShowLuaExampleString{characters.uccodes["ä"]}
\ShowLuaExampleString{characters.shcodes["ä"]}
\ShowLuaExampleString{characters.lcchars["ä"]}
\ShowLuaExampleString{characters.ucchars["ä"]}
\ShowLuaExampleString{characters.shchars["ä"]}

As with other tables, you can use a number instead of an \UTF\ character. Watch
how we get a table for multiple shape codes but a string for multiple shape
characters.

\ShowLuaExampleString{characters.lcchars[0x00C6]}
\ShowLuaExampleString{characters.ucchars[0x00C6]}
\ShowLuaExampleString{characters.shchars[0x00C6]}
\ShowLuaExampleTable {characters.shcodes[0x00C6]}

These codes are used when we manipulate strings. Although there
are \type {upper} and \type {lower} functions in the \type
{string} namespace, the following ones are the real ones to be
used in critical situations.

\ShowLuaExampleString{characters.lower("ÀÁÂÃÄÅàáâãäå")}
\ShowLuaExampleString{characters.upper("ÀÁÂÃÄÅàáâãäå")}
\ShowLuaExampleString{characters.shaped("ÀÁÂÃÄÅàáâãäå")}

A rather special one is the following:

\ShowLuaExampleString{characters.lettered("Only 123 letters + count!")}

With the second argument is true, spaces are kept and collapsed. Leading and
trailing spaces are stripped.

\ShowLuaExampleString{characters.lettered("Only 123 letters + count!",true)}

Access to tables can happen by number or by string, although there are some
limitations when it gets too confusing. Take for instance the number \type {8}
and string \type {"8"}: if we would interpret the string as number we could never
access the entry for the character eight. However, using more verbose hexadecimal
strings works okay. The remappers are also available as functions:

\ShowLuaExampleString{characters.tonumber("a")}
\ShowLuaExampleString{characters.fromnumber(100)}
\ShowLuaExampleString{characters.fromnumber(0x0100)}
\ShowLuaExampleString{characters.fromnumber("0x0100")}
\ShowLuaExampleString{characters.fromnumber("U+0100")}

In addition to the already mentioned category information you can also use a more
direct table approach:

\ShowLuaExampleString{characters.categories["ä"]}
\ShowLuaExampleString{characters.categories[100]}

In a similar fashion you can test if a given character is in a specific category.
This can save a lot of tests.

\ShowLuaExampleBoolean{characters.is_character[characters.categories[67]]}
\ShowLuaExampleBoolean{characters.is_character[67]}
\ShowLuaExampleBoolean{characters.is_character[characters.data[67].category]}
\ShowLuaExampleBoolean{characters.is_letter[characters.data[67].category]}
\ShowLuaExampleBoolean{characters.is_command[characters.data[67].category]}

Another virtual table is the one that provides access to special information, for
instance about how a composed character is made up of components.

\ShowLuaExampleString{characters.specialchars["ä"]}
\ShowLuaExampleString{characters.specialchars[100]}

The outcome is often similar to output that uses the shapecode information.

Although not all the code deep down in \CONTEXT\ is meant for use at the user
level, it sometimes can eb tempting to use data and helpers that are available as
part of the general housekeeping. The next table was used when looking into
sorting Korean. For practical reasons we limit the table to ten entries;
otherwise we would have ended up with hundreds of pages.

\startbuffer
\startluacode
local data = characters.data
local map  = characters.hangul.remapped

local first, last = characters.getrange("hangulsyllables")

last = first + 9 -- for now

context.start()

context.definedfont { "file:unbatang" }

context.starttabulate { "|T||T||T||T||T|" }
for unicode = first, last do
    local character = data[unicode]
    local specials = character.specials
    if specials then
        context.NC()
        context.formatted("%04V",unicode)
        context.NC()
        context.formatted("%c",unicode)
        for i=2,4 do
            local chr = specials[i]
            if chr then
                chr = map[chr] or chr
                context.NC()
                context.formatted("%04V",chr)
                context.NC()
                context.formatted("%c",chr)
            else
                context.NC()
                context.NC()
            end
        end
        context.NC()
        context(character.description)
        context.NC()
        context.NR()
    end
end
context.stoptabulate()

context.stop()
\stopluacode
\stopbuffer

\getbuffer \typebuffer

\stopsection

\startsection[title={Fonts}]

% not discussed (as not too relevant for users):
%
% cache cache_version
% nomath
% units units_per_em
% direction embedding encodingbytes
% boundarychar boundarychar_label
% has_italic has_math
% tounicode sub
% colorscheme (will probably become a hash)
% language script
% spacer
% MathConstants and a few split_names
%
% tables.baselines

There is a lot of code that deals with fonts but most is considered to be a black
box. When a font is defined, its data is collected and turned into a form that
\TEX\ likes. We keep most of that data available at the \LUA\ end so that we can
later use it when needed. In this chapter we discuss some of the possibilities.
More details can be found in the font manual(s) so we don't aim for completeness
here.

A font instance is identified by its id, which is a number where zero is reserved
for the so called \type {nullfont}. The current font id can be requested by the
following function.

\ShowLuaExampleString{fonts.currentid()}

The \type {fonts.current()} call returns the table with data related to the
current id. You can access the data related to any id as follows:

\starttyping
local tfmdata = fonts.identifiers[number]
\stoptyping

Not all entries in the table make sense for the user as some are just meant to
drive the font initialization at the \TEX\ end or the backend. The next table
lists the most important ones. Some of the tables are just shortcuts to en entry
in one of the \type {shared} subtables.

\starttabulate[|l|Tl|p|]
\NC \type{ascender}       \NC number \NC the height of a line conforming the font \NC \NR
\NC \type{descender}      \NC number \NC the depth of a line conforming the font \NC \NR
\NC \type{italicangle}    \NC number \NC the angle of the italic shapes (if present) \NC \NR
\NC \type{designsize}     \NC number \NC the design size of the font (if known) \NC \NR
\ML
\NC \type{size}           \NC number \NC the size in scaled points if the font instance \NC \NR
\NC \type{factor}         \NC number \NC the multiplication factor for unscaled dimensions \NC \NR
\NC \type{hfactor}        \NC number \NC the horizontal multiplication factor \NC \NR
\NC \type{vfactor}        \NC number \NC the vertical multiplication factor \NC \NR
\NC \type{extend}         \NC number \NC the horizontal scaling to be used by the backend \NC \NR
\NC \type{slant}          \NC number \NC the slanting to be applied by the backend \NC \NR
\ML
\NC \type{characters}     \NC table  \NC the scaled character (glyph) information (tfm) \NC \NR
\NC \type{descriptions}   \NC table  \NC the original unscaled glyph information (otf, afm, tfm) \NC \NR
\NC \type{indices}        \NC table  \NC the mapping from unicode slot to glyph index \NC \NR
\NC \type{unicodes}       \NC table  \NC the mapoing from glyph names to unicode \NC \NR
\NC \type{marks}          \NC table  \NC a hash table with glyphs that are marks as entry \NC \NR
\NC \type{parameters}     \NC table  \NC the font parameters as \TEX\ likes them \NC \NR
\NC \type{mathconstants}  \NC table  \NC the \OPENTYPE\ math parameters \NC \NR
\NC \type{mathparameters} \NC table  \NC a reference to the \type {MathConstants} table \NC \NR
\NC \type{shared}         \NC table  \NC a table with information shared between instances \NC \NR
\NC \type{unique}         \NC table  \NC a table with information unique for this instance \NC \NR
\NC \type{unscaled}       \NC table  \NC the unscaled (intermediate) table \NC \NR
\NC \type{goodies}        \NC table  \NC the \CONTEXT\ specific extra font information \NC \NR
\NC \type{fonts}          \NC table  \NC the table with references to other fonts \NC \NR
\NC \type{cidinfo}        \NC table  \NC a table with special information for the backend \NC \NR
\ML
\NC \type{filename}       \NC string \NC the full path of the loaded font \NC \NR
\NC \type{fontname}       \NC string \NC the font name as specified in the font (limited in size) \NC \NR
\NC \type{fullname}       \NC string \NC the complete font name as specified in the font \NC \NR
\NC \type{name}           \NC string \NC the (short) name of the font \NC \NR
\NC \type{psname}         \NC string \NC the (unique) name of the font as used by the backend \NC \NR
\ML
\NC \type{hash}           \NC string \NC the hash that makes this instance unique \NC \NR
\NC \type{id}             \NC number \NC the id (number) that \TEX\ will use for this instance \NC \NR
\ML
\NC \type{type}           \NC string \NC an idicator if the font is \type {virtual} or \type {real} \NC \NR
\NC \type{format}         \NC string \NC a qualification for this font, e.g.\ \type {opentype} \NC \NR
\NC \type{mode}           \NC string \NC the \CONTEXT\ processing mode, \type {node} or \type {base} \NC \NR
\ML
\stoptabulate

The \type {parameters} table contains variables that are used by \TEX\ itself.
You can use numbers as index and these are equivalent to the so called \type
{\fontdimen} variables. More convenient is is to access by name:

\starttabulate[|l|p|]
\NC \type{slant}        \NC the slant per point (seldom used) \NC \NR
\NC \type{space}        \NC the interword space \NC \NR
\NC \type{spacestretch} \NC the interword stretch \NC \NR
\NC \type{spaceshrink}  \NC the interword shrink \NC \NR
\NC \type{xheight}      \NC the x|-|height (not per se the heigth of an x) \NC \NR
\NC \type{quad}         \NC the so called em|-|width (often the width of an emdash)\NC \NR
\NC \type{extraspace}   \NC additional space added in specific situations \NC \NR
\stoptabulate

The math parameters are rather special and explained in the \LUATEX\ manual.
Quite certainly you never have to touch these parameters at the \LUA\ end.

En entry in the \type {characters} table describes a character if we have entries
within the \UNICODE\ range. There can be entries in the private area but these
are normally variants of a shape or special math glyphs.

\starttabulate[|l|p|]
\NC \type{name}             \NC the name of the character \NC \NR
\NC \type{index}            \NC the index in the raw font table \NC \NR
\NC \type{height}           \NC the scaled height of the character \NC \NR
\NC \type{depth}            \NC the scaled depth of the character \NC \NR
\NC \type{width}            \NC the scaled height of the character \NC \NR
\NC \type{tounicode}        \NC a \UTF-16 string representing the conversion back to unicode \NC \NR
\NC \type{expansion_factor} \NC a multiplication factor for (horizontal) font expansion \NC \NR
\NC \type{left_protruding}  \NC a multiplication factor for left side protrusion \NC \NR
\NC \type{right_protruding} \NC a multiplication factor for right side protrusion \NC \NR
\NC \type{italic}           \NC the italic correction \NC \NR
\NC \type{next}             \NC a pointer to the next character in a math size chain \NC \NR
\NC \type{vert_variants}    \NC a pointer to vertical variants conforming \OPENTYPE\ math \NC \NR
\NC \type{horiz_variants}   \NC a pointer to horizontal variants conforming \OPENTYPE\ math \NC \NR
\NC \type{top_accent}       \NC information with regards to math top accents \NC \NR
\NC \type{mathkern}         \NC a table describing stepwise math kerning (following the shape) \NC \NR
\NC \type{kerns}            \NC a table with intercharacter kerning dimensions \NC \NR
\NC \type{ligatures}        \NC a (nested) table describing ligatures that start with this character \NC \NR
\NC \type{commands}         \NC a table with commands that drive the backend code for a virtual shape \NC \NR
\stoptabulate

Not all entries are present for each character. Also, in so called \type {node}
mode, the \type {ligatures} and \type {kerns} tables are empty because in that
case they are dealt with at the \LUA\ end and not by \TEX.

% \startluacode
% local tfmdata = fonts.current()
% context.starttabulate{ "|l|pl|" }
%     for k, v in table.sortedhash(tfmdata) do
%         local tv = type(v)
%         if tv == "string" or tv == "number" or tv == "boolean" then
%             context.NC()
%             string.tocontext(k)
%             context.NC()
%             string.tocontext(tostring(v))
%             context.NC()
%             context.NR()
%         end
%     end
% context.stoptabulate()
% \stopluacode

% \ShowLuaExampleTable{table.sortedkeys(fonts.current())}

Say that you run into a glyph node and want to access the data related to that
glyph. Given that variable \type {n} points to the node, the most verbose way of
doing that is:

\starttyping
local g = fonts.identifiers[n.id].characters[n.char]
\stoptyping

Given the speed of \LUATEX\ this is quite fast. Another method is the following:

\starttyping
local g = fonts.characters[n.id][n.char]
\stoptyping

For some applications you might want faster access to critical
parameters, like:

\starttyping
local quad    = fonts.quads   [n.id][n.char]
local xheight = fonts.xheights[n.id][n.char]
\stoptyping

but that only makes sense when you don't access more than one such variable at
the same time.

Among the shared tables is the feature specification:

\ShowLuaExampleTable{fonts.current().shared.features}

As features are a prominent property of \OPENTYPE\ fonts, there are a few
datatables that can be used to get their meaning.

\ShowLuaExampleString{fonts.handlers.otf.tables.features['liga']}
\ShowLuaExampleString{fonts.handlers.otf.tables.languages['nld']}
\ShowLuaExampleString{fonts.handlers.otf.tables.scripts['arab']}

There is a rather extensive font database built in but discussing its interface
does not make much sense. Most usage happens automatically when you use the \type
{name:} and \type {spec:} methods of defining fonts and the \type {mtx-fonts}
script is built on top of it.

\ctxlua{fonts.names.load()} % could be metatable driven

\ShowLuaExampleTable{table.sortedkeys(fonts.names.data)}

You can load the database (if it's not yet loaded) with:

\starttyping
names.load(reload,verbose)
\stoptyping

When the first argument is true, the database will be rebuild. The second
arguments controls verbosity.

Defining a font normally happens at the \TEX\ end but you can also do it in \LUA.

\starttyping
local id, fontdata = fonts.definers.define {
    lookup = "file",             -- use the filename (file spec name)
    name   = "pagella-regular",  -- in this case the filename
    size   = 10*65535,           -- scaled points
    global = false,              -- define the font globally
    cs     = "MyFont",           -- associate the name \MyFont
    method = "featureset",       -- featureset or virtual (* or @)
    sub    = nil,                -- no subfont specifier
    detail = "whatever",         -- the featureset (or whatever method applies)
}
\stoptyping

In this case the \type {detail} variable defines what featureset has to be
applied. You can define such sets at the \LUA\ end too:

\starttyping
fonts.definers.specifiers.presetcontext (
    "whatever",
    "default",
    {
        mode = "node",
        dlig = "yes",
    }
)
\stoptyping

The first argument is the name of the featureset. The second argument can be an
empty string or a reference to an existing featureset that will be taken as
starting point. The final argument is the featureset. This can be a table or a
string with a comma separated list of key|/|value pairs.

\stopsection

\startsection[title={Nodes}]

Nodes are the building blocks that make a document reality. Nodes are linked into
lists and at various moments in the typesetting process you can manipulate them.
Deep down in \CONTEXT\ we use quite some \LUA\ magic to manipulate lists of
nodes. Therefore it is no surprise that we have some tracing available. Take the
following box.

\startbuffer
\setbox0\hbox{It's in \hbox{\bf all} those nodes.}
\stopbuffer

\typebuffer \getbuffer

This box contains characters and glue between the words. The box is already
constructed. There can also be kerns between characters, but of course only if
the font provides such a feature. Let's inspect this box:

\ShowLuaExampleString{nodes.toutf(tex.box[0])}
\ShowLuaExampleString{nodes.toutf(tex.box[0].list)}

This tracer returns the text and spacing and recurses into nested lists. The next
tracer does not do this and marks non glyph nodes as \type {[-]}:

\ShowLuaExampleString{nodes.listtoutf(tex.box[0])}
\ShowLuaExampleString{nodes.listtoutf(tex.box[0].list)}

A more verbose tracer is the next one. It does show a bit more detailed
information about the glyphs nodes.

\ShowLuaExampleString{nodes.tosequence(tex.box[0])}
\ShowLuaExampleString{nodes.tosequence(tex.box[0].list)}

The fourth tracer does not show that detail and collapses sequences of similar
node types.

\ShowLuaExampleString{nodes.idstostring(tex.box[0])}
\ShowLuaExampleString{nodes.idstostring(tex.box[0].list)}

The number of nodes in a list is identified with the \type {countall} function.
Nested nodes are counted too.

\ShowLuaExampleString{nodes.countall(tex.box[0])}
\ShowLuaExampleString{nodes.countall(tex.box[0].list)}

There are a lot of helpers in the \type {nodes} namespace. In fact, we map all the
helpers provided by the engine itself under \type {nodes} too. These are described
in the \LUATEX\ manual. There are for instance functions to check node types and
node id's:

\starttyping
local str = node.type(1)
local num = node.id("vlist")
\stoptyping

These are basic \LUATEX\ functions. In addition to those we also provide a few more
helpers as well as
mapping tables. There are two tables that map node id's to strings and backwards:

\starttabulate
\NC \type{nodes.nodecodes}    \NC regular nodes, some fo them are sort of private to the engine \NC \NR
\NC \type{nodes.noadcodes}    \NC math nodes that later on are converted into regular nodes  \NC \NR
\stoptabulate

Nodes can have subtypes. Again we have tables that map the subtype numbers onto
meaningfull names and reverse.

\starttabulate
\NC \type{nodes.listcodes}    \NC subtypes of \type {hlist} and \type {vlist} nodes \NC \NR
\NC \type{nodes.kerncodes}    \NC subtypes of \type {kern} nodes \NC \NR
\NC \type{nodes.gluecodes}    \NC subtypes of \type {glue} nodes (skips) \NC \NR
\NC \type{nodes.glyphcodes}   \NC subtypes of \type {glyph} nodes, the subtype can change \NC \NR
\NC \type{nodes.mathcodes}    \NC math specific subtypes \NC \NR
\NC \type{nodes.fillcodes}    \NC these are not really subtypes but indicate the strength of the filler \NC \NR
\NC \type{nodes.whatsitcodes} \NC subtypes of a rather large group of extension nodes \NC \NR
\stoptabulate

Some of the names of types and subtypes have underscores but you can omit them
when you use these tables. You can use tables like this as follows:

\starttyping
local glyph_code = nodes.nodecodes.glyph
local kern_code  = nodes.nodecodes.kern
local glue_code  = nodes.nodecodes.glue

for n in nodes.traverse(list) do
    local id == n.id
    if id == glyph_code then
        ...
    elseif id == kern_code then
        ...
    elseif id == glue_code then
        ...
    else
        ...
    end
end
\stoptyping

You only need to use such temporary variables in time critical code. In spite of
what you might think, lists are not that long and given the speed of \LUA\ (and
successive optimizations in \LUATEX) looping over a paragraphs is rather fast.

Nodes are created using \type {node.new}. If you study the \CONTEXT\ code you
will notice that there are quite some functions in the \type {nodes.pool}
namespace, like:

\starttyping
local g = nodes.pool.glyph(fnt,chr)
\stoptyping

Of course you need to make sure that the font id is valid and that the referred
glyph in in the font. You can use the allocators but don't mess with the code in
the \type {pool} namespace as this might interfere with its usage all over
\CONTEXT.

The \type {nodes} namespace provides a couple of helpers and some of them are
similar to ones provided in the \type {node} namespace. This has practical as
well as historic reasons. For instance some were prototypes functions that were
later built in.

\starttyping
local head, current      = nodes.before (head, current, new)
local head, current      = nodes.after  (head, current, new)
local head, current      = nodes.delete (head, current)
local head, current      = nodes.replace(head, current, new)
local head, current, old = nodes.remove (head, current)
\stoptyping

Another category deals with attributes:

\starttyping
nodes.setattribute      (head, attribute, value)
nodes.unsetattribute    (head, attribute)
nodes.setunsetattribute (head, attribute, value)
nodes.setattributes     (head, attribute, value)
nodes.unsetattributes   (head, attribute)
nodes.setunsetattributes(head, attribute, value)
nodes.hasattribute      (head, attribute, value)
\stoptyping

% context(typesetters.hpack("Hello World!"))
% context(typesetters.hpack("Hello World!",1,100*1024*10))

% nodes.firstchar
% nodes.firstcharinbox

% maybe node-tst
% tasks and so
% number.points (to numbers)

\stopsection

% \startsection[title={Core}]
%     {\em todo}
% \stopsection

\startsection[title={Resolvers}]

All \IO\ is handled by functions in the \type {resolvers} namespace. Most of the
code that you find in the \type {data-*.lua} files is of litle relevance for
users, especially at the \LUA\ end, so we won't discuss it here in great detail.

The resolver code is modelled after the \KPSE\ library that itself implements the
\TEX\ Directory Structure in combination with a configuration file. However, we
go a bit beyond this structure, for instance in integrating support for other
resources that file systems. We also have our own configuration file. But
important is that we still support a similar logic too so that regular
configurations are dealt with.

During a run \LUATEX\ needs files of a different kind: source files, font files,
images, etc. In practice you will probably only deal with source files. The most
fundamental function is \type {findfile}. The first argument is the filename to
be found. A second optional argument indicates the file type.

The following table relates so called formats to suffixes and variables in the
configuration file.

\startluacode
context.starttabulate { "|lp|lp|l|" }
context.NC() context.bold("variable")
context.NC() context.bold("format")
context.NC() context.bold("suffix")
context.NC() context.NR()
context.ML()
for k, v in table.sortedpairs(resolvers.relations.core) do
    local names = v.names
    local variable = v.variable
    local suffixes = v.suffixes
    context.NC()
    if variable then
        context.type(variable)
    end
    context.NC()
    if names then
        for i=1,#names do
            context.type(names[i])
            context.par()
        end
    end
    context.NC()
    if suffixes then
        context.type(table.concat(suffixes, " "))
    end
    context.NC()
    context.NR()
end
context.stoptabulate()
\stopluacode

There are a couple of more formats but these are not that relevant in the
perspective of \CONTEXT.

When a lookup takes place, spaces are ignored and formats are normalized to
lowercase.

\ShowLuaExampleString{file.strip(resolvers.findfile("context.tex"),"tex/")}
\ShowLuaExampleString{file.strip(resolvers.findfile("context.mkiv"),"tex/")}
\ShowLuaExampleString{file.strip(resolvers.findfile("context"),"tex/")}
\ShowLuaExampleString{file.strip(resolvers.findfile("data-res.lua"),"tex/")}
\ShowLuaExampleString{file.strip(resolvers.findfile("lmsans10-bold"),"tex/")}
\ShowLuaExampleString{file.strip(resolvers.findfile("lmsans10-bold.otf"),"tex/")}
\ShowLuaExampleString{file.strip(resolvers.findfile("lmsans10-bold","otf"),"tex/")}
\ShowLuaExampleString{file.strip(resolvers.findfile("lmsans10-bold","opentype"),"tex/")}
\ShowLuaExampleString{file.strip(resolvers.findfile("lmsans10-bold","opentypefonts"),"tex/")}
\ShowLuaExampleString{file.strip(resolvers.findfile("lmsans10-bold","opentype fonts"),"tex/")}

The plural variant of this function returns one or more matches.

\ShowLuaExampleTable{resolvers.findfiles("texmfcnf.lua","cnf")}
\ShowLuaExampleTable{resolvers.findfiles("context.tex","")}

% table.print(resolvers.instance.environment)
% table.print(resolvers.instance.variables)
% table.print(resolvers.instance.expansions)
%
% resolvers.expandbraces
% resolvers.expandpath
% resolvers.expandvar
% resolvers.showpath
% resolvers.var_value
%
% resolvers.getenv
% resolvers.variable()
% resolvers.expansion()
% resolvers.is_variable
% resolvers.is_expansion
%
% resolvers.unexpandedpathlist(str)
% resolvers.unexpandedpath(str)
% resolvers.cleanpathlist
% resolvers.expandpath
% resolvers.expandedpath
% resolvers.expandedpathlistfromvariable
% resolvers.expandpathfromvariable
% resolvers.expandbraces
%
% resolvers.findpath
% resolvers.findgivenfiles
% resolvers.findgivenfile
% resolvers.findwildcardfiles
% resolvers.findwildcardfile
% resolvers.showpath

% data-tre as example
% schemes (data-she)
% caching (containers)
% findbinfile  (open|load)
% variables / environment
% findtexfile opentexfile loadtexfile
% file://

% supp

\stopsection

\startsection[title={Mathematics (math)}]
    {\em todo}
\stopsection

\startsection[title={Graphics (grph)}]
    {\em is a separate chapter}
\stopsection

\startsection[title={Languages (lang)}]
    {\em todo}
\stopsection

\startsection[title={MetaPost (mlib)}]
    {\em todo}
\stopsection

\startsection[title={Lua\TeX\ (luat)}]
    {\em todo}
\stopsection

\startsection[title={Tracing (trac)}]
    {\em todo}
\stopsection

\stopchapter

\stopcomponent
