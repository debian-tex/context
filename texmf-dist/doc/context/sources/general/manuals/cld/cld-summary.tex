% language=uk

\startcomponent cld-summary

\environment cld-environment

\usemodule[s][characters-properties]

\startchapter[title=A sort of summary]

In this chapter we summarize the functionality provided by the \type {context}
namespace. We repeat some of what has been explained in other chapter so that in
fact you can start with this summary.

If you have read this manual (or seen code) you know that you can access all the
core commands using this namespace:

\starttyping
context.somecommand("some argument")
context["somecommand"]("some argument")
\stoptyping

These calls will eventually expand \type {\somecommand} with the given argument.
This interface has been around form the start and has proven to be quite flexible
and robust. In spite of what you might think, the \type {somecommand} is not
really defined in the \type {context} namespace, but in its own one called \type
{core}, accessible via \type {context.core}.

Next we describe the commands that are naturally defined in the \type {context}
namespace. Some have counterparts at the macro level (like \type {bgroup}) but
many haven't (for instance \type {rule}). We tried not to polute the \type
{context} namespace too much but although we could have put the helpers in a
separate namespace it would make usage a bit more unnatural.

\startsection[title=Access to commands]

\startsummary[title={context(".. some text ..")}]

The string is flushed as|-|is:

\starttyping
.. some text ..
\stoptyping

\stopsummary

\startsummary[title={context("format",...)}]

The first string is a format specification according that is passed to the \LUA\
function \type {format} in the \type {string} namespace. Following arguments are
passed too.

\stopsummary

\startsummary[title={context(123,...)}]

The numbers (and following numbers or strings) are flushed without any
formatting.

\starttyping
123... (concatenated)
\stoptyping

\stopsummary

\startsummary[title={context(true)}]

An explicit \type {endlinechar} is inserted, in \TEX\ speak:

\starttyping
^^M
\stoptyping

\stopsummary

\startsummary[title={context(false,...)}]

Strings and numbers are flushed surrounded by curly braces, an indexed table is
flushed as option list, and a hashed table is flushed as parameter set.

\starttyping
multiple {...} or [...] etc
\stoptyping

\stopsummary

\startsummary[title={context(node)}]

The node (or list of nodes) is injected at the spot. Keep in mind that you need
to do the proper memory management yourself.

\stopsummary

\startsummary[title={context["command"] context.core["command"]}]

The function that implements \type {\command}. The \type{core} table is where
these functions realy live.

\stopsummary

\startsummary[title={context["command"](value,...)}]

The value (string or number) is flushed as a curly braced (regular) argument.

\starttyping
\command {value}...
\stoptyping

\stopsummary

\startsummary[title={context["command"]({ value },...)}]

The table is flushed as value set. This can be an identifier,
a list of options, or a directive.

\starttyping
\command [value]...
\stoptyping

\stopsummary

\startsummary[title={context["command"]({ key = val },...)}]

The table is flushed as key|/|value set.

\starttyping
\command [key={value}]...
\stoptyping

\stopsummary

\startsummary[title={context["command"](true)}]

An explicit \type {endlinechar} is inserted.

\starttyping
\command ^^M
\stoptyping

\stopsummary

\startsummary[title={context["command"](node)}]

The node(list) is injected at the spot. Keep in mind that you need to do the
proper memory management yourself.

\starttyping
\command {node(list)}
\stoptyping

\stopsummary

\startsummary[title={context["command"](false,value)}]

The value is flushed without encapsulating tokens.

\starttyping
\command value
\stoptyping

\stopsummary

\startsummary[title={context["command"]({ value }, { key = val }, val, false, val)}]

The arguments are flushed accordingly their nature and the order can be any.

\starttyping
\command [value][key={value}]{value}value
\stoptyping

\stopsummary

\startsummary[title={context.direct(...)}]

The arguments are interpreted the same as if \type {direct} was a command, but no
\type {\direct} is injected in front. Braces are added:

\startbuffer
regular \expandafter \bold \ctxlua{context.direct("bold")} regular
black \expandafter \color \ctxlua{context.direct({"red"})}{red} black
black \expandafter \color \ctxlua{context.direct({"green"},"green")} black
\stopbuffer

\typebuffer

The \type {\expandafter} makes sure that the \type {\bold} and \type {\color}
macros see the following \type{{bold}}, \type {[red]}, and \type {[green]{green}}
arguments.

\startlines\getbuffer\stoplines

\stopsummary

\startsummary[title={context.delayed(...)}]

The arguments are interpreted the same as in a \type {context} call, but instead
of a direct flush, the arguments will be flushed in a next cycle.

\stopsummary

\startsummary[title={context.delayed["command"](...)}]

The arguments are interpreted the same as in a \type {command} call, but instead
of a direct flush, the command and arguments will be flushed in a next cycle.

\stopsummary

\startsummary[title={context.nested["command"]}]

This command returns the command, including given arguments as a string. No
flushing takes place.

\stopsummary

\startsummary[title={context.nested}]

This command returns the arguments as a string and treats them the same as a
regular \type {context} call.

\stopsummary

\startsummary[title={context.formatted["command"]([<regime>,]<format>,<arguments>)}]

This command returns the command that will pass it's arguments to the string
formatter. When the first argument is a number, then it is interpreted as a
catcode regime.

\stopsummary

\startsummary[title={context.formatted([<regime>,]<format>,<arguments>)}]

This command passes it's arguments to the string formatter. When the first
argument is a number, then it is interpreted as a catcode regime.

\stopsummary

\stopsection

\startsection[title=\METAFUN]

\startsummary[title={context.metafun.start()}]

This starts a \METAFUN\ (or \METAPOST) graphic.

\stopsummary

\startsummary[title={context.metafun.stop()}]

This finishes and flushes a \METAFUN\ (or \METAPOST) graphic.

\stopsummary

\startsummary[title={context.metafun("format",...)}]

The argument is appended to the current graphic data but the string formatter is
used on following arguments.

\stopsummary

\startsummary[title={context.metafun.delayed}]

This namespace does the same as \type {context.delayed}: it wraps the code in such
a way that it can be used in a function call.

\stopsummary

\startsection[title=Building blocks]

\startsummary[title={context.bgroup() context.egroup()}]

These are just \type {\bgroup} and \type {\egroup} equivalents and as these are
in fact shortcuts to the curly braced we output these instead.

\stopsummary

\startsummary[title={context.space()}]

This one directly maps onto \type {\space}.

\stopsummary

\startsummary[title={context.par()}]

This one directly maps onto \type {\par}.

\stopsummary

\stopsection

\startsection[title=Basic Helpers]

\startsummary[title={context.rule(wd,ht,dp,direction) context.rule(specification)}]

A rule node is injected with the given properties. A specification is just a
table with the four fields. The rule gets the current attributes.

\stopsummary

\startsummary[title={context.glyph(fontid,n) context.glyph(n)}]

A glyph node is injected with the given font id. When no id is given, the current font
is used. The glyph gets the current attributes.

\stopsummary

\startsummary[title={context.char(n) context.char(str)  context.char(tab)}]

This will inject one or more copies of \type {\char} calls. You can pass a
number, a string representing a number, or a table with numbers.

\stopsummary

\startsummary[title={context.utfchar(n) context.utfchar(str)}]

This injects is \UTF\ character (one or more bytes). You can pass a number
or a string representing a numbers. You need to be aware of special
characters in \TEX, like \type {#}.

\stopsummary

\stopsection

\startsection[title={Registers}]

This is a table that hosts a couple of functions. The following \type {new}
ones are available:

\starttyping
local n = newdimen (name)
local n = newskip  (name)
local n = newcount (name)
local n = newmuskip(name)
local n = newtoks  (name)
local n = newbox   (name)
\stoptyping

These define a register with name \type {name} at the \LUA\ end and \type {\name}
at the \TEX\ end. The registers' number is returned. The next function is like
\type {\chardef}: it defines \type {\name} with value \type {n}.

\starttyping
local n = newchar(name,n)
\stoptyping

It's not likely that you will use any of these commands, if only because when
you're operating from the \LUA\ end using \LUA\ variables is more convenient.

\stopsection

\startsection[title=Catcodes]

Normally we operate under the so called \type {context} catcode regime. This
means that content gets piped to \TEX\ using the same meanings for characters as
you normally use in \CONTEXT. So, a \type {$} starts math. In \in {table}
[tab:catcodes] we show the catcode regimes.

\startplacetable[location=page,title={Catcode regimes},reference=tab:catcodes]
    \showcharactercatcodes
\stopplacetable

\startsummary[title={context.catcodes}]

The \type {context.catcodes} tables contains the internal numbers of the
catcode tables used. The next table shows the names that can be used.

\starttabulate[|lT|cT|lT|]
\BC name     \BC mnemonic \BC \TEX\ command \NC \NR
\NC context  \NC ctx      \NC ctxcatcodes   \NC \NR
\NC protect  \NC prt      \NC prtcatcodes   \NC \NR
\NC plain    \NC tex      \NC texcatcodes   \NC \NR
\NC text     \NC txt      \NC txtcatcodes   \NC \NR
\NC verbatim \NC vrb      \NC vrbcatcodes   \NC \NR
\stoptabulate

\stopsummary

\startsummary[title={context.newindexer(catcodeindex)}]

This function defines a new indexer. You can think of the context command itself
as an indexer. There are two (extra) predefined indexers:

\starttyping
context.verbatim = context.newindexer(context.catcodes.verbatim)
context.puretext = context.newindexer(context.catcodes.text)
\stoptyping

\stopsummary

\startsummary[title={context.pushcatcodes(n) context.popcatcodes()}]

These commands switch to another catcode regime and back. They have to be used
in pairs. Only the regimes atthe \LUA\ end are set.

\stopsummary

\startsummary[title={context.unprotect() context.protect()}]

These commands switch to the protected regime and back. They have to be used in
pairs. Beware: contrary to what its name suggests, the \type {unprotect} enables
the protected regime. These functions also issue an \type {\unprotect} and \type
{\protect} equivalent at the \TEX\ end.

\stopsummary

\startsummary[title={context.verbatim context.puretext}]

The differences between these are subtle:

\startbuffer
\startluacode
    context.verbatim.bold("Why do we use $ for math?") context.par()
    context.verbatim.bold("Why do we use { as start?") context.par()
    context.verbatim.bold("Why do we use } as end?")   context.par()
    context.puretext.bold("Why do we use {\\bi $} at all?")
\stopluacode
\stopbuffer

\typebuffer

Verbatim makes all characters letters while pure text leaves the backslash and
curly braces special.

\startpacked \getbuffer \stoppacked

\stopsummary

\startsummary[title={context.protected}]

The protected namespace is only used for commands that are in the \CONTEXT\
private namespace.

\stopsummary

\startsummary[title={context.escaped(str) context.escape(str)}]

The first command pipes the escaped string to \TEX, while the second one just
returns an unescaped string. The characters \typ {# $ % \ \ { }} are escaped.

\stopsummary

\startsummary[title={context.startcollecting() context.stopcollecting()}]

These two commands will turn flushing to \TEX\ into collecting. This can be handy
when you want to interface commands that grab arguments using delimiters and as
such they are used deep down in some table related interfacing. You probably
don't need them.

\stopsummary

\stopsection

\startsection[title={Templates}]

In addition to the regular template mechanism (part of the utilities) there is a
dedicated template feature in the \type {context} namespace. An example demonstrates
its working:

\startbuffer
\startluacode
  local MyTable = [[
    \bTABLE
      \bTR
         \bTD \bf %one_first% \eTD
         \bTD %[one_second]% \eTD
      \eTR
      \bTR
        \bTD \bf %two_first% \eTD
        \bTD %[two_second]% \eTD
      \eTR
    \eTABLE
  ]]

  context.templates[MyTable] {
    one_first  = "one",
    two_first  = "two",
    one_second = "just one $",
    two_second = "just two $",
  }
\stopluacode
\stopbuffer

\typebuffer

This renders:

\startlinecorrection
    \getbuffer
\stoplinecorrection

You can also use more complex tables. Watch the space before and after the keys:

\startbuffer
\startluacode
  local MyOtherTable = [[
    \bTABLE
      \bTR
        \bTD \bf % ['one']['first'] % \eTD
        \bTD %[ ['one']['second'] ]% \eTD
      \eTR
      \bTR
        \bTD \bf % ['two']['first'] % \eTD
        \bTD %[ ['two']['second'] ]% \eTD
      \eTR
    \eTABLE
  ]]

  local data = {
    one = { first = "one", second = "only 1$" },
    two = { first = "two", second = "only 2$" },
  }

  context.templates[MyOtherTable](data)

  context.templates(MyOtherTable,data)
\stopluacode
\stopbuffer

\typebuffer

We get:

\startlinecorrection
    \getbuffer
\stoplinecorrection

\stopsection

\startsection[title={Management}]

\startsummary[title={context.functions}]

This is private table that hosts managament of functions. You'd better leave this
one alone!

\stopsummary

\startsummary[title={context.nodes}]

Normally you will just use \type {context(<somenode>)} to flush a node and this
private table is more for internal use.

\stopsummary

\stopsection

\startsection[title=String handlers]

These two functions implement handlers that split a given string into lines and
do something with it. We stick to showing their call. They are used for special
purpose flushing, like flushing content to \TEX\ in commands discussed here. The
\XML\ subsystem also used a couple of dedicated handlers.

\starttyping
local foo = newtexthandler {
    content    = function(s) ... end,
    endofline  = function(s) ... end,
    emptyline  = function(s) ... end,
    simpleline = function(s) ... end,
}

local foo = newverbosehandler {
    line    = function(s) ... end,
    space   = function(s) ... end,
    content = function(s) ... end,
    before  = function() ... end,
    after   = function() ... end,
}
\stoptyping

\startsummary[title={context.printlines(str)}]

The low level \type {tex.print} function pipes its content to \TEX\ and thereby
terminates at at \type {\r} (cariage return, \ASCII\ 13), although it depends on
the way catcodes and line endings are set up. In fact, a line ending in \TEX\ is
not really one, as it gets replaced by a space. Only several lines in succession
indicate a new paragraph.

\startbuffer
\startluacode
    tex.print("line 1\n line 2\r line 3")
\stopluacode
\stopbuffer

\typebuffer

This renders only two lines:

\getbuffer

\startbuffer
\startluacode
    context("line 1\n line 2\r line 3")
\stopluacode
\stopbuffer

However, the \type {context} command gives all three lines:

\typebuffer

Like:

\getbuffer

The \type {context.printlines} command is a direct way to print a string in a way
similar to reading from a file. So,

\starttyping
tex.print(io.loaddata(resolvers.findfile("tufte")))
\stoptyping

Gives one line, while:

\starttyping
context.printlines(io.loaddata(resolvers.findfile("tufte")))
\stoptyping

gives them all, as does:

\starttyping
context(io.loaddata(resolvers.findfile("tufte")))
\stoptyping

as does a na\"ive:

\starttyping
tex.print((string.gsub(io.loaddata(resolvers.findfile("tufte")),"\r","\n")))
\stoptyping

But, because successive lines need to become paragraph separators as bit more
work is needed and that is what \type {printlines} and \type {context} do for
you. However, a more convenient alternative is presented next.

\stopsummary

\startsummary[title={context.loadfile(name)}]

This function locates and loads the file with the given name. The leading and
trailing spaces are stripped.

\stopsummary

\startsummary[title={context.runfile(name)}]

This function locates and processes the file with the given name. The assumption
is that it is a valid \LUA\ file! When no suffix is given, the suffix \type {cld}
(\CONTEXT\ \LUA\ document) is used.

\stopsummary

\startsummary[title={context.viafile(data[,tag])}]

The \type {data} is saved to a (pseudo) file with the optional name \type {tag}
and read in again from that file. This is a robust way to make sure that the data
gets processed like any other data read from file. It permits all kind of
juggling with catcodes, verbatim and alike.

\stopsummary

\stopsection

\startsection[title={Helpers}]

\startsummary[title={context.tocontext(variable)}]

For documentation or tracing it can be handy to serialize a variable. The \type
{tocontext} function does this:

\starttyping
context.tocontext(true)
context.tocontext(123)
context.tocontext("foo")
context.tocontext(tonumber)
context.tocontext(nil)
context.tocontext({ "foo", "bar" },true)
context.tocontext({ this = { foo , "bar" } },true)
\stoptyping

Beware, \type {tocontext} is also a table that you can assign to, but that might
spoil serialization. This property makes it possible to extend the serializer.

\stopsummary

\startsummary[title={context.tobuffer(name,str[,catcodes])}]

With this function you can put content in a buffer, optionally under a catcode
regime.

\stopsummary

\startsummary[title={context.tolines(str[,true])}]

This function splits the string in lines and flushes them one by one. When the
second argument is \type {true} leading and trailing spaces are stripped. Each
flushed line always gets one space appended.

\stopsummary

\startsummary[title={context.fprint([regime,]fmt,...),tex.fprint([regime,]fmt,...)}]

The \type {tex.fprint} is just there to complement the other flushers in the
\type {tex} namespace and therefore we also have it in the \type {context}
namespace.

\stopsummary

\stopsection

\startsection[title=Tracing]

\startsummary[title={context.settracing(true or false))}]

You can trace the \TEX\ code that is generated at the \TEX\ end with:

\starttyping
\enabletrackers[context.trace]
\stoptyping

The \LUA\ function sets the tracing from the \LUA\ end. As the \type {context}
command is used a lot in the core, you can expect some more tracing that the code
that you're currently checking.

\stopsummary

\startsummary[title={context.pushlogger(fnc) context.poplogger() context.getlogger()}]

You can provide your own logger if needed. The pushed function receives one string
argument. The getter returns three functions:

\starttyping
local flush, writer, flushdirect = context.getlogger()
\stoptyping

The \type{flush} function is similar to \type {tex.sprint} and appends its
arguments, while \type {flushdirect} treats each argument as a line and behaves
like \type {tex.print}. The \type {flush} function adds braces and paranthesis
around its arguments, apartt from the first one, which is considered to be a
command. Examples are:

\starttyping
flush("one",2,"three") -- catcode, strings|numbers
writer("\\color",{"red"},"this is red")
\stoptyping

and:

\starttyping
flush(context.catcodes.verbatim,"one",2,"three")
writer(context.catcodes.verbatim,"\\color",{"red"},"this is red")
\stoptyping

\stopsummary

\stopsection

\startsection[title=States]

There are several ways to implement alternative code paths in \CONTEXT\ but modes
and conditionals are used mostly. There area few helpers for that.

\startsummary[title={context.conditionals context.setconditional(name,value)}]

Conditionals are used to keep a state. You can set their value using the setter,
but their effect is not immediate but part of the current sequence of commands
which is delegated to \TEX. However, you can easily keep track of your state
at the \LUA\ end with an extra boolean. So, after

\starttyping
if context.conditionals.whatever then
    context.setconditional("dothis",false)
else
    context.setconditional("dothat",true)
end
\stoptyping

the value of \type {dothis} and \type {dothat} conditions are not yet set in
\LUA.

\stopsummary

\startsummary[title={context.modes context.setmode(name,value)}]

As with conditionals, you can (re)set the modes in \LUA\ but their values
get changes as part of the command sequence which is delayed till after the
\LUA\ call.

\stopsummary

\startsummary[title={context.systemmodes context.setsystemmode(name,value)}]

The same applies as for regular modes.

\stopsummary

\startsummary[title={context.trialtypesetting()}]

This function returns \type {true} if we're in trial typesetting mode (used when
for instance prerolling a table).

\stopsummary

\stopsection

\startsection[title=Steps]

The stepper permits stepwise processing of \CONTEXT\ code: after a step contyrol
gets delegated to \CONTEXT\ and afterwards back to \LUA. There main limitation of
this mechanism is that it cannot exceed the number of input levels.

\startsummary[title={context.stepwise() context.step([str])}]

Usage is as follows:

\starttyping
context.stepwise (function()
    ...
    context.step(...)
    ...
    context.step(...)
    ...
    context.stepwise (function()
        ...
        context.step(...)
        ...
    context.step(...)
    ...
    end)
    ...
    context.step(...)
    ...
    context.step(...)
    ...
end)
\stoptyping

\stopsummary

\stopsection

\stopchapter

\stopcomponent
