% language=uk

\startcomponent cld-introduction

\environment cld-environment

\startchapter[title=Introduction]

Sometimes you hear folks complain about the \TEX\ input language, i.e.\ the
backslashed commands that determine your output. Of course, when alternatives are
being discussed every one has a favourite programming language. In practice
coding a document in each of them triggers similar sentiments with regards to
coding as \TEX\ itself does.

So, just for fun, I added a couple of commands to \CONTEXT\ \MKIV\ that permit
coding a document in \LUA. In retrospect it has been surprisingly easy to
implement a feature like this using metatables. Of course it's a bit slower than
using \TEX\ as input language but sometimes the \LUA\ interface is more readable
given the problem at hand.

After a while I decided to use that interface in non|-|critical core \CONTEXT\
code and in styles (modules) and solutions for projects. Using the \LUA\ approach
is sometimes more convenient, especially if the code mostly manipulates data. For
instance, if you process \XML\ files of database output you can use the interface
that is available at the \TEX\ end, or you can use \LUA\ code to do the work, or
you can use a combination. So, from now on, in \CONTEXT\ you can code your style
and document source in (a mixture of) \TEX, \XML, \METAPOST\ and in \LUA.

In the following chapters I will introduce typesetting in \LUA, but as we rely on
\CONTEXT\ it is unavoidable that some regular \CONTEXT\ code shows up. The fact
that you can ignore backslashes does not mean that you can do without knowledge
of the underlying system. I expect that the user is somewhat familiar with this
macro package. Some chapters are follow ups on articles or earlier publications.

Some information (and mechanism) show up in more than one chapter. This is a side
effect of \LUA\ being integrated in many places, so an isolated discussion is a
bit hard.

In the meantime most of the code is rather stable and proven. However, this
manual will never be complete. You can find examples all over the code base, and
duplicating everything here makes no sense. If you find errors, please let me
know. If you think that something is missing, you can try to convince me to add
it. It's hard to keep up with what gets added so input is welcome.

\blank[2*big]

\startlines
Hans Hagen
Hasselt NL
2009 \emdash\ 2016
\stoplines

\stopchapter

\stopcomponent
