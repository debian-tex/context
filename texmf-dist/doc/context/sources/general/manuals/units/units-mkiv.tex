% language=uk

% author    : Hans Hagen
% copyright : PRAGMA ADE & ConTeXt Development Team
% license   : Creative Commons Attribution ShareAlike 4.0 International
% reference : pragma-ade.nl | contextgarden.net | texlive (related) distributions
% origin    : the ConTeXt distribution
%
% comment   : Because this manual is distributed with TeX distributions it comes with a rather
%             liberal license. We try to adapt these documents to upgrades in the (sub)systems
%             that they describe. Using parts of the content otherwise can therefore conflict
%             with existing functionality and we cannot be held responsible for that. Many of
%             the manuals contain characteristic graphics and personal notes or examples that
%             make no sense when used out-of-context.
%
% comment   : Some chapters might have been published in TugBoat, the NTG Maps, the ConTeXt
%             Group journal or otherwise. Thanks to the editors for corrections. Also thanks
%             to users for testing, feedback and corrections.

\usemodule[art-01,abr-02,physics-units]

\definecolor[red]    [darkred]
\definecolor[green]  [darkgreen]
\definecolor[blue]   [darkblue]
\definecolor[yellow] [darkyellow]
\definecolor[magenta][darkmagenta]
\definecolor[cyan]   [darkcyan]

\definecolor[maincolor] [darkcyan]
\definecolor[extracolor][darkmagenta]

\setupbodyfont
  [10pt]

\setuptyping
  [color=extracolor]

\setuptype
  [color=extracolor] % darkyellow

% \setupnumbering
%   [alternative=doublesided]

\setupinteraction
  [hidden]

\setuphead
  [section]
  [color=maincolor]

\usesymbols[cc]

\startdocument
  [metadata:author=Hans Hagen,
   metadata:title=Units,
   author=Hans Hagen,
   affiliation=PRAGMA ADE,
   location=Hasselt NL,
   title=Units,
   extra=ConTeXt MkIV,
   support=www.contextgarden.net,
   website=www.pragma-ade.nl]

\unexpanded\def\UnitsHack#1#2%
  {\setbox\scratchbox\hbox{\bf\strut#1#2/}% kerning and such
   \hbox to \wd\scratchbox{\bf\strut#1\hss/\hss#2}}

\startMPpage

    StartPage ;
        fill Page enlarged 2mm withcolor \MPcolor{darkcyan} ;
        pickup pencircle scaled 2mm ;
        picture p, q, r ;
        p := textext("\ssbf\WORD{\documentvariable{title}}")  xsized (bbheight Page - 2cm) rotated 90 ;
        q := textext("\ssbf\WORD{\documentvariable{author}}") ysized 1cm ;
        r := textext("\ssbf\WORD{\documentvariable{extra}}")  xsized bbwidth q ;
        draw anchored.rt (p, center rightboundary Page shifted (-1cm,   0mm)) withcolor white ;
        draw anchored.lft(q, ulcorner             Page shifted ( 1cm, -84mm)) withcolor white ; % \MPcolor{darkred} ;
        draw anchored.lft(r, ulcorner             Page shifted ( 1cm,-108mm)) withcolor white ; % \MPcolor{darkred} ;
    StopPage ;

\stopMPpage

% \page[empty] \setuppagenumber[start=1]

\startsubject[title={Contents}]

\placelist[section][criterium=all,interaction=all]

\stopsubject

\startsection[title={Introduction}]

In \CONTEXT\ \MKII\ there is a module that implements consistent
typesetting of units (quantities and dimensions). In \MKIV\ this
functionality is now part of the physics core modules. This is
also one of the mechanisms that got a new user interface: instead
of using commands we now parse text. Thanks to those users who
provided input we're more complete now that in \MKII. You can browse
the mailing list archive to get some sense of history.

\stopsection

\startsection[title={The main command}]

The core command is \type {\unit}. The argument to this command gets
parsed and converted into a properly typeset dimension. Normally there
will be a quantity in front.

\starttabulate
\NC \type{10 meter}                   \NC \unit{10 meter}                   \NC \NR
\NC \type{10 meter per second}        \NC \unit{10 meter per second}        \NC \NR
\NC \type{10 square meter per second} \NC \unit{10 square meter per second} \NC \NR
\stoptabulate

The parser knows about special cases, like synonyms:

\starttabulate
\NC \type{10 degree celsius}  \NC \unit{10 degree celsius}  \NC \NR
\NC \type{10 degrees celsius} \NC \unit{10 degrees celsius} \NC \NR
\NC \type{10 celsius}         \NC \unit{10 celsius}         \NC \NR
\stoptabulate

The units can be rather complex, for example:

\startbuffer
\unit{30 kilo pascal square meter / second kelvin}
\stopbuffer

\typebuffer

This comes out as: \ignorespaces\getbuffer\removeunwantedspaces. Depending
on the unit at had, recognition is quite flexible. The following variants
all work out ok.

\starttabulate
\NC \type{10 kilogram}  \NC \unit{10 kilogram}  \NC \NR
\NC \type{10 kilo gram} \NC \unit{10 kilo gram} \NC \NR
\NC \type{10 k gram}    \NC \unit{10 k gram}    \NC \NR
\NC \type{10 kilo g}    \NC \unit{10 kilo g}    \NC \NR
\NC \type{10 k g}       \NC \unit{10 k g}       \NC \NR
\NC \type{10 kg}        \NC \unit{10 kg}        \NC \NR
\NC \type{10 kilog}     \NC \unit{10 kilog}     \NC \NR
\NC \type{10 kgram}     \NC \unit{10 kgram}     \NC \NR
\stoptabulate

Of course being consistent makes sense, so normally you will use
a consistent mix of short or long keywords.

You can provide a qualifier that gets lowered and appended to
the preceding unit.

\startbuffer
\unit{112 decibel (A)}
\stopbuffer

\typebuffer

This gives: \ignorespaces\getbuffer\removeunwantedspaces. Combinations
are also possible:

\starttabulate
\NC \type{5 watt per meter celsius}         \NC \unit{5 watt per meter celsius}         \NC \NR
\NC \type{5 watt per meter degrees celsius} \NC \unit{5 watt per meter degrees celsius} \NC \NR
\NC \type{5 watt per meter kelvin}          \NC \unit{5 watt per meter kelvin}          \NC \NR
\NC \type{5 watt per meter per kelvin}      \NC \unit{5 watt per meter per kelvin}      \NC \NR
\NC \type{10 arcminute}                     \NC \unit{10 arcminute}                     \NC \NR
\NC \type{10 arcminute 20 arcsecond}        \NC \unit{10 arcminute 20 arcsecond}        \NC \NR
\stoptabulate

\stopsection

\startsection[title={Extra units}]

To some extent units can be tuned. You can for instance
influence the spacing between a number and a unit:

\startbuffer
                         \unit{35 kilogram per cubic meter}
\setupunit[space=normal] \unit{35 kilogram per cubic meter}
\setupunit[space=big]    \unit{35 kilogram per cubic meter}
\setupunit[space=medium] \unit{35 kilogram per cubic meter}
\setupunit[space=small]  \unit{35 kilogram per cubic meter}
\setupunit[space=none]   \unit{35 kilogram per cubic meter}
\stopbuffer

\typebuffer

Of course no spacing looks rather bad:

\startlines
\getbuffer
\stoplines

Another parameter is \type {separator}. In order to demonstrate
this we define an extra unit command:

\startbuffer
\defineunit[sunit][separator=small]
\defineunit[nunit][separator=none]
\stopbuffer

\typebuffer \getbuffer

We now have two more commands:

\startbuffer
\unit {35 kilogram cubic meter}
\sunit{35 kilogram cubic meter}
\nunit{35 kilogram cubic meter}
\stopbuffer

\typebuffer

These three commands give different results:

\startlines
\getbuffer
\stoplines

Valid separators are \type {normal}, \type {big}, \type {medium},
\type {small}, \type {none}. You can let units stand out by
applying color or a specific style.

\startbuffer
\setupunit[style=\bi,color=maincolor]
\unit{10 square meter per second}
\stopbuffer

\typebuffer

Keep in mind that all defined units inherit from their parent
definition unless they are set up themselves.

\start \blank \getbuffer \blank \stop

To some extent you can control rendering in text and math mode. As
an example we define an extra instance.

\startbuffer
\defineunit[textunit][alternative=text]
\stopbuffer

\typebuffer \getbuffer

\startbuffer
test  \unit    {10 cubic meter per second}  test
test  \textunit{10 cubic meter per second}  test
test $\unit    {10 cubic meter per second}$ test
test $\textunit{10 cubic meter per second}$ test
test  10 \unit    {cubic meter per second}  test
test  10 \textunit{cubic meter per second}  test
test $10 \unit    {cubic meter per second}$ test
test $10 \textunit{cubic meter per second}$ test
\stopbuffer

\typebuffer

\startlines
\getbuffer
\stoplines

\stopsection

\startsection[title={Labels}]

The units, prefixes and operators are typeset using the label
mechanism which means that they can be made to adapt to a language
and|/|or adapted. Instead of language specific labels you can also
introduce mappings that don't relate to a language at all. As an
example we define some bogus mapping.

\startbuffer
\setupunittext
  [whatever]
  [meter=retem,
   second=dnoces]

\setupprefixtext
  [whatever]
  [kilo=olik]

\setupoperatortext
  [whatever]
  [solidus={ rep }]
\stopbuffer

\typebuffer \getbuffer

Such a mapping can be partial and the current language will
be the default fallback and itselfs falls back on the English
language mapping.

\startbuffer
\unit{10 km/s}
\unit{10 Kilo Meter/s}
\unit{10 kilo Meter/s}
\unit{10 Kilo m/s}
\unit{10 k Meter/s}
\stopbuffer

\typebuffer

When we typeset this we get the normal rendering:

\startlines
\getbuffer
\stoplines

However, when we change the language parameter, we get
a different result:

\startlines
\setupunit[language=whatever]\getbuffer
\stoplines

The alternative rendering is set up as follows:

\starttyping
\setupunit[language=whatever]
\stoptyping

You can also decide to use a special instance of units:

\starttyping
\defineunit[wunit][language=whatever]
\stoptyping

This will define the \type {\wunit} command and leave the original
\type {\unit} command untouched.

\stopsection

\startsection[title={Digits}]

In addition to units we have digits. These can be used independently
but the same functionality is also integrated in the unit commands.
The main purpose of this command is formatting in tables, of which
we give an example below.

\starttabulate[|l|r|]
\NC \type{12,345.67 kilogram} \NC \unit{12,345.67 kilogram} \NR
\NC \type{__,__1.23 kilogram} \NC \unit{__,__1.23 kilogram} \NR
\NC \type{__,___.12 kilogram} \NC \unit{__,___.12 kilogram} \NR
\NC \type{__,__1.== kilogram} \NC \unit{__,__1.== kilogram} \NR
\NC \type{__,___:23 kilogram} \NC \unit{__,___:23 kilogram} \NR
\stoptabulate

The \type {_} character serves as placeholders. There are some
assumptions to how numbers are constructed. In principe the input
assumes a comma to separate thousands and a period to separate the
fraction.

\getbuffer

You can swap periods and commas in the output. In fact there are a
few methods available. For instance we can separate the thousands
with a small space instead of a symbol.

\startbuffer
\starttabulate[|c|r|r|]
\HL
\NC 0 \NC \setupunit[method=0]\unit{00,000.10 kilogram}
      \NC \setupunit[method=0]\unit{@@,@@0.10 kilogram} \NC \NR
\NC 1 \NC \setupunit[method=1]\unit{00,000.10 kilogram}
      \NC \setupunit[method=1]\unit{@@,@@0.10 kilogram} \NC \NR
\NC 2 \NC \setupunit[method=2]\unit{00,000.10 kilogram}
      \NC \setupunit[method=2]\unit{@@,@@0.10 kilogram} \NC \NR
\NC 3 \NC \setupunit[method=3]\unit{00,000.10 kilogram}
      \NC \setupunit[method=3]\unit{@@,@@0.10 kilogram} \NC \NR
\NC 4 \NC \setupunit[method=4]\unit{00,000.10 kilogram}
      \NC \setupunit[method=4]\unit{@@,@@0.10 kilogram} \NC \NR
\NC 5 \NC \setupunit[method=5]\unit{00,000.10 kilogram}
      \NC \setupunit[method=5]\unit{@@,@@0.10 kilogram} \NC \NR
\NC 6 \NC \setupunit[method=6]\unit{00,000.10 kilogram}
      \NC \setupunit[method=6]\unit{@@,@@0.10 kilogram} \NC \NR
\HL
\stoptabulate
\stopbuffer

\typebuffer % [bodyfont=9pt]

\getbuffer

The digit modes can be summarized as::

\startitemize[n,packed]
\item periods/comma
\item commas/period
\item thinmuskips/comma
\item thinmuskips/period
\item thickmuskips/comma
\item thickmuskips/period
\stopitemize

You can reverse the order of commas and period in the input by
setting the parameter \type {order} to \type {reverse}.

The digit parser handles a bunch of special characters as
well as different formats. We strongly suggest you to use
the grouped call.

\starttabulate[|l|l|l|]
\NC \type{.}  \NC , .           \NC comma or period     \NC \NR
\NC \type{,}  \NC , .           \NC comma or period     \NC \NR
\NC \type{:}  \NC               \NC invisible period    \NC \NR
\NC \type{;}  \NC               \NC invisible comma     \NC \NR
\NC \type{_}  \NC               \NC invisible space     \NC \NR
\NC \type{/}  \NC               \NC invisible sign      \NC \NR
\NC \type{-}  \NC $-$           \NC minus sign          \NC \NR
\NC \type{+}  \NC $+$           \NC plus sign           \NC \NR
\NC \type{//} \NC               \NC invisible high sign \NC \NR
\NC \type{--} \NC $\negative$   \NC high minus sign     \NC \NR
\NC \type{++} \NC $\positive$   \NC high plus sign      \NC \NR
\NC \type{=}  \NC $\zeroamount$ \NC zero padding        \NC \NR
\stoptabulate

Let's give some examples:

\starttabulate[|l|r|]
\NC \type{1}                      \NC \ruledhbox{\strut\digits{1}}                      \NC \NR
\NC \type{12}                     \NC \ruledhbox{\strut\digits{12}}                     \NC \NR
\NC \type{12.34}                  \NC \ruledhbox{\strut\digits{12.34}}                  \NC \NR
\NC \type{123,456}                \NC \ruledhbox{\strut\digits{123,456}}                \NC \NR
\NC \type{123,456.78}             \NC \ruledhbox{\strut\digits{123,456.78}}             \NC \NR
\NC \type{12,34}                  \NC \ruledhbox{\strut\digits{12,34}}                  \NC \NR
\NC \type{.1234}                  \NC \ruledhbox{\strut\digits{.1234}}                  \NC \NR
\NC \type{1234}                   \NC \ruledhbox{\strut\digits{1234}}                   \NC \NR
\NC \type{123,456.78^9}           \NC \ruledhbox{\strut\digits{123,456.78^9}}           \NC \NR
\NC \type{123,456.78e9}           \NC \ruledhbox{\strut\digits{123,456.78e9}}           \NC \NR
\NC \type{/123,456.78e-9}         \NC \ruledhbox{\strut\digits{/123,456.78e-9}}         \NC \NR
\NC \type{-123,456.78e-9}         \NC \ruledhbox{\strut\digits{-123,456.78e-9}}         \NC \NR
\NC \type{+123,456.78e-9}         \NC \ruledhbox{\strut\digits{+123,456.78e-9}}         \NC \NR
\NC \type{//123,456.78e-9}        \NC \ruledhbox{\strut\digits{//123,456.78e-9}}        \NC \NR
\NC \type{--123,456.78e-9}        \NC \ruledhbox{\strut\digits{--123,456.78e-9}}        \NC \NR
\NC \type{++123,456.78e-9}        \NC \ruledhbox{\strut\digits{++123,456.78e-9}}        \NC \NR
\NC \type{___,___,123,456,789.00} \NC \ruledhbox{\strut\digits{___,___,123,456,789.00}} \NC \NR
\NC \type{___,___,_12,345,678.==} \NC \ruledhbox{\strut\digits{___,___,_12,345,678.==}} \NC \NR
\stoptabulate

\stopsection

\startsection[title={Adding units}]

It is possible to add extra snippets. This is a two step process:
first some snippet is defined, next a proper label is set up. In the
next example we define a couple of \TEX\ dimensions:

\startbuffer
\registerunit
  [unit]
  [point=point,
   basepoint=basepoint,
   scaledpoint=scaledpoint,
   didot=didot,
   cicero=cicero]
\stopbuffer

\typebuffer \getbuffer

Possible categories are: \type {prefix}, \type {unit}, \type {operator},
\type {suffix}, \type {symbol},\type {packaged}. Next we define labels:

\startbuffer
\setupunittext
  [point=pt,
   basepoint=bp,
   scaledpoint=sp,
   didot=dd,
   cicero=cc]
\stopbuffer

\typebuffer \getbuffer

Now we can use use these:

\startbuffer
\unit{10 point / second}
\stopbuffer

\typebuffer

Of course you can wonder what this means.

\blank \getbuffer \blank

When no label is defined the long name is used:

\startbuffer
\registerunit
  [unit]
  [page=page]
\stopbuffer

\typebuffer \getbuffer

This is used as:

\startbuffer
\unit{10 point / page}
\stopbuffer

\typebuffer

Which gives:

\blank \getbuffer \blank

\stopsection

\startsection[title={Goodies}]

Here are some goodies:

\startbuffer
\type{1 } : \unit  {30^2            meter per second}
\type{2a} : \unit  {30    ± 10      meter per second}
\type{2b} : \unit  {30    – 10      meter per second}
\type{3a} : \unit  {30^2  ± 10^2    meter per second}
\type{3b} : \unit  {30^2  – 10^2    meter per second}
\type{4 } : \unit  {30   (10)       meter per second}
\type{5a} : \unit  {30   (± 10)     meter per second}
\type{5b} : \unit  {30^2 (± 10^2)   meter per second}
\type{6a} : \unit {(30    ± 10)     meter per second}
\type{6b} : \unit {(30^2  ± 10^2)   meter per second}
\type{6c} : \unit {(30^2  – 10^2)   meter per second}
\type{7a} : \unit {(30    ± 10)^2   meter per second}
\type{7b} : \unit {(30    – 10)^2   meter per second}
\type{7c} : \unit {(30^2  – 10^3)^5 meter per second}
\stopbuffer

Instead of \type {±} one can use \type {pm} and \type {to} can be used instead of
\type {–}.

\typebuffer

Their rendering explains their intention:

\startlines
\getbuffer
\stoplines

\stopsection

\startsection[title={Built in keywords}]

A given sequence of keywords is translated in an list of internal
keywords. For instance \type {m}, \type {Meter} and \type {meter}
all become \type {meter} and that one is used when resolving a
label. In the next tables the right column mentions the internal
keyword. The right column shows the Cased variant, but a lowercase
one is built|-|in as well.

The following prefixes are built|-|in:

\showunits[prefixes]

The following units are supported, including some combinations:

\showunits[units]

The amount of operators is small:

\showunits[operators]

There is also a small set of (names) suffixes:

\showunits[suffixes]

Some symbols get a special treatment:

\showunits[symbols]

These are also special:

\showunits[packaged]

\stopsection

\startsection[title={Colofon}]

\starttabulate[|B|p|]
\NC author    \NC \documentvariable{author}, \documentvariable{affiliation}, \documentvariable{location} \NC \NR
\NC version   \NC \currentdate \NC \NR
\NC website   \NC \documentvariable{website} \endash\ \documentvariable{support} \NC \NR
\NC copyright \NC \symbol[cc][cc-by-sa-nc] \NC \NR
\stoptabulate

\stopsection

\stopdocument
