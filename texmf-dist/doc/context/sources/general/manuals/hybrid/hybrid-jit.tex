% language=uk engine=luatex

\startcomponent hybrid-backends

\environment hybrid-environment

\logo[SWIGLIB]  {SwigLib}
\logo[LUAJIT]   {LuaJIT}
\logo[LUAJITTEX]{Luajit\TeX}
\logo[JIT]      {jit}

\startchapter[title={Just in time}]

\startsection [title={Introduction}]

Reading occasional announcements about \LUAJIT,\footnote {\LUAJIT\ is written by
Mike Pall and more information about it and the technology it uses is at \type
{http://luajit.org}, a site also worth visiting for its clean design.} one starts
wondering if just||in||time compilation can speed up \LUATEX. As a side track of
the \SWIGLIB\ project and after some discussion, Luigi Scarso decided to compile
a version of \LUATEX\ that had the \JIT\ compiler as the \LUA\ engine. That's
when our journey into \JIT\ began.

We started with \LINUX\ 32-bit as this is what Luigi used at that time. Some
quick first tests indicated that the \LUAJIT\ compiler made \CONTEXT\ \MKIV\ run
faster but not that much. Because \LUAJIT\ claims to be much faster than stock
\LUA, Luigi then played a bit with \type {ffi}, i.e.\ mixing \CCODE\ and \LUA,
especially data structures. There is indeed quite some speed to gain here;
unfortunately, we would have to mess up the \CONTEXT\ code base so much that one
might wonder why \LUA\ was used in the first place. I could confirm these
observations in a Xubuntu virtual machine in \VMWARE\ running under 32-bit
Windows 8. So, we decided to conduct some more experiments.

A next step was to create a 64-bit binary because the servers at \PRAGMA\ are
\KVM\ virtual machines running a 64-bit OpenSuse 12.1 and 12.2. It took a bit of
effort to get a \JIT\ version compiled because Luigi didn't want to mess up the
regular codebase too much. This time we observed a speedup of about 40\% on some
runs so we decided to move on to \WINDOWS\ to see if we could observe a similar
effect there. And indeed, when we adapted Akira Kakuto's \WINDOWS\ setup a bit we
could compile a version for \WINDOWS\ using the native \MICROSOFT\ compiler. On
my laptop a similar speedup was observed, although by then we saw that in
practice a 25\% speedup was about what we could expect. A bonus is that making
formats and identifying fonts is also faster.

So, in that stage, we could safely conclude that \LUATEX\ combined with \LUAJIT\
made sense if you want a somewhat faster version. But where does the speedup come
from? The easiest way to see if jitting has effect is to turn it on and off.

\starttyping
jit.on()
jit.off()
\stoptyping

To our surprise \CONTEXT\ runs are not much influenced by turning the jitter on
or off. \footnote {We also tweaked some of the fine|-|tuning parameters of
\LUAJIT\ but didn't notice any differences. In due time more tests will
be done.} This means that the improvement comes from other places:

\startitemize[packed,n]
\startitem The virtual machine is a different one, and targets the platforms that
it runs on. This means that regular bytecode also runs faster. \stopitem
\startitem The garbage collector is the one from \LUA\ 5.2, so that can make a
difference. It looks like memory consumption is somewhat lower. \stopitem
\startitem Some standard library functions are recognized and supported in a more
efficient way. Think of \type {math.sin}. \stopitem
\startitem Some built-in functions like \type {type} are probably dealt with in
a more efficient way. \stopitem
\stopitemize

The third item is an important one. We don't use that many standard functions.
For instance, if we need to go from characters to bytes and vice versa, we have
to do that for \UTF\ so we use some dedicated functions or \LPEG. If in \CONTEXT\
we parse strings, we often use \LPEG\ instead of string functions anyway. And if
we still do use string functions, for instance when dealing with simple strings,
it only happens a few times.

The more demanding \CONTEXT\ code deals with node lists, which means frequent
calls to core \LUATEX\ functions. Alas, jitting doesn't help much there unless we
start messing with \type {ffi} which is not on the agenda. \footnote {If we want
to improve these mechanisms it makes much more sense to make more helpers.
However, profiling has shown us that the most demanding code is already quite
optimized.}

\stopsection

\startsection[title=Benchmarks]

Let's look at some of the benchmarks. The first one uses \METAPOST\ and because
we want to see if calculations are faster, we draw a path with a special pen so
that some transformations have to be done in the code that generates the \PDF\
output. We only show the \MSWINDOWS\ and 64-bit \LINUX\ tests here. The 32-bit
tests are consistent with those on \MSWINDOWS\ so we didn't add those timings
here (also because in the meantime Luigi's machine broke down and he moved on
to 64 bits).

\typefile{benchmark-1.tex}

The following times are measured in seconds. They are averages of 5~runs. There
is a significant speedup but jitting doesn't do much.

% mingw crosscompiled 5.2 / new mp : 25.5

\starttabulate[|l|r|r|r|]
\HL
\NC               \NC traditional \NC \JIT\ on \NC \JIT\ off \NC \NR
\HL
\NC \bf Windows 8 \NC 26.0        \NC 20.6     \NC 20.8      \NC \NR
\NC \bf Linux 64  \NC 34.2        \NC 14.9     \NC 14.1      \NC \NR
\HL
\stoptabulate

Our second example uses multiple fonts in a paragraph and adds color as well.
Although well optimized, font||related code involves node list parsing and a
bit of calculation. Color again deals with node lists and the backend
code involves calculations but not that many. The traditional run on \LINUX\ is
somewhat odd, but might have to do with the fact that the \METAPOST\ library
suffers from the 64 bits. It is at least an indication that optimizations make
less sense if there is a different dominant weak spot. We have to look into this
some time.

\typefile{benchmark-2.tex}

Again jitting has no real benefits here, but the overall gain in speed is quite
nice. It could be that the garbage collector plays a role here.

% mingw crosscompiled 5.2 / new mp : 64.3

\starttabulate[|l|r|r|r|]
\HL
\NC               \NC traditional \NC \JIT\ on \NC \JIT\ off \NC \NR
\HL
\NC \bf Windows 8 \NC 54.6        \NC 36.0     \NC 35.9      \NC \NR
\NC \bf Linux 64  \NC 46.5        \NC 32.0     \NC 31.7      \NC \NR
\HL
\stoptabulate

This benchmark writes quite a lot of data to the console, which can have impact on
performance as \TEX\ flushes on a per||character basis. When one runs \TEX\ as a
service this has less impact because in that case the output goes into the void.
There is a lot of file reading going on here, but normally the operating system
will cache data, so after a first run this effect disappears. \footnote {On \MSWINDOWS\
it makes sense to use \type {console2} because due to some clever buffering
tricks it has a much better performance than the default console.}

The third benchmark is one that we often use for testing regression in speed of
the \CONTEXT\ core code. It measures the overhead in the page builder without
special tricks being used, like backgrounds. The document has some 1000 pages.

\typefile{benchmark-3.tex}

These numbers are already quite okay for the normal version but the speedup of
the \LUAJIT\ version is consistent with the expectations we have by now.

% mingw crosscompiled 5.2 / new mp : 6.8

\starttabulate[|l|r|r|r|]
\HL
\NC               \NC traditional \NC \JIT\ on \NC \JIT\ off \NC \NR
\HL
\NC \bf Windows 8 \NC 4.5         \NC 3.6      \NC 3.6       \NC \NR
\NC \bf Linux 64  \NC 4.8         \NC 3.9      \NC 4.0       \NC \NR
\HL
\stoptabulate

The fourth benchmark uses some structuring, which involved \LUA\ tables and
housekeeping, an itemize, which involves numbering and conversions, and a table
mechanism that uses more \LUA\ than \TEX.

\typefile{benchmark-4.tex}

Here it looks like \JIT\ slows down the process, but of course we shouldn't take the last
digit too seriously.

% mingw crosscompiled 5.2 / new mp : 27.4

\starttabulate[|l|r|r|r|]
\HL
\NC               \NC traditional \NC \JIT\ on \NC \JIT\ off \NC \NR
\HL
\NC \bf Windows 8 \NC 20.9        \NC 16.8     \NC 16.5      \NC \NR
\NC \bf Linux 64  \NC 20.4        \NC 16.0     \NC 16.1      \NC \NR
\HL
\stoptabulate

Again, this example does a bit of logging, but not that much reading from file as
buffers are kept in memory.

We should start wondering when \JIT\ does kick in. This is what the fifth
benchmark does.

\typefile{benchmark-5.tex}

Here we see \JIT\ having an effect! First of all the \LUAJIT\ versions are now 4~times
faster. Making the \type {sin} a \type {local} function (the numbers after /) does not
make much of a difference because the math functions are optimized anyway.. See how
we're still faster when \JIT\ is disabled:

% mingw crosscompiled 5.2 / new mp : 2.5/2.1

\starttabulate[|l|r|r|r|]
\HL
\NC               \NC traditional \NC \JIT\ on    \NC \JIT\ off   \NC \NR
\HL
\NC \bf Windows 8 \NC 1.97 / 1.54 \NC 0.46 / 0.45 \NC 0.73 / 0.61 \NC \NR
\NC \bf Linux 64  \NC 1.62 / 1.27 \NC 0.41 / 0.42 \NC 0.67 / 0.52 \NC \NR
\HL
\stoptabulate

Unfortunately this kind of calculation (in these amounts) doesn't happen that
often but maybe some users can benefit.

\stopsection

\startsection[title=Conclusions]

So, does it make sense to complicate the \LUATEX\ build with \LUAJIT ? It does
when speed matters, for instance when \CONTEXT\ is run as a service. Some 25\% gain
in speed means less waiting time, better use of \CPU\ cycles, less energy
consumption, etc. On the other hand, computers are still becoming faster and compared
to those speed|-|ups the 25\% is not that much. Also, as \TEX\ deals with files,
the advance of \SSD\ disks and larger and faster memory helps too. Faster and
larger \CPU\ caches contributes too. On the other hand, multiple cores don't help that
much on a system that only runs \TEX. Interesting is that multi|-|core
architectures tend to run at slower speeds than single cores where more heat can
be dissipated and in that respect servers mostly running \TEX\ are better off with
fewer cores that can run at higher frequencies. But anyhow, 25\% is still better
than nothing and it makes my old laptop feel faster. It prolongs the lifetime
of machines!

Now, say that we cannot speed up \TEX\ itself that much, but that there is still
something to gain at the \LUA\ end \emdash\ what can we reasonably expect? First of all
we need to take into account that only part of the runtime is due to \LUA. Say
that this is 25\% for a document of average complexity.

\startnarrower
runtime\low{tex} + runtime\low{lua} = 100
\stopnarrower

We can consider the time needed by \TEX\ to be constant; so if that is
75\% of the total time (say 100 seconds) to begin with, we have:

\startnarrower
75 + runtime\low{lua} = 100
\stopnarrower

It will be clear that if we bring down the runtime to 80\% (80 seconds) of the
original we end up with:

\startnarrower
75 + runtime\low{lua} = 80
\stopnarrower

And the 25 seconds spent in \LUA\ went down to 5, meaning that \LUA\ processing
got 5 times faster! It is also clear that getting much more out of \LUA\
becomes hard. Of course we can squeeze more out of it, but \TEX\ still needs its
time. It is hard to measure how much time is actually spent in \LUA. We do keep
track of some times but it is not that accurate. These experiments and the gain
in speed indicate that we probably spend more time in \LUA\ than we first
guessed. If you look in the \CONTEXT\ source it's not that hard to imagine that
indeed we might well spend 50\% or more of our time in \LUA\ and|/|or in
transferring control between \TEX\ and \LUA. So, in the end there still might
be something to gain.

Let's take benchmark 4 as an example. At some point we measured for a regular
\LUATEX\ 0.74 run 27.0 seconds and for a \LUAJITTEX\ run 23.3 seconds. If we
assume that the \LUAJIT\ virtual machine is twice as fast as the normal one, some
juggling with numbers makes us conclude that \TEX\ takes some 19.6 seconds of
this. An interesting border case is \type {\directlua}: we sometimes pass quite
a lot of data and that gets tokenized first (a \TEX\ activity) and the resulting
token list is converted into a string (also a \TEX\ activity) and then converted
to bytecode (a \LUA\ task) and when okay executed by \LUA. The time involved in
conversion to byte code is probably the same for stock \LUA\ and \LUAJIT.

In the \LUATEX\ case, 30\% of the runtime for benchmark 4 is on \LUA's tab, and
in \LUAJITTEX\ it's 15\%. We can try to bring down the \LUA\ part even more, but
it makes more sense to gain something at the \TEX\ end. There macro expansion
can be improved (read: \CONTEXT\ core code) but that is already rather
optimized.

Just for the sake of completeness Luigi compiled a stock \LUATEX\ binary for 64-bit
\LINUX\ with the \type {-o3} option (which forces more inlining of functions
as well as a different switch mechanism). We did a few tests and this is the result:

\starttabulate[|lTB|r|r|]
\HL
\NC              \NC \LUATEX\ 0.74 -o2 \NC \LUATEX\ 0.74 - o3 \NC \NR
\HL
\NC benchmark-1  \NC 15.5              \NC 15.0               \NC \NR
\NC benchmark-2  \NC 35.8              \NC 34.0               \NC \NR
\NC benchmark-3  \NC  4.0              \NC  3.9               \NC \NR
\NC benchmark-4  \NC 16.0              \NC 15.8               \NC \NR
\HL
\stoptabulate

This time we used \type {--batch} and \type {--silent} to eliminate terminal
output. So, if you really want to squeeze out the maximum performance you need
to compile with \type {-o3}, use \LUAJITTEX\ (with the faster virtual machine)
but disable \JIT\ (disabled by default anyway).

% tex + jit = 23.3
% tex + lua = 27.0
% lua = 2*jit       % cf roberto
%
% so:
%
% 2*tex + 2*jit = 46.6
%   tex + 2*jit = 27.0
% -------------------- -
%   tex         = 19.6
%
% ratios:
%
% tex : lua = 70 : 30
% tex : jit = 85 : 15

We have no reason to abandon stock \LUA. Also, because during these experiments
we were still using \LUA\ 5.1 we started wondering what the move to 5.2 would
bring. Such a move forward also means that \CONTEXT\ \MKIV\ will not depend on
specific \LUAJIT\ features, although it is aware of it (this is needed because we
store bytecodes). But we will definitely explore the possibilities and see where
we can benefit. In that respect there will be a way to enable and
disable jitting. So, users have the choice to use either stock \LUATEX\ or the
\JIT||aware version but we default to the regular binary.

As we use stock \LUA\ as benchmark, we will use the \type {bit32} library, while
\LUAJIT\ has its own bit library. Some functions can be aliased so that is no big
deal. In \CONTEXT\ we use wrappers anyway. More problematic is that we want to
move on to \LUA\ 5.2 and not all 5.2 features are supported (yet) in \LUAJIT. So,
if \LUAJIT\ is mandatory in a workflow, then users had better make sure that the
\LUA\ code is compatible. We don't expect too many problems in \CONTEXT\ \MKIV.

\stopsection

\startsection[title=About speed]

It is worth mentioning that the \LUA\ version in \LUATEX\ has a patch for
converting floats into strings. Instead of some \type {INF#} result we just
return zero, simply because \TEX\ is integer||based and intercepting incredibly
small numbers is too cumbersome. We had to apply the same patch in the \JIT\
version.

The benchmarks only indicate a trend. In a real document much more happens than
in the above tests. So what are measurements worth? Say that we compile the \TEX
book. This grandparent of all documents coded in \TEX\ is rather plainly coded
(using of course plain \TEX) and compiles pretty fast. Processing does not suffer
from complex expansions, there is no color, hardly any text manipulation, it's
all 8 bit, the pagebuilder is straightforward as is all spacing. Although on my
old machine I can get \CONTEXT\ to run at over 200 pages per second, this quickly
drops to 10\% of that speed when we add some color, backgrounds, headers and
footers, font switches, etc.

So, running documents like the \TEX book for comparing the speed of, say,
\PDFTEX, \XETEX, \LUATEX\ and now \LUAJITTEX\ makes no sense. The first one is
still eight bit, the rest are \UNICODE. Also, the \TEX book uses traditional
fonts with traditional features so effectively that it doesn't rely on anything
that the new engines provide, not even \ETEX\ extensions. On the other hand, a
recent document uses advanced fonts, properties like color and|/|or
transparencies, hyperlinks, backgrounds, complex cover pages or chapter openings,
embeds graphics, etc. Such a document might not even process in \PDFTEX\ or
\XETEX, and if it does, it's still comparing different technologies: eight bit
input and fast fonts in \PDFTEX, frozen \UNICODE\ and wide font support in
\XETEX, instead of additional trickery and control, written in \LUA. So, when we
investigate speed, we need to take into account what (font and input)
technologies are used as well as what complicating layout and rendering features
play a role. In practice speed only matters in an edit|-|view cycle and services
where users wait for some result.

It's rather hard to find a recent document that can be used to compare these
engines. The best we could come up with was the rendering of the user interface
documentation.

\starttabulate[|T|T|T|T||]
\NC texexec \NC --engine=pdftex    \NC --global \NC x-set-12.mkii \NC 5.9 seconds \NC \NR
\NC texexec \NC --engine=xetex     \NC --global \NC x-set-12.mkii \NC 6.2 seconds \NC \NR
\NC context \NC --engine=luatex    \NC --global \NC x-set-12.mkiv \NC 6.2 seconds \NC \NR
\NC context \NC --engine=luajittex \NC --global \NC x-set-12.mkiv \NC 4.6 seconds \NC \NR
\stoptabulate

Keep in mind that \type{texexec} is a \RUBY\ script and uses \type {kpsewhich}
while \type {context} uses \LUA\ and its own (\TDS||compatible) file manager. But
still, it is interesting to see that there is not that much difference if we keep
\JIT\ out of the picture. This is because in \MKIV\ we have somewhat more clever
\XML\ processing, although earlier measurements have demonstrated that in this
case not that much speedup can be assigned to that.

And so recent versions of \MKIV\ already keep up rather well with the older eight
bit world. We do way more in \MKIV\ and the interfacing macros are nicer but
potentially somewhat slower. Some mechanisms might be more efficient because of
using \LUA, but some actually have more overhead because we keep track of more
data. Font feature processing is done in \LUA, but somehow can keep up with the
libraries used in \XETEX, or at least is not that significant a difference,
although I can think of more demanding tasks. Of course in \LUATEX\ we can go
beyond what libraries provide.

No matter what one takes into account, performance is not that much worse in
\LUATEX, and if we enable \JIT\ and so remove some of the traditional \LUA\
virtual machine overhead, we're even better off. Of course we need to add a
disclaimer here: don't force us to prove that the relative speed ratios are the
same for all cases. In fact, it being so hard to measure and compare, performance
can be considered to be something taken for granted as there is not that much we
can do about getting nicer numbers, apart from maybe parallelizing which brings
other complexities into the picture. On our servers, a few other virtual machines
running \TEX\ services kicking in at the same time, using \CPU\ cycles, network
bandwidth (as all data lives someplace else) and asking for disk access have much
more impact than the 25\% we gain. Of course if all processes run faster then
we've gained something.

For what it's worth: processing this text takes some 2.3 seconds on my laptop for
regular \LUATEX\ and 1.8 seconds with \LUAJITTEX, including the extra overhead of
restarting. As this is a rather average example it fits earlier measurements.

Processing a font manual (work in progress) takes \LUAJITTEX\ 15 seconds for 112
pages compared to 18.4 seconds for \LUATEX. The not yet finished manual loads 20
different fonts (each with multiple instances), uses colors, has some \METAPOST\
graphics and does some font juggling. The gain in speed sounds familiar.

\stopsection

\startsection[title=The future]

At the 2012 \LUA\ conference Roberto Ierusalimschy mentioned that the virtual
machine of \LUAJIT\ is about twice as fast due to it being partly done in
assembler while the regular machinery is written in standard \CCODE\ and keeps
portability in mind.

He also presented some plans for future versions of \LUA. There will be some
lightweight helpers for \UTF. Our experiences so far are that only a handful of
functions are actually needed: byte to character conversions and vice versa,
iterators for \UTF\ characters and \UTF\ values and maybe a simple substring
function is probably enough. Currently \LUATEX\ has some extra string iterators
and it will provide the converters as well.

There is a good chance that \LPEG\ will become a standard library (which it
already is in \LUATEX), which is also nice. It's interesting that, especially on
longer sequences, \LPEG\ can beat the string matchers and replacers, although
when in a substitution no match and therefore no replacements happen, the regular
gsub wins. We're talking small numbers here, in daily usage \LPEG\ is about as
efficient as you can wish. In \CONTEXT\ we have a \type {lpeg.UR} and \type
{lpeg.US} and it would be nice to have these as native \UTF\ related methods, but
I must admit that I seldom need them.

This and other extensions coming to the language also have some impact on a \JIT\
version: the current \LUAJIT\ is already not entirely compatible with \LUA\ 5.2
so you need to keep that into account if you want to use this version of \LUATEX.
So, unless \LUAJIT\ follows the mainstream development, as \CONTEXT\ \MKIV\ user
you should not depend on it. But at the moment it's nice to have this choice.

The yet experimental code will end up in the main \LUATEX\ repository in time
before the \TEX\ Live 2013 code freeze. In order to make it easier to run both
versions alongside, we have added the \LUA\ 5.2 built|-|in library \type {bit32}
to \LUAJITTEX. We found out that it's too much trouble to add that library to
\LUA~5.1 but \LUATEX\ has moved on to 5.2 anyway.

\stopsection

\startsection[title=Running]

So, as we will definitely stick to stock \LUA, one might wonder if it makes sense
to officially support jitting in \CONTEXT. First of all, \LUATEX\ is not
influenced that much by the low level changes in the \API\ between 5.1 and 5.2.
Also \LUAJIT\ does support the most important new 5.2 features, so at the moment
we're mostly okay. We expect that eventually \LUAJIT\ will catch up but if not,
we are not in big trouble: the performance of stock \LUA\ is quite okay and above
all, it's portable! \footnote {Stability and portability are important properties
of \TEX\ engines, which is yet another reason for using \LUA. For those doing
number crunching in a document, \JIT\ can come in handy.} For the moment you can
consider \LUAJITTEX\ to be an experiment and research tool, but we will do our
best to keep it production ready.

So how do we choose between the two engines? After some experimenting with
alternative startup scenarios and dedicated caches, the following solution was
reached:

\starttyping
context --engine=luajittex ...
\stoptyping

The usual preamble line also works:

\starttyping
% engine=luajittex
\stoptyping

As the main infrastructure uses the \type {luatex} and related binaries, this
will result in a relaunch: the \type {context} script will be restarted using
\type {luajittex}. This is a simple solution and the overhead is rather minimal,
especially compared to the somewhat faster run. Alternatively you can copy \type
{luajittex} over \type {luatex} but that is more drastic. Keep in mind that \type
{luatex} is the benchmark for development of \CONTEXT, so the \JIT\ aware version
might fall behind sometimes.

Yet another approach is adapting the configuration file, or better, provide (or
adapt) your own \type {texmfcnf.lua} in for instance \type {texmf-local/web2c}
path:

\starttyping
return {
  type    = "configuration",
  version = "1.2.3",
  date    = "2012-12-12",
  time    = "12:12:12",
  comment = "Local overloads",
  author  = "Hans Hagen, PRAGMA-ADE, Hasselt NL",
  content = {
    directives = {
      ["system.engine"] = "luajittex",
    },
  },
}
\stoptyping

This has the same effect as always providing \type {--engine=luajittex} but only
makes sense in well controlled situations as you might easily forget that it's
the default. Of course one could have that file and just comment out the
directive unless in test mode.

Because the bytecode of \LUAJIT\ differs from the one used by \LUA\ itself we
have a dedicated format as well as dedicated bytecode compiled resources (for
instance \type {tmb} instead of \type {tmc}). For most users this is not
something they should bother about as it happens automatically.

Based on experiments, by default we have disabled \JIT\, so we only benefit from
the faster virtual machine. Future versions of \CONTEXT\ might provide some
control over that but first we want to conduct more experiments.

\stopsection

\startsection[title=Addendum]

These developments and experiments took place in November and December 2012. At
the time of this writing we also made the move to \LUA\ 5.2 in stock \LUATEX; the
first version to provide this was 0.74. Here are some measurements on Taco
Hoekwater's 64-bit \LINUX\ machine:

\starttabulate[|lTB|r|r|l|]
\HL
\NC              \NC \LUATEX\ 0.70 \NC \LUATEX\ 0.74 \NC        \NC \NR
\HL
\NC benchmark-1  \NC 23.67         \NC 19.57         \NC faster \NC \NR
\NC benchmark-2  \NC 65.41         \NC 62.88         \NC faster \NC \NR
\NC benchmark-3  \NC  4.88         \NC  4.67         \NC faster \NC \NR
\NC benchmark-4  \NC 23.09         \NC 22.71         \NC faster \NC \NR
\NC benchmark-5  \NC  2.56/2.06    \NC  2.66/2.29    \NC slower \NC \NR
\HL
\stoptabulate

There is a good chance that this is due to improvements of the garbage collector,
virtual machine and string handling. It also looks like memory consumption is a
bit less. Some speed optimizations in reading files have been removed (at least
for now) and some patches to the \type {format} function (in the \type {string}
namespace) that dealt with (for \TEX) unfortunate number conversions have not
been ported. The code base is somewhat cleaner and we expect to be able to split
up the binary in a core program plus some libraries that are loaded on demand.
\footnote {Of course this poses some constraints on stability as components get
decoupled, but this is one of the issues that we hope to deal with properly in
the library project.} In general, we don't expect too many issues in the
transition to \LUA\ 5.2, and \CONTEXT\ is already adapted to support \LUATEX\
with 5.2 as well as \LUAJITTEX\ with an older version.

Running the same tests on a 32-bit \MSWINDOWS\ machine gives this:

\starttabulate[|lTB|r|r|r|]
\HL
\NC              \NC \LUATEX\ 0.70 \NC \LUATEX\ 0.74 \NC        \NC \NR
\HL
\NC benchmark-1  \NC 26.4          \NC 25.5          \NC faster \NC \NR
\NC benchmark-2  \NC 64.2          \NC 63.6          \NC faster \NC \NR
\NC benchmark-3  \NC  7.1          \NC  6.9          \NC faster \NC \NR
\NC benchmark-4  \NC 28.3          \NC 27.0          \NC faster \NC \NR
\NC benchmark-5  \NC  1.95/1.50    \NC  1.84/1.48    \NC faster \NC \NR
\HL
\stoptabulate

The gain is less impressive but the machine is rather old and we can benefit less
from modern \CPU\ properties (cache, memory bandwidth, etc.). I tend to conclude
that there is no significant improvement here but it also doesn't get worse.
However we need to keep in mind that file \IO\ is less optimal in 0.74 so this
might play a role. As usual, runtime is negatively influenced by the relatively
slow speed of displaying messages on the console (even when we use \type
{console2}).

A few days before the end of 2012, Akira Kakuto compiled native \MSWINDOWS\
binaries for both engines. This time I decided to run a comparison inside the
\SCITE\ editor, that has very fast console output. \footnote {Most of my personal
\TEX\ runs are from within \SCITE, while most runs on the servers are in batch
mode, so normally the overhead of the console is acceptable or even neglectable.}

\starttabulate[|lTB|r|r|r|]
\HL
\NC              \NC \LUATEX\ 0.74 (5.2) \NC \LUAJITTEX\ 0.72 (5.1) \NC         \NC \NR
\HL
\NC benchmark-1  \NC 25.4                \NC 25.4                   \NC similar \NC \NR
\NC benchmark-2  \NC 54.7                \NC 36.3                   \NC faster  \NC \NR
\NC benchmark-3  \NC  4.3                \NC  3.6                   \NC faster  \NC \NR
\NC benchmark-4  \NC 20.0                \NC 16.3                   \NC faster  \NC \NR
\NC benchmark-5  \NC  1.93/1.48          \NC  0.74/0.61             \NC faster  \NC \NR
\HL
\stoptabulate

Only the \METAPOST\ library and conversion benchmark didn't show a speedup. The
regular \TEX\ tests 1||3 gain some 15||35\%. Enabling \JIT\ (off by default)
slowed down processing. For the sake of completeness I also timed \LUAJITTEX\
on the console, so here you see the improvement of both engines.

\starttabulate[|lTB|r|r|r|]
\HL
\NC              \NC \LUATEX\ 0.70 \NC \LUATEX\ 0.74 \NC \LUAJITTEX\ 0.72 \NC \NR
\HL
\NC benchmark-1  \NC 26.4          \NC 25.5          \NC  25.9      \NC \NR
\NC benchmark-2  \NC 64.2          \NC 63.6          \NC  45.5      \NC \NR
\NC benchmark-3  \NC 7.1           \NC  6.9          \NC   6.0      \NC \NR
\NC benchmark-4  \NC 28.3          \NC 27.0          \NC  23.3      \NC \NR
\NC benchmark-5  \NC 1.95/1.50     \NC 1.84/1.48     \NC  0.73/0.60 \NC \NR
\HL
\stoptabulate

In this text, the term \JIT\ has come up a lot but you might rightfully wonder if
the observations here relate to \JIT\ at all. For the moment I tend to conclude
that the implementation of the virtual machine and garbage collection have more
impact than the actual just||in||time compilation. More exploration of \JIT\ is
needed to see if we can really benefit from that. Of course the fact that we use
a bit less memory is also nice. In case you wonder why I bother about speed at
all: we happen to run \LUATEX\ mostly as a (remote) service and generating a
bunch of (related) documents takes a bit of time. Bringing the waiting down from
15 to 10 seconds might not sound impressive but it makes a difference when it is
someone's job to generate these sets.

In summary: just before we entered 2013, we saw two rather fundamental updates of
\LUATEX\ show up: an improved traditional one with \LUA\ 5.2 as well as the
somewhat faster \LUAJITTEX\ with a mixture between 5.1 and 5.2. And in 2013 we
will of course try to make them both even more attractive.

\stopsection

\stopchapter

% benchmark-4:
%
% tex + jit = 23.3
% tex + lua = 27.0
% lua = 2*jit       % cf roberto
%
% so:
%
% 2*tex + 2*jit = 46.6
%   tex + 2*jit = 27.0
% -------------------- -
%   tex         = 19.6
%
% ratios:
%
% tex : lua = 70 : 30
% tex : jit = 85 : 15
