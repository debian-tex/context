% language=us runpath=texruns:manuals/luametatex

\environment luametatex-style

\startcomponent luametatex-building

\startchapter[reference=building,title={Boxes, paragraphs and pages}]

\startsection[title={Introduction}]

\topicindex {building}
\topicindex {pages}
\topicindex {paragraphs}
\topicindex {marks}
\topicindex {inserts}

There are some enhancements that relate to the way paragraphs and pages are
built. In this chapter we will cover those. There can be a bit of overlap with
other chapters. These enhancements are still somewhat experimental.

\stopsection

\startsection[title=Directions]

\topicindex {\OMEGA}
\topicindex {\ALEPH}
\topicindex {directions}

\startsubsection[title={Two directions}]

The directional model in \LUAMETATEX\ is a simplified version the the model used
in \LUATEX. In fact, not much is happening at all: we only register a change in
direction.

\stopsubsection

\startsubsection[title={How it works}]

The approach is that we try to make node lists balanced but also try to avoid
some side effects. What happens is quite intuitive if we forget about spaces
(turned into glue) but even there what happens makes sense if you look at it in
detail. However that logic makes in|-|group switching kind of useless when no
properly nested grouping is used: switching from right to left several times
nested, results in spacing ending up after each other due to nested mirroring. Of
course a sane macro package will manage this for the user but here we are
discussing the low level injection of directional information.

This is what happens:

\starttyping
\textdirection 1 nur {\textdirection 0 run \textdirection 1 NUR} nur
\stoptyping

This becomes stepwise:

\startnarrower
\starttyping
injected: [push 1]nur {[push 0]run [push 1]NUR} nur
balanced: [push 1]nur {[push 0]run [pop 0][push 1]NUR[pop 1]} nur[pop 0]
result  : run {RUNrun } run
\stoptyping
\stopnarrower

And this:

\starttyping
\textdirection 1 nur {nur \textdirection 0 run \textdirection 1 NUR} nur
\stoptyping

becomes:

\startnarrower
\starttyping
injected: [+TRT]nur {nur [+TLT]run [+TRT]NUR} nur
balanced: [+TRT]nur {nur [+TLT]run [-TLT][+TRT]NUR[-TRT]} nur[-TRT]
result  : run {run RUNrun } run
\stoptyping
\stopnarrower

Now, in the following examples watch where we put the braces:

\startbuffer
\textdirection 1 nur {{\textdirection 0 run} {\textdirection 1 NUR}} nur
\stopbuffer

\typebuffer

This becomes:

\startnarrower
\getbuffer
\stopnarrower

Compare this to:

\startbuffer
\textdirection 1 nur {{\textdirection 0 run }{\textdirection 1 NUR}} nur
\stopbuffer

\typebuffer

Which renders as:

\startnarrower
\getbuffer
\stopnarrower

So how do we deal with the next?

\startbuffer
\def\ltr{\textdirection 0\relax}
\def\rtl{\textdirection 1\relax}

run {\rtl nur {\ltr run \rtl NUR \ltr run \rtl NUR} nur}
run {\ltr run {\rtl nur \ltr RUN \rtl nur \ltr RUN} run}
\stopbuffer

\typebuffer

It gets typeset as:

\startnarrower
\startlines
\getbuffer
\stoplines
\stopnarrower

We could define the two helpers to look back, pick up a skip, remove it and
inject it after the dir node. But that way we loose the subtype information that
for some applications can be handy to be kept as|-|is. This is why we now have a
variant of \prm {textdirection} which injects the balanced node before the skip.
Instead of the previous definition we can use:

\startbuffer[def]
\def\ltr{\linedirection 0\relax}
\def\rtl{\linedirection 1\relax}
\stopbuffer

\typebuffer[def]

and this time:

\startbuffer[txt]
run {\rtl nur {\ltr run \rtl NUR \ltr run \rtl NUR} nur}
run {\ltr run {\rtl nur \ltr RUN \rtl nur \ltr RUN} run}
\stopbuffer

\typebuffer[txt]

comes out as a properly spaced:

\startnarrower
\startlines
\getbuffer[def,txt]
\stoplines
\stopnarrower

Anything more complex that this, like combination of skips and penalties, or
kerns, should be handled in the input or macro package because there is no way we
can predict the expected behaviour. In fact, the \prm {linedirection} is just a
convenience extra which could also have been implemented using node list parsing.

Directions are complicated by the fact that they often need to work over groups
so a separate grouping related stack is used. A side effect is that there can be
paragraphs with only a local par node followed by direction synchronization
nodes. Paragraphs like that are seen as empty paragraphs and therefore ignored.
Because \prm {noindent} doesn't inject anything but a \prm {indent} injects
an box, paragraphs with only an indent and directions are handles and paragraphs
with content. When indentation is normalized a paragraph with an indentation
skip is seen as content.

\stopsubsection

\startsubsection[title={Normalizing lines}]

The original \TEX\ machinery was never meant to be opened up. As a consequence a
constructed line can have different layouts. There can be left- and/or right
skips and hanging indentation or parshape can result in a shift and adapted
width. In \LUATEX\ glue got subtypes so we can recognize the left-, right and
parfill skips, but still there is no hundred percent certainty about the shape.

In \LUAMETATEX\ lines can be normalized. This is optional because we want to
preserve the original (for comparison) and is controlled by \prm
{normalizelinemode}. That variable actually drives some more. An earlier version
provided a few more granular options (for instance: does a leftskip comes before
or after a left hanging indentation) but in the end that was dropped. Because
this normalization only is seen at the \LUA\ end there is no need to go into much
detail here.

At this moment a line has this pattern: left parfill, left hang, left skip,
indentation, content, right hang, right skip, right parfill. Of course the
indentation and fill skips are not present in every line.

Control over normalization happens via the mentioned mode variable and here is
what the engine provides right now. We use a bitmap:

\starttabulate[|l|l|]
\DB value \BC reported \NC \NR
\TB
\NC \type{0x0001} \NC normalize line as described above            \NC \NR
\NC \type{0x0002} \NC use a skip for parindent instead of a box    \NC \NR
\NC \type{0x0004} \NC swap hangindent in l2r mode                  \NC \NR
\NC \type{0x0008} \NC swap parshape in l2r mode                    \NC \NR
\NC \type{0x0010} \NC put breaks after dir in l2r mode             \NC \NR
\NC \type{0x0020} \NC remove margin kerns (\PDFTEX\ left-over)     \NC \NR
\NC \type{0x0040} \NC if needed clip width and use correction kern \NC \NR
\LL
\stoptabulate

Setting the bit enables the related normalization. More features might be added
in future releases.

% Swapping shapes
%
% Another adaptation to the \ALEPH\ directional model is control over shapes driven
% by \prm {hangindent} and \prm {parshape}. This is controlled by a new parameter
% \prm {shapemode}:
%
% \starttabulate[|c|l|l|]
% \DB value    \BC \prm {hangindent} \BC \prm {parshape} \NC \NR
% \TB
% \BC \type{0} \NC  normal             \NC normal            \NC \NR
% \BC \type{1} \NC  mirrored           \NC normal            \NC \NR
% \BC \type{2} \NC  normal             \NC mirrored          \NC \NR
% \BC \type{3} \NC  mirrored           \NC mirrored          \NC \NR
% \LL
% \stoptabulate
%
% The value is reset to zero (like \prm {hangindent} and \prm {parshape})
% after the paragraph is done with. You can use negative values to prevent
% this. In \in {figure} [fig:shapemode] a few examples are given.
%
% \startplacefigure[reference=fig:shapemode,title={The effect of \type {shapemode}.}]
%     \startcombination[2*3]
%         {\ruledvbox \bgroup \setuptolerance[verytolerant]
%             \hsize .45\textwidth \switchtobodyfont[6pt]
%                 \pardirection 0 \textdirection 0
%                 \hangindent 40pt \hangafter -3
%                 \leftskip10pt \input tufte \par
%          \egroup} {TLT: hangindent}
%         {\ruledvbox \bgroup \setuptolerance[verytolerant]
%             \hsize .45\textwidth \switchtobodyfont[6pt]
%             \pardirection 0 \textdirection 0
%             \parshape 4 0pt .8\hsize 10pt .8\hsize 20pt .8\hsize 0pt \hsize
%             \input tufte \par
%          \egroup} {TLT: parshape}
%         {\ruledvbox \bgroup \setuptolerance[verytolerant]
%             \hsize .45\textwidth \switchtobodyfont[6pt]
%             \pardirection 1 \textdirection 1
%             \hangindent 40pt \hangafter -3
%             \leftskip10pt \input tufte \par
%          \egroup} {TRT: hangindent mode 0}
%         {\ruledvbox \bgroup \setuptolerance[verytolerant]
%             \hsize .45\textwidth \switchtobodyfont[6pt]
%             \pardirection 1 \textdirection 1
%             \parshape 4 0pt .8\hsize 10pt .8\hsize 20pt .8\hsize 0pt \hsize
%             \input tufte \par
%          \egroup} {TRT: parshape mode 0}
%         {\ruledvbox \bgroup \setuptolerance[verytolerant]
%             \hsize .45\textwidth \switchtobodyfont[6pt]
%             \shapemode=3
%             \pardirection 1 \textdirection 1
%             \hangindent 40pt \hangafter -3
%             \leftskip10pt \input tufte \par
%          \egroup} {TRT: hangindent mode 1 & 3}
%         {\ruledvbox \bgroup \setuptolerance[verytolerant]
%             \hsize .45\textwidth \switchtobodyfont[6pt]
%             \shapemode=3
%             \pardirection 1 \textdirection 1
%             \parshape 4 0pt .8\hsize 10pt .8\hsize 20pt .8\hsize 0pt \hsize
%             \input tufte \par
%          \egroup} {TRT: parshape mode 2 & 3}
%     \stopcombination
% \stopplacefigure
%
% We have \type {\pardirection}, \type {\textdirection}, \type {\mathdirection} and
% \type {\linedirection} that is like \type {\textdirection} but with some
% additional (inline) glue checking.

% Controlling glue with \prm {breakafterdirmode}
%
% Glue after a dir node is ignored in the linebreak decision but you can bypass that
% by setting \prm {breakafterdirmode} to~\type {1}. The following table shows the
% difference. Watch your spaces.
%
% \def\ShowSome#1{%
%     \BC \type{#1}
%     \NC \breakafterdirmode\zerocount\hsize\zeropoint#1
%     \NC
%     \NC \breakafterdirmode\plusone\hsize\zeropoint#1
%     \NC
%     \NC \NR
% }
%
% \starttabulate[|l|Tp(1pt)|w(5em)|Tp(1pt)|w(5em)|]
%     \DB
%     \BC \type{0}
%     \NC
%     \BC \type{1}
%     \NC
%     \NC \NR
%     \TB
%     \ShowSome{pre {\textdirection 0 xxx} post}
%     \ShowSome{pre {\textdirection 0 xxx }post}
%     \ShowSome{pre{ \textdirection 0 xxx} post}
%     \ShowSome{pre{ \textdirection 0 xxx }post}
%     \ShowSome{pre { \textdirection 0 xxx } post}
%     \ShowSome{pre {\textdirection 0\relax\space xxx} post}
%     \LL
% \stoptabulate

\stopsubsection

\startsubsection[title=Orientations]

\topicindex {boxes+orientations}

As mentioned, the difference with \LUATEX\ is that we only have numeric
directions and that there are only two: left|-|to|-|right (\type {0}) and
right|-|to|-|left (\type {1}). The direction of a box is set with \type
{direction}.

In addition to that boxes can now have an \type {orientation} keyword followed by
optional \type {xoffset} and|/|or \type {yoffset} keywords. The offsets don't
have consequences for the dimensions. The alternatives \type {xmove} and \type
{ymove} on the contrary are reflected in the dimensions. Just play with them. The
offsets and moves only are accepted when there is also an orientation, so no time
is wasted on testing for these rarely used keywords. There are related primitives
\type {\box...} that set these properties.

As these are experimental it will not be explained here (yet). They are covered
in the descriptions of the development of \LUAMETATEX: articles and|/|or
documents in the \CONTEXT\ distribution. For now it is enough to know that the
orientation can be up, down, left or right (rotated) and that it has some
anchoring variants. Combined with the offsets this permits macro writers to
provide solutions for top|-|down and bottom|-|up writing directions, something
that is rather macro package specific and used for scripts that need
manipulations anyway. The \quote {old} vertical directions were never okay and
therefore not used.

There are a couple of properties in boxes that you can set and query but that
only really take effect when the backend supports them. When usage on \CONTEXT\
shows that is't okay, they will become official, so we just mention them: \prm
{boxdirection}, \prm {boxattribute}, \prm {boxorientation}, \prm {boxxoffset},
\prm {boxyoffset}, \prm {boxxmove}, \prm {boxymove} and \prm {boxtotal}.

{\em This is still somewhat experimental and will be documented in more detail
when I've used it more in \CONTEXT\ and the specification is frozen. This might
take some time (and user input).}

\stopsubsection

\stopsection

\startsection[title={Boxes, rules and leaders}]

\startsubsection[title={\prm {outputbox}}]

\topicindex {output}

This integer parameter allows you to alter the number of the box that will be
used to store the page sent to the output routine. Its default value is 255, and
the acceptable range is from 0 to 65535.

\startsyntax
\outputbox = 12345
\stopsyntax

\stopsubsection

\startsubsection[title={\prm {hrule}, \prm {vrule}, \prm {srule}, \prm {nohrule}, \prm {novrule},
\prm {virtualhrule} and \prm {virtualvrule}}]

\topicindex {rules}

Both rule drawing commands take an optional \type {xoffset} and \type {yoffset}
parameter. The displacement is virtual and not taken into account when the
dimensions are calculated. A rule is specified in the usual way:

\obeydepth

\startbuffer
\blue \vrule
    height 2ex depth 1ex width 10cm
\relax
\stopbuffer

\startlinecorrection
\getbuffer
\stoplinecorrection

There is however a catch. The keyword scanners in \LUAMETATEX\ are implemented
slightly different. When \TEX\ scans a keyword it will (case insensitive) scan
for a whole keyword. So, it scans for \type {height} and when it doesn't find it
it will scan for \type {depth} etc. When it does find a keyword in this case it
expects a dimension next. When that criterium is not met it will issue an error
message.

In order to avoid look ahead failures like that it is recommended to end the
specification with \type {\relax}. A glue specification is an other example where
a \type {\relax} makes sense when look ahead issues are expected and actually
there in traditional scanning the order of keywords can also matter. In any case,
when no valid keyword is seen the characters scanned so far are pushed back in
the input.

The main reason for using an adapted scanner is that we always permit repetition
(consistency) and accept an arbitrary order. Because we have more keywords to
process the scanner quits at a partial failure. This prevents some push back and
also gives an earlier warning. Interesting is that some \CONTEXT\ users ran into
error messages due to a missing \type {\relax} and found out that their style has
a potential flaw with respect to look ahead. One can be lucky for years.

Back to rules, there are some extra keywords, two deal with an offset, and four
provide margins. The margins are a bit special because \type {left} and \type
{top} are the same as are \type {right} and \type {bottom}. They influence the
edges and these depend on it being a horizontal or vertical rule.

\obeydepth

\startbuffer
\blue \vrule
    height 2.0ex depth 1.0ex width 10cm
\relax
\white \vrule
    height 1.0ex depth 0.5ex width  9cm
    xoffset -9.5cm yoffset .25ex
\relax
\blue \vrule
    height .5ex depth 0.25ex width  8cm
    xoffset -18cm yoffset .375ex top 1pt
\relax
\stopbuffer

\startlinecorrection
\getbuffer
\stoplinecorrection

Two new primitives were introduced: \prm {nohrule} and \prm {novrule}. These can
be used to reserve space. This is often more efficient than creating an empty box
with fake dimensions. Of course this assumes that the backend implements them
being invisible but still taking space.

An \prm {srule} is sort of special. In text mode it is just a convenience (we
could do without it for ages) but in math mode it comes in handy when we want to
enforce consistency. \footnote {In \CONTEXT\ there is a lot of focus on
consistent vertical spacing, something that doesn't naturally comes with \TEX\
(you have to pay attention!) and therefore for decades now you can find plenty of
documents with bad spacing of a nature that has seem to have become accepted as
quality. This probably makes these \prm {srule}'s one of the few primitives that
actually targets at \CONTEXT.}

As with all rules, the backend will makes rules span the width or height and
depth of the encapsulating box. An \prm {srule} is just a \prm {vrule} but is set
up such that it can adapt itself:

\startbuffer
\hbox to 3cm {x\leaders\hrule\hfil x}
\hbox{x \vrule width 4cm \relax x}
\hbox{x \srule width 4cm \relax x}
\hbox{x \vrule font \font char `( width 4cm \relax x}
\hbox{x \srule font \font char `( width 4cm \relax x}
\hbox{$x \srule fam \fam  char `( width 4cm \relax x$}
\hbox{$x \vrule fam \fam  char `( width 4cm \relax x$}
\stopbuffer

\typebuffer

You can hard code the height and depth or get it from a font|/|family|/|character
combination. This is especially important in math mode where then can adapt to
(stylistic) circumstances.

\startlines
\showboxes\getbuffer
\stoplines

Because this kind of rules has a dedicated subtype you can intercept it in the backend
if needed. The two virtual variants are special in the way that they are like normal
rules but take no space. Can you figure out how to get this?

\startlinecorrection[blank]
\dontleavehmode \hbox{%
    \hbox{\green before}%
    {\darkblue \virtualvrule width 40pt height -2pt depth 4pt xoffset -20pt\relax}%
    \hbox{\red after}%
}
\stoplinecorrection

% \vskip5pt

% \ruledvbox{\ruledhbox to 10pt{\green x}
% \virtualhrule width 10pt height 2pt depth 2pt yoffset -2pt \relax
% \ruledhbox to 10pt{\red x}}

\stopsubsection

\startsubsection[title={\prm {vsplit}, \prm {tsplit} and \prm {dsplit}}]

\topicindex {splitting}

The \prm {vsplit} primitive has to be followed by a specification of the required
height. As alternative for the \type {to} keyword you can use \type {upto} to get
a split of the given size but result has the natural dimensions then.

\starttyping
\vsplit 123 to   10cm % final box has the required height
\vsplit 123 upto 10cm % final box has its natural height
\stoptyping

The two alternative primitives return a \prm {vtop} or \prm {dbox} instead of a
\prm {vbox}. All three accept the \type {attr} keyword as boxes do.

\stopsubsection

\startsubsection[title={\prm {boxxoffset}, \prm {boxyoffset}, \prm {boxxmove}, \prm {boxymove},
\prm{boxorientation} and \prm{boxgeometry}}]

This repertoire of primitives can be used to do relative positioning. The offsets
are virtual while the moves adapt the dimensions. The orientation bitset can be
used to rotate the box over 90, 180 and 270 degrees. It also influences the
corner, midpoint or baseline.

{\em There is information in the \CONTEXT\ low level manuals and in due time I
will add a few examples here. This feature needs support in the backend when used
(as in \CONTEXT) so it might influence performance.}

\stopsubsection

\startsubsection[title={\prm {boxtotal}}]

The \prm {boxtotal} primitive returns the sum of the height and depth and is less
useful as setter: it just sets the height and depth to half of the given value.

\stopsubsection

\startsubsection[title={\prm {boxshift}}]

In traditional \TEX\ a box has height, depth, width and a shift where the later
relates to \prm {raise}, \prm {lower}, \prm {moveleft} and \prm {moveright}. This
primitive can be used to query and set this property.

\startbuffer
\setbox0\hbox{test test test}
\setbox2\hbox{test test test} \boxshift2 -10pt
\ruledhbox{x \raise10pt\box0\ x}
\ruledhbox{x           \box2\ x}
\stopbuffer

\typebuffer

\stopsubsection

\startsubsection[title={\prm {boxanchor}, \prm {boxanchors}, \prm {boxsource} and \prm {boxtarget}}]

{\em These are experimental.}

\stopsubsection

\startsubsection[title={\prm {boxfreeze}, \prm {boxadapt} and \prm {boxrepack}}]

\topicindex {boxes+postprocessing}

This operation will freeze the glue in the given box, something that normally is
delayed and delegated to the backend.

\startbuffer
\setbox    0 \hbox to 5cm {\hss test}
\setbox    2 \hbox to 5cm {\hss test}
\boxfreeze 2 0
\ruledhbox{\unhbox   0}
\ruledhbox{\unhbox   2}
\stopbuffer

\typebuffer

The second parameter to \prm {boxfreeze} determines recursion. Here we just
freeze the outer level:

\getbuffer

Repacking will take the content of an existing box and add or subtract from it:

\startbuffer
\setbox 0 \hbox        {test test test}
\setbox 2 \hbox {\red   test test test} \boxrepack0 +.2em
\setbox 4 \hbox {\green test test test} \boxrepack0 -.2em
\ruledhbox{\box0} \vskip-\lineheight
\ruledhbox{\box0} \vskip-\lineheight
\ruledhbox{\box0}
\stopbuffer

\typebuffer

\getbuffer

We can use this primitive to check the natural dimensions:

\startbuffer
\setbox 0 \hbox spread 10pt {test test test}
\ruledhbox{\box0} (\the\boxrepack0,\the\wd0)
\stopbuffer

\typebuffer

\getbuffer

Adapting will recalculate the dimensions with a scale factor for the glue:

\startbuffer
\setbox 0 \hbox       {test test test}
\setbox 2 \hbox {\red  test test test} \boxadapt 0   200
\setbox 4 \hbox {\blue test test test} \boxadapt 0  -200
\ruledhbox{\box0} \vskip-\lineheight
\ruledhbox{\box0} \vskip-\lineheight
\ruledhbox{\box0}
\stopbuffer

\typebuffer

\getbuffer

\stopsubsection

\startsubsection[title={\prm {boxvadjust}}]

This primitive binds a \prm {vadjust} to a box and therefore also accepts the
\type {pre} and \type {post} keywords which means that you can prepend and append
as the box itself gets flushed.

\stopsubsection

\startsubsection[title={Overshooting dimensions}]

\topicindex {boxes+overfull}

The \prm {overshoot} primitive reports the most recent amount of overshoot when a
box is packages. It relates to overfull boxes and the then set \prm {badness} of
1000000.

\startbuffer
\hbox to 2cm {does it fit}               \the\overshoot
\hbox to 2cm {does it fit in here}       \the\overshoot
\hbox to 2cm {how much does fit in here} \the\overshoot
\stopbuffer

\typebuffer

This global state variables reports a dimension:

\startlines
\getbuffer
\stoplines

\stopsubsection

\startsubsection[title={Images and reused box objects},reference=sec:imagesandforms]

\topicindex {images}

In original \TEX\ image support is dealt with via specials. It's not a native
feature of the engine. All that \TEX\ cares about is dimensions, so in practice
that meant: using a box with known dimensions that wraps a special that instructs
the backend to include an image. The wrapping is needed because a special itself
is a whatsit and as such has no dimensions.

In \PDFTEX\ a special whatsit for images was introduced and that one {\em has}
dimensions. As a consequence, in several places where the engine deals with the
dimensions of nodes, it now has to check the details of whatsits. By inheriting
code from \PDFTEX, the \LUATEX\ engine also had that property. However, at some
point this approach was abandoned and a more natural trick was used: images (and
box resources) became a special kind of rules, and as rules already have
dimensions, the code could be simplified.

When direction nodes and (formerly local) par nodes also became first class
nodes, whatsits again became just that: nodes representing whatever you want, but
without dimensions, and therefore they could again be ignored when dimensions
mattered. And, because images were disguised as rules, as mentioned, their
dimensions automatically were taken into account. This separation between front
and backend cleaned up the code base already quite a bit.

In \LUAMETATEX\ we still have the image specific subtypes for rules, but the
engine never looks at subtypes of rules. That was up to the backend. This means
that image support is not present in \LUAMETATEX. When an image specification was
parsed the special properties, like the filename, or additional attributes, were
stored in the backend and all that \LUATEX\ does is registering a reference to an
image's specification in the rule node. But, having no backend means nothing is
stored, which in turn would make the image inclusion primitives kind of weird.

Therefore you need to realize that contrary to \LUATEX, {\em in \LUAMETATEX\
support for images and box reuse is not built in}! However, we can assume that
an implementation uses rules in a similar fashion as \LUATEX\ does. So, you can
still consider images and box reuse to be core concepts. Here we just mention the
primitives that \LUATEX\ provides. They are not available in the engine but can
of course be implemented in \LUA.

\starttabulate[|l|p|]
\DB command \BC explanation \NC \NR
\TB
\NC \tex {saveboxresource}             \NC save the box as an object to be included later \NC \NR
\NC \tex {saveimageresource}           \NC save the image as an object to be included later \NC \NR
\NC \tex {useboxresource}              \NC include the saved box object here (by index) \NC \NR
\NC \tex {useimageresource}            \NC include the saved image object here (by index) \NC \NR
\NC \tex {lastsavedboxresourceindex}   \NC the index of the last saved box object \NC \NR
\NC \tex {lastsavedimageresourceindex} \NC the index of the last saved image object \NC \NR
\NC \tex {lastsavedimageresourcepages} \NC the number of pages in the last saved image object \NC \NR
\LL
\stoptabulate

An implementation probably should accept the usual optional dimension parameters
for \type {\use...resource} in the same format as for rules. With images, these
dimensions are then used instead of the ones given to \tex {useimageresource} but
the original dimensions are not overwritten, so that a \tex {useimageresource}
without dimensions still provides the image with dimensions defined by \tex
{saveimageresource}. These optional parameters are not implemented for \tex
{saveboxresource}.

\starttyping
\useimageresource width 20mm height 10mm depth 5mm \lastsavedimageresourceindex
\useboxresource   width 20mm height 10mm depth 5mm \lastsavedboxresourceindex
\stoptyping

Examples or optional entries are \type {attr} and \type {resources} that accept a
token list, and the \type {type} key. When set to non|-|zero the \type {/Type}
entry is omitted. A value of 1 or 3 still writes a \type {/BBox}, while 2 or 3
will write a \type {/Matrix}. But, as said: this is entirely up to the backend.
Generic macro packages (like \type {tikz}) can use these assumed primitives so
one can best provide them. It is probably, for historic reasons, the only more or
less standardized image inclusion interface one can expect to work in all macro
packages.

\stopsubsection

\startsubsection[title={\prm {dbox}}]

This primitive is a variant on \prm {vbox} in the sense that when it gets
appended to a vertical list the height of the topmost line or rule as well as the
depth of the box are taken into account when interline space is calculated.

\stopsubsection


\startsubsection[title={\prm {hpack}, \prm {vpack}, \prm {tpack} and \prm {dpack}}]

\topicindex {packing}

These three primitives are the equivalents of \prm {hbox}, \prm {vbox}, \prm
{vtop} and \prm {dbox} but they don't trigger the packaging related callbacks.
Of course one never know if content needs a treatment so using them should be
done with care. Apart from accepting more keywords (and therefore options) the
normal box behave the same as before.

\stopsubsection

\startsubsection[title={\prm {vcenter}}]

The \prm {vcenter} builder also works in text mode.

\stopsubsection

\startsubsection[title={\prm {unhpack}, \prm {unvpack}}]

\topicindex {unpacking}

These two are somewhat experimental. They ignore the accumulated pre- and
postmigrated material bound to a box. I needed it for some experiment so the
functionality might change when I really need it.

\stopsubsection

\startsubsection[title={\prm {gleaders} and \prm {uleaders}},reference=sec:gleaders]

\topicindex {leaders}

This type of leaders is anchored to the origin of the box to be shipped out. So
they are like normal \prm {leaders} in that they align nicely, except that the
alignment is based on the {\it largest\/} enclosing box instead of the {\it
smallest\/}. The \type {g} stresses this global nature. The \prm {uleaders} are
used for flexible boxes and are discussed elsewhere.

\stopsubsection

\stopsection

\startsection[title={Paragraphs}]

\startsubsection[title=Freezing]

In \LUAMETATEX\ we store quite some properties with a paragraph. Where in traditional
\TEX\ the properties that are set when the paragraph broken into lines are used, here
we can freeze them.

{\em At some point this section will describe \prm {autoparagraphmode}, \prm
{everybeforepar}, \prm {snapshotpar}, \prm {wrapuppar}, etc. For the moment the
manuals that come with \CONTEXT\ have to do.}

% The concept of paragraph in \TEX\ can be a bit confusing, and what follows here
% is even more so. In \LUAMETATEX\ we distinguish three different cases:
%
% \startitemize[packed]
% \starthead {normal}
%     This state is entered when we have an explicit \type {\par}.
% \stophead
% \starthead {inserted}
%     This state is entered when \TEX\ forces a new paragraph which can happen when
%     we automatically change to horizontal mode.
% \stophead
% \starthead{newline}
%     This state is entered when an empty line is encoduntered.
% \stophead
% \stopitemize
%
% When defining macros, a \type {\par} can be a delimiter and an empty line is then
% equivalent to that. With \type {\autoparagraphmode} we can change this behaviour.
% The bits set in this variable determines how \type {\par} tokens are interpreted
% and processed in different situations. It is good to know that this experimental
% feature is pretty much \CONTEXT\ specific. Here are a few characteristics:
%
% \startitemize[packed]
%     \startitem
%         When any bit is set, a par token is appended when with property inserted.
%         The reason behind different properties (mentioned above) is that we can
%         intercept them with callbacks.
%     \stopitem
%     \startitem
%         When bit~1 (text) is set, \type {\par} will be appended to a string when
%         a token list is serialized.
%     \stopitem
%     \startitem
%         When bit~2 (macro) is set when a macro is defined there is an explicit
%         check for a par token when \type {\par} is part of the preamble.
%     \stopitem
%     \startitem
%         When bit~4 (go on) is not set we enter the same state as a new line.
%         After that when bit~1 (text) is set, a regular par token command is
%         injected (with an associated newline state), otherwise the meaning of
%         \type {\par} kicks in (users can have redefined \type {\par}).
%     \stopitem
% \stopitemize
%
% In \CONTEXT\ we currently default to one, because we still have a few \type {\par}
% delimited macros but these will go and then we will set the mode to two. This
% means that in \CONTEXT\ such macro expects an explicit \type {\par} and not an
% empty line which in turn encourages users to use the proper alternatives. We
% anyway don't support redefined \type {\par} tokens. Therefore, when we load for
% instance tikz, we set the mode to zero (normal \TEX) and afterwards back to one.
% This is still an experimental feature that we occasionally review so don't bother
% us with questions about it (just don't set the mode).


\stopsubsection

\startsubsection[title=Penalties]

In addition to the penalties introduced in \ETEX, we also provide \prm
{orphanpenalty} and \prm {orphanpenalties}. When we're shaping a paragraph
an additional \prm {shapingpenalty} can be injected. This penalty gets
injected instead of the usual penalties when the following bits are set in
\prm {shapingpenaltiesmode}:

\starttabulate[|l|l|p|]
\DB value        \BC ignored \NC \NR
\TB
\NC \type {0x01} \NC interlinepenalty \NC \NR
\NC \type {0x02} \NC widowpenalty     \NC \NR
\NC \type {0x04} \NC clubpenalty      \NC \NR
\NC \type {0x08} \NC brokenpenalty    \NC \NR
\LL
\stoptabulate

When none of these is set the shaping penalty will be added. That way one can
prevent a page break inside a shape.

\stopsubsection

\startsubsection[title=Criteria]

The linebreak algorithm uses some heuristics for determining the badness of a
line. In most cases that works quite well. Of course one can run into a bad
result when one has a large document of weird (extreme) constraints and it can be
tempting to mess around with parameters which then of course can lead to bad
results in other places. A solution is is to locally tweak penalties or looseness
but one can also just accept the occasional less optimal result (after all there
are plenty occasions to make a document look bad otherwise so best focus on the
average first). That said, it is tempting to see if changing the hard codes
criteria makes a difference. Experiments with this demonstrated the usual: when
asked what looks best contradictions mix with expectations and being triggered by
events that one related to \TEX, like successive hyphenated lines.

The \prm {linebreakcriterium} parameter can be set to a value made from four bytes. We're
not going to explain the magic numbers because they come from and are discussed in original
\TEX. It is enough to know that we have four criteria:

\starttabulate[|l|l|p|]
\DB magic \BC bound to   \NC bytes      \NC \NR
\TB
\NC 12    \NC semi tight \NC 0x7F...... \NC \NR
\NC 12    \NC decent     \NC 0x..7F.... \NC \NR
\NC 12    \NC semi loose \NC 0x....7F.. \NC \NR
\NC 99    \NC loose      \NC 0x......7F \NC \NR
\LL
\stoptabulate

These four values can be changed according to the above pattern and are limited
to the range 1\endash127 which is plenty especially when one keeps in mind that
the actual useful values sit around the 12 anyway. Values outside the range (and
therefore an all|-|over zero assignment) makes the defaults kick in.

The original decisions are made in the following way:

\starttyping
function loose(badness)
    if badness > loose_criterium then
        return very_loose_fit
    elseif badness > decent_criterium then
        return loose_fit
    else {
        return decent_fit
    end
end

function tight(badness)
    if badness > decent_criterium then
        return tight_fit
    else {
        return decent_fit
    end
end
\stoptyping

while in \LUAMETATEX\ we use (again in \LUA speak):

\starttyping
function loose(badness)
    if badness > loose then
        return very_loose_fit
    elseif badness > semi_loose then
        return semi_loose_fit
    elseif badness > decent then
        return loose_fit
    else
        return decent_fit
    end
end

function tight(badness)
    if badness > semi_tight then
        return semi_tight_fit
    else if badness > decent then
        return tight_fit
    else
        return decent_fit
    end
end
\stoptyping

So we have a few more steps to play with. But don't be disappointed when it
doesn't work out as you expect. Don Knuth did a good job on the heuristics and
after many decades there is no real need to change something. Consider it a
playground.

The parameter \prm {ignoredepthcriterium} is set to -1000pt at startup and is a
special signal for \prm {prevdepth}. You can change the value locally for
educational purposes but best not mess with this standard value in production
code unless you want special effects.

\stopsubsection

\stopsection

\startsection[title={Inserts}]

Inserts are tightly integrated into the page builder. Depending on penalties and
available space they end up on the same page as were they got injected or they
move to following pages, either or not split.

In traditional \TEX\ inserts are controlled by registers. A quadruple of box,
skip, dimen and count registers with the same number acts as an insert class.
Details can be found in the \TEX book. A side effect of this is that we only have
these four properties bound to class, other properties of inserts are driven by
shared parameters. Another side effect is that register management has to make
sure that these foursome get \quote {allocates} as set and not clashes with other
register allocations.

In \LUAMETATEX\ you can set the \prm {insertmode} to a non zero value in which case
inserts are not using the register pool but have their own (global) resources. For
now this is mode driven (for compatibility reasons) and once set or when an
insert has been accessed, this mode is frozen, so  this parameter can be set
very early in the macro package loading process.


\starttabulate[|l|l|p|]
\DB primitive               \BC traditional            \BC explanation \NC \NR
\TB
\NC \prm {insertdistance}   \NC skip                   \NC the space before the first instance (on a page) \NC \NR
\NC \prm {insertmultiplier} \NC count                  \NC a factor that is used to calculate the height used \NC \NR
\NC \prm {insertlimit}      \NC dimen                  \NC the maximum amount of space on a page to be taken \NC \NR
\NC \prm {insertpenalty}    \NC \prm {insertpenalties} \NC the floating penalty (used when set) \NC \NR
\NC \prm {insertmaxdepth}   \NC \prm {maxdepth}        \NC the maximum split depth (used when set) \NC \NR
\NC \prm {insertstorage}    \NC                        \NC signals that the insert has to be stored for later \NC \NR
\NC \prm {insertheight}     \NC \prm {ht} box / index  \NC the accumulated height of the inserts so far \NC \NR
\NC \prm {insertdepth}      \NC \prm {dp} box / index  \NC the current depth of the inserts so far \NC \NR
\NC \prm {insertwidth}      \NC \prm {wd} box / index  \NC the width of the inserts \NC \NR
\NC \prm {insertbox}        \NC box / index            \NC the boxed content \NC \NR
\NC \prm {insertcopy}       \NC box / index            \NC a copy of the boxed content \NC \NR
\NC \prm {insertunbox}      \NC box / index            \NC the unboxed content \NC \NR
\NC \prm {insertuncopy}     \NC box / index            \NC a copy of the unboxed content \NC \NR
\NC \prm {insertuncopy}     \NC box / index            \NC a copy of the unboxed content \NC \NR
\NC \prm {insertprogress}   \NC box / index            \NC the currently accumulated height \NC \NR
\LL
\stoptabulate

These primitives takes an insert class number. The \prm {insertpenalties}
primitives is unchanged, as is the \LUATEX\ \prm {insertheights} one. When \prm
{insertstoring} is set 1, all inserts that have their storage flag set will be
saved. Think of a multi column setup where inserts have to end up in the last
column. If there are three columns, the first two will store inserts. Then when
the last column is dealt with \prm {insertstoring} can be set to 2 and that will
signal the builder that we will inject the inserts. In both cases, the value of
this register will be set to zero so that it doesn't influence further
processing. You can use \prm {ifinsert} to check if an insert box is void. More
details about these (probably experimental for a while) features can be found in
documents that come with \CONTEXT.

A limitation of inserts is that when they are buried too deep, a property they
share with inserts, they become invisible This can be dealt with by the migration
feature described in an upcoming section.

The \LUAMETATEX\ engine has some tracing built in that is enabled by setting \prm
{tracinginserts} to a positive value.

\stopsection

\startsection[title={Marks}]

\topicindex {marks}

Marks are kind of signal nodes in the list that refer to stored token lists. When
a page has been split off and is handed over to the output routine these signals
are resolved into first, top and bottom mark references that can (for instance)
be used for running headers.

In \ETEX\ the standard \TEX\ primitives \prm {mark}, \prm {firstmark}, \prm
{topmark}, \prm {botmark}, \prm {splitfirstmark} and \prm {splitbotmark} have
been extended with plural forms that accent a number before the token list. That
number indicates a mark class.

In addition to the mark fetch commands, we also have access to the last set
mark in the given class with \prm {currentmarks}:

\startsyntax
\currentmarks <16-bit number>
\stopsyntax

A problem with marks is that one cannot really reset them. Mark states are kept
in the node lists and only periodically the state is snapshot into the global
state variables. The \LUATEX\ engine can reset these global states with \prm
{clearmarks} but that's only half a solution. In \LUAMETATEX\ we have \prm
{flushmarks} which, like \prm {marks}, puts a node in the list that does a reset.
This permits implementing controlled resets of specific marks at the cost of a
possible interfering mode, but that can normally be dealt with rather well.

The \prm {clearmarks} primitive complements the \ETEX\ mark primitives and clears
a mark class completely, resetting all three connected mark texts to empty. It is
an immediate command (no synchronization node is used).

\startsyntax
\clearmarks <16-bit number>
\stopsyntax

The \prm {flushmarks} variant is delayed but puts a (mark) node in the list as
signal (we could have gone for a keyword to \prm {marks} instead).

\startsyntax
\flushmarks <16-bit number>
\stopsyntax

Another problem with marks is that when they are buried too deep, a property they
share with inserts, they become invisible. This can be dealt with by the
migration feature described in the next section.

The \LUAMETATEX\ engine has some tracing built in that is enabled by setting \prm
{tracingmarks} to a positive value. When set to~1 the page builder shows the set
values, and when set to a higher value details about collecting them are shown.

\stopsection

\startsection[title={Adjusts}]

The \prm {vadjust} primitive injects something in the vertical list after the
line where it ends up. In \PDFTEX\ the \type {pre} keyword was added so that one
could force something before a previous line (actually this was something that we
needed in \CONTEXT\ \MKII). The \LUAMETATEX\ engine also supports the \type {post}
keyword.

We support a few more keywords: \type {before} will prepend the adjustment to the
already given one, and \type {after} will append it. The \type {index} keyword
expects an integer and relates that to the current adjustment. This index is
passed to an (optional) callback when the adjustment is finally moved to the
vertical list. That move is actually delayed because like inserts and marks these
(vertical) adjustments can migrate to the \quote {outer} vertical level.

The main reason for the index having no influence on the order is that this
primitive already could be used multiple times and order is determined by usage.
\footnote {Under consideration is to let the callback mess with the flushing
order.}

The \LUAMETATEX\ engine has some tracing built in that is enabled by setting \prm
{tracingadjusts} to a positive value. Currently there is not that much tracing
which is why the value has to be at least 2 in order to be compatible with other
(detailed) tracers.

\stopsection

\startsection[title={Migration}]

There are a few injected node types that are used to track information: marks,
inserts and adjusts (see previous sections). Marks are token lists that can be
used to register states like section numbers and titles they are synchronized in
the page builder when a page is shipped out. Inserts are node lists that get
rendered and relate to specific locations and these are flushed with the main
vertical list which also means that in calculating page breaks they need to be
taken into account. An Adjust is material that gets injected before or after a
line. Strictly spoke local boxes also in this repertoire but they are dealt with
in the par builder.

A new primitive \prm {automigrationmode} can be used to let deeply burried marks
and inserts bubble up to the outer level.

\starttabulate[|c|p|]
\DB value \BC explanation \NC \NR
\TB
\NC \the\markautomigrationcode   \NC migrate marks in the par builder \NC \NR
\NC \the\insertautomigrationcode \NC migrate inserts in the par builder  \NC \NR
\NC \the\adjustautomigrationcode \NC migrate adjusts in the par builder  \NC \NR
\NC \the\preautomigrationcode    \NC migrate prebox material in the page builder \NC \NR
\NC \the\postautomigrationcode   \NC migrate postbox material in the page builder \NC \NR
\LL
\stoptabulate

If you want to migrate marks and inserts you need to set all these flags. Migrated
marks and inserts end up as post|-|box properties and will be handled in the page
builder as such. At the \LUA\ end you can add pre- and post|-|box material too.

The primitive register \prm {holdingmigrations} is a bitset that can be used to temporarily
disable migrations. It is a generalization of \prm {holdinginserts}.

\starttabulate[|cT|p|]
\DB value \BC explanation \NC \NR
\TB
\NC 0x01  \NC marks   \NC \NR
\NC 0x02  \NC inserts \NC \NR
\NC 0x04  \NC adjusts \NC \NR
\LL
\stoptabulate

Migrates material is bound to boxes so boxed material gets unboxed it is taken
into account, but you should be aware of potential side effects. But then, marks,
inserts and adjusts always demanded care.

\stopsection

\startsection[title={Pages}]

The page builder can be triggered by (for instance) a penalty but you can also
use \prm {pageboundary}. This will trigger the page builder but not leave
anything behind.

{\em In due time we will discuss \prm {pagevsize}, \prm {pageextragoal} and \prm
{lastpageextra} but for now we treat them as very experimental and they will be
tested in \CONTEXT, also in discussion with users.}

\stopsection

\startsection[title={Paragraphs}]

The numeric primitive \prm {lastparcontext} inspector reports the current context
in which a paragraph triggering commands happened. The numbers can be queried
with \type {tex.getparcontextvalues()} and currently are: \showvaluelist
{tex.getparcontextvalues()}. As with the other \type {\last...} primitives this
variable is global.

Traditional \TEX\ has the \prm {parfillskip} parameter that determines the way
the last line is filled. In \LUAMETATEX\ we also have \prm {parfillleftskip}. The
counterparts for the first line are \prm {parinitleftskip} and \prm
{parinitrightskip}.

\startbuffer
\leftskip        2em
\rightskip       \leftskip
\parfillskip     \zeropoint plus 1 fill
\parfillleftskip \parfillskip
\parinitleftskip \parfillleftskip
\parinitrightskip\parfillleftskip
\input ward
\stopbuffer

\typebuffer This results in: \par \start \em \getbuffer \par \stop

An additional tracing primitive \prm {tracingfullboxes} reports details about the
encountered overfull boxes. This can be rather verbose!

Normally \TEX\ will insert an empty hbox when paragraph indentation is requested
but when the second bit in \prm {normalizelinemode} has been set \LUAMETATEX\
will in a glue node instead. You can zero the set value with \prm {undent} unless
of course some more has been inserted already.

\startbuffer
\parinitleftskip1cm \parindent 1cm \indent test \par
\parinitleftskip1cm \parindent 1cm \undent test \par
\parinitleftskip1cm \parindent 1cm \indent \undent test \par
\parinitleftskip1cm \parindent 1cm \indent \strut \undent test \par
\stopbuffer

\typebuffer \startpacked \getbuffer \stoppacked

By setting \prm {tracingpenalties} to a positive value penalties related to
windows, clubs, lines etc.\ get reported to the output channels.

\stopsection

\startsection[title={Local boxes}]

As far as I know the \OMEGA/\ALEPH\ local box mechanism is mostly in those
engines in order to support repetitive quotes. In \LUATEX\ this mechanism has
been made more robust and in \LUAMETATEX\ it became more tightly integrated in
the paragraph properties. In order for it to be more generic and useful, it got
more features. For instance it is a bit painful to manage with respect to
grouping (which is a reason why it's not that much used). The most interesting
property is that the dimensions are taken into account when a paragraph is
broken into lines.

There are three commands: \prm {localleftbox}, \prm {localrightbox} and the
\LUAMETATEX\ specific \prm {localmiddlebox} which is basically a right box but
when we pass these boxes to a callback they can be distinguished (we could have
used the index but this was a cheap extra signal so we keep it).

These commands take optional keywords. The \type {index} keyword has to be
followed by an integer. This index determines the order which doesn't introduce a
significant compatibility issue: local boxes are hardly used and originally had
only one instance.

The \type {par} keyword forces the box to be added to the current paragraph head.
This permits setting them when a paragraph has already started. The
implementation of these boxes is done via so called (local) paragraph nodes and
there is one at the start of each paragraph.

The \type {local} keyword tells this mechanism not to update the registers that
keep these boxes. In that case a next paragraph will start fresh. The \type
{keep} option will do the opposite and retain the box after a group ends.

The commands: \prm {localleftboxbox}, \prm {localrightboxbox} and \prm
{localmiddleboxbox} return a copy of the current related register content.

\stopsection

\startsection[title={Leaders}]

Leaders are flexible content that are basically just seen as glue and it is up to
the backend to apply the effective glue to the result as seen in the backend
(like a rule of box). This means that the frontend doesn't do anything with the
fact that we have a regular \prm {leaders}, a \prm {gleaders}, \prm {xleaders} or
\prm {cleaders}. The \prm {uleaders} that has been added in \LUAMETATEX\ is just
that: an extra leader category. The main difference is that the width of the
given box is added to the glue. That way we create a stretchable box.

\startbuffer
\unexpandedloop 1 30 1 {x             \hbox{1 2 3}                                                           x }
\unexpandedloop 1 30 1 {x {\uleaders \hbox{1 2 3}\hskip 0pt plus 10pt               minus 10pt\relax}        x }
\unexpandedloop 1 30 1 {x {\uleaders \hbox{1 2 3}\hskip 0pt plus  \interwordstretch minus \interwordshrink}  x }
\unexpandedloop 1 30 1 {x {\uleaders \hbox{1 2 3}\hskip 0pt plus 2\interwordstretch minus 2\interwordshrink} x }
\stopbuffer

\typebuffer

Here are some examples:

\startlines
\getbuffer
\stoplines

So the flexibility fo the box plays a role in the line break calculations. But in
the end the backend has to do the work.

\startbuffer[a]
{\green \hrule width \hsize} \par \vskip2pt
\vbox to 40pt {
    {\red\hrule width \hsize} \par \vskip2pt
    \vbox {
        \vskip2pt {\blue\hrule width \hsize} \par
        \vskip 10pt plus 10pt minus 10pt
        {\blue\hrule width \hsize} \par \vskip2pt
    }
    \vskip2pt {\red\hrule width \hsize} \par
}
\vskip2pt {\green \hrule width \hsize} \par
\stopbuffer

\startbuffer[b]
{\green \hrule width \hsize} \par \vskip2pt
\vbox to 40pt {
    {\red\hrule width \hsize} \par \vskip2pt
    \uleaders\vbox {
        \vskip2pt {\blue\hrule width \hsize} \par
        \vskip 10pt plus 10pt minus 10pt
        {\blue\hrule width \hsize} \par \vskip2pt
    }\vskip 0pt plus 10pt minus 10pt
    \vskip2pt {\red\hrule width \hsize} \par
}
\vskip2pt {\green \hrule width \hsize} \par
\stopbuffer

\typebuffer[a]

with

\typebuffer[b]

In the first case we get the this:

\startlinecorrection
\getbuffer[a]
\stoplinecorrection

but with \prm {uleaders} we get:

\startlinecorrection
\normalizeparmode\zerocount
\getbuffer[b]
\stoplinecorrection

or this:

\startlinecorrection
\normalizeparmode"FF
\getbuffer[b]
\stoplinecorrection

In the second case we flatten the leaders in the engine by setting the second bit
in the \prm {normalizeparmode} parameter (\type {0x2}). We actually do the same
with \prm {normalizelinemode} where bit 10 is set (\type {0x200}). The \type
{delay} keyword can be passed with a box to prevent flattening. If we don't do
this in the engine, the backend has to take care of it. In principle this permits
implementing variants in a macro package. Eventually there will be plenty examples in
the \CONTEXT\ code base and documentation. Till then, consider this experimental.

\stopsection

\startsection[title=Alignments]

The primitive \prm {alignmark} duplicates the functionality of \type {#} inside
alignment preambles, while \prm {aligntab} duplicates the functionality of \type
{&}. The \prm {aligncontent} primitive directly refers to an entry so that one
does not get repeated.

Alignments can be traced with \prm {tracingalignments}. When set to~1 basics
usage is shown, for instance of \prm {noalign} but more interesting is~2 or more:
you then get the preambles reported.

The \prm {halign} (tested) and \prm {valign} (yet untested) primitives accept a
few keywords in addition to \type {to} and \type {spread}:

\starttabulate[|l|p|]
\DB keyword \BC explanation \NC \NR
\TB
\NC \type {attr}     \NC set the given attribute to the given value \NC \NR
\NC \type {callback} \NC trigger the \type {alignment_filter} callback \NC \NR
\NC \type {discard}  \NC discard zero \prm {tabskip}'s \NC \NR
\NC \type {noskips}  \NC don't even process zero \prm {tabskip}'s \NC \NR
\NC \type {reverse}  \NC reverse the final rows \NC \NR
\LL
\stoptabulate

In the preamble the \prm {tabsize} primitive can be used to set the width of a
column. By doing so one can avoid using a box in the preamble which, combined
with the sparse tabskip features, is a bit easier on memory when you produce
tables that span hundreds of pages and have a dozen columns.

The \prm {everytab} complements the \prm {everycr} token register but is sort of
experimental as it might become more selective and powerful some day.

The two primitives \prm {alignmentcellsource} and \prm {alignmentwrapsource} that
associate a source id (integer) to the current cell and row (line). Sources and
targets are experimental and are being explored in \CONTEXT\ so we'll see where
that ends up in.

{\em todo: callbacks}

\stopsection

\stopchapter

\stopcomponent
