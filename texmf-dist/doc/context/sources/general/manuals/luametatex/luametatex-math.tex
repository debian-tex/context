% language=us runpath=texruns:manuals/luametatex

\environment luametatex-style

\startcomponent luametatex-math

\startchapter[reference=math,title={Math}]

\startsection[title={Traditional alongside \OPENTYPE}]

\topicindex {math}

Because we started from \LUATEX, by the end of 2021 this chapter started with
this, even if we already reworked the engine:

\startquotation
At this point there is no difference between \LUAMETATEX\ and \LUATEX\ with
respect to math. \footnote {This might no longer be true because we have more
control options that define default behavior and also have a more extensive
scaling model. Anyway, it should not look worse, and maybe even a bit better.}
The handling of mathematics in \LUATEX\ differs quite a bit from how \TEX82 (and
therefore \PDFTEX) handles math. First, \LUATEX\ adds primitives and extends some
others so that \UNICODE\ input can be used easily. Second, all of \TEX82's
internal special values (for example for operator spacing) have been made
accessible and changeable via control sequences. Third, there are extensions that
make it easier to use \OPENTYPE\ math fonts. And finally, there are some
extensions that have been proposed or considered in the past that are now added
to the engine.

You might be surprised that we don't use all these new control features in
\CONTEXT\ \LMTX\ but who knows what might happen because users drive it. The main
reason for adding so much is that I decided it made more sense to be complete now
than gradually add more and more. At some point we should be able to say \quote
{This is it}. Also, when looking at these features, you need to keep in mind that
when it comes to math, \LATEX\ is the dominant macro package and it never needed
these engine features, so most are probably just here for exploration purposes.
\stopquotation

Although we still process math as \TEX\ does, there have been some fundamental
changes to the machinery. Most of that is discussed in documents that come with
\CONTEXT\ and in Mikael Sundqvist math manual. Together we explored some new ways
to deal with math spacing, penalties, fencing, operators, fractions, atoms and
other features of the \TEX\ engine. We started from the way \CONTEXT\ used the
already present functionality combine with sometimes somewhat dirty (but on the
average working well) tricks.

It will take a while before this chapter is updated. If you find errors or things
missing, let me know. A lot of pairwise spacing primitives were dropped but also
quite a bit of new ones introduced to control matters. Much in \LUAMETATEX\ math
handling is about micro|-|typography and for us the results are quite visible.
But, as far as we know, there have never been complaints or demands in the
direction of the features discussed here. Also, \TEX\ math usage outside
\CONTEXT\ is rather chiselled in stone (already for nearly three decades) so we
don't expect other macro packages to use the new features anyway.

\stopsection

\startsection[title=Intermezzo]

It is important to understand a bit how \TEX\ handles math. The math engine
is a large subsystem and basically can be divided in two parts: convert
sequential input into a list of nodes where math related ones actually are
sort of intermediate and therefore called noads.

In text mode entering \type {abc} results in three glyph nodes and \type {a b
c} in three glyph nodes separated by (spacing) glue. Successive glyphs can be
transformed in the font engine later on, just as hyphenation directive can
be added. Eventually one (normally) gets a mix of glyphs, font kerns from
a sequence of glyphs

In math mode \type {abc} results in three simple ordinary noads and \type {a b c}
is equivalent to that: three noads. But \type {a bc} results in two ordinary
noads where the second one has a sublist of two ordinary noads. Because
characters have class properties, \type {( a + b = c )} results in a simple open
noad, a simple ordinary, a simple binary, a simple ordinary, a simple relation, a
simple ordinary and simple close noad. The next samples show a bit of this; in
order to see the effects of spacing between ordinary atoms set it to \type {9mu}.

\startbuffer
\typebuffer[sample]
% \tracingmath1\tracingonline1
\startlinecorrection
\setmathspacing\mathordinarycode\mathordinarycode\allmathstyles9mu
\mathgroupingmode\zerocount
\scale[scale=2000]{\showmakeup[mathglue]\showboxes\mathspacingmode\plusone\getbuffer[sample]}
\stoplinecorrection
\blank[2*line]
\stopbuffer

\startbuffer[sample]
$a b c$ \quad $a bc$ \quad $abc$
\stopbuffer

\getbuffer

With \type {\tracingmath 1} we get this logged:

\starttyping
> \inlinemath=
\noad[ord][...]
.\nucleus
..\mathchar[ord][...], family "0, character "61
\noad[ord][...]
.\nucleus
..\mathchar[ord][...], family "0, character "62
\noad[ord][...]
.\nucleus
..\mathchar[ord][...], family "0, character "63

> \inlinemath=
\noad[ord][...]
.\nucleus
..\mathchar[ord][...], family "0, character "61
\noad[ord][...]
.\nucleus
..\mathchar[ord][...], family "0, character "62
\noad[ord][...]
.\nucleus
..\mathchar[ord][...], family "0, character "63

> \inlinemath=
\noad[ord][...]
.\nucleus
..\mathchar[ord][...], family "0, character "61
\noad[ord][...]
.\nucleus
..\mathchar[ord][...], family "0, character "62
\noad[ord][...]
.\nucleus
..\mathchar[ord][...], family "0, character "63
\stoptyping

\startbuffer[sample]
${a} {b} {c}$ \quad ${a} {bc}$ \quad ${abc}$
\stopbuffer

\getbuffer

If the previous log surprises you, that might be because in \CONTEXT\ we set up the
engine differently: curly braces don't create ordinary atoms. However, when we
set \type {\mathgroupingmode 0} we return to what the engine normally does.

\starttyping
> \inlinemath=
\noad[ord][...]
.\nucleus
..\mathchar[ord][...], family "0, character "61
\noad[ord][...]
.\nucleus
..\mathchar[ord][...], family "0, character "62
\noad[ord][...]
.\nucleus
..\mathchar[ord][...], family "0, character "63

> \inlinemath=
\noad[ord][...]
.\nucleus
..\mathchar[ord][...], family "0, character "61
\noad[ord][...]
.\nucleus
..\submlist[0][...][tracing depth 5 reached]

> \inlinemath=
\noad[ord][...]
.\nucleus
..\submlist[0][...][tracing depth 5 reached]
\stoptyping

From the first example you can imagine what these sub lists look like: a list of
ordinary atoms. The final list that is mix of nodes and yet unprocessed noads get
fed into the math|-|to|-|hlist function and eventually the noads become glyphs,
boxes, kerns, glue and whatever makes sense. A lot goes on there: think scripts,
fractions, fences, accents, radicals, spacing, break control.

An example of more tricky scanning is shown here:

\starttyping
a +   1 \over 2   + b
a +  {1}\over{2}  + b
a + {{1}\over{2}} + b
\stoptyping

In this case the \type {\over} makes \TEX\ reconsider the last noad, remove if
from the current list and save it for later, then scan for a following atom a
single character turned atom or a braced sequence that then is an ordinary noad.
In the end a fraction noad is made. When that gets processed later specific
numerator and denominator styles get applied (explicitly entered style nodes of
course overload this for the content). The fact that this construct is all about
(implicit) ordinary noads, themselves captured in noads, combined with the wish
for enforced consistent positioning of numerator and denominator, plus style
overload, color support and whatever comes to mind means that in practice one
will use a \type {\frac} macro that provides all that control. \footnote {There
are now a \prm {Uover} primitives that look ahead and then of course still
treat curly braces as math lists to be picked up.}

A similar tricky case is this:

\starttyping
      ( a +       ( b - c        ) + d        )
\left ( a + \left ( b - c \right ) + d \right )
\stoptyping

Here the first line creates a list of noads but the second line create a fenced
structure that is handled as a whole in order to make the fences match. \footnote
{Actually instead of such a structure there could have been delimiters with
backlinks but one never knows what happens with these links when processing
passes are made so that fragility is avoided.} A fence noad will not break across
lines as it is boxed and that is the reason why macro packages have these \type
{\bigg} macros: they explicitly force a size using some trickery. In \LUAMETATEX\
a fence object can actually be unpacked when the class is configured as such. It
is one of the many extensions we have.

There are some peculiar cases that one can run into but that actually are
mentioned in the \TEX\ book. Often these reasons for intentional side effects
become clear when one thinks of the average usage but unless one is willing to
spend time on the \quote {fine points of math} they can also interfere with
intentions. The next bits of code are just for the reader to look at. Try to
predict the outcome. Watch out: in \LMTX\ the outcome is not what one gets by
default in \LUATEX, \PDFTEX\ or regular \TEX. \footnote {One can set \typ
{\mathgroupingmode = 0} to get close.}

\starttyping
$ 1 {\red +} 2$\par
$ 1 \color[red]{+} 2$\par
$ 1 \mathbin{\red +} 2$\par
$ a + - b + {- b} $
$ a \pm - b - {+ b} $
$ - b $
$ {- b} $
\stoptyping

The message here is that when a user is coding the mindset with respect to
grouping using curly braces has to be switched to math mode too. And how many
users really read the relevant chapters of the \TEX\ book a couple of times (as
much makes only sense after playing with math in \TEX)? Even if one doesn't grasp
everything it's a worthwhile read. Also consider this: did you really ask for an
ordinary atom when you uses curly braces where no lists were expected? And what
would have happened when ordinary related spacing had been set to non|-|zero?

All the above (and plenty more) is why in \CONTEXT\ \LMTX\ we make extensive use
of some \LUAMETATEX\ features, like: additional atom classes, configurable inter
atom spacing and penalties, pairwise atom rules that can change classes, class
based rendering options, more font parameters, configurable style instead of hard
coded ones in constructs, more granular spacing, etc. That way we get quite
predictable results but also drop some older (un)expected behavior and side
effects. It is also why we cannot show many examples in the \LUAMETATEX\ manual:
it uses \CONTEXT\ and we see no reason to complicate out lives (and spend energy
on) turning off all the nicely cooperating features (and then for sure forgetting
one) just for the sake of demos. It also gave us the opportunity to improve
existing mechanisms and|/|or at least simplify their sometimes complex code.

One last word here about sequences of ordinary atoms: the traditional code path
feeds ordinary atoms into a ligature and kerning routine and does that when it
encounters one. However, in \OPENTYPE\ we don't have ligatures not (single) kerns
so there that doesn't apply. As we're not aware of traditional math fonts with
ligatures and no one is likely to use these fonts with \LUAMETATEX\ the ligature
code has been disabled. \footnote {It might show up in a different way if we feel
the need in which case it's more related to runtime patches to fonts and class
bases ligature building.} The kerning has been redone a bit so that it permits us
to fine tune spacing (which in \CONTEXT\ we control with goodie files). The
mentioned routine can also add italic correction, but that happens selectively
because it is driven by specifications and circumstances. It is one of the places
where the approach differs from the original, if only for practical reasons.

\stopsection

\startsection[title={Grouping with \prm {beginmathgroup} and \prm {endmathgroup}}]

These two primitives behave like \prm {begingroup} and \prm {endgroup} but
restore a style change inside the group. Style changes are actually injecting s
special style noad which makes them sort of persistent till the next explicit
change which can be confusing. This additional grouping model compensates for
that.

\stopsection

\startsection[title={Unicode math characters}]

\topicindex {math+\UNICODE}
\topicindex {\UNICODE+math}

For various reasons we need to encode a math character in a 32 bit number and
because we often also need to keep track of families and classes the range of
characters is limited to 20 bits. There are upto 64 classes (which is a lot more
than in \LUATEX) and 64 families (less than in \LUATEX). The upper limit of
characters is less that what \UNICODE\ offers but for math we're okay. If needed
we can provide less families.

The math primitives from \TEX\ are kept as they are, except for the ones that
convert from input to math commands: \type {mathcode}, and \type {delcode}. These
two now allow for the larger character codes argument on the left hand side of
the equals sign. The number variants of some primitives might be dropped in favor
of the primitives that read more than one separate value (class, family and
code), for instance:

\starttyping
\def\overbrace{\Umathaccent 0 1 "23DE }
\stoptyping

The altered \TEX82 primitives are:

\starttabulate[|l|l|r|c|l|r|]
\DB primitive       \BC min \BC max    \BC \kern 2em \BC min \BC max    \NC \NR
\TB
\NC \prm {mathcode} \NC 0   \NC 10FFFF \NC =         \NC 0   \NC 8000   \NC \NR
\NC \prm {delcode}  \NC 0   \NC 10FFFF \NC =         \NC 0   \NC FFFFFF \NC \NR
\LL
\stoptabulate

The unaltered ones are:

\starttabulate[|l|l|r|]
\DB primitive          \BC min \BC max     \NC \NR
\TB
\NC \prm {mathchardef} \NC 0   \NC    8000 \NC \NR
\NC \prm {mathchar}    \NC 0   \NC    7FFF \NC \NR
\NC \prm {mathaccent}  \NC 0   \NC    7FFF \NC \NR
\NC \prm {delimiter}   \NC 0   \NC 7FFFFFF \NC \NR
\NC \prm {radical}     \NC 0   \NC 7FFFFFF \NC \NR
\LL
\stoptabulate

% In \LUATEX\ we support the single number primitives *with \type {num} in their
% name) conforming the \XETEX\ method. For the moment that still works but you need
% to figure out the number yourself. The split number variants are more future safe
% with respect to classes and families. We don't document \prm {Umathcharnumdef},
% \prm {Umathcharnum}, \prm {Umathcodenum} and \prm {Udelcodenum} here any longer.

In \LUATEX\ we support the single number primitives *with \type {num} in their
name) conforming the \XETEX\ method. These primitives have been dropped in
\LUAMETATEX\ because we use different ranges and properties, so these numbers
have no (stable) meaning.

\starttabulate[|l|l|c|c|c|]
\DB primitive           \BC        \BC class \BC family \BC character \NC \NR
\TB
\NC \prm {Umathchardef} \NC csname \NC "40   \NC "40    \NC "FFFFF     \NC \NR
\NC \prm {Umathcode}    \NC        \NC "40   \NC "40    \NC "FFFFF     \NC \NR
\NC \prm {Udelcode}     \NC "FFFFF \NC "40   \NC "40    \NC "FFFFF     \NC \NR
\NC \prm {Umathchar}    \NC        \NC "40   \NC "40    \NC "FFFFF     \NC \NR
\NC \prm {Umathaccent}  \NC        \NC "40   \NC "40    \NC "FFFFF     \NC \NR
\NC \prm {Udelimiter}   \NC        \NC "40   \NC "40    \NC "FFFFF     \NC \NR
\NC \prm {Uradical}     \NC        \NC "40   \NC "40    \NC "FFFFF     \NC \NR
\LL
\stoptabulate

So, there are upto 64 classes of which at this moment about 20 are predefined so,
taking some future usage by the engine into account,you can assume 32 upto 60 to
be available for any purpose. The number of families has been reduced from 256 to
64 which is plenty for daily use in an \OPENTYPE\ setup. If we ever need to
expand the \UNICODE\ range there will be less families or we just go for a larger
internal record. The values of begin and end classes and the number of classes
can be fetched from the \LUA\ status table.

Given the above, specifications typically look like:

\starttyping
\Umathchardef \xx = "1 "0 "456
\Umathcode    123 = "1 "0 "789
\stoptyping

The new primitives that deal with delimiter|-|style objects do not set up a
\quote {large family}. Selecting a suitable size for display purposes is expected
to be dealt with by the font via the \prm {Umathoperatorsize} parameter. Old
school fonts can still be handled but you need to set up the engine to do that;
this can be done per font. In principle we assume that \OPENTYPE\ fonts are used,
which is no big deal because loading fonts is already under \LUA\ control. At
that moment the distinction between small and large delimiters will be gone. Of
course an alternative is to support a specific large size but that is unlikely to
happen.

This means that future versions of \LUAMETATEX\ might drop for instance the large
family in delimiters, if only because we assume a coherent setup where
extensibles come from the same font so that we don't need to worry about clashing
font parameters. This is a condition that we can easily meet in \CONTEXT, which is
the reference for \LUAMETATEX.

% Constructor related primitives like \prm {Umathaccent} accepts optional keywords
% to control various details regarding their treatment and rendering. See for
% instance \in {section} [mathacc] for details. Some keywords are specific, but
% some are shared between the math nodes (aka noads).

There are more new primitives and most of these will be explained in following
sections. For instance these are variants of radicals and delimiters all are
set the same:

\starttabulate[|l|c|c|c|]
\DB primitive              \BC class \BC family \NC character \NC \NR
\TB
\NC \prm {Uroot}           \NC "40   \NC "40    \NC "FFFFF    \NC \NR
\NC \prm {Uoverdelimiter}  \NC "40   \NC "40    \NC "FFFFF    \NC \NR
\NC \prm {Uunderdelimiter} \NC "40   \NC "40    \NC "FFFFF    \NC \NR
\NC \prm {Udelimiterover}  \NC "40   \NC "40    \NC "FFFFF    \NC \NR
\NC \prm {Udelimiterunder} \NC "40   \NC "40    \NC "FFFFF    \NC \NR
\LL
\stoptabulate

In addition there are \prm {Uvextensible} and \prm {Uoperator} and extended
versions of fenced: \prm {Uleft}, \prm {Uright} and \prm {Umiddle}. There is also
\prm {Uover} and similar primitives that expect the numerator and denominator
after the primitive. In addition to regular scripts there are prescripts and a
dedicated prime script. Many of these \type {U} primitives can be controlled by
options and keywords.

\stopsection

\startsection[title=Setting up the engine]

\startsubsection[title={Control}]

\topicindex{math+control}

Rendering math has long been dominated by \TEX\ but that changed when \MICROSOFT\
came with \OPENTYPE\ math: an implementation as well as a font. Some of that was
modelled after \TEX\ and some was dictated (we think) by the way word processors
deal with math. For instance, traditional \TEX\ math has a limited set of glyph
properties and therefore has a somewhat complex interplay between width and
italic correction. There are no kerns, contrary to \OPENTYPE\ math fonts that
provides staircase kerns. Interestingly \TEX\ does have some ligature building
going on in the engine.

In traditional \TEX\ italic correction gets added to the width and selectively
removed later (or compensated by some shift and|/|or cheating with the box
width). When we started with \LUATEX\ we had to gamble quite a bit about how to
apply parameters and glyph properties which resulted in different code paths,
heuristics, etc. That worked on the average but fonts are often not perfect and
when served as an example for another one the bad bits can be inherited. That
said, over time the descriptions improved and this is what the \OPENTYPE\
specification has to say about italic correction now \footnote {\type
{https://docs.microsoft.com/en-us/typography/opentype/spec/math}}:

\startitemize [n]
    \startitem
        When a run of slanted characters is followed by a straight character
        (such as an operator or a delimiter), the italics correction of the last
        glyph is added to its advance width.
    \stopitem
    \startitem
        When positioning limits on an N-ary operator (e.g., integral sign), the
        horizontal position of the upper limit is moved to the right by half the
        italics correction, while the position of the lower limit is moved to the
        left by the same distance.
    \stopitem
    \startitem
        When positioning superscripts and subscripts, their default horizontal
        positions are also different by the amount of the italics correction of
        the preceding glyph.
    \stopitem
\stopitemize

The first rule is complicated by the fact that \quote {followed} is vague: in
\TEX\ the sequence \type {$ a b c def $} results in six separate atoms, separated
by inter atom spacing. The characters in these atoms are the nucleus and there
can be a super- and|/|or subscript attached and in \LUAMETATEX\ also a prime,
superprescript and/or subprescript.

The second rule comes from \TEX\ and one can wonder why the available top accent
anchor is not used. Maybe because bottom accent anchors are missing? Anyway,
we're stuck with this now.

The third rule also seems to come from \TEX. Take the \quote {\it f} character:
in \TEX\ fonts that one has a narrow width and part sticks out (in some even at
the left edge). That means that when the subscript gets attached it will move
inwards relative to the real dimensions. Before the superscript an italic
correction is added so what that correction is non|-|zero the scripts are
horizontally shifted relative to each other.

Now look at this specification of staircase kerns \footnote {Idem.}:

\startnarrower
    The \type {MathKernInfo} table provides mathematical kerning values used for
    kerning of subscript and superscript glyphs relative to a base glyph. Its
    purpose is to improve spacing in situations such as omega with superscript f
    or capital V with subscript capital A.

    Mathematical kerning is height dependent; that is, different kerning amounts
    can be specified for different heights within a glyph’s vertical extent. For
    any given glyph, different values can be specified for four corner positions,
    top|-|right, to|-|left, etc., allowing for different kerning adjustments
    according to whether the glyph occurs as a subscript, a superscript, a base
    being kerned with a subscript, or a base being kerned with a superscript.
\stopnarrower

Again we're talking super- and subscripts and should we now look at the italic
correction or assume that the kerns do the job? This is a mixed bag because
scripts are not always (single) characters. We have to guess a bit here. After
years of experimenting we came to the conclusion that it will never be okay so
that's why we settled on controls and runtime fixes to fonts.

This means that processing math is controlled by \prm {mathfontcontrol}, a
numeric bitset parameter. The recommended bits are marked with a star but it
really depends on the macro package to set up the machinery well. Of course one
can just enable all and see what happens. \footnote {This model was more granular
and could even be font (and character) specific but that was dropped because
fonts are too inconsistent and an occasional fit is more robust that a generally
applied rule.}

\starttabulate[|l|l|c|]
\DB bit             \BC name                              \NC \NC \NR
\TB
\NC \type {0x000001} \NC \type {usefontcontrol}           \NC       \NR
\NC \type {0x000002} \NC \type {overrule}                 \NC \star \NR
\NC \type {0x000004} \NC \type {underrule}                \NC \star \NR
\NC \type {0x000008} \NC \type {radicalrule}              \NC \star \NR
\NC \type {0x000010} \NC \type {fractionrule}             \NC \star \NR
\NC \type {0x000020} \NC \type {accentskewhalf}           \NC \star \NR
\NC \type {0x000040} \NC \type {accentskewapply}          \NC \star \NR
\NC \type {0x000080} \NC \type {applyordinarykernpair}    \NC \star \NR
\NC \type {0x000100} \NC \type {applyverticalitalickern}  \NC \star \NR
\NC \type {0x000200} \NC \type {applyordinaryitalickern}  \NC \star \NR
\NC \type {0x000400} \NC \type {applycharitalickern}      \NC       \NR
\NC \type {0x000800} \NC \type {reboxcharitalickern}      \NC       \NR
\NC \type {0x001000} \NC \type {applyboxeditalickern}     \NC \star \NR
\NC \type {0x002000} \NC \type {staircasekern}            \NC \star \NR
\NC \type {0x004000} \NC \type {applytextitalickern}      \NC \star \NR
\NC \type {0x008000} \NC \type {checktextitalickern}      \NC \star \NR
\NC \type {0x010000} \NC \type {checkspaceitalickern}     \NC       \NR
\NC \type {0x020000} \NC \type {applyscriptitalickern}    \NC \star \NR
\NC \type {0x040000} \NC \type {analysescriptnucleuschar} \NC \star \NR
\NC \type {0x080000} \NC \type {analysescriptnucleuslist} \NC \star \NR
\NC \type {0x100000} \NC \type {analysescriptnucleusbox}  \NC \star \NR
\LL
\stoptabulate

So, to summarize: the reason for this approach is that traditional and \OPENTYPE\
fonts have different approaches (especially when it comes to dealing with the
width and italic corrections) and is even more complicated by the fact that the
fonts are often inconsistent (within and between). In \CONTEXT\ we deal with this
by runtime fixes to fonts. In any case the Cambria font is taken as reference.

\stopsubsection

\startsubsection[title={Analyzing the script nucleus}]

\topicindex {math+kerning}
\topicindex {math+scripts}

The three analyze option relate to staircase kerns for which we need to look into the
nucleus to get to the first character. In principle we only need to look into simple
characters and lists but we can also look into boxes.There can be interference with
other kinds spacing as well as italic corrections, which is why it is an option. These three
are not bound to fonts because we don't know if have a font involved.

% We keep the next text commented for historic reasons. In \CONTEXT\ we clean up
% fonts in the font goodie files so the examples would not be honest anyway. But it
% shows a bit where we come from and what alternatives we tried.

% If you want to typeset text in math macro packages often provide something \type
% {\text} which obeys the script sizes. As the definition can be anything there is
% a good chance that the kerning doesn't come out well when used in a script. Given
% that the first glyph ends up in an \prm {hbox} we have some control over this.
% And, as a bonus we also added control over the normal sublist kerning. The \prm
% {mathscriptboxmode} parameter defaults to~1.
%
% \starttabulate[|c|l|]
% \DB value     \BC meaning \NC \NR
% \TB
% \NC \type {0} \NC forget about kerning \NC \NR
% \NC \type {1} \NC kern math sub lists with a valid glyph \NC \NR
% \NC \type {2} \NC also kern math sub boxes that have a valid glyph \NC \NR
% \NC \type {3} \NC only kern math sub boxes with a boundary node present\NC \NR
% \LL
% \stoptabulate
%
% Here we show some examples. Of course this doesn't solve all our problems, if
% only because some fonts have characters with bounding boxes that compensate for
% italics, while other fonts can lack kerns.
%
% \startbuffer[1]
%     $T_{\tf fluff}$
% \stopbuffer
%
% \startbuffer[2]
%     $T_{\text{fluff}}$
% \stopbuffer
%
% \startbuffer[3]
%     $T_{\text{\boundary1 fluff}}$
% \stopbuffer
%
% \unexpanded\def\Show#1#2#3%
%   {\doifelsenothing{#3}
%      {\small\tx\typeinlinebuffer[#1]}
%      {\doifelse{#3}{-}
%         {\small\bf\tt mode #2}
%         {\switchtobodyfont[#3]\showfontkerns\showglyphs\mathscriptboxmode#2\relax\inlinebuffer[#1]}}}
%
% \starttabulate[|lBT|c|c|c|c|c|]
%     \NC          \NC \Show{1}{0}{}         \NC\Show{1}{1}{}         \NC \Show{2}{1}{}         \NC \Show{2}{2}{}         \NC \Show{3}{3}{}         \NC \NR
%     \NC          \NC \Show{1}{0}{-}        \NC\Show{1}{1}{-}        \NC \Show{2}{1}{-}        \NC \Show{2}{2}{-}        \NC \Show{3}{3}{-}        \NC \NR
%     \NC modern   \NC \Show{1}{0}{modern}   \NC\Show{1}{1}{modern}   \NC \Show{2}{1}{modern}   \NC \Show{2}{2}{modern}   \NC \Show{3}{3}{modern}   \NC \NR
%     \NC lucidaot \NC \Show{1}{0}{lucidaot} \NC\Show{1}{1}{lucidaot} \NC \Show{2}{1}{lucidaot} \NC \Show{2}{2}{lucidaot} \NC \Show{3}{3}{lucidaot} \NC \NR
%     \NC pagella  \NC \Show{1}{0}{pagella}  \NC\Show{1}{1}{pagella}  \NC \Show{2}{1}{pagella}  \NC \Show{2}{2}{pagella}  \NC \Show{3}{3}{pagella}  \NC \NR
%     \NC cambria  \NC \Show{1}{0}{cambria}  \NC\Show{1}{1}{cambria}  \NC \Show{2}{1}{cambria}  \NC \Show{2}{2}{cambria}  \NC \Show{3}{3}{cambria}  \NC \NR
%     \NC dejavu   \NC \Show{1}{0}{dejavu}   \NC\Show{1}{1}{dejavu}   \NC \Show{2}{1}{dejavu}   \NC \Show{2}{2}{dejavu}   \NC \Show{3}{3}{dejavu}   \NC \NR
% \stoptabulate
%
% Kerning between a character subscript is controlled by \prm {mathscriptcharmode}
% which also defaults to~1.
%
% Here is another example. Internally we tag kerns as italic kerns or font kerns
% where font kerns result from the staircase kern tables. In 2018 fonts like Latin
% Modern and Pagella rely on cheats with the boundingbox, Cambria uses staircase
% kerns and Lucida a mixture. Depending on how fonts evolve we might add some more
% control over what one can turn on and off.
%
% \def\MathSample#1#2#3%
%   {\NC
%    #1 \NC
%    #2 \NC
%    \showglyphdata \switchtobodyfont[#2,17.3pt]$#3T_{f}$         \NC
%    \showglyphdata \switchtobodyfont[#2,17.3pt]$#3\gamma_{e}$    \NC
%    \showglyphdata \switchtobodyfont[#2,17.3pt]$#3\gamma_{ee}$   \NC
%    \showglyphdata \switchtobodyfont[#2,17.3pt]$#3T_{\tf fluff}$ \NC
%    \NR}
%
% \starttabulate[|Tl|Tl|l|l|l|l|]
%     \FL
%     \MathSample{normal}{modern}  {\mr}
%     \MathSample{}      {pagella} {\mr}
%     \MathSample{}      {cambria} {\mr}
%     \MathSample{}      {lucidaot}{\mr}
%     \ML
%     \MathSample{bold}  {modern}  {\mb}
%     \MathSample{}      {pagella} {\mb}
%     \MathSample{}      {cambria} {\mb}
%     \MathSample{}      {lucidaot}{\mb}
%     \LL
% \stoptabulate

\stopsubsection

\stopsection

\startsection[title={Math styles}]

\startsubsection[title={\prm {mathstyle}, \prm {mathstackstyle} and \prm {givenmathstyle}}]

\topicindex {math+styles}

It is possible to discover the math style that will be used for a formula in an
expandable fashion (while the math list is still being read). To make this
possible, \LUATEX\ adds the new primitive: \prm {mathstyle}. This is a \quote
{convert command} like e.g. \prm {romannumeral}: its value can only be read,
not set. Beware that contrary to \LUATEX\ this is now a proper number so you need
to use \type {\number} or \type {\the} in order to serialize it.

The returned value is between 0 and 7 (in math mode), or $-1$ (all other modes).
For easy testing, the eight math style commands have been altered so that they can
be used as numeric values, so you can write code like this:

\starttyping
\ifnum\mathstyle=\textstyle
    \message{normal text style}
\else \ifnum\mathstyle=\crampedtextstyle
    \message{cramped text style}
\fi \fi
\stoptyping

Sometimes you won't get what you expect so a bit of explanation might help to
understand what happens. When math is parsed and expanded it gets turned into a
linked list. In a second pass the formula will be build. This has to do with the
fact that in order to determine the automatically chosen sizes (in for instance
fractions) following content can influence preceding sizes. A side effect of this
is for instance that one cannot change the definition of a font family (and
thereby reusing numbers) because the number that got used is stored and used in
the second pass (so changing \type {\fam 12} mid|-|formula spoils over to
preceding use of that family).

The style switching primitives like \prm {textstyle} are turned into nodes so the
styles set there are frozen. The \prm {mathchoice} primitive results in four
lists being constructed of which one is used in the second pass. The fact that
some automatic styles are not yet known also means that the \prm {mathstyle}
primitive expands to the current style which can of course be different from the
one really used. It's a snapshot of the first pass state. As a consequence in the
following example you get a style number (first pass) typeset that can actually
differ from the used style (second pass). In the case of a math choice used
ungrouped, the chosen style is used after the choice too, unless you group.

\startbuffer[1]
    [a:\number\mathstyle]\quad
    \bgroup
    \mathchoice
        {\bf \scriptstyle       (x:d :\number\mathstyle)}
        {\bf \scriptscriptstyle (x:t :\number\mathstyle)}
        {\bf \scriptscriptstyle (x:s :\number\mathstyle)}
        {\bf \scriptscriptstyle (x:ss:\number\mathstyle)}
    \egroup
    \quad[b:\number\mathstyle]\quad
    \mathchoice
        {\bf \scriptstyle       (y:d :\number\mathstyle)}
        {\bf \scriptscriptstyle (y:t :\number\mathstyle)}
        {\bf \scriptscriptstyle (y:s :\number\mathstyle)}
        {\bf \scriptscriptstyle (y:ss:\number\mathstyle)}
    \quad[c:\number\mathstyle]\quad
    \bgroup
    \mathchoice
        {\bf \scriptstyle       (z:d :\number\mathstyle)}
        {\bf \scriptscriptstyle (z:t :\number\mathstyle)}
        {\bf \scriptscriptstyle (z:s :\number\mathstyle)}
        {\bf \scriptscriptstyle (z:ss:\number\mathstyle)}
    \egroup
    \quad[d:\number\mathstyle]
\stopbuffer

\startbuffer[2]
    [a:\number\mathstyle]\quad
    \begingroup
    \mathchoice
        {\bf \scriptstyle       (x:d :\number\mathstyle)}
        {\bf \scriptscriptstyle (x:t :\number\mathstyle)}
        {\bf \scriptscriptstyle (x:s :\number\mathstyle)}
        {\bf \scriptscriptstyle (x:ss:\number\mathstyle)}
    \endgroup
    \quad[b:\number\mathstyle]\quad
    \mathchoice
        {\bf \scriptstyle       (y:d :\number\mathstyle)}
        {\bf \scriptscriptstyle (y:t :\number\mathstyle)}
        {\bf \scriptscriptstyle (y:s :\number\mathstyle)}
        {\bf \scriptscriptstyle (y:ss:\number\mathstyle)}
    \quad[c:\number\mathstyle]\quad
    \begingroup
    \mathchoice
        {\bf \scriptstyle       (z:d :\number\mathstyle)}
        {\bf \scriptscriptstyle (z:t :\number\mathstyle)}
        {\bf \scriptscriptstyle (z:s :\number\mathstyle)}
        {\bf \scriptscriptstyle (z:ss:\number\mathstyle)}
    \endgroup
    \quad[d:\number\mathstyle]
\stopbuffer

\typebuffer[1]

% \typebuffer[2]

This gives:

\blank $\displaystyle \getbuffer[1]$ \blank
\blank $\textstyle    \getbuffer[1]$ \blank

Using \prm {begingroup} \unknown\ \prm {endgroup} instead gives:

\blank $\displaystyle \getbuffer[2]$ \blank
\blank $\textstyle    \getbuffer[2]$ \blank

This might look wrong but it's just a side effect of \prm {mathstyle} expanding
to the current (first pass) style and the number being injected in the list that
gets converted in the second pass. It all makes sense and it illustrates the
importance of grouping. In fact, the math choice style being effective afterwards
has advantages. It would be hard to get it otherwise.

So far for the more \LUATEX ish approach. One problem with \prm {mathstyle} is
that when you got it, and want to act upon it, you need to remap it onto say \prm
{scriptstyle} which can be done with an eight branched \prm {ifcase}. This is
why we also have a more efficient alternative that you can use in macros:

\starttyping
\normalexpand{ ... \givenmathstyle\the\mathstyle      ...}
\normalexpand{ ... \givenmathstyle\the\mathstackstyle ...}
\stoptyping

This new primitive \prm {givenmathstyle} accepts a numeric value. The \prm
{mathstackstyle} primitive is just a bonus (it complements \prm {mathstack}).

The styles that the different math components and their subcomponents start out
with are no longer hard coded but can be set at runtime:

\starttabulate
\DB primitive name                        \BC default \NC \NR
\TB
\NC \prm {Umathoverlinevariant}           \NC cramped           \NC \NR
\NC \prm {Umathunderlinevariant}          \NC normal            \NC \NR
\NC \prm {Umathoverdelimitervariant}      \NC small             \NC \NR
\NC \prm {Umathunderdelimitervariant}     \NC small             \NC \NR
\NC \prm {Umathdelimiterovervariant}      \NC normal            \NC \NR
\NC \prm {Umathdelimiterundervariant}     \NC normal            \NC \NR
\NC \prm {Umathhextensiblevariant}        \NC normal            \NC \NR
\NC \prm {Umathvextensiblevariant}        \NC normal            \NC \NR
\NC \prm {Umathfractionvariant}           \NC cramped           \NC \NR
\NC \prm {Umathradicalvariant}            \NC cramped           \NC \NR
\NC \prm {Umathdegreevariant}             \NC doublesuperscript \NC \NR
\NC \prm {Umathaccentvariant}             \NC cramped           \NC \NR
\NC \prm {Umathtopaccentvariant}          \NC cramped           \NC \NR
\NC \prm {Umathbottomaccentvariant}       \NC cramped           \NC \NR
\NC \prm {Umathoverlayaccentvariant}      \NC cramped           \NC \NR
\NC \prm {Umathnumeratorvariant}          \NC numerator         \NC \NR
\NC \prm {Umathdenominatorvariant}        \NC denominator       \NC \NR
\NC \prm {Umathsuperscriptvariant}        \NC superscript       \NC \NR
\NC \prm {Umathsubscriptvariant}          \NC subscript         \NC \NR
\NC \prm {Umathprimevariant}              \NC superscript       \NC \NR
\NC \prm {Umathstackvariant}              \NC numerator         \NC \NR
\LL
\stoptabulate

These defaults remap styles are as follows:

\starttabulate[|Tl|l|l|]
\DB default           \BC result                    \BC mapping \NC \NR
\TB
\NC cramped           \NC cramp the style           \NC D' D' T' T' S'  S'  SS' SS' \NC \NR
\NC subscript         \NC smaller and cramped       \NC S' S' S' S' SS' SS' SS' SS' \NC \NR
\NC small             \NC smaller                   \NC S  S  S  S  SS  SS  SS  SS  \NC \NR
\NC superscript       \NC smaller                   \NC S  S  S  S  SS  SS  SS  SS  \NC \NR
\NC smaller           \NC smaller unless already SS \NC S  S' S  S' SS  SS' SS  SS' \NC \NR
\NC numerator         \NC smaller unless already SS \NC S  S' S  S' SS  SS' SS  SS' \NC \NR
\NC denominator       \NC smaller, all cramped      \NC S' S' S' S' SS' SS' SS' SS' \NC \NR
\NC doublesuperscript \NC smaller, keep cramped     \NC S  S' S  S' SS  SS' SS  SS' \NC \NR
\LL
\stoptabulate

The main reason for opening this up was that it permits experiments and removed
hard coded internal values. But as these defaults served well for decades there
are no real reasons to change them.

\stopsubsection

\startsubsection[title={\prm {mathstack}}]

\topicindex {math+stacks}

There are a few math commands in \TEX\ where the style that will be used is not
known straight from the start. These commands (\prm {over}, \prm {atop},
\prm {overwithdelims}, \prm {atopwithdelims}) would therefore normally return
wrong values for \prm {mathstyle}. To fix this, \LUATEX\ introduces a special
prefix command: \prm {mathstack}:

\starttyping
$\mathstack {a \over b}$
\stoptyping

The \prm {mathstack} command will scan the next brace and start a new math group
with the correct (numerator) math style. The \prm {mathstackstyle} primitive
relates to this feature.

\stopsubsection

\startsubsection[title={The new \type {\cramped...style} commands}]

\topicindex {math+styles}
\topicindex {math+spacing}
\topicindex {math+cramped}

\LUATEX\ has four new primitives to set the cramped math styles directly:

\starttyping
\crampeddisplaystyle
\crampedtextstyle
\crampedscriptstyle
\crampedscriptscriptstyle
\stoptyping

These additional commands are not all that valuable on their own, but they come
in handy as arguments to the math parameter settings that will be added shortly.

Because internally the eight styles are represented as numbers some of the new
primnitives that relate to them also work with numbers and often you can use them
mixed. The \prm {tomathstyle} prefix converts a symbolic style into a number so
\typ {\number \tomathstyle \crampedscriptstyle} gives~\number \tomathstyle
\crampedscriptstyle.

In Eijkhouts \quotation {\TEX\ by Topic} the rules for handling styles in scripts
are described as follows:

\startitemize
\startitem
    In any style superscripts and subscripts are taken from the next smaller style.
    Exception: in display style they are in script style.
\stopitem
\startitem
    Subscripts are always in the cramped variant of the style; superscripts are only
    cramped if the original style was cramped.
\stopitem
\startitem
    In an \type {..\over..} formula in any style the numerator and denominator are
    taken from the next smaller style.
\stopitem
\startitem
    The denominator is always in cramped style; the numerator is only in cramped
    style if the original style was cramped.
\stopitem
\startitem
    Formulas under a \type {\sqrt} or \prm {overline} are in cramped style.
\stopitem
\stopitemize

In \LUATEX\ one can set the styles in more detail which means that you sometimes
have to set both normal and cramped styles to get the effect you want. (Even) if
we force styles in the script using \prm {scriptstyle} and \prm
{crampedscriptstyle} we get this:

\startbuffer[demo]
\starttabulate
\DB style         \BC example \NC \NR
\TB
\NC default       \NC $b_{x=xx}^{x=xx}$ \NC \NR
\NC script        \NC $b_{\scriptstyle x=xx}^{\scriptstyle x=xx}$ \NC \NR
\NC crampedscript \NC $b_{\crampedscriptstyle x=xx}^{\crampedscriptstyle x=xx}$ \NC \NR
\LL
\stoptabulate
\stopbuffer

\getbuffer[demo]

Now we set the following parameters using \prm {setmathspacing} that accepts two
class identifier, a style and a value.

\startbuffer[setup]
\setmathspacing 0 3 \scriptstyle = 30mu
\setmathspacing 0 3 \scriptstyle = 30mu
\stopbuffer

\typebuffer[setup]

This gives a different result:

\start\getbuffer[setup,demo]\stop

But, as this is not what is expected (visually) we should say:

\startbuffer[setup]
\setmathspacing 0 3 \scriptstyle        = 30mu
\setmathspacing 0 3 \scriptstyle        = 30mu
\setmathspacing 0 3 \crampedscriptstyle = 30mu
\setmathspacing 0 3 \crampedscriptstyle = 30mu
\stopbuffer

\typebuffer[setup]

Now we get:

\start\getbuffer[setup,demo]\stop

\stopsubsection

\stopsection

\startsection[title={Math parameter settings}]

\startsubsection[title={Many new \tex {Umath*} primitives}]

\topicindex {math+parameters}

In \LUATEX, the font dimension parameters that \TEX\ used in math typesetting are
now accessible via primitive commands. In fact, refactoring of the math engine
has resulted in turning some hard codes properties into parameters.

{\em The next needs checking ...}

\starttabulate
\DB primitive name                 \BC description \NC \NR
\TB
\NC \prm {Umathquad}               \NC the width of 18 mu's \NC \NR
\NC \prm {Umathaxis}               \NC height of the vertical center axis of
                                       the math formula above the baseline \NC \NR
\NC \prm {Umathoperatorsize}       \NC minimum size of large operators in display mode \NC \NR
\NC \prm {Umathoverbarkern}        \NC vertical clearance above the rule \NC \NR
\NC \prm {Umathoverbarrule}        \NC the width of the rule \NC \NR
\NC \prm {Umathoverbarvgap}        \NC vertical clearance below the rule \NC \NR
\NC \prm {Umathunderbarkern}       \NC vertical clearance below the rule \NC \NR
\NC \prm {Umathunderbarrule}       \NC the width of the rule \NC \NR
\NC \prm {Umathunderbarvgap}       \NC vertical clearance above the rule \NC \NR
\NC \prm {Umathradicalkern}        \NC vertical clearance above the rule \NC \NR
\NC \prm {Umathradicalrule}        \NC the width of the rule \NC \NR
\NC \prm {Umathradicalvgap}        \NC vertical clearance below the rule \NC \NR
\NC \prm {Umathradicaldegreebefore}\NC the forward kern that takes place before placement of
                                       the radical degree \NC \NR
\NC \prm {Umathradicaldegreeafter} \NC the backward kern that takes place after placement of
                                       the radical degree \NC \NR
\NC \prm {Umathradicaldegreeraise} \NC this is the percentage of the total height and depth of
                                       the radical sign that the degree is raised by; it is
                                       expressed in \type {percents}, so 60\% is expressed as the
                                       integer $60$ \NC \NR
\NC \prm {Umathstackvgap}          \NC vertical clearance between the two
                                       elements in an \prm {atop} stack \NC \NR
\NC \prm {Umathstacknumup}         \NC numerator shift upward in \prm {atop} stack \NC \NR
\NC \prm {Umathstackdenomdown}     \NC denominator shift downward in \prm {atop} stack \NC \NR
\NC \prm {Umathfractionrule}       \NC the width of the rule in a \prm {over} \NC \NR
\NC \prm {Umathfractionnumvgap}    \NC vertical clearance between the numerator and the rule \NC \NR
\NC \prm {Umathfractionnumup}      \NC numerator shift upward in \prm {over} \NC \NR
\NC \prm {Umathfractiondenomvgap}  \NC vertical clearance between the denominator and the rule \NC \NR
\NC \prm {Umathfractiondenomdown}  \NC denominator shift downward in \prm {over} \NC \NR
\NC \prm {Umathfractiondelsize}    \NC minimum delimiter size for \type {\...withdelims} \NC \NR
\NC \prm {Umathlimitabovevgap}     \NC vertical clearance for limits above operators \NC \NR
\NC \prm {Umathlimitabovebgap}     \NC vertical baseline clearance for limits above operators \NC \NR
\NC \prm {Umathlimitabovekern}     \NC space reserved at the top of the limit \NC \NR
\NC \prm {Umathlimitbelowvgap}     \NC vertical clearance for limits below operators \NC \NR
\NC \prm {Umathlimitbelowbgap}     \NC vertical baseline clearance for limits below operators \NC \NR
\NC \prm {Umathlimitbelowkern}     \NC space reserved at the bottom of the limit \NC \NR
\NC \prm {Umathoverdelimitervgap}  \NC vertical clearance for limits above delimiters \NC \NR
\NC \prm {Umathoverdelimiterbgap}  \NC vertical baseline clearance for limits above delimiters \NC \NR
\NC \prm {Umathunderdelimitervgap} \NC vertical clearance for limits below delimiters \NC \NR
\NC \prm {Umathunderdelimiterbgap} \NC vertical baseline clearance for limits below delimiters \NC \NR
\NC \prm {Umathsubshiftdrop}       \NC subscript drop for boxes and subformulas \NC \NR
\NC \prm {Umathsubshiftdown}       \NC subscript drop for characters \NC \NR
\NC \prm {Umathsupshiftdrop}       \NC superscript drop (raise, actually) for boxes and subformulas \NC \NR
\NC \prm {Umathsupshiftup}         \NC superscript raise for characters \NC \NR
\NC \prm {Umathsubsupshiftdown}    \NC subscript drop in the presence of a superscript \NC \NR
\NC \prm {Umathsubtopmax}          \NC the top of standalone subscripts cannot be higher than this
                                       above the baseline \NC \NR
\NC \prm {Umathsupbottommin}       \NC the bottom of standalone superscripts cannot be less than
                                       this above the baseline \NC \NR
\NC \prm {Umathsupsubbottommax}    \NC the bottom of the superscript of a combined super- and subscript
                                       be at least as high as this above the baseline \NC \NR
\NC \prm {Umathsubsupvgap}         \NC vertical clearance between super- and subscript \NC \NR
\NC \prm {Umathspaceafterscript}   \NC additional space added after a super- or subscript \NC \NR
\NC \prm {Umathconnectoroverlapmin}\NC minimum overlap between parts in an extensible recipe \NC \NR
\LL
\stoptabulate

In addition to the above official \OPENTYPE\ font parameters we have these (the
undefined will get presets, quite likely zero):

\starttabulate
\DB primitive name                             \BC description \NC \NR
\TB
\NC \prm {Umathconnectoroverlapmin}            \NC \NC \NR
\NC \prm {Umathsubsupshiftdown}                \NC \NC \NR
\NC \prm {Umathfractiondelsize}                \NC \NC \NR
\NC \prm {Umathnolimitsupfactor}               \NC a multiplier for the way limits are shifted up and down \NC \NR
\NC \prm {Umathnolimitsubfactor}               \NC a multiplier for the way limits are shifted up and down \NC \NR
\NC \prm {Umathaccentbasedepth}                \NC the complement of \prm {Umathaccentbaseheight} \NC \NR
\NC \prm {Umathflattenedaccentbasedepth}       \NC the complement of \prm {Umathflattenedaccentbaseheight} \NC \NR
\NC \prm {Umathspacebeforescript}              \NC \NC \NR
\NC \prm {Umathprimeraise}                     \NC \NC \NR
\NC \prm {Umathprimeraisecomposed}             \NC \NC \NR
\NC \prm {Umathprimeshiftup}                   \NC the prime variant of \prm {Umathsupshiftup} \NC \NR
\NC \prm {Umathprimespaceafter}                \NC the prescript variant of \prm {Umathspaceafterscript} \NC \NR
\NC \prm {Umathprimeshiftdrop}                 \NC the prime variant of \prm {Umathsupshiftdrop} \NC \NR
\NC \prm {Umathprimewidth}                     \NC the percentage of width that gets added \NC \NR
\NC \prm {Umathskeweddelimitertolerance}       \NC \NC \NR
\NC \prm {Umathaccenttopshiftup}               \NC the amount that a top accent is shifted up \NC \NR
\NC \prm {Umathaccentbottomshiftdown}          \NC the amount that a bottom accent is shifted down \NC \NR
\NC \prm {Umathaccenttopovershoot}             \NC \NC \NR
\NC \prm {Umathaccentbottomovershoot}          \NC \NC \NR
\NC \prm {Umathaccentsuperscriptdrop}          \NC \NC \NR
\NC \prm {Umathaccentsuperscriptpercent}       \NC \NC \NR
\NC \prm {Umathaccentextendmargin}             \NC margins added to automatically extended accents \NC \NR
\NC \prm {Umathflattenedaccenttopshiftup}      \NC the amount that a wide top accent is shifted up \NC \NR
\NC \prm {Umathflattenedaccentbottomshiftdown} \NC the amount that a wide bottom accent is shifted down \NC \NR
\NC \prm {Umathdelimiterpercent}               \NC \NC \NR
\NC \prm {Umathdelimitershortfall}             \NC \NC \NR
\NC \prm {Umathradicalextensiblebefore}        \NC \NC \NR
\NC \prm {Umathradicalextensibleafter}         \NC \NC \NR
\stoptabulate

These relate to the font parameters and in \CONTEXT\ we assign some different
defaults and tweak them in the goodie files:

\starttabulate[|T|T|c|]
\DB font parameter                    \BC primitive name                             \BC default   \NC \NR
\TB
\NC MinConnectorOverlap               \NC \prm {Umathconnectoroverlapmin}            \NC 0         \NC \NR
\NC SubscriptShiftDownWithSuperscript \NC \prm {Umathsubsupshiftdown}                \NC inherited \NC \NR
\NC FractionDelimiterSize             \NC \prm {Umathfractiondelsize}                \NC undefined \NC \NR
\NC FractionDelimiterDisplayStyleSize \NC \prm {Umathfractiondelsize}                \NC undefined \NC \NR
\NC NoLimitSubFactor                  \NC \prm {Umathnolimitsupfactor}               \NC 0         \NC \NR
\NC NoLimitSupFactor                  \NC \prm {Umathnolimitsubfactor}               \NC 0         \NC \NR
\NC AccentBaseDepth                   \NC \prm {Umathaccentbasedepth}                \NC reserved  \NC \NR
\NC FlattenedAccentBaseDepth          \NC \prm {Umathflattenedaccentbasedepth}       \NC reserved  \NC \NR
\NC SpaceBeforeScript                 \NC \prm {Umathspacebeforescript}              \NC 0         \NC \NR
\NC PrimeRaisePercent                 \NC \prm {Umathprimeraise}                     \NC 0         \NC \NR
\NC PrimeRaiseComposedPercent         \NC \prm {Umathprimeraisecomposed}             \NC 0         \NC \NR
\NC PrimeShiftUp                      \NC \prm {Umathprimeshiftup}                   \NC 0         \NC \NR
\NC PrimeShiftUpCramped               \NC \prm {Umathprimeshiftup}                   \NC 0         \NC \NR
\NC PrimeSpaceAfter                   \NC \prm {Umathprimespaceafter}                \NC 0         \NC \NR
\NC PrimeBaselineDropMax              \NC \prm {Umathprimeshiftdrop}                 \NC 0         \NC \NR
\NC PrimeWidthPercent                 \NC \prm {Umathprimewidth}                     \NC 0         \NC \NR
\NC SkewedDelimiterTolerance          \NC \prm {Umathskeweddelimitertolerance}       \NC 0         \NC \NR
\NC AccentTopShiftUp                  \NC \prm {Umathaccenttopshiftup}               \NC undefined \NC \NR
\NC AccentBottomShiftDown             \NC \prm {Umathaccentbottomshiftdown}          \NC undefined \NC \NR
\NC AccentTopOvershoot                \NC \prm {Umathaccenttopovershoot}             \NC 0         \NC \NR
\NC AccentBottomOvershoot             \NC \prm {Umathaccentbottomovershoot}          \NC 0         \NC \NR
\NC AccentSuperscriptDrop             \NC \prm {Umathaccentsuperscriptdrop}          \NC 0         \NC \NR
\NC AccentSuperscriptPercent          \NC \prm {Umathaccentsuperscriptpercent}       \NC 0         \NC \NR
\NC AccentExtendMargin                \NC \prm {Umathaccentextendmargin}             \NC 0         \NC \NR
\NC FlattenedAccentTopShiftUp         \NC \prm {Umathflattenedaccenttopshiftup}      \NC undefined \NC \NR
\NC FlattenedAccentBottomShiftDown    \NC \prm {Umathflattenedaccentbottomshiftdown} \NC undefined \NC \NR
\NC DelimiterPercent                  \NC \prm {Umathdelimiterpercent}               \NC 0         \NC \NR
\NC DelimiterShortfall                \NC \prm {Umathdelimitershortfall}             \NC 0         \NC \NR
\stoptabulate

These parameters not only provide a bit more control over rendering, they also
can be used in compensating issues in font, because no font is perfect. Some are
the side effects of experiments and they have CamelCase companions in the \type
{MathConstants} table. For historical reasons the names are a bit inconsistent as
some originate in \TEX\ so we prefer to keep those names. Not many users will
mess around with these font parameters anyway. \footnote {I wonder if some names
should change, so that decision is pending.}

Each of the parameters in this section can be set by a command like this:

\starttyping
\Umathquad\displaystyle=1em
\stoptyping

they obey grouping, and you can use \type {\the\Umathquad\displaystyle} if
needed.

There are quite some parameters that can be set and there are eight styles, which means a lot
of keying in. For that reason is is possible to set parameters groupwise:

\starttabulate[|l|c|c|c|c|c|c|c|c|]
\DB primitive name               \BC D \BC D' \BC T \BC T' \BC S \BC S' \BC SS \BC SS' \NC \NR
\TB
\NC \prm {alldisplaystyles}      \NC$+$\NC$+ $\NC   \NC    \NC   \NC    \NC    \NC     \NC \NR
\NC \prm {alltextstyles}         \NC   \NC    \NC$+$\NC$+ $\NC   \NC    \NC    \NC     \NC \NR
\NC \prm {allscriptstyles}       \NC   \NC    \NC   \NC    \NC$+$\NC$+ $\NC    \NC     \NC \NR
\NC \prm {allscriptscriptstyles} \NC   \NC    \NC   \NC    \NC   \NC    \NC$+ $\NC$+  $\NC \NR
\NC \prm {allmathstyles}         \NC$+$\NC$+ $\NC$+$\NC$+ $\NC$+$\NC$+ $\NC$+ $\NC$+  $\NC \NR
\NC \prm {allmainstyles}         \NC   \NC    \NC   \NC    \NC   \NC    \NC    \NC     \NC \NR
\NC \prm {allsplitstyles}        \NC$+$\NC$+ $\NC$+$\NC$+ $\NC$-$\NC$- $\NC$- $\NC$-  $\NC \NR
\NC \prm {allunsplitstyles}      \NC   \NC    \NC   \NC    \NC$+$\NC$+ $\NC$+ $\NC$+  $\NC \NR
\NC \prm {alluncrampedstyles}    \NC$+$\NC    \NC$+$\NC    \NC$+$\NC    \NC$+ $\NC     \NC \NR
\NC \prm {allcrampedstyles}      \NC   \NC$+ $\NC   \NC$+ $\NC   \NC$+ $\NC    \NC$+  $\NC \NR
\LL
\stoptabulate

These groups are especially handy when you set up inter atom spacing, pre- and
post atom penalties and atom rules.

\stopsubsection

\startsubsection[title={Font|-|based math parameters}]

\topicindex {math+parameters}

We already introduced the font specific math parameters but we tell abit more
about them and how they relate to the original \TEX\ font dimensions.

While it is nice to have these math parameters available for tweaking, it would
be tedious to have to set each of them by hand. For this reason, \LUATEX\
initializes a bunch of these parameters whenever you assign a font identifier to
a math family based on either the traditional math font dimensions in the font
(for assignments to math family~2 and~3 using \TFM|-|based fonts like \type
{cmsy} and \type {cmex}), or based on the named values in a potential \type
{MathConstants} table when the font is loaded via Lua. If there is a \type
{MathConstants} table, this takes precedence over font dimensions, and in that
case no attention is paid to which family is being assigned to: the \type
{MathConstants} tables in the last assigned family sets all parameters.

In the table below, the one|-|letter style abbreviations and symbolic tfm font
dimension names match those used in the \TeX book. Assignments to \prm
{textfont} set the values for the cramped and uncramped display and text styles,
\prm {scriptfont} sets the script styles, and \prm {scriptscriptfont} sets the
scriptscript styles, so we have eight parameters for three font sizes. In the
\TFM\ case, assignments only happen in family~2 and family~3 (and of course only
for the parameters for which there are font dimensions).

Besides the parameters below, \LUATEX\ also looks at the \quote {space} font
dimension parameter. For math fonts, this should be set to zero.

\def\MathLine#1#2#3#4#5%
  {\TB
   \NC \llap{\high{\tx #2\enspace}}\ttbf \prm {#1} \NC \tt #5 \NC \NR
   \NC \tx #3 \NC \tt #4 \NC \NR}

\starttabulate[|l|l|]
\DB variable / style \BC tfm / opentype \NC \NR
\MathLine{Umathaxis}                     {}   {}                     {AxisHeight}                              {axis_height}
\MathLine{Umathaccentbaseheight}         {}   {}                     {AccentBaseHeight}                        {xheight}
\MathLine{Umathflattenedaccentbaseheight}{}   {}                     {FlattenedAccentBaseHeight}               {xheight}
\MathLine{Umathoperatorsize}             {6}  {D, D'}                {DisplayOperatorMinHeight}                {\emdash}
\MathLine{Umathfractiondelsize}          {9}  {D, D'}                {FractionDelimiterDisplayStyleSize}       {delim1}
\MathLine{Umathfractiondelsize}          {9}  {T, T', S, S', SS, SS'}{FractionDelimiterSize}                   {delim2}
\MathLine{Umathfractiondenomdown}        {}   {D, D'}                {FractionDenominatorDisplayStyleShiftDown}{denom1}
\MathLine{Umathfractiondenomdown}        {}   {T, T', S, S', SS, SS'}{FractionDenominatorShiftDown}            {denom2}
\MathLine{Umathfractiondenomvgap}        {}   {D, D'}                {FractionDenominatorDisplayStyleGapMin}   {3*default_rule_thickness}
\MathLine{Umathfractiondenomvgap}        {}   {T, T', S, S', SS, SS'}{FractionDenominatorGapMin}               {default_rule_thickness}
\MathLine{Umathfractionnumup}            {}   {D, D'}                {FractionNumeratorDisplayStyleShiftUp}    {num1}
\MathLine{Umathfractionnumup}            {}   {T, T', S, S', SS, SS'}{FractionNumeratorShiftUp}                {num2}
\MathLine{Umathfractionnumvgap}          {}   {D, D'}                {FractionNumeratorDisplayStyleGapMin}     {3*default_rule_thickness}
\MathLine{Umathfractionnumvgap}          {}   {T, T', S, S', SS, SS'}{FractionNumeratorGapMin}                 {default_rule_thickness}
\MathLine{Umathfractionrule}             {}   {}                     {FractionRuleThickness}                   {default_rule_thickness}
\MathLine{Umathskewedfractionhgap}       {}   {}                     {SkewedFractionHorizontalGap}             {math_quad/2}
\MathLine{Umathskewedfractionvgap}       {}   {}                     {SkewedFractionVerticalGap}               {math_x_height}
\MathLine{Umathlimitabovebgap}           {}   {}                     {UpperLimitBaselineRiseMin}               {big_op_spacing3}
\MathLine{Umathlimitabovekern}           {1}  {}                     {0}                                       {big_op_spacing5}
\MathLine{Umathlimitabovevgap}           {}   {}                     {UpperLimitGapMin}                        {big_op_spacing1}
\MathLine{Umathlimitbelowbgap}           {}   {}                     {LowerLimitBaselineDropMin}               {big_op_spacing4}
\MathLine{Umathlimitbelowkern}           {1}  {}                     {0}                                       {big_op_spacing5}
\MathLine{Umathlimitbelowvgap}           {}   {}                     {LowerLimitGapMin}                        {big_op_spacing2}
\MathLine{Umathoverdelimitervgap}        {}   {}                     {StretchStackGapBelowMin}                 {big_op_spacing1}
\MathLine{Umathoverdelimiterbgap}        {}   {}                     {StretchStackTopShiftUp}                  {big_op_spacing3}
\MathLine{Umathunderdelimitervgap}       {}   {}                     {StretchStackGapAboveMin}                 {big_op_spacing2}
\MathLine{Umathunderdelimiterbgap}       {}   {}                     {StretchStackBottomShiftDown}             {big_op_spacing4}
\MathLine{Umathoverbarkern}              {}   {}                     {OverbarExtraAscender}                    {default_rule_thickness}
\MathLine{Umathoverbarrule}              {}   {}                     {OverbarRuleThickness}                    {default_rule_thickness}
\MathLine{Umathoverbarvgap}              {}   {}                     {OverbarVerticalGap}                      {3*default_rule_thickness}
\MathLine{Umathquad}                     {1}  {}                     {<font_size(f)>}                          {math_quad}
\MathLine{Umathradicalkern}              {}   {}                     {RadicalExtraAscender}                    {default_rule_thickness}
\MathLine{Umathradicalrule}              {2}  {}                     {RadicalRuleThickness}                    {<not set>}
\MathLine{Umathradicalvgap}              {3}  {D, D'}                {RadicalDisplayStyleVerticalGap}          {default_rule_thickness+abs(math_x_height)/4}
\MathLine{Umathradicalvgap}              {3}  {T, T', S, S', SS, SS'}{RadicalVerticalGap}                      {default_rule_thickness+abs(default_rule_thickness)/4}
\MathLine{Umathradicaldegreebefore}      {2}  {}                     {RadicalKernBeforeDegree}                 {<not set>}
\MathLine{Umathradicaldegreeafter}       {2}  {}                     {RadicalKernAfterDegree}                  {<not set>}
\MathLine{Umathradicaldegreeraise}       {2,7}{}                     {RadicalDegreeBottomRaisePercent}         {<not set>}
\MathLine{Umathspaceafterscript}         {4}  {}                     {SpaceAfterScript}                        {script_space}
\MathLine{Umathstackdenomdown}           {}   {D, D'}                {StackBottomDisplayStyleShiftDown}        {denom1}
\MathLine{Umathstackdenomdown}           {}   {T, T', S, S', SS, SS'}{StackBottomShiftDown}                    {denom2}
\MathLine{Umathstacknumup}               {}   {D, D'}                {StackTopDisplayStyleShiftUp}             {num1}
\MathLine{Umathstacknumup}               {}   {T, T', S, S', SS, SS'}{StackTopShiftUp}                         {num3}
\MathLine{Umathstackvgap}                {}   {D, D'}                {StackDisplayStyleGapMin}                 {7*default_rule_thickness}
\MathLine{Umathstackvgap}                {}   {T, T', S, S', SS, SS'}{StackGapMin}                             {3*default_rule_thickness}
\MathLine{Umathsubshiftdown}             {}   {}                     {SubscriptShiftDown}                      {sub1}
\MathLine{Umathsubshiftdrop}             {}   {}                     {SubscriptBaselineDropMin}                {sub_drop}
\MathLine{Umathsubsupshiftdown}          {8}  {}                     {SubscriptShiftDownWithSuperscript}       {\emdash}
\MathLine{Umathsubtopmax}                {}   {}                     {SubscriptTopMax}                         {abs(math_x_height*4)/5}
\MathLine{Umathsubsupvgap}               {}   {}                     {SubSuperscriptGapMin}                    {4*default_rule_thickness}
\MathLine{Umathsupbottommin}             {}   {}                     {SuperscriptBottomMin}                    {abs(math_x_height/4)}
\MathLine{Umathsupshiftdrop}             {}   {}                     {SuperscriptBaselineDropMax}              {sup_drop}
\MathLine{Umathsupshiftup}               {}   {D}                    {SuperscriptShiftUp}                      {sup1}
\MathLine{Umathsupshiftup}               {}   {T, S, SS,}            {SuperscriptShiftUp}                      {sup2}
\MathLine{Umathsupshiftup}               {}   {D', T', S', SS'}      {SuperscriptShiftUpCramped}               {sup3}
\MathLine{Umathsupsubbottommax}          {}   {}                     {SuperscriptBottomMaxWithSubscript}       {abs(math_x_height*4)/5}
\MathLine{Umathunderbarkern}             {}   {}                     {UnderbarExtraDescender}                  {default_rule_thickness}
\MathLine{Umathunderbarrule}             {}   {}                     {UnderbarRuleThickness}                   {default_rule_thickness}
\MathLine{Umathunderbarvgap}             {}   {}                     {UnderbarVerticalGap}                     {3*default_rule_thickness}
\MathLine{Umathconnectoroverlapmin}      {5}  {}                     {MinConnectorOverlap}                     {0}
\LL
\stoptabulate

A few notes:

\startitemize[n]

\startitem
    \OPENTYPE\ fonts set \prm {Umathlimitabovekern} and \prm
    {Umathlimitbelowkern} to zero and set \prm {Umathquad} to the font size of
    the used font, because these are not supported in the \type {MATH} table.
\stopitem

\startitem
    Traditional \TFM\ fonts do not set \prm {Umathradicalrule} because \TEX82\
    uses the height of the radical instead. When this parameter is indeed not set
    when \LUATEX\ has to typeset a radical, a backward compatibility mode will
    kick in that assumes that an oldstyle \TEX\ font is used. Also, they do not
    set \prm {Umathradicaldegreebefore}, \prm {Umathradicaldegreeafter}, and \prm
    {Umathradicaldegreeraise}. These are then automatically initialized to
    $5/18$quad, $-10/18$quad, and 60.
\stopitem

\startitem
    If \TFM\ fonts are used, then the \prm {Umathradicalvgap} is not set until
    the first time \LUATEX\ has to typeset a formula because this needs
    parameters from both family~2 and family~3. This provides a partial backward
    compatibility with \TEX82, but that compatibility is only partial: once the
    \prm {Umathradicalvgap} is set, it will not be recalculated any more.
\stopitem

\startitem
    When \TFM\ fonts are used a similar situation arises with respect to \prm
    {Umathspaceafterscript}: it is not set until the first time \LUATEX\ has to
    typeset a formula. This provides some backward compatibility with \TEX82. But
    once the \prm {Umathspaceafterscript} is set, \prm {scriptspace} will never
    be looked at again.
\stopitem

\startitem
    Traditional \TFM\ fonts set \prm {Umathconnectoroverlapmin} to zero because
    \TEX82\ always stacks extensibles without any overlap.
\stopitem

\startitem
    The \prm {Umathoperatorsize} is only used in \prm {displaystyle}, and is only
    set in \OPENTYPE\ fonts. In \TFM\ font mode, it is artificially set to one
    scaled point more than the initial attempt's size, so that always the \quote
    {first next} will be tried, just like in \TEX82.
\stopitem

\startitem
    The \prm {Umathradicaldegreeraise} is a special case because it is the only
    parameter that is expressed in a percentage instead of a number of scaled
    points.
\stopitem

\startitem
    \type {SubscriptShiftDownWithSuperscript} does not actually exist in the
    \quote {standard} \OPENTYPE\ math font Cambria, but it is useful enough to be
    added.
\stopitem

\startitem
    \type {FractionDelimiterDisplayStyleSize} and \type {FractionDelimiterSize}
    do not actually exist in the \quote {standard} \OPENTYPE\ math font Cambria,
    but were useful enough to be added.
\stopitem

\stopitemize

As this mostly refers to \LUATEX\ there is more to tell about how \LUAMETATEX\
deals with it. However, it is enough to know that much more behavior is
configurable.

You can let the engine ignore parameter with \prm {setmathignore}, like:

\starttyping
\setmathignore \Umathspacebeforescript 1
\setmathignore \Umathspaceafterscript  1
\stoptyping

Be aware of the fact that a global setting can get unnoticed by users because
there is no warning that some parameter is ignored.

\stopsubsection

\stopsection

\startsection[title={Extra parameters}]

\startsubsection[title={Style related parameters}]

There are a couple of parameters that don't relate to the font but are more generally
influencing the appearances. Some were added for experimenting.

\starttabulate[|l|l|]
\DB primitive \BC meaning \NC \NR
\prm {Umathextrasubpreshift}    \NC \NR
\prm {Umathextrasubprespace}    \NC \NR
\prm {Umathextrasubshift}       \NC \NR
\prm {Umathextrasubspace}       \NC \NR
\prm {Umathextrasuppreshift}    \NC \NR
\prm {Umathextrasupprespace}    \NC \NR
\prm {Umathextrasupshift}       \NC \NR
\prm {Umathextrasupspace}       \NC \NR
\prm {Umathsubshiftdistance}    \NC \NR
\prm {Umathsupshiftdistance}    \NC \NR
\prm {Umathpresubshiftdistance} \NC \NR
\prm {Umathpresupshiftdistance} \NC \NR
\prm {Umathprimeshiftdrop}      \NC \NR
\LL
\stoptabulate

\stopsubsection

\startsubsection[title={Math struts}]

Todo:

\starttabulate[|l|l|]
\DB primitive \BC meaning \NC \NR
\NC \prm {Umathruleheight} \NC \NR
\NC \prm {Umathruledepth}  \NC \NR
\LL
\stoptabulate

\stopsubsection

\stopsection

\startsection[title={Math spacing}]

\startsubsection[reference=spacing:surround,title={Setting inline surrounding space with \prm {mathsurround} and \prm {mathsurroundskip}}]

\topicindex {math+spacing}

Inline math is surrounded by (optional) \prm {mathsurround} spacing but that is a fixed
dimension. There is now an additional parameter \prm {mathsurroundskip}. When set to a
non|-|zero value (or zero with some stretch or shrink) this parameter will replace
\prm {mathsurround}. By using an additional parameter instead of changing the nature
of \prm {mathsurround}, we can remain compatible. In the meantime a bit more
control has been added via \prm {mathsurroundmode}. This directive can take 6 values
with zero being the default behavior.

\start

\def\MathHack#1{\mathsurroundmode#1\relax\inlinebuffer}

\def\OneLiner#1#2%
  {\NC \type{#1}
   \NC \dontleavehmode\inframed[align=normal,offset=0pt,frame=off]{\hsize 100pt  x$\MathHack{#1}x$x}
   \NC \dontleavehmode\inframed[align=normal,offset=0pt,frame=off]{\hsize 100pt x $\MathHack{#1}x$ x}
   \NC #2
   \NC \NR}

\startbuffer
\mathsurround    10pt
\mathsurroundskip20pt
\stopbuffer

\typebuffer

\starttabulate[|c|c|c|pl|]
\DB mode \BC x\$x\$x \BC x \$x\$ x \BC effect \NC \NR
\TB
\OneLiner{0}{obey \prm {mathsurround} when \prm {mathsurroundskip} is 0pt}
\OneLiner{1}{only add skip to the left}
\OneLiner{2}{only add skip to the right}
\OneLiner{3}{add skip to the left and right}
\OneLiner{4}{ignore the skip setting, obey \prm {mathsurround}}
\OneLiner{5}{disable all spacing around math}
\OneLiner{6}{only apply \prm {mathsurroundskip} when also spacing}
\OneLiner{7}{only apply \prm {mathsurroundskip} when no spacing}
\LL
\stoptabulate

\stop

Anything more fancy, like checking the beginning or end of a paragraph (or edges
of a box) would not be robust anyway. If you want that you can write a callback
that runs over a list and analyzes a paragraph. Actually, in that case you could
also inject glue (or set the properties of a math node) explicitly. So, these
modes are in practice mostly useful for special purposes and experiments (they
originate in a tracker item). Keep in mind that this glue is part of the math
node and not always treated as normal glue: it travels with the begin and end
math nodes. Also, method 6 and 7 will zero the skip related fields in a node when
applicable in the first occasion that checks them (linebreaking or packaging).

\stopsubsection

\startsubsection[title={Pairwise spacing}]

\topicindex {math+spacing}

Besides the parameters mentioned in the previous sections, there are also
primitives to control the math spacing table (as explained in Chapter~18 of the
\TEX book). This happens per class pair. Because we have many possible classes,
we no longer have the many primitives that \LUATEX\ has but you can define then
using the generic \prm {setmathspacing} primitive:

\starttyping
\def\Umathordordspacing     {\setmathspacing 0 0 }
\def\Umathordordopenspacing {\setmathspacing 0 4 }
\stoptyping

These parameters are (normally) of type \prm {muskip}, so setting a parameter can
be done like this:

\starttyping
\setmathspacing 1 0 \displaystyle=4mu plus 2mu % op ord Umathopordspacing
\stoptyping

The atom pairs known by the engine are all initialized by \type {initex} to the
values mentioned in the table in Chapter~18 of the \TEX book.

For ease of use as well as for backward compatibility, \prm {thinmuskip}, \prm
{medmuskip} and \prm {thickmuskip} are treated specially. In their case a pointer
to the corresponding internal parameter is saved, not the actual \prm {muskip}
value. This means that any later changes to one of these three parameters will be
taken into account. As a bonus we also introduced the \prm {tinymuskip} and \prm
{pettymuskip} primitives, just because we consider these fundamental, but they
are not assigned internally to atom spacing combinations.

In \LUAMETATEX\ we go a bit further. Any named dimension, glue and mu glue
register as well as the constants with these properties can be bound to a pair by
prefixing \prm {setmathspacing} by \prm {inherited}.

Careful readers will realize that there are also primitives for the items marked
\type {*} in the \TEX book. These will actually be used because we pose no
restrictions. However, you can enforce the remapping rules to conform to the
rules of \TEX\ (or yourself).

Every class has a set of spacing parameters and the more classes you define the more
pairwise spacing you need to define. However, you can default to an existing class.
By default all spacing is zero and you can get rid of the defaults inherited from
good old \TEX\ with \prm {resetmathspacing}. You can alias class spacing to an exiting
class with \prm {letmathspacing}:

\starttyping
\letmathspacing class displayclass textclass scriptclass scriptscriptclass
\stoptyping

Instead you can copy spacing with \prm {copymathspacing}:

\starttyping
\copymathspacing class parentclass
\stoptyping

Specific paring happens with \prm {setmathspacing}:

\starttyping
\setmathspacing leftclass rightclass style value
\stoptyping

Unless we have a frozen parameter, the prefix \prm {inherited} makes it possible
to have a more dynamic relationship: the used value resolves to the current value
of the given register. Possible values are the usual mu skip register, a regular
skip or dimension register, or just some mu skip value.

A similar set of primitives deals with rules. These remap pairs onto other pairs, so
\prm {setmathatomrule} looks like:

\starttyping
\setmathatomrule oldleftclass oldrightclass newleftclass newrightclass
\stoptyping

The \prm {letmathatomrule} and \prm {copymathatomrule} primitives take two
classes where the second is the parent.

Some primitives are still experimental and might evolve, like \prm
{letmathparent} and \prm {copymathparent} that take numbers as in:

\starttyping
\letmathatomrule class spacingclass prepenaltyclass postpenaltyclass options reserved
\stoptyping

Primitives like this were used when experimenting and when re use them in \CONTEXT\
eventually they will become stable.

The \prm {setmathprepenalty} and \prm {setmathpostpenalty} primitives take a
class and penalty (integer) value. These are injected before and after atoms with
the given class where a penalty of 10000 is a signal to ignore it.

The engine control options for a class can be set with \prm {setmathoptions}. The
possible options are discussed elsewhere. This primitive takes a class number and
an integer (bitset). For all these setters the \CONTEXT\ math setup gives examples.

\stopsubsection

% \setdefaultmathcodes
% \setmathignore        parameter

\startsubsection[title={Local \prm {frozen} settings with}]

Math is processed in two passes. The first pass is needed to intercept for
instance \type {\over}, one of the few \TEX\ commands that actually has a
preceding argument. There are often lots of curly braces used in math and these
can result in a nested run of the math sub engine. However, you need to be aware
of the fact that some properties are kind of global to a formula and the last
setting (for instance a family switch) wins. This also means that a change (or
again, the last one) in math parameters affects the whole formula. In
\LUAMETATEX\ we have changed this model a bit. One can argue that this introduces
an incompatibility but it's hard to imagine a reason for setting the parameters
at the end of a formula run and assume that they also influence what goes in
front.

\startbuffer
$
                                               x \subscript  {-}
     \frozen\Umathsubshiftdown\textstyle  0pt  x \subscript  {0}
    {\frozen\Umathsubshiftdown\textstyle  5pt  x \subscript  {5}}
                                               x \subscript  {0}
    {\frozen\Umathsubshiftdown\textstyle 15pt  x \subscript {15}}
                                               x \subscript  {0}
    {\frozen\Umathsubshiftdown\textstyle 20pt  x \subscript {20}}
                                               x \subscript  {0}
     \frozen\Umathsubshiftdown\textstyle 10pt  x \subscript {10}
                                               x \subscript  {0}
$
\stopbuffer

\typebuffer

The \type {\frozen} prefix does the magic: it injects information in the
math list about the set parameter.

In \LUATEX\ 1.10+ the last setting, the \type {10pt} drop wins, but in
\LUAMETATEX\ you will see each local setting taking effect. The implementation
uses a new node type, parameters nodes, so you might encounter these in an
unprocessed math list. The result looks as follows:

\blank \getbuffer \blank

\stopsubsection

\startsubsection[title={Arbitrary atoms with \prm {mathatom} etc.}]

% The original \TEX\ engine has primitives like \prm {mathord} and a limited set of
% possible atoms. In \LUAMETATEX\ we have many more built in and you can add more.
% It will take a while before we have documented all the new math features and more
% details can be found in the manuals that come with \CONTEXT\ for which all this
% was implemented. In addition to \prm {mathordinary} (aka \prm {mathord}), \prm
% {mathoperator} (aka \prm {mathop}), \prm {mathbinary} (aka \prm {mathbin}), \prm
% {mathrelation} (aka \prm {mathrel}), \prm {mathopen}, \prm {mathclose}, \prm
% {mathpunctuation} (aka {mathpunct}) and \prm {mathinner} we have \prm
% {mathfraction}, \prm {mathradical}, \prm {mathmiddle}, \prm {mathaccent}, \prm
% {mathfenced}, \prm {mathghost} and the existing \prm {mathunderline} (aka \prm
% {underline}) and \prm {mathoverline} (aka \prm {overline}) class driven atoms.

The \prm {mathatom} primitive is the generic one and it accepts a couple of
keywords:

\starttabulate[|lT|l|l|]
\DB keyword    \BC argument \NC meaning \NC \NR
\TB
\NC attr       \NC int int \NC attributes to be applied to this atom \NC \NR
\NC leftclass  \NC class   \NC the left edge class that determines spacing etc \NC \NR
\NC rightclass \NC class   \NC the right edge class that determines spacing etc \NC \NR
\NC class      \NC class   \NC the general class \NC \NR
\NC unpack     \NC         \NC unpack this atom in inline math \NC \NR
\NC source     \NC int     \NC a symbolic index of the resulting box \NC \NR
\NC textfont   \NC         \NC use the current text font \NC \NR
\NC mathfont   \NC         \NC use the current math font \NC \NR
\NC limits     \NC         \NC put scripts on top and below \NC \NR
\NC nolimits   \NC         \NC force scripts to be postscripts \NC \NR
\NC nooverflow \NC         \NC keep (extensible) within target dimensions \NC \NR
\NC options    \NC int     \NC bitset with options \NC \NR
\NC void       \NC         \NC discard content and ignore dimensions \NC \NR
\NC phantom    \NC         \NC discard content but retain dimensions \NC \NR
\LL
\stoptabulate

To what extend the options kick in depends on the class as well where and how the
atom is used.

The original \TEX\ engines has three atom modifiers: \prm {displaylimits}, \prm
{limits}, and \prm {nolimits}. These look back to the last atom and set a limit
related signal. Just to be consistent we have some more of that: \prm
{Umathadapttoleft}, \prm {Umathadapttoright}, \prm {Umathuseaxis}, \prm
{Umathnoaxis}, \prm {Umathphantom}, \prm {Umathvoid}, \prm {Umathsource}, \prm
{Umathopenupheight}, \prm {Umathopenupdepth}, \prm {Umathlimits}, \prm
{Umathnolimits}. The last two are equivalent to the lowercase ones with the
similar names. Al these modifiers are cheap primitives and one can wonder if they
are needed but that also now also applies to the original three. We could stick
to one modifier that takes an integer but let's not diverge too much from the
original concept.

The \prm {nonscript} primitive injects a glue node that signals that the next
glue is to be ignored when we are in script or scriptscript mode. The \prm
{noatomruling} does the same but this time the signal is that no inter|-|atom
rules need to be applied.

\stopsubsection

\startsubsection[title={Checking a state with \prm {ifmathparameter}}]

When you adapt math parameters it might make sense to see if they are set at all.
When a parameter is unset its value has the maximum dimension value and you might
for instance mistakenly multiply that value to open up things a bit, which gives
unexpected side effects. For that reason there is a convenient checker: \prm
{ifmathparameter}. This test primitive behaves like an \prm {ifcase}, with:

\starttabulate[|c|l|]
\DB value \BC meaning \NC \NR
\TB
\NC 0 \NC the parameter value is zero \NC \NR
\NC 1 \NC the parameter is set \NC \NR
\NC 2 \NC the parameter is unset \NC \NR
\LL
\stoptabulate

\stopsubsection

\startsubsection[title={Forcing fixed scripts with \prm {mathscriptsmode}}]

We have three parameters that are used for this fixed anchoring:

\starttabulate[|c|l|]
\DB parameter \BC register \NC \NR
\NC $d$ \NC \prm {Umathsubshiftdown}    \NC \NR
\NC $u$ \NC \prm {Umathsupshiftup}      \NC \NR
\NC $s$ \NC \prm {Umathsubsupshiftdown} \NC \NR
\LL
\stoptabulate

When we set \prm {mathscriptsmode} to a value other than zero these are used
for calculating fixed positions. This is something that is needed for instance
for chemistry. You can manipulate the mentioned variables to achieve different
effects.

\def\SampleMath#1%
  {$\mathscriptsmode#1\mathupright CH_2 + CH^+_2 + CH^2_2$}

\starttabulate[|c|c|c|p|]
\DB mode \BC down          \BC up            \BC example        \NC \NR
\TB
\NC 0    \NC dynamic       \NC dynamic       \NC \SampleMath{0} \NC \NR
\NC 1    \NC $d$           \NC $u$           \NC \SampleMath{1} \NC \NR
\NC 2    \NC $s$           \NC $u$           \NC \SampleMath{2} \NC \NR
%
% In \LUATEX\ but dropped in \LUAMETATEX:
%
%NC 3    \NC $s$           \NC $u + s - d$   \NC \SampleMath{3} \NC \NR
%NC 4    \NC $d + (s-d)/2$ \NC $u + (s-d)/2$ \NC \SampleMath{4} \NC \NR
%NC 5    \NC $d$           \NC $u + s - d$   \NC \SampleMath{5} \NC \NR
\LL
\stoptabulate

The value of this parameter obeys grouping and is applied to character atoms only
(but that might evolve as we go).

\stopsubsection

\startsubsection[title={Penalties: \prm {mathpenaltiesmode}}]

\topicindex {math+penalties}

Only in inline math penalties will be added in a math list. You can force
penalties (also in display math) by setting:

\starttyping
\mathpenaltiesmode = 1
\stoptyping

This primnitive is not really needed in \LUATEX\ because you can use the callback
\cbk {mlist_to_hlist} to force penalties by just calling the regular routine
with forced penalties. However, as part of opening up and control this primitive
makes sense. As a bonus we also provide two extra penalties:

\starttyping
\prebinoppenalty = -100 % example value
\prerelpenalty   =  900 % example value
\stoptyping

They default to inifinite which signals that they don't need to be inserted. When
set they are injected before a binop or rel noad. This is an experimental feature.

\stopsubsection

\startsubsection[title={Equation spacing: \prm {matheqnogapstep}}]

By default \TEX\ will add one quad between the equation and the number. This is
hard coded. A new primitive can control this:

\startsyntax
\matheqnogapstep = 1000
\stopsyntax

Because a math quad from the math text font is used instead of a dimension, we
use a step to control the size. A value of zero will suppress the gap. The step
is divided by 1000 which is the usual way to mimmick floating point factors in
\TEX.

\stopsubsection

\stopsection

\startsection[title={Math constructs}]

\startsubsection[title={Cheating with fences}]

\topicindex {math+fences}

Sometimes you might want to act upon the size of a delimiter, something that is
not really possible because of the fact that they are calculated {\em after} most
has been typeset already. For this we have two keyword: \type {phantom} and
\type {void}. In both cases the symbol is replaced by an empty rule, in the first
case all three dimensions are preserved in the last case only the height and depth.

\startbuffer
\startformula
    x\mathlimop{\Uvextensible         \Udelimiter 5 0 "222B}_1^2 x
\stopformula
\vskip-9ex
\startformula \red
    x\mathlimop{\Uvextensible phantom \Udelimiter 5 0 "222B}_1^2 x
\stopformula
\vskip-9ex
\startformula \blue
    x\mathlimop{\Uvextensible void    \Udelimiter 5 0 "222B}_1^2 x
\stopformula
\stopbuffer

\typebuffer

In typeset form this looks like:

\getbuffer

Normally fences need to be matched, that is: when a left fence is seen, there has
to be a right fence. When you set \prm {mathcheckfencesmode} to non|-|zero the
scanner silently recovers from this.

\stopsubsection

\startsubsection[title={Accent handling with \prm {Umathaccent}},reference=mathacc]

\topicindex {math+accents}

\LUATEX\ supports both top accents and bottom accents in math mode, and math
accents stretch automatically (if this is supported by the font the accent comes
from, of course). Bottom and combined accents as well as fixed-width math accents
are controlled by optional keywords following \prm {Umathaccent}.

The keyword \type {bottom} after \prm {Umathaccent} signals that a bottom accent
is needed, and the keyword \type {both} signals that both a top and a bottom
accent are needed (in this case two accents need to be specified, of course).

Then the set of three integers defining the accent is read. This set of integers
can be prefixed by the \type {fixed} keyword to indicate that a non-stretching
variant is requested (in case of both accents, this step is repeated).

A simple example:

\starttyping
\Umathaccent both fixed 0 0 "20D7 fixed 0 0 "20D7 {example}
\stoptyping

If a math top accent has to be placed and the accentee is a character and has a
non-zero \type {top_accent} value, then this value will be used to place the
accent instead of the \prm {skewchar} kern used by \TEX82.

The \type {top_accent} value represents a vertical line somewhere in the
accentee. The accent will be shifted horizontally such that its own \type
{top_accent} line coincides with the one from the accentee. If the \type
{top_accent} value of the accent is zero, then half the width of the accent
followed by its italic correction is used instead.

The vertical placement of a top accent depends on the \type {x_height} of the
font of the accentee (as explained in the \TEX book), but if a value turns out
to be zero and the font had a \type {MathConstants} table, then \type
{AccentBaseHeight} is used instead.

The vertical placement of a bottom accent is straight below the accentee, no
correction takes place.

Possible locations are \type {top}, \type {bottom}, \type {both} and \type
{center}. When no location is given \type {top} is assumed. An additional
parameter \nod {fraction} can be specified followed by a number; a value of for
instance 1200 means that the criterium is 1.2 times the width of the nucleus. The
fraction only applies to the stepwise selected shapes and is mostly meant for the
\type {overlay} location. It also works for the other locations but then it
concerns the width.

\stopsubsection

\startsubsection[title={Building radicals with \prm {Uradical}, \prm {Uroot} and \prm {Urooted}}]

\topicindex {math+radicals}
\topicindex {math+roots}

The new primitive \prm {Uroot} allows the construction of a radical noad
including a degree field. Its syntax is an extension of \prm {Uradical}:

\starttyping
\Uradical
    <fam integer> <left char integer>
    <content>
\Uroot
    <fam integer> <left char integer>
    <degree>
    <content>
\Urooted
    <fam integer> <left char integer>
    <fam integer> <right char integer>
    <degree>
    <content>
\stoptyping

The placement of the degree is controlled by the math parameters \prm
{Umathradicaldegreebefore}, \prm {Umathradicaldegreeafter}, and \prm
{Umathradicaldegreeraise}. The degree will be typeset in \prm
{scriptscriptstyle}.

In \CONTEXT\ we use \prm {Urooted} to wrap something in an \quote {annuity}
umbrella where there is a symbol at the end that has to behave like the radical
does at the left end: adapt its size. In order to support variants this primitive
supports two delimiters.

{\em todo: mention optional keywords}

\stopsubsection

\startsubsection[title={Tight delimiters with \prm {Udelimited}}]

\topicindex {math+radicals}
\topicindex {math+delimiters}

This new primitive is like \prm {Urooted} in that it takes two delimiters but it
takes no degree and no rule is drawn.

\starttyping
\Udelimited
    <fam integer> <left char integer>
    <fam integer> <right char integer>
    <content>
\stoptyping

In \CONTEXT\ we use it for fourier notations in which case there is only a right
symbol (like a hat).

{\em todo: mention optional keywords}

\stopsubsection

\startsubsection[title={Super- and subscripts}]

The character fields in a \LUA|-|loaded \OPENTYPE\ math font can have a \quote
{mathkern} table. The format of this table is the same as the \quote {mathkern}
table that is returned by the \type {fontloader} library, except that all height
and kern values have to be specified in actual scaled points.

When a super- or subscript has to be placed next to a math item, \LUATEX\ checks
whether the super- or subscript and the nucleus are both simple character items.
If they are, and if the fonts of both character items are \OPENTYPE\ fonts (as
opposed to legacy \TEX\ fonts), then \LUATEX\ will use the \OPENTYPE\ math
algorithm for deciding on the horizontal placement of the super- or subscript.

This works as follows:

\startitemize
    \startitem
        The vertical position of the script is calculated.
    \stopitem
    \startitem
        The default horizontal position is flat next to the base character.
    \stopitem
    \startitem
        For superscripts, the italic correction of the base character is added.
    \stopitem
    \startitem
        For a superscript, two vertical values are calculated: the bottom of the
        script (after shifting up), and the top of the base. For a subscript, the two
        values are the top of the (shifted down) script, and the bottom of the base.
    \stopitem
    \startitem
        For each of these two locations:
        \startitemize
            \startitem
                find the math kern value at this height for the base (for a subscript
                placement, this is the bottom_right corner, for a superscript
                placement the top_right corner)
            \stopitem
            \startitem
                find the math kern value at this height for the script (for a
                subscript placement, this is the top_left corner, for a superscript
                placement the bottom_left corner)
            \stopitem
            \startitem
                add the found values together to get a preliminary result.
            \stopitem
        \stopitemize
    \stopitem
    \startitem
        The horizontal kern to be applied is the smallest of the two results from
        previous step.
    \stopitem
\stopitemize

The math kern value at a specific height is the kern value that is specified by the
next higher height and kern pair, or the highest one in the character (if there is no
value high enough in the character), or simply zero (if the character has no math kern
pairs at all).

\stopsubsection

\startsubsection[title={Scripts on extensibles: \prm {Uunderdelimiter}, \prm {Uoverdelimiter},
\prm {Udelimiterover}, \prm {Udelimiterunder} and \prm {Uhextensible}}]

\topicindex {math+scripts}
\topicindex {math+delimiters}
\topicindex {math+extensibles}

The primitives \prm {Uunderdelimiter} and \prm {Uoverdelimiter} allow the
placement of a subscript or superscript on an automatically extensible item and
\prm {Udelimiterunder} and \prm {Udelimiterover} allow the placement of an
automatically extensible item as a subscript or superscript on a nucleus. The
input:

% these produce radical noads .. in fact the code base has the numbers wrong for
% quite a while, so no one seems to use this

\startbuffer
$\Uoverdelimiter  0 "2194 {\hbox{\strut  overdelimiter}}$
$\Uunderdelimiter 0 "2194 {\hbox{\strut underdelimiter}}$
$\Udelimiterover  0 "2194 {\hbox{\strut  delimiterover}}$
$\Udelimiterunder 0 "2194 {\hbox{\strut delimiterunder}}$
\stopbuffer

\typebuffer will render this:

\blank \startnarrower \getbuffer \stopnarrower \blank

The vertical placements are controlled by \prm {Umathunderdelimiterbgap}, \prm
{Umathunderdelimitervgap}, \prm {Umathoverdelimiterbgap}, and \prm
{Umathoverdelimitervgap} in a similar way as limit placements on large operators.
The superscript in \prm {Uoverdelimiter} is typeset in a suitable scripted style,
the subscript in \prm {Uunderdelimiter} is cramped as well.

These primitives accepts an optional \type {width} specification. When used the
also optional keywords \type {left}, \type {middle} and \type {right} will
determine what happens when a requested size can't be met (which can happen when
we step to successive larger variants).

An extra primitive \prm {Uhextensible} is available that can be used like this:

\startbuffer
$\Uhextensible width 10cm 0 "2194$
\stopbuffer

\typebuffer This will render this:

\blank \startnarrower \getbuffer \stopnarrower \blank

Here you can also pass options, like:

\startbuffer
$\Uhextensible width 1pt middle 0 "2194$
\stopbuffer

\typebuffer This gives:

\blank \startnarrower \getbuffer \stopnarrower \blank

\LUATEX\ internally uses a structure that supports \OPENTYPE\ \quote
{MathVariants} as well as \TFM\ \quote {extensible recipes}. In most cases where
font metrics are involved we have a different code path for traditional fonts end
\OPENTYPE\ fonts.

\stopsubsection

\startsubsection[title={Fractions and the new \prm {Ustretched} and \prm {Ustretchedwithdelims}}]

\topicindex {math+fractions}

These commands are similar the regular rule separated fractions but expect a delimiter
that then will be used instead. This permits for instance the use of horizontal
extensible arrows. When no extensible is possible (this is a font property) the given
glyph is centered.

Normally one will pass a specific delimiter and not a character, if only because
these come from the non \ASCII\ ranges:

\starttyping
{ \Ustretched <delimiter> <options> {1} {2} }
{ \Ustretchedwithdelims <delimiter> () <options> {1} {2} }
\stoptyping

\stopsubsection

\startsubsection[title={Fractions and the new \prm {Uskewed} and \prm {Uskewedwithdelims}}]

\topicindex {math+fractions}

The \prm {abovewithdelims} command accepts a keyword \type {exact}. When issued
the extra space relative to the rule thickness is not added. One can of course
use the \type {\Umathfraction..gap} commands to influence the spacing. Also the
rule is still positioned around the math axis.

\starttyping
$$ { {a} \abovewithdelims() exact 4pt {b} }$$
\stoptyping

The math parameter table contains some parameters that specify a horizontal and
vertical gap for skewed fractions. Of course some guessing is needed in order to
implement something that uses them. And so we now provide a primitive similar to the
other fraction related ones but with a few options so that one can influence the
rendering. Of course a user can also mess around a bit with the parameters
\prm {Umathskewedfractionhgap} and \prm {Umathskewedfractionvgap}.

The syntax used here is:

\starttyping
{ \Uskewed / <options> {1} {2} }
{ \Uskewedwithdelims / () <options> {1} {2} }
\stoptyping

where the options can be \type {noaxis} and \type {exact}. By default we add half
the axis to the shifts and by default we zero the width of the middle character.
For Latin Modern the result looks as follows:

\def\ShowA#1#2#3{$x + { \Uskewed           /    #3 {#1} {#2} } + x$}
\def\ShowB#1#2#3{$x + { \Uskewedwithdelims / () #3 {#1} {#2} } + x$}

\start
    \switchtobodyfont[modern]
    \starttabulate[||||||]
        \NC \NC
            \ShowA{a}{b}{} \NC
            \ShowA{1}{2}{} \NC
            \ShowB{a}{b}{} \NC
            \ShowB{1}{2}{} \NC
        \NR
        \NC \type{exact} \NC
            \ShowA{a}{b}{exact} \NC
            \ShowA{1}{2}{exact} \NC
            \ShowB{a}{b}{exact} \NC
            \ShowB{1}{2}{exact} \NC
        \NR
        \NC \type{noaxis} \NC
            \ShowA{a}{b}{noaxis} \NC
            \ShowA{1}{2}{noaxis} \NC
            \ShowB{a}{b}{noaxis} \NC
            \ShowB{1}{2}{noaxis} \NC
        \NR
        \NC \type{exact noaxis} \NC
            \ShowA{a}{b}{exact noaxis} \NC
            \ShowA{1}{2}{exact noaxis} \NC
            \ShowB{a}{b}{exact noaxis} \NC
            \ShowB{1}{2}{exact noaxis} \NC
        \NR
    \stoptabulate
\stop

The \type {\over} and related primitives have the form:

\starttyping
{{top}\over{bottom}}
\stoptyping

For convenience, which also avoids some of the trickery that makes this
\quote {looking back} possible, the \LUAMETATEX\ also provides this variant:

\starttyping
\Uover{top}{bottom}
\stoptyping

The optional arguments are also supported but we have one extra option: \type
{style}. The style is applied to the numerator and denominator.

\starttyping
\Uover style \scriptstyle {top} {bottom}
\stoptyping

The complete list of these commands is: \prm {Uabove}, \prm {Uatop}, \prm
{Uover}, \prm {Uabovewithdelims}, \prm {Uatopwithdelims}, \prm {Uoverwithdelims},
\prm {Uskewed}, \prm {Uskewedwithdelims}. As with other extensions we use a
leading \type {U}. Here are a few examples:

\startbuffer
$\Uover {                   1234} {                   5678} $\quad
$\Uover {\textstyle         1234} {\textstyle         5678} $\quad
$\Uover {\scriptstyle       1234} {\scriptstyle       5678} $\quad
$\Uover {\scriptscriptstyle 1234} {\scriptscriptstyle 5678} $\blank

$\Uover                          {1234} {5678} $\quad
$\Uover style \textstyle         {1234} {5678} $\quad
$\Uover style \scriptstyle       {1234} {5678} $\quad
$\Uover style \scriptscriptstyle {1234} {5678} $\blank
\stopbuffer

\typebuffer

These render as: \getbuffer

\stopsubsection

\startsubsection[title={Math styles: \prm {givenmathstyle}}]

This primitive accepts a style identifier:

\starttyping
\givenmathstyle \displaystyle
\stoptyping

This in itself is not spectacular because it is equivalent to

\starttyping
\displaystyle
\stoptyping

Both commands inject a style node and change the current style. However, as in other
places where \LUAMETATEX\ expects a style you can also pass a number in the range
zero upto seven (like the ones reported by the primitive \prm {mathstyle}). So, the
next few lines give identical results:

\startbuffer
$\givenmathstyle0                         \number\mathstyle
 \givenmathstyle7                         \number\mathstyle$
$\givenmathstyle\displaystyle             \number\mathstyle
 \givenmathstyle\crampedscriptscriptstyle \number\mathstyle$
$\displaystyle                            \number\mathstyle
 \crampedscriptscriptstyle                \number\mathstyle$
\stopbuffer

Like: \inlinebuffer . Values outside the valid range are ignored.

There is an extra option \type {norule} that can be used to suppress the rule while
keeping the spacing compatible.

\stopsubsection

\startsubsection[title={Delimiters: \prm {Uleft}, \prm {Umiddle} and \prm {Uright}}]

\topicindex {math+delimiters}

Normally you will force delimiters to certain sizes by putting an empty box or
rule next to it. The resulting delimiter will either be a character from the
stepwise size range or an extensible. The latter can be quite differently
positioned than the characters as it depends on the fit as well as the fact
whether the used characters in the font have depth or height. Commands like
(plain \TEX s) \type {\big} need to use this feature. In \LUATEX\ we provide a bit
more control by three variants that support optional parameters \type {height},
\type {depth} and \type {axis}. The following example uses this:

\startbuffer
\Uleft   height 30pt depth 10pt      \Udelimiter "0 "0 "000028
\quad x\quad
\Umiddle height 40pt depth 15pt      \Udelimiter "0 "0 "002016
\quad x\quad
\Uright  height 30pt depth 10pt      \Udelimiter "0 "0 "000029
\quad \quad \quad
\Uleft   height 30pt depth 10pt axis \Udelimiter "0 "0 "000028
\quad x\quad
\Umiddle height 40pt depth 15pt axis \Udelimiter "0 "0 "002016
\quad x\quad
\Uright  height 30pt depth 10pt axis \Udelimiter "0 "0 "000029
\stopbuffer

\typebuffer

\startlinecorrection
\ruledhbox{\mathematics{\getbuffer}}
\stoplinecorrection

The keyword \type {exact} can be used as directive that the real dimensions
should be applied when the criteria can't be met which can happen when we're
still stepping through the successively larger variants. When no dimensions are
given the \type {noaxis} command can be used to prevent shifting over the axis.

You can influence the final class with the keyword \type {class} which will
influence the spacing. The numbers are the same as for character classes.

\stopsubsection

\stopsection

\startsection[title={Extracting values}]

\startsubsection[title={Codes and using \prm {Umathcode}, \prm {mathcharclass}, \prm
{mathcharfam} and \prm {mathcharslot}}]

\topicindex {math+codes}

You should not really depend on the number that comes from \prm {Umathcode} because
the engine can (at some point) use a different amount of families and classes. Given this,
you can extract the components of a math character. Say that we have defined:

\starttyping
\Umathcode 1 2 3 4
\stoptyping

then

\starttyping
[\mathcharclass\Umathcode1] [\Umathcharfam\Umathcode1] [\Umathcharslot\Umathcode1]
\stoptyping

which will return:

\starttyping
[2] [3] [4]
\stoptyping

You can of course store the code in for instance a register and use that as
argument. The three commands also accept a specification (and maybe more in the
future).

These commands are provided as convenience. Before they became available you
could do the following:

\starttyping
\def\mathcharclass{\numexpr
    \directlua{tex.print(tex.getmathcode(token.scan_int())[1])}
\relax}
\def\Umathcharfam{\numexpr
    \directlua{tex.print(tex.getmathcode(token.scan_int())[2])}
\relax}
\def\Umathcharslot{\numexpr
    \directlua{tex.print(tex.getmathcode(token.scan_int())[3])}
\relax}
\stoptyping

\stopsubsection

\startsubsection[title={Last lines and \prm{predisplaygapfactor}}]

\topicindex {math+last line}

There is a new primitive to control the overshoot in the calculation of the
previous line in mid|-|paragraph display math. The default value is 2 times
the em width of the current font:

\starttyping
\predisplaygapfactor=2000
\stoptyping

If you want to have the length of the last line independent of math i.e.\ you don't
want to revert to a hack where you insert a fake display math formula in order to
get the length of the last line, the following will often work too:

\starttyping
\def\lastlinelength{\dimexpr
    \directlua {tex.sprint (
        (nodes.dimensions(node.tail(tex.lists.page_head).list))
    )}sp
\relax}
\stoptyping

\stopsubsection

\stopsection

\startsection[title={Math mode and scripts}]

\startsubsection[title={Entering and leaving math mode with \prm {Ustartmathmode}
and \prm {Ustopmathmode}}]

These commands are variants on the single and double (usually) dollar signs that
make us enter math mode and later leave it. The start command expects a style
identifier that determines in what style we end up in.

\stopsubsection

\startsubsection[title={Verbose versions of single|-|character math commands like \prm {superscript}
and \prm {subscript}}]

\topicindex {math+styles}

\LUATEX\ defines six new primitives that have the same function as
\type {^}, \type {_}, \type {$}, and \type {$$}:

\starttabulate[|l|l|]
\DB primitive                \BC explanation \NC \NR
\TB
\NC \prm {superscript}      \NC duplicates the functionality of \type {^} \NC \NR
\NC \prm {subscript}        \NC duplicates the functionality of \type {_} \NC \NR
\NC \prm {Ustartmath}        \NC duplicates the functionality of \type {$}, % $
                                   when used in non-math mode. \NC \NR
\NC \prm {Ustopmath}         \NC duplicates the functionality of \type {$}, % $
                                   when used in inline math mode. \NC \NR
\NC \prm {Ustartdisplaymath} \NC duplicates the functionality of \type {$$}, % $$
                                   when used in non-math mode. \NC \NR
\NC \prm {Ustopdisplaymath}  \NC duplicates the functionality of \type {$$}, % $$
                                   when used in display math mode. \NC \NR
\LL
\stoptabulate

The \prm {Ustopmath} and \prm {Ustopdisplaymath} primitives check if the current
math mode is the correct one (inline vs.\ displayed), but you can freely intermix
the four mathon|/|mathoff commands with explicit dollar sign(s).

\stopsubsection

\startsubsection[title={Script commands \prm {nosuperscript}, \prm {nosubscript}, \prm {nosuperprescript} and \prm {nosubprescript}}]

\topicindex {math+styles}
\topicindex {math+scripts}

These commands result in super- and subscripts but with the current style (at the
time of rendering). So,

\startbuffer[script]
$
    x\superscript  {1}\subscript  {2} =
    x\nosuperscript{1}\nosubscript{2} =
    x\superscript  {1}\nosubscript{2} =
    x\nosuperscript{1}\subscript  {2}
$
\stopbuffer

\typebuffer[script]

results in \inlinebuffer[script].

\stopsubsection

\startsubsection[title={Script commands \prm {shiftedsuperscript}, \prm {shiftedsubscript}, \prm {shiftedsuperprescript} and \prm {shiftedsubprescript}}]

\topicindex {math+styles}
\topicindex {math+scripts}
\topicindex {math+indices}

Sometimes a script acts as an index in which case is should be anchored
differently. For that we have four extra primitives. Here the shifted postscripts
are shown:

\startbuffer[script]
$
    x\superscript       {1}\subscript       {2} =
    x\shiftedsuperscript{1}\shiftedsubscript{2} =
    x\superscript       {1}\shiftedsubscript{2} =
    x\shiftedsuperscript{1}\subscript       {2}
$
\stopbuffer

\typebuffer[script]

results in \inlinebuffer[script].

\stopsubsection

\startsubsection[title={Injecting primes with \prm {primescript}}]

This one is a bit special. In \LUAMETATEX\ a prime is a native element of a
nucleus, alongside the two prescript and two postscripts. The most confusing
combination of primes and postscripts is the case where we have a prime and
superscript. In that case we assume that in order to avoid confusion parenthesis
are applied so we don't covert it. That leaves three cases:

\startbuffer[script]
$
    a \primescript{1} \superscript{2} \subscript {3} +
    b \subscript  {3} \primescript{1} +
    c \primescript{1} \subscript  {3} = d
$
\stopbuffer

\typebuffer[script]

This gets rendered as: \inlinebuffer[script]. In this case a subscript is handled as if
it were an index.

\stopsubsection

\startsubsection[title={Prescripts with \prm {superprescript} and \prm {subprescript}}]

\startbuffer
\hbox{$
    {\tf X}^1_2^^^3___4 \quad
    {\tf X}^1  ^^^3    \quad
    {\tf X}  _1    ___4 \quad
    {\tf X}    ^^^3    \quad
    {\tf X}        ___4 \quad
    {\tf X}^^^3    ___4
$}
\stopbuffer

\typebuffer

The problem with the circumflex is that it is also used for escaping character
input. Normally that only happens in a format file so you can safely disable
that. Alternatives are using active characters that adapt. In \CONTEXT\ we make
them regular (other) characters in text mode and set \prm {supmarkmode} to~1 to
disable the normal multiple \type {^} treatment (a value larger than 1 will also
disable that in text mode). In math mode we make them active and behave as
expected.

\blank \getbuffer \blank

The more explicit commands are:

\startbuffer
\hbox{$
{\tf X}\superscript{1}                                               \quad
{\tf X}               \subscript{2}                                  \quad
{\tf X}\superscript{1}\subscript{2}                                  \quad
{\tf X}\superscript{1}             \superprescript{3}                \quad
{\tf X}               \subscript{2}                  \subprescript{4}\quad
{\tf X}\superscript{1}\subscript{2}\superprescript{3}\subprescript{4}\quad
{\tf X}                            \superprescript{3}                \quad
{\tf X}                                              \subprescript{4}\quad
{\tf X}                            \superprescript{3}\subprescript{4}
$}
\stopbuffer

\typebuffer

These more verbose triggers can be used to build interfaces:

\blank \getbuffer \blank

\stopsubsection

\startsubsection[title={Allowed math commands in non|-|math modes}]

\topicindex {math+text}
\topicindex {text+math}

The commands \prm {mathchar}, and \prm {Umathchar} and control sequences that are
the result of \prm {mathchardef} or \prm {Umathchardef} are also acceptable in
the horizontal and vertical modes. In those cases, the \prm {textfont} from the
requested math family is used.

\stopsubsection

\stopsection

% \startsection[title={Flattening: \prm {mathflattenmode}}]
%
% \topicindex {math+flattening}
%
% The \TEX\ math engine collapses \type {ord} noads without sub- and superscripts
% and a character as nucleus, which has the side effect that in \OPENTYPE\ mode
% italic corrections are applied (given that they are enabled).
%
% \startbuffer[sample]
% \switchtobodyfont[modern]
% $V \mathbin{\mathbin{v}} V$\par
% $V \mathord{\mathord{v}} V$\par
% \stopbuffer
%
% \typebuffer[sample]
%
% This renders as:
%
% \blank \start \mathflattenmode\plusone \getbuffer[sample] \stop \blank
%
% When we set \prm {mathflattenmode} to 31 we get:
%
% \blank \start \mathflattenmode\numexpr1+2+4+8+16\relax \getbuffer[sample] \stop \blank
%
% When you see no difference, then the font probably has the proper character
% dimensions and no italic correction is needed. For Latin Modern (at least till
% 2018) there was a visual difference. In that respect this parameter is not always
% needed unless of course you want efficient math lists anyway.
%
% You can influence flattening by adding the appropriate number to the value of the
% mode parameter. The default value is~1.
%
% \starttabulate[|Tc|c|]
% \DB mode \BC class \NC \NR
% \TB
% \NC  1   \NC ord   \NC \NR
% \NC  2   \NC bin   \NC \NR
% \NC  4   \NC rel   \NC \NR
% \NC  8   \NC punct \NC \NR
% \NC 16   \NC inner \NC \NR
% \LL
% \stoptabulate
%
% \stopsubsection


\startsection[title={Tracing}]

\topicindex {math+tracing}

\startsubsection[title={Assignments}]

Because there are quite some math related parameters and values, it is possible
to limit tracing. Only when \type {tracingassigns} and|/|or \type
{tracingrestores} are set to~2 or more they will be traced.

\stopsubsection

\startsubsection[title={\prm{tracingmath}}]

The \TEX\ engine has a of places where tracing information can be generated so
one can see what gets read and what comes out, but the math machinery is a black
box. In \LUAMETATEX\ the math engine has been extended with tracing too.

A value of one shows only the initial list, but a value of two also shows the
intermediate lists as well as applied rules, injected spacing, injected penalties
and parameter initialization. A value of three shows the result and larger values
will do so with maximum breadth and depth.

If you also want to see something on the console make sure to set \prm
{tracingonline} to more than one.

\stopsubsection

\stopsection

\startsection[title={Classes}]

\startsubsection[title={Forcing classes with \prm {mathclass}}]

You can change the class of a math character on the fly:

\startbuffer
$x\mathopen   {!}+123+\mathclose   {!}x$
$x\mathclass4 `! +123+\mathclass5 `! x$
$x             ! +123+             ! x$
$x\mathclose  {!}+123+\mathopen   {!}x$
$x\mathclass5 `! +123+\mathclass4 `! x$
\stopbuffer

\typebuffer

Watch how the spacing changes:

\startlines
\getbuffer
\stoplines

The \TEX\ engines deal with active characters in math differently as in text.
When a character has class~8 it will be fed back into the machinery with an
active catcode which of course assumes that there is some meaning attached.

A variant on this is the use of \prm {amcode}. A character that has that code set
and that is active when we are in math mode, will be fed back with that code as
catcode which can be one of alignment tab, superscript, subscript, letter, other
char, or active. This feature is still experimental. Watch out: when an already
active character remains active we get a loop. Also, math characters are checked
for this property too, which can then turn them active.

\stopsubsection

\startsubsection[title={Checking class states}]

When a formula is typeset it starts out with a begin class and finishes with an end class.
This is done by adding two \quote {fake} atoms.  here are two global state variables that tell
what the most recent edge classes are and two variables that act like registers and are
local. There are also two registers that can be set to values that will force begin and end
classes.

The values of \prm {mathbeginclass} and \prm {mathendclass} are used when a
formula starts and afterwards they are reset. Afterwards \prm {mathleftclass} and
\prm {mathrightclass} have the effective edge classes. The global \prm
{lastleftclass} and \prm {lastrightclass} variables also have the last edge
classes but them being global they might not always reflect what you expect.

\stopsubsection

\startsubsection[title={Getting class spacing}]

You can query the pairwise spacing of atoms with \prm {mathatomglue} and inject it with
\prm {mathatomskip}, as in:

\startbuffer
\the\mathatomglue 5 4 \displaystyle : $[\mathatomskip 5 4 \displaystyle]$
\stopbuffer

\typebuffer

and get: \inlinebuffer.

\stopsubsection

\startsubsection[title={Default math codes}]

The probably not that useful primitive (but who knows) \prm {setdefaultmathcodes}
initializes the mathcodes of digits to family zero and the lowercase and
uppercase letters to family one, just as standard \TEX\ does. Don't do this in in
\CONTEXT.

\stopsubsection

\stopsection

\startsection[title={Modes}]

\startsubsection[title={Introduction}]

For most cases the math engine acts the same as in regular \TEX, apart of course
from some font specific features that need to be supported out of the box. There
are however ways to divert, which we do in \CONTEXT. The following paragraphs are
therefore rather \CONTEXT\ driven and not that relevant otherwise. Some modes have
been removed and became default and|/|ro were replaced by more granular options.

\stopsubsection

\startsubsection[title={Controlling display math with \prm {mathdisplaymode}}]

By setting \prm {mathdisplaymode} larger than zero double math shift characters
(normally the dollar sign) are disabled. The reason for this feature is rather
\CONTEXT\ specific: we pay a lot attention to spacing and the build in heuristics
don't work well with that. We also need to initialize display math as well as
deal with whatever has to be done with respect to finalizing. Because users use
the high level commands anyway, disabling is okay for \CONTEXT\ and less likely
to be done for other macro packages, so be careful with this one.

\stopsubsection

\startsubsection[title={Skips around display math and \prm {mathdisplayskipmode}}]

\topicindex {math+spacing}

The injection of \prm {abovedisplayskip} and \prm {belowdisplayskip} is not
symmetrical. An above one is always inserted, also when zero, but the below is
only inserted when larger than zero. Especially the latter makes it sometimes hard
to fully control spacing. Therefore \LUATEX\ comes with a new directive: \prm
{mathdisplayskipmode}. The following values apply:

\starttabulate[|c|l|]
\DB value \BC meaning \NC \NR
\TB
\NC 0 \NC normal \TEX\ behavior \NC \NR
\NC 1 \NC always (same as 0) \NC \NR
\NC 2 \NC only when not zero \NC \NR
\NC 3 \NC never, not even when not zero \NC \NR
\LL
\stoptabulate

\stopsubsection

\startsubsection[title={Scripts}]

The regular superscript trigger \type {^} and subscript trigger \type {_} are
quite convenient and although we do have verbose aliases in regular text these
two will be used. A multiple superscript character sequence is used for accessing
characters by number unless you disable that via catcode manipulations. In
\CONTEXT\ the super- and subscript characters are regular characters and only in
math mode they have a special meaning. We can have upto for script characters and
they indicate pre- and postscripts.

\starttabulate[|c|c|c|c|]
\NC \type {^}    \NC         \NC super \NC post \NC \NR
\NC \type {_}    \NC         \NC sub   \NC post \NC \NR
\NC \type {^^}   \NC         \NC super \NC post \NC \NR
\NC \type {__}   \NC         \NC sub   \NC post \NC \NR
\NC \type {^^^}  \NC shifted \NC super \NC pre  \NC \NR
\NC \type {___}  \NC shifted \NC sub   \NC pre  \NC \NR
\NC \type {^^^^} \NC shifted \NC super \NC pre  \NC \NR
\NC \type {____} \NC shifted \NC sub   \NC pre  \NC \NR
\stoptabulate

The shifted variants force a script to be marked as index. In versions upto 2.10
a subscript was moved to after the superscript but when we introduced continued
scripts that feature was disabled (we might bring it back as configurable
option). The shifted variants now behave the same but at the \LUA\ end one can
check if they carry this flag. Since 2.11 the one and two character are for post
while three and four handle prescripts (indices are more common than prescripts).
The \prm {noscript} command signals that we're up for a new continuation.

Related to this is the issue of double scripts. The regular \TEX\ is to issue
an error message, inject an ordinary node and carry on when asked to. Here we have
\prm {mathdoublescriptmode} as escape: when set to a zero or positive value it will
also inject an atom but with class properties determined by its value. There will
be no error message.

\starttyping
\mathdoublescriptmode"MMLLRR % main left right
\stoptyping

% Another variable that influences rendering is \prm {mathscriptcharmode} that in
% \CONTEXT\ we default to~1. The \prm {mathscriptboxmode} parameter determines if a
% boxed nucleus is analyzed and is also set to~1 in \CONTEXT. Both parameters are a
% left|-|over from the split code path approach and might still be handy for
% experiments. They might go away some day of replaced by a different control
% option. Older versions if \LUAMETATEX\ had optional behavior for different values
% but that code was removed.

\stopsubsection

\startsubsection[title={Grouping}]

When set to non zero \prm {mathgroupingmode} will make stand alone \quote {list}
as in \typ {a {bc} d} behave like grouping instead of creating composite atoms.
In \CONTEXT\ indeed we set it to a positive value. Although it was ot strictly
necessary it is nicer when users don't get side effects if they revert to low
level source coding.

\stopsubsection

\startsubsection[title={Slack}]

The \prm {mathslackmode} parameters controls removal of accidental left and|/|or
right space added to a formula. Of course we enable this in \CONTEXT. There is
more detailed control possible at the atom label as well as with class options.

\stopsubsection

\startsubsection[title={Limit fitting \prm {mathlimitsmode}}]

\topicindex {math+limits}

When set, this parameter avoids too wide limits to stick out too much by sort of
centering them.

\stopsubsection

\startsubsection[title={Nolimit correction with \prm {mathnolimitsmode}}]

\topicindex {math+limits}

There are two extra math parameters \prm {Umathnolimitsupfactor} and \prm
{Umathnolimitsubfactor} that were added to provide some control over how limits
are spaced (for example the position of super and subscripts after integral
operators). They relate to an extra parameter \prm {mathnolimitsmode}. The half
corrections are what happens when scripts are placed above and below. The
problem with italic corrections is that officially that correction italic is used
for above|/|below placement while advanced kerns are used for placement at the
right end. The question is: how often is this implemented, and if so, do the
kerns assume correction too. Anyway, with this parameter one can control it.

\starttabulate[|l|ck1|ck1|ck1|ck1|ck1|ck1|]
    \NC
        \NC \mathnolimitsmode0    $\displaystyle\int\nolimits^0_1$
        \NC \mathnolimitsmode1    $\displaystyle\int\nolimits^0_1$
        \NC \mathnolimitsmode2    $\displaystyle\int\nolimits^0_1$
        \NC \mathnolimitsmode3    $\displaystyle\int\nolimits^0_1$
        \NC \mathnolimitsmode4    $\displaystyle\int\nolimits^0_1$
        \NC \mathnolimitsmode8000 $\displaystyle\int\nolimits^0_1$
    \NC \NR
    \TB
    \BC mode
        \NC \tttf 0
        \NC \tttf 1
        \NC \tttf 2
        \NC \tttf 3
        \NC \tttf 4
        \NC \tttf 8000
    \NC \NR
    \BC superscript
        \NC 0
        \NC font
        \NC 0
        \NC 0
        \NC +ic/2
        \NC 0
    \NC \NR
    \BC subscript
        \NC -ic
        \NC font
        \NC 0
        \NC -ic/2
        \NC -ic/2
        \NC 8000ic/1000
    \NC \NR
\stoptabulate

When the mode is set to one, the math parameters are used. This way a macro
package writer can decide what looks best. Given the current state of fonts in
\CONTEXT\ we currently use mode 1 with factor 0 for the superscript and 750 for
the subscripts. Positive values are used for both parameters but the subscript
shifts to the left. A \prm {mathnolimitsmode} larger that 15 is considered to
be a factor for the subscript correction. This feature can be handy when
experimenting.

\stopsubsection

\startsubsection[title={Some spacing control with \prm {mathsurroundmode}, \prm {mathspacingmode} and \prm {mathgluemode}}]

See \in {section} [spacing:surround] for more about inline spacing. The \prm
{mathsurroundmode} parameter just permits the glue variant to kick in and indeed
we enable it in \CONTEXT.

The \prm {mathspacingmode} parameter is for tracing: normally zero inter atom
glue is not injected but when this parameter is set to non|-|zero even zero
spacing will show up. This permits us to check the applied inter atom spacing.

The \prm {mathgluemode} bitset controls if glue can stretch and|/|or shrink. It
is used in some of the upgraded \CONTEXT\ high level math alignment command so
probably more qualifies as a feature specific for that usage.

\starttabulate[|c|l|]
\DB value \BC meaning \NC \NR
\TB
\NC 0x01  \NC obey stretch component \NC \NR
\NC 0x02  \NC obey shrink  component \NC \NR
\LL
\stoptabulate

\stopsubsection

\stopsection

\startsection[title={Experiments}]

There are a couple of experimental features. They will stay but details might
change, for instance more control over spacing. We just show some examples and
let your imagination work it out.

\startsubsection[title={Scaling spacing with \prm {Umathxscale} and  \prm {Umathyscale}}]

These two primitives scale the horizontal and vertical scaling related
parameters. They are set by style. There is no combined scaling primitive.

\startbuffer
$\Umathxscale\textstyle  800 a + b + x + d + e = f $\par
$\Umathxscale\textstyle 1000 a + b + x + d + e = f $\par
$\Umathxscale\textstyle 1200 a + b + x + d + e = f $\blank

$\Umathyscale\textstyle  800 \sqrt[2]{x+1}$\quad
$\Umathyscale\textstyle 1000 \sqrt[2]{x+1}$\quad
$\Umathyscale\textstyle 1200 \sqrt[2]{x+1}$\blank
\stopbuffer

\typebuffer

Normally only small deviations from 1000 make sense but here we want to show the
effect and use a 20\percent\ scaling:

\getbuffer

\startsubsection[title={Scaling with \prm {scaledmathstyle}}]

Because styles put a style switching node in the stream we have a
scaling primitive that uses such a style node to signal dynamic
scaling. Thisis still somewhat experimental.

\startbuffer
$
    {\scaledmathstyle  500 x + x}\quad
    {\scaledmathstyle 1000 x + x}\quad
    {\scaledmathstyle 1500 x + x}
$
\stopbuffer

\typebuffer

You get differently sized math but of course you then probably also need to
handle spacing differently, although for small deviations from 1000 is should
come out acceptable.

\stopsubsection

\startsubsection[title={Spreading math with \prm {maththreshold}}]

Small formulas that don't get unpacked can be made to act like glue, that is,
they become a sort of leader and permit the par builder to prevent excessive
spacing because the embedded inter atom glue can now participate in the
calculations. The \prm {maththreshold} primitive is a regular glue parameter.

\stopsubsection

\startsubsection[title={\prm {everymathatom} and \prm {lastatomclass}}]

Just for completeness we have \prm {everymathatom} as companion for \prm {everyhbox}
and friends and it is probably just as useful. The next example shows how it works:

\startbuffer
\everymathatom
  {\begingroup
   \scratchcounter\lastatomclass
   \everymathatom{}%
   \mathghost{\hbox to 0pt yoffset -1ex{\smallinfofont \setstrut\strut \the\scratchcounter\hss}}%
   \endgroup}

$ a = \mathatom class 4 {b} + \mathatom class 5 {c} $
\stopbuffer

\typebuffer

We get a formula with open- and close atom spacing applied to~$b$ and~$c$:

{\getbuffer}

This example shows bit of all: we want the number to be invisible to the math
machinery so we ghost it. So, we need to make sure we don't get recursion due to
nested injection and expansion of \prm {everymathatom} and of course we need to
store the number. The \prm {lastatomclass} state variable is only meaningful
inside an explicit atom wrapper like \prm {mathatom} and \prm {mathatom}.

\stopsubsection

\startsubsection[title={\prm {postinlinepenalty} and \prm {preinlinepenalty}}]

In horizontal lists math is embedded in a begin and end math node. These nodes
also carry information about the surrounding space, and the in \LUAMETATEX\
optional glue. We also store a penalty so that we can let that play a role in the
decisions to be made; these two internal integer registers control this. Just
like the mentioned spacing they are not visible as nodes in the list.

\stopsubsection

\startsubsection[title={\prm {mathforwardpenalties} and \prm {mathbackwardpenalties}}]

These penalties are experimental and deltas added to the regular penalties
between atoms. Here is an example, as with other primitives that take more
arguments the first number indicates how much follows.

\startbuffer
$ a + b + c + d + e + f + g + h = x $\par
\mathforwardpenalties  3 300 200 100
\mathbackwardpenalties 3 250 150  50
$ a + b + c + d + e + f + g + h = x $\par
\stopbuffer

\typebuffer

You'll notice that we apply more severe penalties at the edges:

{\showmakeup[penalty]\multiply\glyphscale\plustwo \getbuffer}

\stopsubsection

\startsubsection[title={\prm {Umathdiscretionary} and \prm {hmcode}}]

\topicindex {math+discretionaries}

The usual \prm {discretionary} command is supported in math mode but it has the
disadvantage that one needs to make sure that the content triplet does the math
right (especially the style). This command takes an optional class specification.

\starttyping
\Umathdiscretionary [class n] {+} {+} {+}
\stoptyping

It uses the same logic as \prm {mathchoice} but in this case we handle three
snippets in the current style.

A fully automatic mechanism kicks in when a character has a \prm {hmcode} set:

\starttabulate[|c|l|p|]
\DB bit \BC meaning \BC explanation \NC \NR
\TB
\NC 1   \NC normal  \NC a discretionary is created with the same components \NC \NR
\NC 2   \NC italic  \NC following italic correction is kept with the component \NC \NR
\LL
\stoptabulate

So we can say:

\starttyping
\hmcode `+ 3
\stoptyping

When the \type {italic} bit is set italic correction is kept at a linebreak.
\stopsubsection

\stopsection

\stopchapter

\stopcomponent
