% language=us runpath=texruns:manuals/luametatex

% lua.newtable

\environment luametatex-style

\startcomponent luametatex-pdf

\startchapter[reference=pdf,title={The \PDF\ related libraries}]

\startsection[title={The \type {pdfe} library}][library=pdfe]

\startsubsection[title={Introduction}]

\topicindex{\PDF+objects}

\topicindex{\PDF+analyze}
\topicindex{\PDF+\type{pdfe}}

The \type {pdfe} library replaces the \type {epdf} library and provides an
interface to \PDF\ files. It uses the same code as is used for \PDF\ image
inclusion. The \type {pplib} library by Paweł Jackowski replaces the \type
{poppler} (derived from \type {xpdf}) library.

A \PDF\ file is basically a tree of objects and one descends into the tree via
dictionaries (key/value) and arrays (index/value). There are a few topmost
dictionaries that start at root that are accessed more directly.

Although everything in \PDF\ is basically an object we only wrap a few in so
called userdata \LUA\ objects.

\starttabulate[|l|l|]
\DB type          \BC mapping \NC \NR
\TB
\BC \PDF          \BC \LUA \NC \NR
\NC null          \NC nil \NC \NR
\NC boolean       \NC boolean \NC \NR
\NC integer       \NC integer \NC \NR
\NC float         \NC number \NC \NR
\NC name          \NC string \NC \NR
\NC string        \NC string \NC \NR
\NC array         \NC array userdatum \NC \NR
\NC dictionary    \NC dictionary userdatum \NC \NR
\NC stream        \NC stream userdatum (with related dictionary) \NC \NR
\NC reference     \NC reference userdatum \NC \NR
\LL
\stoptabulate

The regular getters return these \LUA\ data types but one can also get more
detailed information.

\stopsubsection

\startsubsection[title={\type {open}, \type {openfile}, \type {new}, \type {getstatus}, \type {close}, \type {unencrypt}}]

\libindex {open}
\libindex {new}
\libindex {new}
\libindex {getstatus}
\libindex {close}
\libindex {unencrypt}

A document is loaded from a file (by name or handle) or string:

\starttyping
<pdfe document> = pdfe.open(filename)
<pdfe document> = pdfe.openfile(filehandle)
<pdfe document> = pdfe.new(somestring,somelength)
\stoptyping

Such a document is closed with:

\starttyping
pdfe.close(<pdfe document>)
\stoptyping

You can check if a document opened well by:

\starttyping
pdfe.getstatus(<pdfe document>)
\stoptyping

The returned codes are:

\starttabulate[|c|l|]
\DB value       \BC explanation \NC \NR
\TB
\NC \type {-2}  \NC the document failed to open \NC \NR
\NC \type {-1}  \NC the document is (still) protected \NC \NR
\NC \type {0}   \NC the document is not encrypted \NC \NR
\NC \type {2}   \NC the document has been unencrypted \NC \NR
\LL
\stoptabulate

An encrypted document can be unencrypted by the next command where instead of
either password you can give \type {nil}:

\starttyping
pdfe.unencrypt(<pdfe document>,userpassword,ownerpassword)
\stoptyping

\stopsubsection

\startsubsection[title={\type {getsize}, \type {getversion}, \type {getnofobjects}, \type {getnofpages}}]

\libindex {getsize}
\libindex {getversion}
\libindex {getnofobjects}
\libindex {getnofpages}

A successfully opened document can provide some information:

\starttyping
bytes = getsize(<pdfe document>)
major, minor = getversion(<pdfe document>)
n = getnofobjects(<pdfe document>)
n = getnofpages(<pdfe document>)
bytes, waste = getnofpages(<pdfe document>)
\stoptyping

\stopsubsection

\startsubsection[title={\type {get[catalog|trailer|info]}}]

\libindex {getcatalog}
\libindex {gettrailer}
\libindex {getinfo}

For accessing the document structure you start with the so called catalog, a
dictionary:

\starttyping
<pdfe dictionary> = pdfe.getcatalog(<pdfe document>)
\stoptyping

The other two root dictionaries are accessed with:

\starttyping
<pdfe dictionary> = pdfe.gettrailer(<pdfe document>)
<pdfe dictionary> = pdfe.getinfo(<pdfe document>)
\stoptyping

\stopsubsection

\startsubsection[title={\type {getpage}, \type {getbox}}]

\libindex {getpage}
\libindex {getbox}

A specific page can conveniently be reached with the next command, which
returns a dictionary.

\starttyping
<pdfe dictionary> = pdfe.getpage(<pdfe document>,pagenumber)
\stoptyping

Another convenience command gives you the (bounding) box of a (normally page)
which can be inherited from the document itself. An example of a valid box name
is \type {MediaBox}.

\starttyping
pages = pdfe.getbox(<pdfe dictionary>,boxname)
\stoptyping

\stopsubsection

\startsubsection[title={\type {get[string|integer|number|boolean|name]}}]

\libindex {getstring}
\libindex {getinteger}
\libindex {getnumber}
\libindex {getboolean}
\libindex {getname}

Common values in dictionaries and arrays are strings, integers, floats, booleans
and names (which are also strings) and these are also normal \LUA\ objects:

\starttyping
s = getstring (<pdfe array|dictionary>,index|key)
i = getinteger(<pdfe array|dictionary>,index|key)
n = getnumber (<pdfe array|dictionary>,index|key)
b = getboolean(<pdfe array|dictionary>,index|key)
n = getname   (<pdfe array|dictionary>,index|key)
\stoptyping

The \type {getstring} function has two extra variants:

\starttyping
s, h = getstring (<pdfe array|dictionary>,index|key,false)
s    = getstring (<pdfe array|dictionary>,index|key,true)
\stoptyping

The first call returns the original string plus a boolean indicating if the
string is hex encoded. The second call returns the unencoded string.

\stopsubsection

\startsubsection[title={\type {get[dictionary|array|stream]}}]

\libindex {getdictionary} \libindex {getfromdictionary}
\libindex {getarray}      \libindex {getfromarray}
\libindex {getstream}     \libindex {getfromstream}

Normally you will use an index in an array and key in a dictionary but dictionaries
also accept an index. The size of an array or dictionary is available with the
usual \type {#} operator.

\starttyping
<pdfe dictionary>   = getdictionary(<pdfe array|dictionary>,index|key)
<pdfe array>        = getarray     (<pdfe array|dictionary>,index|key)
<pdfe stream>,
<pdfe dictionary>   = getstream    (<pdfe array|dictionary>,index|key)
\stoptyping

These commands return dictionaries, arrays and streams, which are dictionaries
with a blob of data attached.

Before we come to an alternative access mode, we mention that the objects provide
access in a different way too, for instance this is valid:

\starttyping
print(pdfe.open("foo.pdf").Catalog.Type)
\stoptyping

At the topmost level there are \type {Catalog}, \type {Info}, \type {Trailer}
and \type {Pages}, so this is also okay:

\starttyping
print(pdfe.open("foo.pdf").Pages[1])
\stoptyping

\stopsubsection

\startsubsection[title={\type {[open|close|readfrom|whole|]stream}}]

\libindex {openstream}
\libindex {closestream}
\libindex {readfromstream}
\libindex {readfromwholestream}

Streams are sort of special. When your index or key hits a stream you get back a
stream object and dictionary object. The dictionary you can access in the usual
way and for the stream there are the following methods:

\starttyping
okay   = openstream(<pdfe stream>,[decode])
         closestream(<pdfe stream>)
str, n = readfromstream(<pdfe stream>)
str, n = readwholestream(<pdfe stream>,[decode])
\stoptyping

You either read in chunks, or you ask for the whole. When reading in chunks, you
need to open and close the stream yourself. The \type {n} value indicates the
length read. The \type {decode} parameter controls if the stream data gets
uncompressed.

As with dictionaries, you can access fields in a stream dictionary in the usual
\LUA\ way too. You get the content when you \quote {call} the stream. You can
pass a boolean that indicates if the stream has to be decompressed.

% pdfe.objectcodes      = objectcodes
% pdfe.stringcodes      = stringcodes
% pdfe.encryptioncodes  = encryptioncodes

\stopsubsection

\startsubsection[title={\type {getfrom[dictionary|array]}}]

\libindex {getfromdictionary}
\libindex {getfromarray}

In addition to the interface described before, there is also a bit lower level
interface available.

\starttyping
key, type, value, detail = getfromdictionary(<pdfe dictionary>,index)
type, value, detail = getfromarray(<pdfe array>,index)
\stoptyping

\starttabulate[|c|l|l|l|]
\DB type       \BC meaning    \BC value            \BC detail \NC \NR
\NC \type {0}  \NC none       \NC nil              \NC \NC \NR
\NC \type {1}  \NC null       \NC nil              \NC \NC \NR
\NC \type {2}  \NC boolean    \NC boolean          \NC \NC \NR
\NC \type {3}  \NC integer    \NC integer          \NC \NC \NR
\NC \type {4}  \NC number     \NC float            \NC \NC \NR
\NC \type {5}  \NC name       \NC string           \NC \NC \NR
\NC \type {6}  \NC string     \NC string           \NC hex \NC \NR
\NC \type {7}  \NC array      \NC arrayobject      \NC size \NC \NR
\NC \type {8}  \NC dictionary \NC dictionaryobject \NC size \NC \NR
\NC \type {9}  \NC stream     \NC streamobject     \NC dictionary size \NC \NR
\NC \type {10} \NC reference  \NC integer          \NC \NC \NR
\LL
\stoptabulate

A \type {hex} string is (in the \PDF\ file) surrounded by \type {<>} while plain
strings are bounded by \type {<>}.

\stopsubsection

\startsubsection[title={\type {[dictionary|array]totable}}]

\libindex {dictionarytotable}
\libindex {arraytotable}

All entries in a dictionary or table can be fetched with the following commands
where the return values are a hashed or indexed table.

\starttyping
hash = dictionarytotable(<pdfe dictionary>)
list = arraytotable(<pdfe array>)
\stoptyping

You can get a list of pages with:

\starttyping
{ { <pdfe dictionary>, size, objnum }, ... } = pagestotable(<pdfe document>)
\stoptyping

\stopsubsection

\startsubsection[title={\type {getfromreference}}]

\libindex {getfromreference}

Because you can have unresolved references, a reference object can be resolved
with:

\starttyping
type, <pdfe dictionary|array|stream>, detail = getfromreference(<pdfe reference>)
\stoptyping

So, as second value you get back a new \type {pdfe} userdata object that you can
query.

\stopsubsection

\stopsection

\startsection[title={Memory streams}][library=pdfe]

\topicindex{\PDF+memory streams}

\libindex {new}

The \type {pdfe.new} function takes three arguments:

\starttabulate
\DB value           \BC explanation      \NC \NR
\TB
\NC \type {stream}  \NC this is a (in low level \LUA\ speak) light userdata
                        object, i.e.\ a pointer to a sequence of bytes \NC \NR
\NC \type {length}  \NC this is the length of the stream in bytes (the stream can
                        have embedded zeros) \NC \NR
\NC \type {name}    \NC optional, this is a unique identifier that is used for
                        hashing the stream \NC \NR
\LL
\stoptabulate

The third argument is optional. When it is not given the function will return a
\type {pdfe} document object as with a regular file, otherwise it will return a
filename that can be used elsewhere (e.g.\ in the image library) to reference the
stream as pseudo file.

Instead of a light userdata stream (which is actually fragile but handy when you
come from a library) you can also pass a \LUA\ string, in which case the given
length is (at most) the string length.

The function returns a \type {pdfe} object and a string. The string can be used in
the \type {img} library instead of a filename. You need to prevent garbage
collection of the object when you use it as image (for instance by storing it
somewhere).

Both the memory stream and it's use in the image library is experimental and can
change. In case you wonder where this can be used: when you use the swiglib
library for \type {graphicmagick}, it can return such a userdata object. This
permits conversion in memory and passing the result directly to the backend. This
might save some runtime in one|-|pass workflows. This feature is currently not
meant for production and we might come up with a better implementation.

\stopsection

\startsection[title={The \type {pdfscanner} library}][library=pdfscanner]

This library is not available in \LUAMETATEX.

\stopsection

% \startsection[title={The \type {pdfscanner} library}][library=pdfscanner]
%
% \topicindex{\PDF+scanner}
%
% \libindex {scan}
%
% The \type {pdfscanner} library allows interpretation of \PDF\ content streams and
% \type {/ToUnicode} (cmap) streams. You can get those streams from the \type
% {pdfe} library, as explained in an earlier section. There is only a single
% top|-|level function in this library:
%
% \startfunctioncall
% pdfscanner.scan (<pdfe stream>, <table> operatortable, <table> info)
% pdfscanner.scan (<pdfe array>, <table> operatortable, <table> info)
% pdfscanner.scan (<string>, <table> operatortable, <table> info)
% \stopfunctioncall
%
% The first argument should be a \LUA\ string or a stream or array onject coming
% from the \type {pdfe} library. The second argument, \type {operatortable}, should
% be a \LUA\ table where the keys are \PDF\ operator name strings and the values
% are \LUA\ functions (defined by you) that are used to process those operators.
% The functions are called whenever the scanner finds one of these \PDF\ operators
% in the content stream(s). The functions are called with two arguments: the \type
% {scanner} object itself, and the \type {info} table that was passed are the third
% argument to \type {pdfscanner.scan}.
%
% Internally, \type {pdfscanner.scan} loops over the \PDF\ operators in the
% stream(s), collecting operands on an internal stack until it finds a \PDF\
% operator. If that \PDF\ operator's name exists in \type {operatortable}, then the
% associated function is executed. After the function has run (or when there is no
% function to execute) the internal operand stack is cleared in preparation for the
% next operator, and processing continues.
%
% The \type {scanner} argument to the processing functions is needed because it
% offers various methods to get the actual operands from the internal operand
% stack.
%
% A simple example of processing a \PDF's document stream could look like this:
%
% \starttyping
% local operatortable = { }
%
% operatortable.Do = function(scanner,info)
%     local resources = info.resources
%     if resources then
%         local val     = scanner:pop()
%         local name    = val[2]
%         local xobject = resources.XObject
%         print(info.space .. "Uses XObject " .. name)
%         local resources = xobject.Resources
%         if resources then
%             local newinfo =  {
%                 space     = info.space .. "  ",
%                 resources = resources,
%             }
%             pdfscanner.scan(entry, operatortable, newinfo)
%         end
%     end
% end
%
% local function Analyze(filename)
%     local doc = pdfe.open(filename)
%     if doc then
%         local pages = doc.Pages
%         for i=1,#pages do
%             local page = pages[i]
%             local info = {
%               space     = "  " ,
%               resources = page.Resources,
%             }
%             print("Page " .. i)
%          -- pdfscanner.scan(page.Contents,operatortable,info)
%             pdfscanner.scan(page.Contents(),operatortable,info)
%         end
%     end
% end
%
% Analyze("foo.pdf")
% \stoptyping
%
% This example iterates over all the actual content in the \PDF, and prints out the
% found \type {XObject} names. While the code demonstrates quite some of the \type
% {pdfe} functions, let's focus on the type \type {pdfscanner} specific code
% instead.
%
% From the bottom up, the following line runs the scanner with the \PDF\ page's
% top|-|level content given in the first argument.
%
% The third argument, \type {info}, contains two entries: \type {space} is used to
% indent the printed output, and \type {resources} is needed so that embedded \type
% {XForms} can find their own content.
%
% The second argument, \type {operatortable} defines a processing function for a
% single \PDF\ operator, \type {Do}.
%
% The function \type {Do} prints the name of the current \type {XObject}, and then
% starts a new scanner for that object's content stream, under the condition that
% the \type {XObject} is in fact a \type {/Form}. That nested scanner is called
% with new \type {info} argument with an updated \type {space} value so that the
% indentation of the output nicely nests, and with a new \type {resources} field
% to help the next iteration down to properly process any other, embedded \type
% {XObject}s.
%
% Of course, this is not a very useful example in practice, but for the purpose of
% demonstrating \type {pdfscanner}, it is just long enough. It makes use of only
% one \type {scanner} method: \type {scanner:pop()}. That function pops the top
% operand of the internal stack, and returns a \LUA\ table where the object at index
% one is a string representing the type of the operand, and object two is its
% value.
%
% The list of possible operand types and associated \LUA\ value types is:
%
% \starttabulate[|l|l|]
% \DB types           \BC type      \NC \NR
% \TB
% \NC \type{integer}  \NC <number>  \NC \NR
% \NC \type{real}     \NC <number>  \NC \NR
% \NC \type{boolean}  \NC <boolean> \NC \NR
% \NC \type{name}     \NC <string>  \NC \NR
% \NC \type{operator} \NC <string>  \NC \NR
% \NC \type{string}   \NC <string>  \NC \NR
% \NC \type{array}    \NC <table>   \NC \NR
% \NC \type{dict}     \NC <table>   \NC \NR
% \LL
% \stoptabulate
%
% In case of \type {integer} or \type {real}, the value is always a \LUA\ (floating
% point) number. In case of \type {name}, the leading slash is always stripped.
%
% In case of \type {string}, please bear in mind that \PDF\ actually supports
% different types of strings (with different encodings) in different parts of the
% \PDF\ document, so you may need to reencode some of the results; \type {pdfscanner}
% always outputs the byte stream without reencoding anything. \type {pdfscanner}
% does not differentiate between literal strings and hexadecimal strings (the
% hexadecimal values are decoded), and it treats the stream data for inline images
% as a string that is the single operand for \type {EI}.
%
% In case of \type {array}, the table content is a list of \type {pop} return
% values and in case of \type {dict}, the table keys are \PDF\ name strings and the
% values are \type {pop} return values.
%
% \libindex{pop}
% \libindex{popnumber}
% \libindex{popname}
% \libindex{popstring}
% \libindex{poparray}
% \libindex{popdictionary}
% \libindex{popboolean}
% \libindex{done}
%
% There are a few more methods defined that you can ask \type {scanner}:
%
% \starttabulate[|l|p|]
% \DB method               \BC explanation \NC \NR
% \TB
% \NC \type{pop}           \NC see above \NC \NR
% \NC \type{popnumber}     \NC return only the value of a \type {real} or \type {integer} \NC \NR
% \NC \type{popname}       \NC return only the value of a \type {name} \NC \NR
% \NC \type{popstring}     \NC return only the value of a \type {string} \NC \NR
% \NC \type{poparray}      \NC return only the value of a \type {array} \NC \NR
% \NC \type{popdictionary} \NC return only the value of a \type {dict} \NC \NR
% \NC \type{popboolean}    \NC return only the value of a \type {boolean} \NC \NR
% \NC \type{done}          \NC abort further processing of this \type {scan()} call \NC \NR
% \LL
% \stoptabulate
%
% The \type {pop*} are convenience functions, and come in handy when you know the
% type of the operands beforehand (which you usually do, in \PDF). For example, the
% \type {Do} function could have used \type {local name = scanner:popname()}
% instead, because the single operand to the \type {Do} operator is always a \PDF\
% name object.
%
% The \type {done} function allows you to abort processing of a stream once you
% have learned everything you want to learn. This comes in handy while parsing
% \type {/ToUnicode}, because there usually is trailing garbage that you are not
% interested in. Without \type {done}, processing only ends at the end of the
% stream, possibly wasting \CPU\ cycles.
%
% {\em We keep the older names \type {popNumber}, \type {popName}, \type
% {popString}, \type {popArray}, \type {popDict} and \type {popBool} around.}
%
% \stopsection

\stopchapter

\stopcomponent
