% language=uk

\environment luametatex-style

\startcomponent luametatex-primitives

\startchapter[reference=primitives,title={Primitives aka commands}]

\startsection[title=Introduction]

The starting point of \LUATEX\ is \PDFTEX, which itself contains regular \TEX\
and \ETEX. Because directional support was needed we also took some code from
\ALEPH\ (\OMEGA). In a later stage the backend specific commands were isolated in
its own namespace which resulted in a cleaner code base where the backend code no
longer was interwoven with the normal frontend primitives. We also promoted some
generic constructs (like box resources and directions) to core functionality.

Some of the \PDFTEX\ support primitives have been around from the start but when
\LUA\ integration became better and when a token scanner library was added, not
all of those made sense as primitives. In previous chapters we already mentioned
what is gone from the core. Deep down some more has changed but not all is
reflected at the primitive level. Because there is still a considerable amount of
new primitives, a summary is given below.

\stopsection

\startsection[title=Languages]

\starttabulate[|||p|]
\NC \type {automatichyphenmode}       \NC integer       \NC \NC \NR
\NC \type {automatichyphenpenalty}    \NC integer       \NC \NC \NR
\NC \type {hyphenpenaltymode}         \NC integer       \NC \NC \NR
\NC \type {compoundhyphenmode}        \NC integer       \NC \NC \NR
\NC \type {exceptionpenalty}          \NC integer       \NC \NC \NR
\NC \type {explicithyphenpenalty}     \NC integer       \NC \NC \NR
\NC \type {hyphenationbounds}         \NC integer       \NC \NC \NR
\NC \type {hjcode}                    \NC charactercode \NC \NC \NR
\NC \type {hyphenationmin}            \NC charactercode \NC \NC \NR
\NC \type {postexhyphenchar}          \NC charactercode \NC \NC \NR
\NC \type {posthyphenchar}            \NC charactercode \NC \NC \NR
\NC \type {preexhyphenchar}           \NC charactercode \NC \NC \NR
\NC \type {prehyphenchar}             \NC charactercode \NC \NC \NR
\stoptabulate

\stopsection

\startsection[title=Fonts]

\starttabulate[|||p|]
\NC \type {tracingfonts}              \NC integer            \NC \NC \NR
\NC \type {suppressfontnotfounderror} \NC integer            \NC \NC \NR
\NC \type {setfontid}                 \NC integer            \NC \NC \NR
\NC \type {fontid}                    \NC font               \NC \NC \NR
\NC \type {efcode}                    \NC font charactercode \NC \NC \NR
\NC \type {lpcode}                    \NC font charactercode \NC \NC \NR
\NC \type {rpcode}                    \NC font charactercode \NC \NC \NR
\stoptabulate

\stopsection

\startsection[title=Math]

\starttabulate[|||p|]
\NC \type {matholdmode}               \NC integer          \NC \NC \NR
\NC \type {mathstyle}                 \NC integer          \NC \NC \NR
\NC \type {matheqnogapstep}           \NC integer          \NC \NC \NR
\NC \type {Uskewed}                   \NC                  \NC \NC \NR
\NC \type {Uskewedwithdelims}         \NC                  \NC \NC \NR
\NC \type {Ustartdisplaymath}         \NC                  \NC \NC \NR
\NC \type {Ustartmath}                \NC                  \NC \NC \NR
\NC \type {Ustopdisplaymath}          \NC                  \NC \NC \NR
\NC \type {Ustopmath}                 \NC                  \NC \NC \NR
\NC \type {crampeddisplaystyle}       \NC                  \NC \NC \NR
\NC \type {crampedtextstyle}          \NC                  \NC \NC \NR
\NC \type {crampedscriptstyle}        \NC                  \NC \NC \NR
\NC \type {crampedscriptscriptstyle}  \NC                  \NC \NC \NR
\NC \type {Umathchardef}              \NC                  \NC \NC \NR
\NC \type {Umathcharnumdef}           \NC                  \NC \NC \NR
\NC \type {mathdisplayskipmode}       \NC integer          \NC \NC \NR
\NC \type {mathscriptsmode}           \NC integer          \NC \NC \NR
\NC \type {mathnolimitsmode}          \NC integer          \NC \NC \NR
\NC \type {mathitalicsmode}           \NC integer          \NC \NC \NR
\NC \type {mathrulesmode}             \NC integer          \NC \NC \NR
\NC \type {mathrulesfam}              \NC integer          \NC \NC \NR
\NC \type {mathdelimitersmode}        \NC integer          \NC \NC \NR
\NC \type {mathflattenmode}           \NC integer          \NC \NC \NR
\NC \type {mathpenaltiesmode}         \NC integer          \NC \NC \NR
\NC \type {mathrulethicknessmode}     \NC integer          \NC \NC \NR
\NC \type {mathscriptboxmode}         \NC integer          \NC \NC \NR
\NC \type {mathscriptcharmode}        \NC integer          \NC \NC \NR
\NC \type {mathsurroundmode}          \NC integer          \NC \NC \NR
\NC \type {nokerns}                   \NC integer          \NC \NC \NR
\NC \type {noligs}                    \NC integer          \NC \NC \NR
\NC \type {prebinoppenalty}           \NC integer          \NC \NC \NR
\NC \type {predisplaygapfactor}       \NC integer          \NC \NC \NR
\NC \type {prerelpenalty}             \NC integer          \NC \NC \NR
\NC \type {Usuperscript}              \NC command          \NC \NC \NR
\NC \type {Usubscript}                \NC command          \NC \NC \NR
\NC \type {Unosuperscript}            \NC command          \NC \NC \NR
\NC \type {Unosubscript}              \NC command          \NC \NC \NR
\NC \type {Umathcode}                 \NC                  \NC \NC \NR
\NC \type {Umathcodenum}              \NC                  \NC \NC \NR
\NC \type {Udelcode}                  \NC                  \NC \NC \NR
\NC \type {Udelcodenum}               \NC                  \NC \NC \NR
\NC \type {Umathaxis}                 \NC family dimension \NC \NC \NR
\NC \type {Umathbinbinspacing}        \NC family dimension \NC \NC \NR
\NC \type {Umathbinclosespacing}      \NC family dimension \NC \NC \NR
\NC \type {Umathbininnerspacing}      \NC family dimension \NC \NC \NR
\NC \type {Umathbinopenspacing}       \NC family dimension \NC \NC \NR
\NC \type {Umathbinopspacing}         \NC family dimension \NC \NC \NR
\NC \type {Umathbinordspacing}        \NC family dimension \NC \NC \NR
\NC \type {Umathbinpunctspacing}      \NC family dimension \NC \NC \NR
\NC \type {Umathbinrelspacing}        \NC family dimension \NC \NC \NR
\NC \type {Umathclosebinspacing}      \NC family dimension \NC \NC \NR
\NC \type {Umathcloseclosespacing}    \NC family dimension \NC \NC \NR
\NC \type {Umathcloseinnerspacing}    \NC family dimension \NC \NC \NR
\NC \type {Umathcloseopenspacing}     \NC family dimension \NC \NC \NR
\NC \type {Umathcloseopspacing}       \NC family dimension \NC \NC \NR
\NC \type {Umathcloseordspacing}      \NC family dimension \NC \NC \NR
\NC \type {Umathclosepunctspacing}    \NC family dimension \NC \NC \NR
\NC \type {Umathcloserelspacing}      \NC family dimension \NC \NC \NR
\NC \type {Umathconnectoroverlapmin}  \NC family dimension \NC \NC \NR
\NC \type {Umathfractiondelsize}      \NC family dimension \NC \NC \NR
\NC \type {Umathfractiondenomdown}    \NC family dimension \NC \NC \NR
\NC \type {Umathfractiondenomvgap}    \NC family dimension \NC \NC \NR
\NC \type {Umathfractionnumup}        \NC family dimension \NC \NC \NR
\NC \type {Umathfractionnumvgap}      \NC family dimension \NC \NC \NR
\NC \type {Umathfractionrule}         \NC family dimension \NC \NC \NR
\NC \type {Umathinnerbinspacing}      \NC family dimension \NC \NC \NR
\NC \type {Umathinnerclosespacing}    \NC family dimension \NC \NC \NR
\NC \type {Umathinnerinnerspacing}    \NC family dimension \NC \NC \NR
\NC \type {Umathinneropenspacing}     \NC family dimension \NC \NC \NR
\NC \type {Umathinneropspacing}       \NC family dimension \NC \NC \NR
\NC \type {Umathinnerordspacing}      \NC family dimension \NC \NC \NR
\NC \type {Umathinnerpunctspacing}    \NC family dimension \NC \NC \NR
\NC \type {Umathinnerrelspacing}      \NC family dimension \NC \NC \NR
\NC \type {Umathlimitabovebgap}       \NC family dimension \NC \NC \NR
\NC \type {Umathlimitabovekern}       \NC family dimension \NC \NC \NR
\NC \type {Umathlimitabovevgap}       \NC family dimension \NC \NC \NR
\NC \type {Umathlimitbelowbgap}       \NC family dimension \NC \NC \NR
\NC \type {Umathlimitbelowkern}       \NC family dimension \NC \NC \NR
\NC \type {Umathlimitbelowvgap}       \NC family dimension \NC \NC \NR
\NC \type {Umathnolimitsubfactor}     \NC family dimension \NC \NC \NR
\NC \type {Umathnolimitsupfactor}     \NC family dimension \NC \NC \NR
\NC \type {Umathopbinspacing}         \NC family dimension \NC \NC \NR
\NC \type {Umathopclosespacing}       \NC family dimension \NC \NC \NR
\NC \type {Umathopenbinspacing}       \NC family dimension \NC \NC \NR
\NC \type {Umathopenclosespacing}     \NC family dimension \NC \NC \NR
\NC \type {Umathopeninnerspacing}     \NC family dimension \NC \NC \NR
\NC \type {Umathopenopenspacing}      \NC family dimension \NC \NC \NR
\NC \type {Umathopenopspacing}        \NC family dimension \NC \NC \NR
\NC \type {Umathopenordspacing}       \NC family dimension \NC \NC \NR
\NC \type {Umathopenpunctspacing}     \NC family dimension \NC \NC \NR
\NC \type {Umathopenrelspacing}       \NC family dimension \NC \NC \NR
\NC \type {Umathoperatorsize}         \NC family dimension \NC \NC \NR
\NC \type {Umathopinnerspacing}       \NC family dimension \NC \NC \NR
\NC \type {Umathopopenspacing}        \NC family dimension \NC \NC \NR
\NC \type {Umathopopspacing}          \NC family dimension \NC \NC \NR
\NC \type {Umathopordspacing}         \NC family dimension \NC \NC \NR
\NC \type {Umathoppunctspacing}       \NC family dimension \NC \NC \NR
\NC \type {Umathoprelspacing}         \NC family dimension \NC \NC \NR
\NC \type {Umathordbinspacing}        \NC family dimension \NC \NC \NR
\NC \type {Umathordclosespacing}      \NC family dimension \NC \NC \NR
\NC \type {Umathordinnerspacing}      \NC family dimension \NC \NC \NR
\NC \type {Umathordopenspacing}       \NC family dimension \NC \NC \NR
\NC \type {Umathordopspacing}         \NC family dimension \NC \NC \NR
\NC \type {Umathordordspacing}        \NC family dimension \NC \NC \NR
\NC \type {Umathordpunctspacing}      \NC family dimension \NC \NC \NR
\NC \type {Umathordrelspacing}        \NC family dimension \NC \NC \NR
\NC \type {Umathoverbarkern}          \NC family dimension \NC \NC \NR
\NC \type {Umathoverbarrule}          \NC family dimension \NC \NC \NR
\NC \type {Umathoverbarvgap}          \NC family dimension \NC \NC \NR
\NC \type {Umathoverdelimiterbgap}    \NC family dimension \NC \NC \NR
\NC \type {Umathoverdelimitervgap}    \NC family dimension \NC \NC \NR
\NC \type {Umathpunctbinspacing}      \NC family dimension \NC \NC \NR
\NC \type {Umathpunctclosespacing}    \NC family dimension \NC \NC \NR
\NC \type {Umathpunctinnerspacing}    \NC family dimension \NC \NC \NR
\NC \type {Umathpunctopenspacing}     \NC family dimension \NC \NC \NR
\NC \type {Umathpunctopspacing}       \NC family dimension \NC \NC \NR
\NC \type {Umathpunctordspacing}      \NC family dimension \NC \NC \NR
\NC \type {Umathpunctpunctspacing}    \NC family dimension \NC \NC \NR
\NC \type {Umathpunctrelspacing}      \NC family dimension \NC \NC \NR
\NC \type {Umathquad}                 \NC family dimension \NC \NC \NR
\NC \type {Umathradicaldegreeafter}   \NC family dimension \NC \NC \NR
\NC \type {Umathradicaldegreebefore}  \NC family dimension \NC \NC \NR
\NC \type {Umathradicaldegreeraise}   \NC family dimension \NC \NC \NR
\NC \type {Umathradicalkern}          \NC family dimension \NC \NC \NR
\NC \type {Umathradicalrule}          \NC family dimension \NC \NC \NR
\NC \type {Umathradicalvgap}          \NC family dimension \NC \NC \NR
\NC \type {Umathrelbinspacing}        \NC family dimension \NC \NC \NR
\NC \type {Umathrelclosespacing}      \NC family dimension \NC \NC \NR
\NC \type {Umathrelinnerspacing}      \NC family dimension \NC \NC \NR
\NC \type {Umathrelopenspacing}       \NC family dimension \NC \NC \NR
\NC \type {Umathrelopspacing}         \NC family dimension \NC \NC \NR
\NC \type {Umathrelordspacing}        \NC family dimension \NC \NC \NR
\NC \type {Umathrelpunctspacing}      \NC family dimension \NC \NC \NR
\NC \type {Umathrelrelspacing}        \NC family dimension \NC \NC \NR
\NC \type {Umathskewedfractionhgap}   \NC family dimension \NC \NC \NR
\NC \type {Umathskewedfractionvgap}   \NC family dimension \NC \NC \NR
\NC \type {Umathspaceafterscript}     \NC family dimension \NC \NC \NR
\NC \type {Umathstackdenomdown}       \NC family dimension \NC \NC \NR
\NC \type {Umathstacknumup}           \NC family dimension \NC \NC \NR
\NC \type {Umathstackvgap}            \NC family dimension \NC \NC \NR
\NC \type {Umathsubshiftdown}         \NC family dimension \NC \NC \NR
\NC \type {Umathsubshiftdrop}         \NC family dimension \NC \NC \NR
\NC \type {Umathsubsupshiftdown}      \NC family dimension \NC \NC \NR
\NC \type {Umathsubsupvgap}           \NC family dimension \NC \NC \NR
\NC \type {Umathsubtopmax}            \NC family dimension \NC \NC \NR
\NC \type {Umathsupbottommin}         \NC family dimension \NC \NC \NR
\NC \type {Umathsupshiftdrop}         \NC family dimension \NC \NC \NR
\NC \type {Umathsupshiftup}           \NC family dimension \NC \NC \NR
\NC \type {Umathsupsubbottommax}      \NC family dimension \NC \NC \NR
\NC \type {Umathunderbarkern}         \NC family dimension \NC \NC \NR
\NC \type {Umathunderbarrule}         \NC family dimension \NC \NC \NR
\NC \type {Umathunderbarvgap}         \NC family dimension \NC \NC \NR
\NC \type {Umathunderdelimiterbgap}   \NC family dimension \NC \NC \NR
\NC \type {Umathunderdelimitervgap}   \NC family dimension \NC \NC \NR
\NC \type {Udelimiter}                \NC                  \NC \NC \NR
\NC \type {Umathaccent}               \NC                  \NC \NC \NR
\NC \type {Umathchar}                 \NC                  \NC \NC \NR
\NC \type {Umathcharnum}              \NC                  \NC \NC \NR
\NC \type {Ustack}                    \NC                  \NC \NC \NR
\NC \type {Uradical}                  \NC                  \NC \NC \NR
\NC \type {Uroot}                     \NC                  \NC \NC \NR
\NC \type {Uunderdelimiter}           \NC                  \NC \NC \NR
\NC \type {Uoverdelimiter}            \NC                  \NC \NC \NR
\NC \type {Udelimiterunder}           \NC                  \NC \NC \NR
\NC \type {Udelimiterover}            \NC                  \NC \NC \NR
\NC \type {Uhextensible}              \NC                  \NC \NC \NR
\NC \type {Uchar}                     \NC                  \NC \NC \NR
\NC \type {Umathcharclass}            \NC                  \NC \NC \NR
\NC \type {Umathcharfam}              \NC                  \NC \NC \NR
\NC \type {Umathcharslot}             \NC                  \NC \NC \NR
\NC \type {Uleft}                     \NC                  \NC \NC \NR
\NC \type {Umiddle}                   \NC                  \NC \NC \NR
\NC \type {Uright}                    \NC                  \NC \NC \NR
\NC \type {Uvextensible}              \NC                  \NC \NC \NR
\stoptabulate

\stopsection

\startsection[title=Boxes and directions]

\starttabulate[|||p|]
\NC \type {pardirection}              \NC direction       \NC \NC \NR
\NC \type {textdirection}             \NC direction       \NC \NC \NR
\NC \type {mathdirection}             \NC direction       \NC \NC \NR
\NC \type {linedirection}             \NC direction       \NC \NC \NR
\NC \type {breakafterdirmode}         \NC integer         \NC \NC \NR
\NC \type {shapemode}                 \NC integer         \NC \NC \NR
\NC \type {fixupboxesmode}            \NC integer         \NC \NC \NR
\NC \type {boxdirection}              \NC box direction   \NC \NC \NR
\NC \type {boxorientation}            \NC box orientation \NC rotation over 90, 180, 270 degrees \NC \NR
\NC \type {boxxoffset}                \NC box xoffset     \NC leaves dimensions untounched \NC \NR
\NC \type {boxyoffset}                \NC box yoffset     \NC leaves dimensions untounched \NC \NR
\NC \type {boxxmove}                  \NC box xmove       \NC offsets that reflect on dimensions \NC \NR
\NC \type {boxymove}                  \NC box ymove       \NC offsets that reflect on dimensions \NC \NR
\NC \type {boxtotal}                  \NC box ht+dp       \NC height plus depth (and when assigned halfs) \NC \NR
\NC \type {boxattr}                   \NC box attr value  \NC (sets) a specific attribute to a value \NC \NR
\stoptabulate

\stopsection

\startsection[title=Scanning]

\starttabulate[|||p|]
\NC \type {aftergrouped}              \NC text                    \NC like aftergroup but for given list \NC \NR
\NC \type {alignmark}                 \NC                         \NC equivalent to hash token \NC \NR
\NC \type {aligntab}                  \NC                         \NC equivalent to tab token  \NC \NR
\NC \type {begincsname}               \NC command                 \NC variant of \type {\csname} that ignores undefined commands \NC \NR
\NC \type {catcodetable}              \NC integer                 \NC switch to catcode table \NC \NR
\NC \type {csstring}                  \NC command                 \NC the command without preceding escape character \NC \NR
\NC \type {endlocalcontrol}           \NC command                 \NC switches back to the main control loop \NC \NR
\NC \type {etoksapp}                  \NC tokenregister text      \NC append expanded text to given tokenregister \NC \NR
\NC \type {etokspre}                  \NC tokenregister text      \NC prepend expanded text to given tokenregister \NC \NR
\NC \type {expanded}                  \NC text                    \NC expands the given text \NC \NR
\NC \type {frozen}                    \NN prefix                  \NC \NC \NR
\NC \type {futureexpand}              \NN token command command   \NC expands second ot third token depending on first match \NC \NR
\NC \type {futureexpandis}            \NN token command command   \NC as \type {futureexpand} but also skips pars \NC \NR
\NC \type {futureexpandisap}          \NN token command command   \NC same as idem but doesn't push back skipped spaces \NC \NR
\NC \type {gtoksapp}                  \NC tokenregister text      \NC globally append text to given tokenregister \NC \NR
\NC \type {gtokspre}                  \NC tokenregister text      \NC globally prepend text to given tokenregister \NC \NR
\NC \type {ifabsdim}                  \NC dimension <=> dimension \NC test the absolute value of the given dimension \NC \NR
\NC \type {ifabsnum}                  \NC integer <=> integer     \NC test the absolute value of the given integer \NC \NR
\NC \type {ifcondition}               \NC command                 \NC assume the next token is a test (so skip as if) \NC \NR
\NC \type {ifdimen}                   \NC possibly a dimension    \NC acts like an \type {\ifcase} with 1 for valid and 2 for invalid \NC \NR
\NC \type {ifincsname}                \NC command                 \NC check if we're inside a csname expansion \NC \NR
\NC \type {ifnumval}                  \NC                         \NC \NC \NR
\NC \type {ifdimval}                  \NC                         \NC \NC \NR
\NC \type {ifchknum}                  \NC                         \NC \NC \NR
\NC \type {ifchkdim}                  \NC                         \NC \NC \NR
\NC \type {ifcmpnum}                  \NC                         \NC \NC \NR
\NC \type {ifcmpdim}                  \NC                         \NC \NC \NR
\NC \type {ifusercmd}                 \NC command                 \NC \NC \NR
\NC \type {ifprotected}               \NC command                 \NC \NC \NR
\NC \type {iffrozen}                  \NC command                 \NC \NC \NR
\NC \type {iftok}                     \NC                         \NC \NC \NR
\NC \type {ifcstok}                   \NC                         \NC \NC \NR
\NC \type {internalcodesmode}         \NC integer                 \NC \NC \NR
%NC \type {ifprimitive}               \NC command                 \NC check if the given command is a primitive \NC \NR
\NC \type {immediateassigned}         \NC command                 \NC (todo) expand the following assignment now\NC \NR
\NC \type {immediateassignment}       \NC command                 \NC (todo) expand the following assignment now\NC \NR
\NC \type {initcatcodetable}          \NC integer                 \NC initialize catcode table \NC \NR
\NC \type {lastnamedcs}               \NC command                 \NC last found command of \type {\ifcsname} construction \NC \NR
\NC \type {nospaces}                  \NC integer                 \NC don't inject spaces \NC \NR
%NC \type {primitive}                 \NC command                 \NC expands the next primitive equivalent \NC \NR
\NC \type {orelse}                    \NC condition               \NC \NC \NR
\NC \type {pxdimen}                   \NC dimension               \NC multiplier for the \type {px} unit \NC \NR
\NC \type {savecatcodetable}          \NC integer                 \NC save catcode table \NC \NR
\NC \type {scantextokens}             \NC text                    \NC \type {\scantokens} without file side effects \NC \NR
\NC \type {suppressifcsnameerror}     \NC integer                 \NC recover from issues in csname testing \NC \NR
\NC \type {suppresslongerror}         \NC integer                 \NC make \type {\long} a nop \NC \NR
\NC \type {suppressmathparerror}      \NC integer                 \NC accepts \type {\par} and empty lines in math \NC \NR
\NC \type {suppressoutererror}        \NC integer                 \NC make \type {\outer} a nop \NC \NR
\NC \type {suppressprimitiveerror}    \NC integer                 \NC don't report an invalid \type {\primitive} \NC \NR
\NC \type {toksapp}                   \NC tokenregister text      \NC append text to given tokenregister \NC \NR
\NC \type {tokspre}                   \NC tokenregister text      \NC prepend text to given tokenregister \NC \NR
\NC \type {xtoksapp}                  \NC tokenregister text      \NC globally append expanded text to given tokenregister \NC \NR
\NC \type {xtokspre}                  \NC tokenregister text      \NC globally prepend expanded text to given tokenregister \NC \NR
% new (some end-of-the-year experiment)
\NC \type {letfrozen}                 \NC macro                   \NC sets the frozen property of a macro \NC \NR
\NC \type {letprotected}              \NC macro                   \NC sets the protected property of a macro \NC \NR
\NC \type {unletfrozen}               \NC macro                   \NC unsets the frozen property of a macro \NC \NR
\NC \type {unletprotected}            \NC macro                   \NC unsets the protected property of a macro \NC \NR
\stoptabulate

\stopsection

\startsection[title=Typesetting]

\starttabulate[|||p|]
\NC \type {protrudechars}             \NC integer       \NC \NC \NR
\NC \type {localbrokenpenalty}        \NC integer       \NC \NC \NR
\NC \type {localinterlinepenalty}     \NC integer       \NC \NC \NR
\NC \type {adjustspacing}             \NC integer       \NC \NC \NR
\NC \type {boundary}                  \NC command       \NC \NC \NR
\NC \type {noboundary}                \NC command       \NC \NC \NR
\NC \type {protrusionboundary}        \NC command       \NC \NC \NR
\NC \type {wordboundary}              \NC command       \NC \NC \NR
\NC \type {nohrule}                   \NC command       \NC \NC \NR
\NC \type {novrule}                   \NC command       \NC \NC \NR
\NC \type {insertht}                  \NC number        \NC \NC \NR
\NC \type {quitvmode}                 \NC command       \NC \NC \NR
\NC \type {leftmarginkern}            \NC dimension     \NC \NC \NR
\NC \type {rightmarginkern}           \NC dimension     \NC \NC \NR
\NC \type {localleftbox}              \NC box           \NC \NC \NR
\NC \type {localrightbox}             \NC box           \NC \NC \NR
\NC \type {gleaders}                  \NC command       \NC \NC \NR
\stoptabulate

\stopsection

\startsection[title=\LUA]

\starttabulate[|||p|]
\NC \type {luacopyinputnodes}         \NC integer \NC \NC \NR
\NC \type {luadef}                    \NC         \NC \NC \NR
\NC \type {luabytecodecall}           \NC         \NC \NC \NR
\NC \type {luafunctioncall}           \NC         \NC \NC \NR
\NC \type {latelua}                   \NC         \NC \NC \NR
\NC \type {lateluafunction}           \NC         \NC \NC \NR
\NC \type {luabytecode}               \NC         \NC \NC \NR
\NC \type {luaescapestring}           \NC         \NC \NC \NR
\NC \type {luafunction}               \NC         \NC \NC \NR
\stoptabulate

\stopsection

\startsection[title=Management]

\starttabulate[|||p|]
\NC \type {outputbox}                 \NC integer \NC \NC \NR
\NC \type {clearmarks}                \NC         \NC \NC \NR
\NC \type {attribute}                 \NC         \NC \NC \NR
\NC \type {glet}                      \NC         \NC \NC \NR
\NC \type {letcharcode}               \NC         \NC \NC \NR
\NC \type {attributedef}              \NC         \NC \NC \NR
\stoptabulate

\stopsection

\startsection[title=Miscellaneous]

\starttabulate[|||p|]
\NC \type {luatexversion}             \NC         \NC \NC \NR
\NC \type {formatname}                \NC         \NC \NC \NR
\NC \type {luatexbanner}              \NC         \NC \NC \NR
\NC \type {luatexrevision}            \NC         \NC \NC \NR
\stoptabulate

\stopsection

\stopchapter

\stopcomponent
