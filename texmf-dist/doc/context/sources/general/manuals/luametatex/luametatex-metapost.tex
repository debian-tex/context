% language=us runpath=texruns:manuals/luametatex

\environment luametatex-style

\startcomponent luametatex-metapost

\startluacode
    function document.showmpcodes(what)
        context.type("mplib."..what.."()")
        context(": ")
        local t = mplib[what]()
        for i=0,#t do
            if i > 0 then
                context(", ")
            end
            context("%i: %s",i,t[i])
        end
    end
\stopluacode

\startchapter[reference=metapost,title={The \METAPOST\ library \type {mplib}}]

\startsection[title={Introduction}][library=mplib]

The library used in \LUAMETATEX\ differs from the one used in \LUATEX. There are
for instance no backends and the binary number model is not available. There is
also no textual output. There are scanners and injectors that make it possible to
enhance the language and efficiently feed back into \METAPOST. File handling is
now completely delegated to \LUA, so there are more callbacks.

{\em Some functionality is experimental and therefore documentation is limited.
Also, details are discussed in articles.}

\stopsection

\startsection[title={Process management}][library=mplib]

\topicindex {\METAPOST}
\topicindex {\METAPOST+mplib}
\topicindex {images+mplib}
\topicindex {images+\METAPOST}

\libindex{version}

The \METAPOST\ library interface registers itself in the table \type {mplib}. It
is based on \MPLIB\ version \ctxlua {context(mplib.version())} (\LUATEX\ used
version 2+). Not all functionality is described here. Once we're out of the
experimental stage some more information will be added. Using the library boils
down to initializing an instance, executing statements and picking up assembled
figures in the form of \LUA\ user data objects (and from there on \LUA\ variables
like tables).


\startsubsection[title={\type {new}}]

\libindex{new}

To create a new \METAPOST\ instance, call

\startfunctioncall
<mpinstance> mp = mplib.new({...})
\stopfunctioncall

This creates the \type {mp} instance object. The argument is a hash table that
can have a number of different fields, as follows:

\starttabulate[|l|l|pl|pl|]
\DB name                 \BC type     \BC description                \BC default           \NC \NR
\TB
\NC \type {error_line}   \NC number   \NC error line width           \NC 79                \NC \NR
\NC \type {print_line}   \NC number   \NC line length in ps output   \NC 100               \NC \NR
\NC \type {random_seed}  \NC number   \NC the initial random seed    \NC variable          \NC \NR
\NC \type {math_mode}    \NC string   \NC the number system to use:
                                          \type {scaled},
                                          \type {double} or
                                        % \type {binary} or
                                          \type {decimal}            \NC \type {scaled}    \NC \NR
\NC \type {interaction}  \NC string   \NC the interaction mode:
                                         \type {batch},
                                         \type {nonstop},
                                         \type {scroll} or
                                         \type {errorstop}          \NC \type {errorstop}  \NC \NR
\NC \type {job_name}     \NC string   \NC a compatibility value      \NC                   \NC \NR
\NC \type {utf8_mode}    \NC boolean  \NC permit characters in the
                                          range 128 upto 255 to be
                                          part of names              \NC \type {false}     \NC \NR
\NC \type {text_mode}    \NC boolean  \NC permit characters 2 and 3
                                          as fencing string literals \NC \type {false}     \NC \NR
\NC \type {tolerance}    \NC number   \NC the value used as criterium
                                          for straight lines         \NC \type {131/65536} \NC \NR
\NC \type {extensions}   \NC boolean  \NC enable all extensions      \NC (might go)        \NC \NR
\LL
\stoptabulate

The binary mode is no longer available in the \LUAMETATEX\ version of \MPLIB. It
offers no real advantage and brings a ton of extra libraries with platform
specific properties that we can now avoid. We might introduce a high resolution
scaled variant at some point but only when it pays of performance wise.

In addition to the above we need to provide functions that helps \METAPOST\
communicate to the outside world.

\starttabulate[|l|l|pl|pl|]
\DB name                 \BC type     \BC argument(s)                    \BC result       \NC \NR
\TB
\NC \type {find_file}    \NC function \NC string, string, string         \NC string       \NC \NR
\NC                      \NC function \NC string, string, number         \NC string       \NC \NR
\NC \type {open_file}    \NC function \NC string, string, string         \NC table        \NC \NR
\NC                      \NC function \NC string, string, number         \NC table        \NC \NR
\NC \type {run_logger}   \NC function \NC number, string                 \NC              \NC \NR
\NC \type {run_script}   \NC function \NC string                         \NC whatever [, boolean] \NC \NR
\NC                      \NC function \NC number                         \NC whatever [, boolean] \NC \NR
\NC \type {make_text}    \NC function \NC string, number                 \NC string       \NC \NR
\NC \type {run_internal} \NC function \NC number, number, number, string \NC              \NC \NR
\NC \type {run_overload} \NC function \NC number, string, number         \NC boolean      \NC \NR
\NC \type {run_error}    \NC function \NC string, string, number         \NC              \NC \NR
\LL
\stoptabulate

The \type {find_file} and \type {open_file} functions should be of this form:

\starttyping
<string> found   = find_file (<string> name, <string> mode, <string> type)
<table>  actions = open_file (<string> name, <string> mode, <string> type)
\stoptyping

where the mode is \type {r} or \type {w} and the type is \type {mp}, \type
{data}, \type {terminal} or a number, The finder is supposed to return the full
path name of the found file, or \type {nil} if the file cannot be found. The
\type {open_file} is supposed to return a table with a \type {close} and \type
{read} function. This is similar to the way we do it in \TEX. The special name
\type {terminal} is used for interactive input. A numeric type indicates a
specific read or write channel.

The \type {run_logger} callback gets a target and a string. A target \type {1}
means log, a value \type {2} means and \type {3} means both.

The \type {run_script} function gets either a number or a string. The string
represents a script, the number can be used as reference to something stored. The
return value can be a boolean, number, string or table. Booleans and numbers are
injected directly, strings and concatenated tables are fed into scantokens. When
the second argument is true, the strings are also injected directly and tables
are injected as pairs, colors, paths, transforms, depending on how many elements
there are.

The \type {run_internal} function triggers when internal \METAPOST\ variables
flagged with \type {runscript} are initialized, saved or restored. The first
argument is an index, the second the action. When initialized a third and fourth
argument are passed. This is an experimental feature.

The experimental \type {run_overload} callback kicks in when a variable (or
macro) with a property other than zero is redefined. It gets a property, name and
the value of \type {overloadmode} passed and when the function returns \type
{true} redefinition is permitted.

The \type {run_error} callback gets the error message, help text and current
interaction mode passed. Normally it's best to just quit and let the user fix
the code.

When you are processing a snippet of text starting with \type {btex} or \type
{verbatimtex} and ending with \type {etex}, the \METAPOST\ \type {texscriptmode}
parameter controls how spaces and newlines get honoured. The default value is~1.
Possible values are:

\starttabulate[|l|p|]
\DB name      \BC meaning \NC \NR
\TB
\NC \type {0} \NC no newlines \NC \NR
\NC \type {1} \NC newlines in \type {verbatimtex} \NC \NR
\NC \type {2} \NC newlines in \type {verbatimtex} and \type {etex} \NC \NR
\NC \type {3} \NC no leading and trailing strip in \type {verbatimtex} \NC \NR
\NC \type {4} \NC no leading and trailing strip in \type {verbatimtex} and \type {btex} \NC \NR
\LL
\stoptabulate

That way the \LUA\ handler (assigned to \type {make_text}) can do what it likes.
An \type {etex} has to be followed by a space or \type {;} or be at the end of a
line and preceded by a space or at the beginning of a line. The \type {make_text}
function can return a string that gets fed into scantokens.

\stopsubsection

\startsubsection[title={\type {getstatistics}}]

\libindex{statistics}

You can request statistics with:

\startfunctioncall
<table> stats = mp:getstatistics()
\stopfunctioncall

This function returns the vital statistics for an \MPLIB\ instance. Some are
useful, others make more sense when debugging.

\starttabulate[|l|l|p|]
\DB field  \BC type \BC explanation \NC \NR
\TB
\NC \type {memory}     \NC number \NC bytes of node memory      \NC \NR
\NC \type {hash}       \NC number \NC size of the hash          \NC \NR
\NC \type {parameters} \NC number \NC allocated parameter stack \NC \NR
\NC \type {input}      \NC number \NC allocated input stack     \NC \NR
\NC \type {tokens}     \NC number \NC number of token nodes     \NC \NR
\NC \type {pairs}      \NC number \NC number of pair nodes      \NC \NR
\NC \type {knots}      \NC number \NC number of knot nodes      \NC \NR
\NC \type {nodes}      \NC number \NC number of value nodes     \NC \NR
\NC \type {symbols}    \NC number \NC number of symbolic nodes  \NC \NR
\NC \type {characters} \NC number \NC number of string bytes    \NC \NR
\NC \type {strings}    \NC number \NC number of strings         \NC \NR
\NC \type {internals}  \NC number \NC number of internals       \NC \NR
\LL
\stoptabulate

Note that in the new version of \MPLIB, this is informational only. The objects
are all allocated dynamically, so there is no chance of running out of space
unless the available system memory is exhausted.

\stopsubsection

\startsubsection[title={\type {execute}}]

\libindex{execute}

You can ask the \METAPOST\ interpreter to run a chunk of code by calling

\startfunctioncall
<table> rettable = execute(mp,"metapost code")
\stopfunctioncall

for various bits of \METAPOST\ language input. Be sure to check the \type
{rettable.status} (see below) because when a fatal \METAPOST\ error occurs the
\MPLIB\ instance will become unusable thereafter.

Generally speaking, it is best to keep your chunks small, but beware that all
chunks have to obey proper syntax, like each of them is a small file. For
instance, you cannot split a single statement over multiple chunks.

In contrast with the normal stand alone \type {mpost} command, there is \notabene
{no} implied \quote{input} at the start of the first chunk. When no string is
passed to the execute function, there will still be one triggered because it then
expects input from the terminal and you can emulate that channel with the
callback you provide.

\stopsubsection

\startsubsection[title={\type {finish}}]

\libindex{finish}

Once you create an instance it is likely that you will keep it open for
successive processing, if only because you want to avoid loading a format each
time. If for some reason you want to stop using an \MPLIB\ instance while
processing is not yet actually done, you can call \type {finish}.

\startfunctioncall
<table> rettable = finish(mp)
\stopfunctioncall

Eventually, used memory
will be freed and open files will be closed by the \LUA\ garbage collector, but
an explicit \type {finish} is the only way to capture the final part of the
output streams.

\stopsubsection

\startsubsection[title={\type {settolerance} and \type {gettolerance}}]

\libindex{gettolerance}
\libindex{settolerance}

These two functions relate to the bend tolerance, a value that is used when the
export determines if a path has straight lines (like a rectangle has).

\stopsubsection

\startsubsection[title={Errors}]

\libindex{showcontext}

In case of an error you can get the context where it happened with \type
{showcontext}.

\stopsubsection

\startsubsection[title={The scanner status}]

\libindex {getstatus}
\libindex {getstates}

When processing a graphic an instance is in a specific state and again we have a
getter for the (internal) values \ctxlua {document.showmpcodes ("getstates")}.
The current status can be queried with \type {getstatus}.

\stopsubsection

\startsubsection[title={The hash}]

\libindex {gethashentries}
\libindex {gethashentry}

Macro names and variable names are stored in a hash table. You can get a list
with entries with \type {gethashentries}, which takes an instance as first
argument. When the second argument is \type {true} more details will be provided.
With \type {gethashentry} you get info about the given macro or variable.

\stopsubsection

\startsubsection[title={Callbacks}]

\libindex{getcallbackstate}

Some statistics about the number of calls to the callbacks can be queried with
\type {getcallbackstate}, This function expects a valid instance.

\stopsubsection

\stopsection

\startsection[title={The end result}]

\startsubsection[title={The figure}]

\libindex {fields}
\libindex {stacking}

The return value of \type {execute} and \type {finish} is a table with a
few possible keys (only \type {status} is always guaranteed to be present).

\starttabulate[|l|l|p|]
\DB field  \BC type \BC explanation \NC \NR
\TB
\NC \type{status} \NC number \NC the return value:
                                 \type {0} = good,
                                 \type {1} = warning,
                                 \type {2} = errors,
                                 \type {3} = fatal error \NC \NR
\NC \type{fig}    \NC table  \NC an array of generated figures (if any) \NC \NR
\LL
\stoptabulate

When \type {status} equals~3, you should stop using this \MPLIB\ instance
immediately, it is no longer capable of processing input.

If it is present, each of the entries in the \type {fig} array is a userdata
representing a figure object, and each of those has a number of object methods
you can call:

You can check if a figure uses stacking with the \type {stacking} function. When
objects are fetched, memory gets freed so no information about stacking is
available then. You can get the used bend tolerance of an object with \type
{tolerance}.

\starttabulate[|l|l|p|]
\DB field  \BC type \BC explanation \NC \NR
\TB
\NC \type{boundingbox}  \NC function \NC returns the bounding box, as an array of 4
                                         values \NC \NR
\NC \type{objects}      \NC function \NC returns the actual array of graphic objects in
                                         this \type {fig} \NC \NR
\NC \type{filename}     \NC function \NC the filename this \type {fig}'s \POSTSCRIPT\
                                         output would have written to in stand alone
                                         mode \NC \NR
\NC \type{width}        \NC function \NC the \type {fontcharwd} value \NC \NR
\NC \type{height}       \NC function \NC the \type {fontcharht} value \NC \NR
\NC \type{depth}        \NC function \NC the \type {fontchardp} value \NC \NR
\NC \type{italic}       \NC function \NC the \type {fontcharit} value \NC \NR
\NC \type{charcode}     \NC function \NC the (rounded) \type {charcode} value \NC \NR
\NC \type{stacking}     \NC function \NC is there a non|-|zero stacking  \NC \NR
\LL
\stoptabulate

Note: you can call \type {fig:objects()} only once for any one \type {fig}
object! Some information, like stacking, can only be queried when the complete
figure is still present and calling up objects will free elements in the original
once they are transferred.

When the boundingbox represents a \quote {negated rectangle}, i.e.\ when the
first set of coordinates is larger than the second set, the picture is empty.

Graphical objects come in various types: \type {fill}, \type {outline}, \type
{text}, \type {start_clip}, \type {stop_clip}, \type {start_bounds}, \type
{stop_bounds}, \type {start_group} and \type {stop_group}. Each type has a
different list of accessible values.

There is a helper function (\type {mplib.fields(obj)}) to get the list of
accessible values for a particular object, but you can just as easily use the
tables given below.

All graphical objects have a field \type {type} that gives the object type as a
string value; it is not explicit mentioned in the following tables. In the
following, \type {number}s are \POSTSCRIPT\ points (base points in \TEX\ speak)
represented as a floating point number, unless stated otherwise. Field values
that are of type \type {table} are explained in the next section.

\stopsubsection

\startsubsection[title={fill}]

\starttabulate[|l|l|p|]
\DB field  \BC type \BC explanation \NC \NR
\TB
\NC \type{path}       \NC table  \NC the list of knots \NC \NR
\NC \type{htap}       \NC table  \NC the list of knots for the reversed trajectory \NC \NR
\NC \type{pen}        \NC table  \NC knots of the pen \NC \NR
\NC \type{color}      \NC table  \NC the object's color \NC \NR
\NC \type{linejoin}   \NC number \NC line join style (bare number)\NC \NR
\NC \type{miterlimit} \NC number \NC miterlimit\NC \NR
\NC \type{prescript}  \NC string \NC the prescript text \NC \NR
\NC \type{postscript} \NC string \NC the postscript text \NC \NR
\NC \type{stacking}   \NC number \NC the stacking (level) \NC \NR
\LL
\stoptabulate

The entries \type {htap} and \type {pen} are optional.

\stopsubsection

\startsubsection[title={outline}]

\starttabulate[|l|l|p|]
\DB field  \BC type \BC explanation \NC \NR
\TB
\NC \type{path}       \NC table  \NC the list of knots \NC \NR
\NC \type{pen}        \NC table  \NC knots of the pen \NC \NR
\NC \type{color}      \NC table  \NC the object's color \NC \NR
\NC \type{linejoin}   \NC number \NC line join style (bare number) \NC \NR
\NC \type{miterlimit} \NC number \NC miterlimit \NC \NR
\NC \type{linecap}    \NC number \NC line cap style (bare number) \NC \NR
\NC \type{dash}       \NC table  \NC representation of a dash list \NC \NR
\NC \type{prescript}  \NC string \NC the prescript text \NC \NR
\NC \type{postscript} \NC string \NC the postscript text \NC \NR
\NC \type{stacking}   \NC number \NC the stacking (level) \NC \NR
\LL
\stoptabulate

The entry \type {dash} is optional.

\stopsubsection

\startsubsection[title={start_bounds, start_clip, start_group}]

\starttabulate[|l|l|p|]
\DB field  \BC type \BC explanation \NC \NR
\TB
\NC \type{path}     \NC table \NC the list of knots \NC \NR
\NC \type{stacking} \NC number \NC the stacking (level) \NC \NR
\LL
\stoptabulate

\stopsubsection

\startsubsection[title={stop_bounds, stop_clip, stop_group}]

Here we have only one key:

\starttabulate[|l|l|p|]
\DB field  \BC type \BC explanation \NC \NR
\TB
\NC \type{stacking} \NC number \NC the stacking (level) \NC \NR
\LL
\stoptabulate

\stopsubsection

\stopsection

\startsection[title={Subsidiary table formats}]

\startsubsection[title={Paths and pens}]

Paths and pens (that are really just a special type of paths as far as \MPLIB\ is
concerned) are represented by an array where each entry is a table that
represents a knot.

\starttabulate[|l|l|p|]
\DB field  \BC type \BC explanation \NC \NR
\TB
\NC \type{left_type}  \NC string \NC when present: endpoint, but usually absent \NC \NR
\NC \type{right_type} \NC string \NC like \type {left_type} \NC \NR
\NC \type{x_coord}    \NC number \NC X coordinate of this knot \NC \NR
\NC \type{y_coord}    \NC number \NC Y coordinate of this knot \NC \NR
\NC \type{left_x}     \NC number \NC X coordinate of the precontrol point of this knot \NC \NR
\NC \type{left_y}     \NC number \NC Y coordinate of the precontrol point of this knot \NC \NR
\NC \type{right_x}    \NC number \NC X coordinate of the postcontrol point of this knot \NC \NR
\NC \type{right_y}    \NC number \NC Y coordinate of the postcontrol point of this knot \NC \NR
\LL
\stoptabulate

There is one special case: pens that are (possibly transformed) ellipses have an
extra key \type {type} with value \type {elliptical} besides the array part
containing the knot list.

\stopsubsection

\startsubsection[title={Colors}]

A color is an integer array with 0, 1, 3 or 4 values:

\starttabulate[|l|l|p|]
\DB field  \BC type \BC explanation \NC \NR
\TB
\NC \type{0} \NC marking only \NC no values                                                     \NC \NR
\NC \type{1} \NC greyscale    \NC one value in the range $(0,1)$, \quote {black} is $0$         \NC \NR
\NC \type{3} \NC \RGB         \NC three values in the range $(0,1)$, \quote {black} is $0,0,0$  \NC \NR
\NC \type{4} \NC \CMYK        \NC four values in the range $(0,1)$, \quote {black} is $0,0,0,1$ \NC \NR
\LL
\stoptabulate

If the color model of the internal object was \type {uninitialized}, then it was
initialized to the values representing \quote {black} in the colorspace \type
{defaultcolormodel} that was in effect at the time of the \type {shipout}.

\stopsubsection

\startsubsection[title={Transforms}]

Each transform is a six|-|item array.

\starttabulate[|l|l|p|]
\DB index  \BC type \BC explanation \NC \NR
\TB
\NC \type{1} \NC number \NC represents x  \NC \NR
\NC \type{2} \NC number \NC represents y  \NC \NR
\NC \type{3} \NC number \NC represents xx \NC \NR
\NC \type{4} \NC number \NC represents yx \NC \NR
\NC \type{5} \NC number \NC represents xy \NC \NR
\NC \type{6} \NC number \NC represents yy \NC \NR
\LL
\stoptabulate

Note that the translation (index 1 and 2) comes first. This differs from the
ordering in \POSTSCRIPT, where the translation comes last.

\stopsubsection

\startsubsection[title={Dashes}]

Each \type {dash} is a hash with two items. We use the same model as \POSTSCRIPT\
for the representation of the dashlist. \type {dashes} is an array of \quote {on}
and \quote {off}, values, and \type {offset} is the phase of the pattern.

\starttabulate[|l|l|p|]
\DB field  \BC type \BC explanation \NC \NR
\TB
\NC \type{dashes} \NC hash   \NC an array of on-off numbers \NC \NR
\NC \type{offset} \NC number \NC the starting offset value  \NC \NR
\LL
\stoptabulate

\stopsubsection

\startsubsection[title={Pens and \type {peninfo}}]

\libindex{peninfo}

There is helper function (\type {peninfo(obj)}) that returns a table containing
a bunch of vital characteristics of the used pen (all values are floats):

\starttabulate[|l|l|p|]
\DB field  \BC type \BC explanation \NC \NR
\TB
\NC \type{width} \NC number \NC width of the pen \NC \NR
\NC \type{sx}    \NC number \NC $x$ scale        \NC \NR
\NC \type{rx}    \NC number \NC $xy$ multiplier  \NC \NR
\NC \type{ry}    \NC number \NC $yx$ multiplier  \NC \NR
\NC \type{sy}    \NC number \NC $y$ scale        \NC \NR
\NC \type{tx}    \NC number \NC $x$ offset       \NC \NR
\NC \type{ty}    \NC number \NC $y$ offset       \NC \NR
\LL
\stoptabulate

\stopsubsection

\startsubsection[title={Character size information}]

\libindex{char_width}
\libindex{char_height}
\libindex{char_depth}

These functions find the size of a glyph in a defined font. The \type {fontname}
is the same name as the argument to \type {infont}; the \type {char} is a glyph
id in the range 0 to 255; the returned \type {w} is in AFM units.

\startfunctioncall
<number> w = char_width(mp,<string> fontname, <number> char)
<number> h = char_height(mp,<string> fontname, <number> char)
<number> d = char_depth(mp,<string> fontname, <number> char)
\stopfunctioncall

\stopsubsection

\stopsection

\startsection[title=Scanners]

After a relative long period of testing the scanners are now part of the
interface. That doesn't mean that there will be no changes: depending on the
needs and experiences details might evolve. The summary below is there
still preliminary and mostly provided as reminder.

\starttabulate[|l|l|p|]
\DB scanner \BC argument \BC returns \NC \NR
\TB
\NC \type{scannext}       \NC instance, keep         \NC token, mode, type \NC \NR
\NC \type{scanexpression} \NC instance, keep         \NC type \NC \NR
\NC \type{scantoken}      \NC instance, keep         \NC token, mode, kind \NC \NR
\NC \type{scansymbol}     \NC instance, keep, expand \NC string \NC \NR
\NC \type{scannumeric}    \NC instance, type         \NC number \NC \NR
\NC \type{scaninteger}    \NC instance, type         \NC integer \NC \NR
\NC \type{scanboolean}    \NC instance, type         \NC boolean \NC \NR
\NC \type{scanstring}     \NC instance, type         \NC string \NC \NR
\NC \type{scanpair}       \NC instance, hashed, type \NC table or two numbers \NC \NR
\NC \type{scancolor}      \NC instance, hashed, type \NC table or three numbers \NC \NR
\NC \type{scancmykcolor}  \NC instance, hashed, type \NC table or four numbers \NC \NR
\NC \type{scantransform}  \NC instance, hashed, type \NC table or six numbers \NC \NR
\NC \type{scanpath}       \NC instance, hashed, type \NC table with hashes or arrays \NC \NR
\NC \type{scanpen}        \NC instance, hashed, type \NC table with hashes or arrays \NC \NR
\NC \type{scanproperty}   \NC {\em todo}             \NC \NC \NR
\HL
\NC \type{skiptoken}      \NC {\em todo}             \NC \NC \NR
\LL
\stoptabulate

The types and token codes are numbers but they actually depend on the
implementation (although changes are unlikely). The types of data structures can
be queried with \ctxlua {document.showmpcodes ("gettypes")}, and command codes
with \ctxlua {document.showmpcodes ("getcodes")}

Now, if you really want to use these, keep in mind that the internals of
\METAPOST\ are not trivial, especially because expression scanning can be
complex. So you need to experiment a bit. In \CONTEXT\ all is (and will be)
hidden below an abstraction layer so users are not bothered by all these
look|-|ahead and push|-|back issues that originate in the way \METAPOST\ scans
its input.

The supported color models are: \ctxlua {document.showmpcodes ("getcolormodels")}.

If you want the internal codes of the possible fields in a graphic object use
\ctxlua {document.showmpcodes ("getobjecttypes")}. You can query the id of a
graphic object with the \type {gettype} function.

\startluacode
local t = mplib.getobjecttypes()
local f = mplib.getfields()
context.starttabulate { "|T|T|Tpl|" }
    context.DB()
        context("id")
    context.BC()
        context("object")
    context.BC()
        context("fields")
    context.NC()
    context.NR()
    for i=1,#t do
        context.NC()
            context(i)
        context.NC()
            context(t[i])
        context.NC()
            context("% t",f[i])
        context.NC()
        context.NR()
    end
    context.LL()
context.stoptabulate()
\stopluacode

\stopsection

\startsection[title=Injectors]

It is important to know that piping code into the library is pretty fast and
efficient. Most processing time relates to memory management, calculations and
generation of output can not be neglected either. Out of curiousity I added some
functions that directly push data into the library but the gain is not that
large. \footnote {The main motivation was checking of huge paths could be
optimized. The other data structures were then added for completeness.}

\starttabulate[|l|l|]
\DB scanner \BC argument \NC \NR
\TB
\NC \type{injectnumeric}   \NC instance, number \NC \NR
\NC \type{injectinteger}   \NC instance, number \NC \NR
\NC \type{injectboolean}   \NC instance, boolean \NC \NR
\NC \type{injectstring}    \NC instance, string \NC \NR
\NC \type{injectpair}      \NC instance, (table with) two numbers \NC \NR
\NC \type{injectcolor}     \NC instance, (table with) three numbers \NC \NR
\NC \type{injectcmykcolor} \NC instance, (table with) four numbers \NC \NR
\NC \type{injecttransform} \NC instance, (table with) six numbers \NC \NR
\NC \type{injectpath}      \NC instance, table with hashes or arrays, cycle, variant \NC \NR
\NC \type{injectwhatever}  \NC instance, ont of the above depending on type and size \NC \NR
\LL
\stoptabulate

The path injector takes a table with subtables that are either hashed (like the
path solver) or arrays with two, four or six entries. When the third argument has
the value \type {true} the path is closed. When the fourth argument is \type
{true} the path is constructed out of straight lines (as with \type {--}) by
setting the \type {curl} values to~1 automatically. \footnote {This is all
experimental so future versions might provide more control.}

This is the simplest path definition:

\starttyping
{
    { x, y },
    ...,
    cycle = true
}
\stoptyping

and this one also has the control points:

\starttyping
{
    { x0, y0, x1, y1, x2, y2 },
    ...,
    cycle = true
}
\stoptyping

A very detailed specification is this but you have to make sure that the
parameters make sense.

\starttyping
{
    {
        x_coord       = ...,
        y_coord       = ...,
        left_x        = ...,
        left_y        = ...,
        right_x       = ...,
        right_y       = ...,
        left_tension  = ...,
        right_tension = ...,
        left_curl     = ...,
        right_curl    = ...,
        direction_x   = ...,
        direction_y   = ...,
        left_type     = ...,
        right_type    = ...,
    },
    ...,
    cycle = true
}
\stoptyping

Instead of the optional keyword \type {cycle} you can use \type {close}.

\stopsection

\startsection[title={To be checked}]

\starttyping
% solvepath
% expandtex
\stoptyping

\stopsection

\stopchapter

\stopcomponent
