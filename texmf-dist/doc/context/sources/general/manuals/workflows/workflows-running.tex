% language=us runpath=texruns:manuals/workflows

\environment workflows-style

\startcomponent workflows-running

\startchapter[title={Running}]

\startsection[title={Errors}]

A \CONTEXT\ runs normally spits out quite some information to the console. In
fact, even more information can go to the log file. It makes sense to have a look
at the end of the log file occasionally because there you can find information
about the (file) structure loaded, modules, issues with references and|/|or
fonts, etc.

One problem with the terminal is that you can miss an issue easily, but there is
a way out of this:

\starttyping
\enabledirectives[logs.errors]
\stoptyping

The command line argument \type {--errors} has the same consequence. If you want
to quit in an error, you can say for instance:

\starttyping
\enabledirectives[logs.errors=missing characters]
\stoptyping

If you run \CONTEXT\ with \type {--trackers} or \type {--directives} you get some
information about the possible extra tracing. It might be illustrative to run a
file with:

\starttyping
\enabletrackers[*]
\stoptyping

There are all kind of trackers and directives and you can get a list with:

\starttyping
context --trackers --directives
\stoptyping

An example is:

\starttyping
\enabletrackers[fonts.missing=replace]
\stoptyping

or just:

\starttyping
\enabletrackers[fonts.missing]
\stoptyping


\stopsection

\startsection[title={Silent}]

The \type {--silent} options blocks most message. You can also pass a list (or
pattern) of categories to silence. The \type {--noconsole} option only disables
logging to the console. The error reporting mentioned in the previous section
will never be silenced.

\stopsection

\stopchapter

\stopcomponent

