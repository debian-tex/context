% language=us runpath=texruns:manuals/ontarget

\startcomponent ontarget-introduction

\environment ontarget-style

\startchapter[title={Introduction}]

This is the seventh wrapup of the \LUATEX\ and \LUAMETATEX\ development cycle. It
is dedicated to all those users who kept up with developments and are always
willing to test the new features. Without them a project like this would not be
possible.

At the time this introduction is written the \LUAMETATEX\ code base is rather
stable and quite a bit of the \MKIV\ code base has been adapted to new situation.
But, as usual, there are always new possibilities to explore, so I expect that
this document will grow over time as did the others. I'm not going to repeat all
that has been done because that's what the previous episodes are about.

As the title suggest, we're still on target. When the \LUAMETATEX\ project
started there actually was no deadline formulated so in fact we're always on
target. The core components \TEX, \METAPOST, and \LUA\ are all long term efforts
so we're in no hurry at all. However, this is the year that a fast pace will
become a slow pace with respect to the \LUAMETATEX\ code base. There are still
some things on the agenda but in principle the goals are reached. One problem in
today's code development is that useability and quality seems to relate to the
amount of changes in code. No update can mean old, unusable and uninteresting.
It's probably why some sources get this silly yearly copyright year update.
However, the update cycle of good old \TEX\ has an decade interval by now while
it is still a pretty useable program. It would be nice to end up in such a long
term cycle with \LUAMETATEX: bug fixes only.

Although \CONTEXT\ has always adapted early to new developments (color, graphics,
\PDF, \METAPOST, \ETEX, \PDFTEX, \LUATEX, \UTF, fonts) the effects on the
\CONTEXT\ code base are mostly hidden for users. There have been some changes
between \MKII\ and \MKIV, simply because there has been a shift from specific
eight bit encodings to \UTF\ and \TYPEONE\ to \OPENTYPE\ fonts. Both had an
impact on important subsystems: input encodings, font definitions and features,
language and script support. On other subsystems the impact was hardly
noticeable, like for instance backend related features (these have always been
kind of abstract). That doesn't mean that these haven't changed deep down, they
definitely have. Some mechanisms became better in \MKIV, simply because less
hackery was needed. My experience is that when users see that it gets better or
easier, they are also willing to adapt the few lines in their document source
that benefit from it. Of course the impact on the \METAPOST\ integration in
\CONTEXT\ had a real large impact, especially in terms of performance.

The upgrade to \LMTX, the version of \CONTEXT\ for \LUAMETATEX, is even less
visible although already some new mechanisms showed up. This time a couple of
engine specific features have been improved and made more flexible. In fact, the
whole code base of the engine has been overhauled. This happened stepwise because
we had to make sure all things kept working. As a first step code was made
independent of the compilation infrastructure and the dependencies, other than a
very few small ones, have been removed. The result is a rather lean and mean
setup, even when we consider what has been added at the primitive level and
traditional subsystems. A benefit is that in the meantime the \LUAMETATEX\ \LMTX\
combination outperforms \LUATEX\ with \MKIV, something that was not ensured when
the built|-|in \PDF\ backend was removed and delegated to \LUA. By binding
development closely to \CONTEXT\ we also hope that the code base stays clean of
arbitrary extensions.

Because in the end, \TEX\ is also a programming language, there have been
extensions that make programming easier. There is already a stable middle layer
of auxiliary macros in \CONTEXT\ that help the user who likes to program but
doesn't like real low level primitives and dirty tricks, but by extending the
primitive repertoire a bit users can now stay closer to the original \TEX\
concepts. Adding more and more layers of indirectness makes no sense if we can
improve the bottom programming layer. It also makes coding a bit more natural
(the \TEX\ look), apart from offering performance benefits. This is where you can
see differences between the \MKIV\ and \LMTX\ code base which for that reason is
now nearly split completely. The \METAPOST\ subsystem has been extended with
proper scanners so that we can enhance the interfaces in a natural way and as a
result we also have an upgraded code base there. We also moved to \LUA 5.4 and
will keep up as long as compatibility is no issue. Some \LUA\ code is likely to
remain common between \MKIV\ and \LMTX, for instance font handling and helpers
but we'll see where that ends.

The \LUAMETATEX\ engine provides control over most internals and there are all
kind of new interesting features. Decades of \CONTEXT\ development are behind
that. Also, in the days that there were discussions about extending \TEX,
\CONTEXT\ was not that much of influence and on the road to and from user group
meetings, Taco and I often discussed what we'd like to see added (and some was
actually implemented in \type {eetex} but that only lived on our machines. One
can consider \LUATEX\ to be a follow up on that, and \LUAMETATEX\ in turn follows
up on that project, which we both liked doing a lot. In some way \LUATEX\ lowered
the boundary for implementing some of the more intrusive extensions in
\LUAMETATEX\ and the follow up on \MPLIB. And once you start along that road
small steps become large steps and one can as well be try to be as complete as
possible. We've come a long way but eventually arrived at the destination.
Personally I think we got there by not being in a hurry.

But even targets that are reached can eventually move,

\blank[2*big]

\startlines
Hans Hagen
Hasselt NL
August 2021\high{++}
\stoplines

\stopchapter

\stopcomponent
