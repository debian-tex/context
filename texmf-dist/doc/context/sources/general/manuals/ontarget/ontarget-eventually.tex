% language=us runpath=texruns:manuals/ontarget

\startcomponent ontarget-eventually

\environment ontarget-style

\startchapter[title={Eventually 1.0}]

\startsection[title=Reflection]

This is just a short reflection on how we came to version 1.0 of \LUAMETATEX.
Much has already been said in articles and history documents. There is nothing
in here that is new but I just occasionally like to wrap up the current state.
At the time of writing, which happens to be the \CONTEXT\ 2021 meeting, we're
somewhere between 0.9 and 1.0 and as usual it reflects a current state of mind.

\stopsection

\startsection[title=Introduction]

The development on \LUAMETATEX\ took a bit more time than I had in planned when I
started with it. I presume that it also relates to the way the \TEX\ program is
looked at: a finished program that converges to a bugless state. But, with
version 1.0 near by it makes sense to reflect on the process. Before I go into
details I want to remark that when I wrote \CONTEXT\ I looked at this program
from the macro end. I had no real reason to look into the code, and figuring out
what happens in a black box is a challenge (and kind of game) in itself. At the
time I started using \TEX\ I had done my share of complex and relatively large
scale programming in \PASCAL\ and \MODULA\ so it's not that I was afraid of
languages. It was before the Internet took off and not being in academia and
connected one had to figure things out anyway. I did have Don's 5 volume \TEX\
series but stuck to the \TEX\ book. Being on \MSDOS\ I couldn't compile the
program anyway, definitely not without the source at hand. I did read the first
chapters of the \METAFONT\ book, but apart from being intrigued by it, it was not
before I ran into \METAPOST\ that knowing that language took off. Of course I had
browsed \TEX\ the program but not in a systematic way.

I was involved with \PDFTEX\ development but stayed at my end of the line: needs,
applications, testing and suggestions. With \LUATEX\ that line got crossed,
triggered by the \LUA\ interfaces, but while I focussed on the \TEX\ end, Taco
did the \CCODE, and we had pleasant and intense daily discussion on how to move
forward. I could not get away any longer with the abstraction but had to deal
with nodes and such, which was okay as we were hit the boundaries of convenience
programming solutions in \CONTEXT.

When we started our \LUATEX\ journey the \TEX\ follow|-|up most widely used,
\PDFTEX, did have some \ETEX\ extensions but in retrospect only a few of those
were of relevance to us, like the concept of \type {\protected} macros \footnote
{In \CONTEXT\ we always had a protection mechanism and from the \LUATEX\ source I
learned that the macro bases solution was basically the same as the one used in
the engine.} and the larger set of registers. And the \ETEX\ project, in spite of
occasional discussions, never became a continuous effort. The \NTS\ project that
was related to \ETEX\ and had as objective an extensible successor produced a
\JAVA\ implementation but that one was never useful (as a starter, its
performance was such that it could not be used) and I didn't really look forward
to spending time on \JAVA\ anyway. Taco and I played with an extended \ETEX\ but
lack of time made that one end up in the archive.

There were some programmatic additions to \PDFTEX\ but it's main attributes were
protrusion, expansion and a \PDF\ backend (\THANH's thesis subject). Features
like position tracking were handy but basically just a built|-|in variant of a
concept we already had come up with at the \DVI\ level (using a postprocessing
script that later became \type {dvipos}). There was \OMEGA\ with a directional
model but this engine was always more of an academic project, not a production
system. \footnote {\ALEPH\ was more reliable but never took off, if only because
\PDFTEX\ had a backend.} It was \XETEX\ that moved the \TEX\ world into the
\UNICODE\ domain and opened the engine up to new font technologies. Although
\UTF8\ was already doable in earlier engines (which is why \CONTEXT\ used it
already for some internals), native support was way more convenient.

It was clear that if we wanted to move on we had to make more fundamental steps,
but in such a way that it still fit in with what people expect from \TEX. While
it started an a playground by embedding the \LUA\ interpreter, it quickly became
clear that we could open up the internals in fundamental ways, thereby also
getting around the discussion about to what extent \TEX\ could and should be
extended: that discussion could be and was postponed by the opening up. Because
we already foresaw some of possibilities it was decided to freeze \CONTEXT\ for
the older engines. It was around the first \CONTEXT\ meeting that the \MKII\ and
\MKIV\ tags showed up, around the same time that \LUATEX\ became useable. More
than a decade later, when \LUATEX\ basically had become frozen, at another
meeting it was decided to move on with \LUAMETATEX: the \LUATEX\ project was
pretty much a \CONTEXT\ projects and that follow up would be even more driven by
\CONTEXT\ users and usage. But how does it all feel 15 years later? I'll try to
summarize that below. It will also explain why I got more audacious in extending
the \LUATEX\ engine into what is now \LUAMETATEX. This also related to the fact
that at some point I realized that progress just demands taking decisions, and it
happens that we can make these in the perspective of \CONTEXT\ without side
effects for other \TEX\ usage. It is also fun to experiment.

\stopsection

\startsection[title=Extending necessary parts]

The \PDFTEX\ program, having a backend built in already supports the usage of
wide \TRUETYPE\ but it was \XETEX\ that first provided using them directly in the
frontend. But that happened within the concept of traditional \TEX, especially
when it comes to math. There are some extra primitives to deal with scripts and
languages but (and this is personally) I decided that these didn't really fit in
the way \CONTEXT\ looks at things so \MKII\ doesn't support anything beyond the
fonts. The \XETEX\ program first was available on Apple computers and font
support was closely related to its technology as well as technologies that relate
to where the program originates. Later other operating systems became supported
too.

We decided in \LUATEX\ to delegate \quote {everything fonts} to \LUA, for a good
reason: we didn't want to be platform dependent. And using libraries has the
danger of periodical enforced fundamental changes because in these times software
politics and fashion have short cycles. The fact that \XETEX\ later changed the
font engine proved that this was a good decision. At some point \LATEX\ decided
to use a special version of \LUATEX\ that uses a font library as alternative,
which is fine, but that also introduces a dependency (and frequent updating of
the binary). The \LUATEX\ engine has a slim variant of the \FONTFORGE\ library
built in for reading various font formats and its backend can embed subsets of
\OPENTYPE, \TYPEONE\ and traditional bitmap fonts. At some point \CONTEXT\
switched to its own \LUA\ based font file interpreter and experimented with a
\LUA\ based backend that later became exclusive for \LUAMETATEX. It became clear
that we could do with less code in the engine and thereby less dependencies.

In this perspective it is also good to notice that the \LUATEX\ engine has no
real concept of \UNICODE: it just expects \UTF8\ and that's it. All internals
provide enough granularity to support \UNICODE. The rest has to come from the
macro package, as we know that each one does it its own way. There are no
dependencies on \UNICODE\ libraries. You only have to look at what ends up on
your system when you install a program that just juggles bytes to notice that by
including one library a whole lot gets drawn in, most of which is not relevant to
the program and we don't want that. It might start small but who knows where
one ends up. If we want users to be able to compile the program, we don't want
to end up in dependency hell.

The \LUATEX\ project was, apart from curiosity and potential usage in \CONTEXT,
initially also driven by the Oriental \TEX\ project that aimed at high quality
bidirectional typesetting. There the focus was on fonts as well as processing
paragraphs. That triggered all kinds of opening up of internals and once
\CONTEXT\ started swapping (and adding) mechanisms using \LUA\ more came to
fruit. In the end it took a decade to reach version 1.0 and we could have stopped
there knowing that we're quite prepared for the future.

Although the whole \TEX\ concept didn't change, there were some fundamental
changes. From the documentation by Don Knuth it becomes clear that interpreting
is closely interwoven with typesetting: the so called main interpretation loop
calls out to font processing, ligature building, hyphenation, kerning, breaking
lines, processing pages, etc. In \LUATEX\ these steps became more independent
simply because the processing of fonts (via \LUA) came down to feeding a linked
list of nodes to a callback function. That list should be hyphenated if needed (a
now separated step) and if needed the traditional font processing could be
applied (ligature building and kerning). But, although one can say that we
already got away from the way \TEX\ works internally, most documentation to the
original program still applied, simply because the fundamental approach was the
same. We didn't feel too guilty about it and I don't think anyone objected. By
the way, the same is true for the math subsystem: we had to adapt it to
\OPENTYPE\ parameters and formula construction and although that was inspired by
\TEX\ it definitely was different, even to the extend that the math fonts that
evolved in the community are now a strange hybrid of old and new.

\stopsection

\startsection[title=Getting around the frozen machinery]

So why did the \LUAMETATEX\ project started at all? There has been plenty written
on how \LUATEX\ evolved and the same is true for \LUAMETATEX\ so I'm not going to
repeat that here. It is enough to know that the demand for a stable and frozen
\LUATEX\ by other users than \CONTEXT\ simply doesn't go well with further
experiments and we still had plenty ideas. Because at some point Taco had no time
I was already responsible for quite some additions to the \LUATEX\ program so it
was no big deal to switch to a an even more extensive mix of working with
\quotation {\TEX\ the macro language} and \quotation {\TEX\ the program}.

The first priorities were with some basic cleanup: remove unused font code, get
rid of some ever changing libraries and remove the backend related code. I could
do that because I already had a \LUA\ driven backend in \MKIV\ (which was removed
later on) and font handling was already all done in \LUA. The idea was to go lean
and mean, and indeed, even with all kind of extensions, the binary is much
smaller than its predecessor, which is nice because it is also a \LUA\ engine.
Simplifying the build so that users can easily compile themselves was also of
high priority because I considered the rather large and complex setup as a time
bomb. And I also had my doubts if we could prevent the \LUATEX\ engine to evolve
over time in a way that made it less useable for \CONTEXT.

But, interestingly all this extending and pruning didn't feel like I was
violating the concept of a long term stable engine. In fact, original \TEX\ has
no backend either, just a simple binary serialization of output (\DVI). And by
removing some font related frontend code we actually came closer to the original.
I suppose that these decisions slowly made me aware of the fact that there was no
reason to not consider more drastic extensions. After all, wasn't the \ETEX\
project also about extending. \footnote {Although non of the ideas that Taco and
I discussed on our numerous trips to meetings all over the world ever made it
into that engine.}

When we look at \LUAMETATEX\ 1.0 we still see the expected machinery there but
many subsystems have been extended. Once I made the decision that it's now or
never, each subsystem got evaluated against my long term wish list and usage in
\CONTEXT. Now, let's be clear: I basically can do all I want in \LUATEX\ but that
doesn't mean it's always a pretty solution. And to make the \CONTEXT\ code base
better to understand for users, even if it is already rather consistent and set
up to be readable, is one of my objectives. I spend a lot of time on readability:
I cannot stand a bad looking source and over time the look and feel is also
determined by the way the \CONTEXT\ interfaces and related syntax highlighting
evolved, especially the \TEX, \METAPOST, \LUA\ mix. This is why \LUAMETATEX\ has
some extensions to the macro language.

So, while some might argue that \quotation {It can already be done.} I decided to
ignore that argument when the actual solutions came too close to \quotation {See
how well I can do this using dirty tricks!}. If we can do better, without harming
the system, let's do it: \LUA\ did it, \CCODE\ did it and even Don Knuth switched
from \PASCAL\ to \CCODE . If we want we can put all the extensions under the
\quotation {\TEX\ is meant to be extended} umbrella, as long as we call it
different, which is what we do. But I admit that one has to (emotionally) cross a
boundary of feeling comfortable with fundamental additions to a program like
\TEX. But I've been around long enough to not feel guilty about it.

So in the end that means that for instance marks were extended, inserts got more
options, glyphs and boxes have way more properties, (the result and handling of)
paragraphs can be better controlled, page breaking got hooks (and might be
extended), local boxes got redone, adjustments were extended, the math machinery
has been completely opened up, hyphenation became more powerful, the font
mechanism got more control and new scaling features, alignments got some
extensions, we can do more with boxes, etc. But often I still first had to
convince myself that it's okay to do so. After all, none of this had happened
before and to my knowledge also has not been considered in ways that resulted in
an implementation (but I might be wrong here). It helps that I can test out
experiments in production versions of \LMTX\ and that users are quite willing to
test.

\stopsection

\startsection[title=Extending the macro language]

In the previous section some mechanisms were mentioned, but before \TEX\ even
ends up there macros and primitives come into play. The \LUATEX\ engine already
has some handy extras, like ways to prepend and append tokens and a limited so
called \quote {local control} mechanism (think of nested main loops). There are
some new look head and expansions related primitives and csname related tricks.
There are a few more conditionals too. Details can be found in manual and
articles.

In \LUAMETATEX\ some more got added and some of these mechanism could be improved
and the reason again is that I aim at readable code. Most programming languages
for instance have conditionals with some kind of continuation (like \type
{elseif}) and so I added that to \TEX\ too \type {\orelse}. Actually, there are
even more new conditionals than in \LUATEX. Yes, we don't really need these,
especially because in \LUAMETATEX\ we can now extend the primitive language via
\LUA, but I wanted to improve readability deep down in \CONTEXT. It also reduces
the clutter when logging, although logging itself has been quite a bit
overhauled. There is less need for intermediate (often not that natural)
intermediate layers when we can do it properly in primitive \TEX\ lingua.

More fundamental was extending the way \TEX\ deals with macro arguments. Although
the extensions to parsing them are using specifiers that make them upward
compatible I admit that even I have to consult a list of possibilities every now
and then but in the end they make things better (performance wise with less
code). As a side effect the macro machinery could be optimized a bit (expansion
as well as the save stacking).

There are a few more ways to store integers and dimensions (these fit in nicely),
there are new into grouping, some primitives have more keywords and
therefore scanners have been extended, the \ETEX\ expression handlers have
alternative variants.

Although this is a sensitive aspect of \TEX\ when it comes to compatibility, at
some point I decided that it made no sense to not expose more details about
nodes, input, and nesting states. The grouping and input related stacks had been
optimized in the meantime so reporting in that area was already not compatible.
Improving logging is an ongoing effort and I don't really loose sleep over it not
being compatible, as long as it gets better. There is now also some tracing for
marks, inserts, math and alignments.

\stopsection

\startsection[title=Refactoring the code base]

This is again an emotionally laden decision: what to touch and keep. For sure we
keep the original comments but that doesn't make it literate. We started out with
a \CCODE\ base that came from converted \PASCAL\ \WEB.

The input machinery is a bit different due to the fact that \LUA\ can (and often
has to) kick in. In \LUAMETATEX\ it's even more different because even more goes
via \LUA. We cannot even run the engine without a basic set of callbacks
assigned: if you don't like that, use \LUATEX. Does this violate the \TEX\
concept? Not really, because system dependencies are explicitly mentioned as such
in the source code. We have to adapt to the way an operating system sees files
anyway (eight bit, \UTF8, \UTF16).

We still have many global variables (a practical Knuth thing I guess) but now
they are grouped into structures so that we can more clearly see where they
belong. This involved quite a but of shuffling and editing but I got there. In
\LUAMETATEX\ all constants (coded in macros) became enumerations, and all hard
coded values too which was quite a bit of work too. Probably no one will notice
or realize that, but starting from an existing code base is more work than
starting from scratch, which is what I always did so far. When possible we use
case statements. Most macros became (inline) functions. Complex functions got
better variable names. All functions are in name spaces. This was (and is) a
stepwise process that takes lots of time, especially because \CONTEXT\ users
expect a reasonable stable system and changes like that are sensitive for errors.

Talking of errors, the error and reporting system has been overhauled, so for
instance we have now a dedicated string formatter. This all happened in several
steps: normalization, consistency, abstraction, formatters, etc. Keep in mind
that we not only have the original messages but also new ones. And we have \TEX,
\LUA\ and \METAPOST\ communicating with the user. Where in \LUATEX\ we have to
conform more to the traditional engine, because that is what other macro packages
rely on, In \CONTEXT\ we have more freedom, so we can make it better and more
detailed. Of course it could all be controlled by configurations but at some
point I decided to kick out variables doing that because it made no sense to
complicate the code base.

Memory management has been overhauled (more dynamic) as has dumping to the (more
efficient) format file. With what is mentioned in the previous paragraphs we can
safely say that in the meantime back porting to \LUATEX\ (which I had in mind)
makes no sense any longer. There is occasionally some pressure to let \LUATEX\ do
the same as other engines (new common features) and that doesn't always fit into
the model. There is no need for \LUAMETATEX\ to follow up on that because often
we already have plenty of possibilities. There is of course still work todo, for
instance I still have to make some variable names in functions more verbose but
that is not fundamental. I also have to go over the documentation in the code. I
might make some interfaces more consistent anyway, so that also would demand
adaptations. And of course the documentation in general always lags behind.

So far I only mentioned dealing with \TEX, but keep in mind that in \LUAMETATEX\
we also have an upgraded \METAPOST: only a \LUA\ backend (we can produce \PDF\
from that other output), no font code, a couple of extensions, more callbacks,
\IO\ via \LUA. Scanners make extending the language possible and injectors make
for efficient piping back to \METAPOST. Such extensions are also possible in
\TEX\ and the \LUAMETATEX\ scanning interfaces have been improved and extended
too. We have extra callbacks (but some were dropped), more helpers (most
noticeable in the node namespace), libraries that improve dealing with binary
files, a reworked token library (which in turn lead to a reorganization of
command codes in the \TEX\ engine), a few more extensions if \LUA\ file handling
and string manipulations. We got decimal math, complex math, new compression
libraries, better (\LUA) memory management, a few optional library interfaces,
etc. Fortunately that all didn't bloat the binary.

So, because in the meantime \LUAMETATEX\ is quite different from \LUATEX, we can
consider the last one to be a prototype for the real deal.

\stopsection

\startsection[title=Simplifying the build]

This was one of the first things I did. It was a curious process of removing more
and more of the original build (all kind of dependencies) which is not entirely
trivial because of the way the \LUATEX\ build is set up. I admit that I did try
to stay within the regular source build concept but after a while I realized that
this made no sense so we (Mojca was involved in that) made the move to \CMAKE.
Shortly after that I started using Visual Studio as editing environment (which
saves time and is rather convenient) and native compilation under \MSWINDOWS\
became possible without any special measures (in fact, setting up the build for
\ARM\ processors was more work).

A side effect is that right from the start we could provide binaries for various
platforms via the compile farm on the \CONTEXT\ garden maintained by Mojca, who
also does daily \TEX\ live builds there. On my machine I use the Windows Linux
Subsystem for cross compilation but we can also do native builds. And, with my
laptop being a robust 2013 old timer I force myself to make sure that
\LUAMETATEX\ keeps performing well.

\stopsection

\startsection[title=Because it just makes sense]

So, in the end \LUAMETATEX\ is likely the engine most different from the Knuthian
original but from the above one can conclude that this was a graduate process
where I got more audacious over time. In the end the only thing that matters (and
I believe that Don Knuth agrees with this) that you like writing the code, feel
confident that the code is all right, explore the possibilities, try to improve
the quality and understanding and that successive rewrites can reduce obscurity.
And in my opinion we didn't loose the \TEX\ look and feel and still can operate
well within the established boundaries of the \TEX\ ecosystem. The fact that most
\CONTEXT\ users in the meantime use \LUAMETATEX\ and the related \LMTX\ variant
is an indication that they are okay with it, and that is what matters most.

\stopsection

\stopchapter

\stopcomponent
