% language=us runpath=texruns:manuals/musings

% \showfontkerns

\startcomponent musings-manuals

\environment musings-style

\ifdefined\startnarrowtyping \else
    \definetyping[narrowtyping][style=\ttx]
\fi

\startchapter[title={About what \CONTEXT\ isn't}]

\startsection[title={Introduction}]

It really puzzles me why, when someone someplace asks if \CONTEXT\ is suitable
for her or is his needs, there are answers like: \quotation {You need to think of
\CONTEXT\ as being kind of plain \TEX: you have to define everything yourself.}
That answer probably stems from the fact that for \LATEX\ you load some style
that defines a lot, which you then might need to undefine or redefine, but that's
not part of the answer.

In the following sections I will go into a bit more detail of what plain \TEX\ is
and how it influences macro packages, especially \CONTEXT . I'm sure I have
discussed this before so consider this another go at it.

The \type {plain.tex} file start with the line:

\starttyping
% This is the plain TeX format that's described in The TeXbook.
\stoptyping

A few lines later we read:

\starttyping
% And don't modify the file under any circumstances.
\stoptyping

So, this format related to the \TEX\ reference. It serves as a template for what
is called a macro package. Here I will not go into the details of macro programming
but an occasional snippet of code can be illustrative.

\stopsection

\startsection[title={Getting started}]

The first code we see in the plain file is:

\startnarrowtyping
\catcode`\{=1 % left brace is begin-group character
\catcode`\}=2 % right brace is end-group character
\catcode`\$=3 % dollar sign is math shift
\catcode`\&=4 % ampersand is alignment tab
\catcode`\#=6 % hash mark is macro parameter character
\catcode`\^=7 \catcode`\^^K=7 % circumflex and uparrow are for superscripts
\catcode`\_=8 \catcode`\^^A=8 % underline and downarrow are for subscripts
\catcode`\^^I=10                          % ascii tab is a blank space
\chardef\active=13    \catcode`\~=\active % tilde is active
\catcode`\^^L=\active \outer\def^^L{\par} % ascii form-feed is "\outer\par"
\stopnarrowtyping

Assigning catcodes to the braces and hash are needed in order to make it possible
to define macros. The dollar is set to enter math mode and the ampersand becomes
a separator in tables. The superscript and subscript also relate to math. Nothing
demands these bindings but they are widely accepted. In this respect \CONTEXT\ is
indeed like plain.

The tab is made equivalent to a space and a tilde is made active which means that
later on we need to give it some meaning. It is quite normal to make that an
unbreakable space, and one with the width of a digit when we're doing tables.
Now, nothing demands that we have to assume \ASCII\ input but for practical
reasons the formfeed character is made equivalent to a \type {\par}.

Now what do these \type {^^K} and similar triplets represent? The \type {^^A}
represents character zero and normally all these control characters below decimal
32 (space) are special. The \type {^^I} is the \ASCII\ tab character, and \type
{^^L} the formfeed. But, the ones referred to as uparrow and downarrow in the
comments have only meaning on certain keyboards. So these are typical definitions
that only made sense for Don Knuth at that time and are not relevant in other
macro packages that aim at standardized input media.

\startnarrowtyping
% We had to define the \catcodes right away, before the message line, since
% \message uses the { and } characters. When INITEX (the TeX initializer) starts
% up, it has defined the following \catcode values:
%
% \catcode`\^^@=9  % ascii null is ignored
% \catcode`\^^M=5  % ascii return is end-line
% \catcode`\\=0    % backslash is TeX escape character
% \catcode`\%=14   % percent sign is comment character
% \catcode`\ =10   % ascii space is blank space
% \catcode`\^^?=15 % ascii delete is invalid
% \catcode`\A=11 ... \catcode`\Z=11 % uppercase letters
% \catcode`\a=11 ... \catcode`\z=11 % lowercase letters
% all others are type 12 (other)
\stopnarrowtyping

The comments above speak for themselves. Changing catcodes is one way to adapt
interpretation. For instance, in verbatim mode most catcodes can best be made
letter or other. In \CONTEXT\ we always had so called catcode regimes: for
defining macros, for normal text, for \XML, for verbatim, etc. In \MKIV\ this
mechanism was adapted to the new catcode table mechanism available in that
engine. It was one of the first things we added to \LUATEX. So, again, although
we follow some standards (expectations) \CONTEXT\ differs from plain.

\startnarrowtyping
% We make @ signs act like letters, temporarily, to avoid conflict between user
% names and internal control sequences of plain format.

\catcode`@=11
\stopnarrowtyping

In \CONTEXT\ we went a step further and when defining macros also adapted
the catcode of \type {!} and \type {?} and later in \MKIV\ \type {_}. When
we're in unprotected mode this applies. In addition to regular text
input math is dealt with:

\startnarrowtyping
% INITEX sets up \mathcode x=x, for x=0..255, except that
%
% \mathcode x=x+"7100, for x = `A to `Z and `a to `z;
% \mathcode x=x+"7000, for x = `0 to `9.

% The following changes define internal codes as recommended in Appendix C of
% The TeXbook:

\mathcode`\^^@="2201 % \cdot
\mathcode`\^^A="3223 % \downarrow
\mathcode`\^^B="010B % \alpha
\mathcode`\^^C="010C % \beta
....................
\mathcode`\|="026A
\mathcode`\}="5267
\mathcode`\^^?="1273 % \smallint
\stopnarrowtyping

Here we see another set of definitions but the alphabetic ones are not defined in
\CONTEXT, they are again bindings to the authors special keyboard.

\startnarrowtyping
% INITEX sets \sfcode x=1000 for all x, except that \sfcode`X=999 for uppercase
% letters. The following changes are needed:

\sfcode`\)=0 \sfcode`\'=0 \sfcode`\]=0

% The \nonfrenchspacing macro will make further changes to \sfcode values.
\stopnarrowtyping

Definitions like this depend on the language. Because original \TEX\ was mostly
meant for typesetting English, these things are hard coded. In \CONTEXT\ such
definitions relate to languages.

I show these definitions because they also illustrate what \TEX\ is about:
typesetting math:

\startnarrowtyping
% Finally, INITEX sets all \delcode values to -1, except \delcode`.=0

\delcode`\(="028300
\delcode`\)="029301
\delcode`\[="05B302
\delcode`\]="05D303
\delcode`\<="26830A
\delcode`\>="26930B
\delcode`\/="02F30E
\delcode`\|="26A30C
\delcode`\\="26E30F

% N.B. { and } should NOT get delcodes; otherwise parameter grouping fails!
\stopnarrowtyping

Watch the last comment. One of the complications of \TEX\ is that because some
characters have special meanings, we also need to deal with exceptions. It also
means that arbitrary input is not possible. For instance, unless the percent
character is made a letter, everything following it till the end of a line will
be discarded. This is an areas where macro packages can differ but in \MKII\ we
followed these rules. In \MKIV\ we made what we called \type {\nonknuthmode}
default which means that ampersands are just that and scripts are only special in
math (there was also \type {\donknuthmode}). So, \CONTEXT\ is not like plain
there.

\stopsection

\startsection[title=Housekeeping]

The next section defines some numeric shortcuts. Here the fact is used that a
defined symbolic character can act as counter value. When the number is larger
than 255 a math character is to be used. In \LUATEX, which is a \UNICODE\ engine
character codes can be much larger.

\startnarrowtyping
% To make the plain macros more efficient in time and space, several constant
% values are declared here as control sequences. If they were changed, anything
% could happen; so they are private symbols.

\chardef\@ne=1
\chardef\tw@=2
\chardef\thr@@=3
\chardef\sixt@@n=16
\chardef\@cclv=255
\mathchardef\@cclvi=256
\mathchardef\@m=1000
\mathchardef\@M=10000
\mathchardef\@MM=20000
\stopnarrowtyping

In \CONTEXT\ we still support these shortcuts but never use them ourselves. We
have plenty more variables and constants and nowadays always use verbose names.
(There was indeed a time when each extra characters depleted string memory more
and more so then using short command names made sense.) The comment is right that
using such variables is more efficient, for instance once loaded a macro is a
sequence of tokens, so \type {\@one} takes one memory slot. In the case of the
first three the saving is zero and even interpreting a single character token
\type {3} is not less efficient than \type {\thr@@}, but in the case of \type
{\@cclv} the three tokens \type {255} take more memory and also trigger the number
scanner which is much slower than simply taking the meaning of the \type
{\chardef}'d token. However, the \CONTEXT\ variable \type {\plusone} is as
efficient as the \type {\@ne} and it looks prettier in code too (and I'm very
sensitive for that). So, here \CONTEXT\ is definitely different!

It makes no sense to show the next section here: it deals with managing
registers, like counters and dimensions and token lists. Traditional \TEX\ has
255 registers per category. Associating a control sequence (name) with a specific
counter is done with \type {\countdef} but I don't think that you will find a
macro package that expects a user to use that primitive. Instead it will provide
a \type {\newcount} macro. So yes, here \CONTEXT\ is like plain.

Understanding these macros is a test case for understanding \TEX. Take the
following snippet:

\startnarrowtyping
\let\newtoks=\relax % we do this to allow plain.tex to be read in twice
\outer\def\newhelp#1#2{\newtoks#1#1\expandafter{\csname#2\endcsname}}
\outer\def\newtoks{\alloc@5\toks\toksdef\@cclvi}
\stopnarrowtyping

The \type {\outer} prefix flags macros as to be used at the outermost level and
because the \type {\newtoks} is in the macro body of \type {\newtoks} it has to
be relaxed first. Don't worry if you don't get it. In \CONTEXT\ we have no outer
macros so the definitions differ there.

The plain format assumes that the first 10 registers are used for scratch
purposes, so best also assume this to be the case in other macro packages. There
is no need for \CONTEXT\ to differ from plain here. The definitions of box
registers and inserts are special: there is no \type {\boxdef} and inserts use
multiple registers. Especially the allocation of inserts is macro package
specific. Anyway, \CONTEXT\ users never see such details because inserts are used
as building blocks deep down.

Right after defining the allocators some more constants are defined:

\startnarrowtyping
% Here are some examples of allocation.

\newdimen\maxdimen \maxdimen=16383.99999pt % the largest legal <dimen>
\stopnarrowtyping

We do have that one, as it's again a standard but we do have more such constants.
This definition is kind of interesting as it assumes knowledge about what is
acceptable for \TEX\ as dimension:

\startbuffer
{\dimen0=16383.99999pt \the\dimen0 \quad \number\dimen0}
{\dimen0=16383.99998pt \the\dimen0 \quad \number\dimen0}
\stopbuffer

\typebuffer

\startlines
\getbuffer
\stoplines

Indeed it is the largest legal dimension but the real largest one is slightly
less. We could also have said the following, which also indicates what the
maximum cardinal is:

\startnarrowtyping
\newdimen\maxdimen \maxdimen=1073741823sp
\stopnarrowtyping

We dropped some of the others defined in plain. So, \CONTEXT\ is a bit like plain
but differs substantially. In fact, \MKII\ already used a different allocator
implementation and \MKIV\ is even more different. We also have more \type {\new}
things.

The \type {\newif} definition also differs. Now that definition is quite special
in plain \TEX, so if you want a challenge, look it up. It defines three macros as
the comment says:

\startnarrowtyping
% For example, \newif\iffoo creates \footrue, \foofalse to go with \iffoo.
\stopnarrowtyping

The \type {\iffoo} is either equivalent to \type {\iftrue} or \type {\iffalse}
because that is what \TEX\ needs to see in order to be able to skip nested
conditional branches. In \CONTEXT\ we have so called conditionals, which are more
efficient. So, yes, you will find such defined ifs in the \CONTEXT\ source but
way less than you'd expect in such a large macro package: \CONTEXT\ code doesn't
look much like plain code I fear.

\stopsection

\startsection[title=Parameters]

A next stage sets the internal parameters:

\startnarrowtyping
% All of TeX's numeric parameters are listed here, but the code is commented out
% if no special value needs to be set. INITEX makes all parameters zero except
% where noted.
\stopnarrowtyping

We use different values for many of them. The reason is that the plain \TEX\ format
is set up for a 10 point Computer Modern font system, and for a particular kind
of layout, so we use different values for:

\startnarrowtyping
\hsize=6.5in
\vsize=8.9in
\maxdepth=4pt
\stopnarrowtyping

and

\startnarrowtyping
\abovedisplayskip=12pt plus 3pt minus 9pt
\abovedisplayshortskip=0pt plus 3pt
\belowdisplayskip=12pt plus 3pt minus 9pt
\belowdisplayshortskip=7pt plus 3pt minus 4pt
\stopnarrowtyping

No, here \CONTEXT\ is not like plain. But, there is one aspect that we do inherit and
that is the ratio. Here a 10 point relates to 12 point and this 1.2 factor is carried
over in some defaults in \CONTEXT. So, in the end we're a bit like plain.

After setting up the internal quantities plain does this:

% We also define special registers that function like parameters:

\startnarrowtyping
\newskip\smallskipamount \smallskipamount=3pt plus 1pt minus 1pt
\newskip\medskipamount \medskipamount=6pt plus 2pt minus 2pt
\newskip\bigskipamount \bigskipamount=12pt plus 4pt minus 4pt
\newskip\normalbaselineskip \normalbaselineskip=12pt
\newskip\normallineskip \normallineskip=1pt
\newdimen\normallineskiplimit \normallineskiplimit=0pt
\newdimen\jot \jot=3pt
\newcount\interdisplaylinepenalty \interdisplaylinepenalty=100
\newcount\interfootnotelinepenalty \interfootnotelinepenalty=100
\stopnarrowtyping

The first three as well as the following three related variables are not internal
quantities but preallocated registers. These are not used in the engine but in
macros. In \CONTEXT\ we do provide them but the first three are never used that
way. The last three are not defined at all. So, \CONTEXT\ provides a bit what
plain provides, just in case.

\stopsection

\startsection[title=Fonts]

The font section is quite interesting. I assume that one reason why some want to
warn users against using \CONTEXT\ is because it supports some of the font
switching commands found in plain. We had no reasons to come up with different ones
but they do different things anyway, for instance adapting to situations. So, in
\CONTEXT\ you will not find the plain definitions:

\startnarrowtyping
\font\tenrm=cmr10 % roman text
\font\preloaded=cmr9
\font\preloaded=cmr8
\font\sevenrm=cmr7
\font\preloaded=cmr6
\font\fiverm=cmr5
\stopnarrowtyping

There is another thing going on here. Some fonts are defined \type {\preloaded}. So,
\type {cmr9} is defined, and then \type {cmr8} and \type {cmr6}. But they all use the
same name. Later on we see:

\startnarrowtyping
\let\preloaded=\undefined % preloaded fonts must be declared anew later.
\stopnarrowtyping

If you never ran into the relevant part of the \TEX\ book or read the program
source of \TEX, you won't realize that preloading means that it stays in memory
which in turn means that when it gets (re)defined later, the font data doesn't
come from disk. In fact, as the plain format is normally dumped for faster reload
later on, the font data is also retained. So, preloading is a speed up hack. In
\CONTEXT\ font loading has always been delayed till the moment a font is really
used. This permits plenty of definitions and gives less memory usage. Of course
we do reuse fonts once loaded. All this, plus the fact that we have a a system of
related sizes, collections of families, support multiple font encodings
alongside, collect definitions in so called typescript, etc. makes that the
\CONTEXT\ font subsystem is far from what plain provides. Only some of the
command stick, like \type {\rm} and \type {\bf}.

The same is true for math fonts, where we can have different math font setups in
one document. Definitely in \MKII\ times, we also had to work around limitations
in the number of available math families, which again complicated the code. In
\MKIV\ things are even more different, one can even consider the implementation
somewhat alien for a standard macro package, but that's for another article (if
at all).

\stopsection

\startsection[title=Macros]

Of course \CONTEXT\ comes with macros, but these are organized in setups,
environments, instances, etc. The whole process and setup is keyword driven. Out
of the box all things work: nothing needs to be loaded. If you want it different,
you change some settings, but you don't need to load something. Maybe that last
aspect is what is meant with \CONTEXT\ being like plain: you don't (normally)
load extra stuff. You just adapt the system to your needs. So there we proudly
follow up on plain \TEX.

In the plain macro section we find definitions like:

\startnarrowtyping
\def\frenchspacing{\sfcode`\.\@m \sfcode`\?\@m \sfcode`\!\@m
  \sfcode`\:\@m \sfcode`\;\@m \sfcode`\,\@m}
\def\nonfrenchspacing{\sfcode`\.3000\sfcode`\?3000\sfcode`\!3000%
  \sfcode`\:2000\sfcode`\;1500\sfcode`\,1250 }
\stopnarrowtyping

and:

\startnarrowtyping
\def\space{ }
\def\empty{}
\def\null{\hbox{}}

\let\bgroup={
\let\egroup=}
\stopnarrowtyping

and:

\startnarrowtyping
\def\nointerlineskip{\prevdepth-1000\p@}
\def\offinterlineskip{\baselineskip-1000\p@
  \lineskip\z@ \lineskiplimit\maxdimen}
\stopnarrowtyping

Indeed we also provide these, but apart from the two grouping related aliases
their implementation is different in \CONTEXT. There is no need to reinvent
names.

For a while we kept (and did in \MKII) some of the plain helper macros, for
instance those that deal with tabs, but we have several more extensive table
models that are normally used. We always had our own code for float placement,
and we also have more options there. Footnotes are supported but again we have
multiple classes, placements, options, etc. Idem for itemized lists, one of the
oldest mechanisms in \CONTEXT. We don't have \type {\beginsection} but of course
we do have sectioning commands, and have no \type {\proclaim} but provide lots of
descriptive alternatives, so many that I forgot about most of them by now (so
plain is a winner in terms of knowing a macro package inside out).

The fact that we use tables, floats and footnotes indeed makes \CONTEXT\ to act
like plain, but that's then also true for other macro packages. A fact is that
plain sets the standard for how to think about these matters! The same is true
for naming characters:

\startnarrowtyping
\chardef\%=`\%
\chardef\&=`\&
\chardef\#=`\#
\chardef\$=`\$
\chardef\ss="19
\chardef\ae="1A
\chardef\oe="1B
\chardef\o="1C
\chardef\AE="1D
\chardef\OE="1E
\chardef\O="1F
\chardef\i="10 \chardef\j="11 % dotless letters
\stopnarrowtyping

But we have many more and understandable the numbers are different in \CONTEXT\
because we use different font (encodings). Their implementation is more adaptive.
The same is true for accented characters:

\startnarrowtyping
\def\`#1{{\accent18 #1}}
\def\'#1{{\accent19 #1}}
\stopnarrowtyping

The definitions in \MKII\ are different (in most cases we use native glyphs) and
in \MKIV\ we use \UNICODE\ anyway. I think that the \type {\accent} command is
only used in a few exceptional cases (like very limited fonts) in \MKII\ and never
in \MKIV. The implementation of for instance accents (and other pasted together
symbols) in math is also quite different.

There are also definitions that seem to be commonly used in macro packages but
that we never use in \CONTEXT\ because they interfere badly with all kind of
other mechanisms, so you will find no usage of

\startnarrowtyping
\def\leavevmode{\unhbox\voidb@x} % begins a paragraph, if necessary
\stopnarrowtyping

in \CONTEXT. In order to stress that we provide \type {\dontleavehmode}, a wink
to not using the one above.

The macro section ends with lots of math definitions. Most of the names used are
kind of standard so again here \CONTEXT\ is like plain, but the implementation
can differ as does the level of control.

\stopsection

\startsection[title=Output]

Once a page is ready it gets wrapped up and shipped out. Here \CONTEXT\ is very
different from plain. The amount of code in plain is not that large but the
possibilities aren't either, which is exactly what the objectives demand: a
simple (example) format that can be described in the \TEX book. But, as with
other aspects of plain, it steered the way macro packages started out as it
showed the way. As did many examples in the \TEX\ book.

\stopsection

\startsection[title=Hyphenation]

As an afterthought, the plain format ends with loading hyphenation patterns, that
is the English ones. That said it will be clear that \CONTEXT\ is not like plain:
we support many languages, and the subsystem deals with labels, specific
typesetting properties, etc.\ too.

\startnarrowtyping
\lefthyphenmin=2 \righthyphenmin=3 % disallow x- or -xx breaks
\input hyphen
\stopnarrowtyping

We don't even use these patterns as we switched to \UTF\ long ago (also in \MKII)
if only because we had to deal with a mix of font encodings. But we did preload the
lot there. In \MKIV\ again things are much different.

\stopsection

\startsection[title=Conclusion]

The plain format does (and provides) what it is supposed to do. It is a showcase
of possibilities and part of the specification. In that respect it's nice that
\CONTEXT\ is considered to be like plain. But if it wasn't more, there was no
reason for its existence. Like more assumptions about \CONTEXT\ it demonstrates
that those coming up with answers and remarks like that probably missed something
in assessing \CONTEXT. Just let users find out themselves what suits best (and
for some that actually might be plain \TEX).

Let me make on thing clear. If you look at the documents that describe the
development of \TEX, \METAFONT\ and the related fonts, you can only awe at what
was done on hardware that doesn't come close to what we hold now in the palm of
our hand. And it was done in a relative short time span. The fact that plain
\TEX\ ran on it the way it did is amazing. Anyone who considers criticizing \TEX\
and plain \TEX\ should think (more than) twice.

\stopsection

\stopchapter

\stopcomponent
