% language=us

\startcomponent followingup-tex

\environment followingup-style

\startchapter[title={\TEX}]

\startsection[title={Prefixes}]

The fact that we merged \ETEX, a bit of \PDFTEX\ and some of \ALEPH\ into
\LUATEX, already makes it a non|-|standard \TEX\ engine. In \LUAMETATEX\ we go a
bit further. Completely outsourcing the backend has the side effect that some
(extension related) primitives have to be implemented explicitly. The fact that
\LUA\ is integrated has consequences for, for instance, initialization.
Defaulting to \UTF-8 input makes it different too. And delegating many font
matters to \LUA\ also doesn't make it behave like good old \TEX.

Here I discuss another difference. One can argue that this definitely makes it
less \TEX, but in practice this is not that problematic. We're talking prefixes
here. Traditional \TEX\ has only prefixes:

\startitemize[n]
\startitem
    \type {\global}: when used, it will make the next definition a global one.
    The \type {\globaldefs} parameter can be used to force global or local
    definitions.
\stopitem
\startitem
    \type {\long}: when applied, this will make a macro bark on a \type {\par}
    (or its equivalent) when grabbing an argument. In \LUATEX\ this check can be
    disabled. \footnote {In a similar fashion barking about a \type {\par} in
    math mode can be disabled. Such warnings made much sense when a \TEX\ run
    took much time and was triggered and traced on relative slow output devices.}
\stopitem
\startitem
    \type {\outer}: when applied the macro can only be used at the outer level.
\stopitem
\stopitemize

Multiple prefixes can be given and their effects accumulate. The \ETEX\ extension
adds another one:

\startitemize[continue]
\startitem
    \type {\protected}: this will make a macro unexpandable inside an \type
    {\edef}, an \type {\xdef} or token list serialization.
\stopitem
\stopitemize

In \CONTEXT\ we never use(d) \type {\outer} and I can't even think of a useful
application in a large macro package. in \MKII\ most interface macros are defined
as \type {\long}, and because in \MKIV\ we block the complaints, we don't need
this prefix either. On the other hand, many macros are defined \type
{\protected}. \footnote {Or in \CONTEXT\ speak, they are defined as \typ
{\unexpanded}, because we already had \typ {\protected} as well as \typ
{\unexpanded} before these were introduces as primitives.}

When you look at the implementation, \type {\long} and \type {\outer} are
properties of the so called command code: we have normal, long, outer and long
outer macros, and each has a unique command code. For some reason \type
{\protected} is not implemented with command codes, which would have doubled the
number to eight, but as special token injected in front of the macro preamble.
Using a command code would have made more sense as there is no real speed penalty
in that, while the special token indicating is a macro (body) is protected now
has to be intercepted in some cases.

Anyhow, already for a while I wondered if I should drop \type {\long} and \type
{\outer} (making them no|-|ops). I also had on my agenda to promote \type
{\protected} to a normal command code. And, already for a long time I wanted to
play with a new prefix: \footnote {This is a typical example of a feature that I
like playing with, before deciding if it will stay (as such).}

\startitemize[continue]
\startitem
    \type {\frozen}: this will protect a macro (for now only a macro) against
    redefinition, which provides a bit of protection for a user.
\stopitem
\stopitemize

Promoting \type {\protected} brings the set of call commands from four to eight,
and a \type {\frozen} property would bump it to sixteen. This is still okay, but
in some places it would involve mode testing. However, dropping \type {\long} and
\type {\outer} would not only keep the set small (just four) but also rid it of
some tests. There is no performance penalty either (even a bit of gain in case of
many protected macros as we no longer need to skip the special signal token) and
it even saves some memory (but not that much).

As a bonus there are a few more conditionals: \type {\ifprotected}, \type
{\iffrozen}, and, very experimental, \type {\ifusercmd}, which can be used to
check if something is user defined (often not a primitive). These probably only
make sense for diagnostic purposes.

In the end, the implementation was not that hard. In the process I also removed
the \type {\suppress...} parameters so \type {\par} no longer plays havoc. If this
new prefix \type {\frozen} stays of will affect more definitions, we'll see.

\stopsection

\startsection[title={Conditionals}]

Another domain where there have been some extensions is conditions. In a previous
chapter I mentioned \type {\iftok} already. As this is not a manual I will not go
into details about other new conditionals. For instance we have a few that can be
used to check for valid dimensions and numbers. This can lead to a bit cleaner
code, although for instance in \CONTEXT\ we always used support macros for this.
We seldom needed more than we had but when interfacing with \METAPOST\ it helps a
little.

Another, maybe interesting one is \type {\ifcondition} which when \TEX\ is in
jump over branches mode is seen as a valid \type {\if<cmd>} token but when it
comes to expansion the following macro determines a true or false state. A second
nice experiment is \type {\orelse} which is to be followed by a valid \type
{\if<cmd>} token and makes for less nesting which sometimes looks nicer and also
has some advantages.

I might wrap up these and other extensions in articles once they are considered
stable and useful. But first I'll test them in real situation, which in practice
means that \CONTEXT\ users will test them, probably without noticing.

\stopsection

\stopchapter

\stopcomponent
