% language=us

\startcomponent followingup-performance

\environment followingup-style

\startchapter[title={Performance}]

\startsection[title={Introduction}]

Those who've read the other documents describing the development of \LUATEX, know
that performance is always on my radar. A decent performance is a must for a
useable workflow, especially because typesetting is a multi|-|pass process.
\footnote {I'm often baffled by reports of (non|-|\CONTEXT) \LUATEX\ users about
the performance of \LUATEX. It seems easier to blame an engine than ones own
macros or setup and most of those tests make no sense anyway. Believe it or not,
but if performance of \CONTEXT\ \MKIV\ was much worse than \MKII\ (using \PDFTEX\
or \XETEX) it would have backfired and the project would never have taken of.
Just think of this: would Hans really use \LUATEX\ and continue with development
if it were that slow?} One page reference changing from two digits to three
digits can influence whatever follows and we're not only talking of a different
page break, even a change in line breaks can have consequences. The core engine
cannot be made much faster. When the (single core) run has the whole cpu
available not much can be gained. But multiple processes are run at the same
time, the cache has to be shared and misses can become an issue. So, efficiency
of code is still important. Occasionally a (tiny) improvement can be made, but
only the accumulation of such improvements can make a dent. The feeling is that
over time \LUATEX\ has not become slower but we keep an eye on possible other
improvements. The memory footprint is also something to keep an eye on. \footnote
{Of course this is all becoming less relevant now that having e.g. a browser open
in the background will set you back with a constant 5\endash10\% cpu load and
slowly accumulating gigabyte memory usage. That actually was something I had to
keep in mind when running \LUAMETATEX\ benchmarks.}

The more we delegate to \LUA, the less we can benefit from for instance \CPU\
improvements: in that case the \LUA\ virtual machine is the bottleneck. And there
is not much we can do about that. This also means that when we delegate more to
\LUA\ we sacrifice performance. Sometimes things can be done more efficient in
\LUA, but those are often tasks that are not performed frequently. That said, I'm
convinced most of the \CONTEXT\ code is quite efficient and not much can be
gained.

The biggest change in \LUAMETATEX\ is the backend. We gain some efficiency in
terms of speed, performance and output in some cases, while in other cases we
loose a bit. On the average the small performance hit is bearable. Because
\CONTEXT\ users don't complain about performance I think that I have some slack
here.

\stopsection

\startsection[title={An example}]

There are a few places where \LUATEX\ looks ahead to check something and goes back
when the condition is not met. Take these:

\starttyping
\hbox   {...}
\hbox   to 10cm {...}
\hrule  width 10cm height 10cm \relax
\dimen0 =10cm
\dimen0 10cm
\mydimen 10cm
\toks0  {...}
\toks0  \toks2
\stoptyping

Spaces and sometimes \type {\relax} after the trigger (\type {\hbox}, \type
{\dimen}, etc.) are skipped and in some case there can be an optional \type {=}
sign. So, there are quite some cases where there is first a check for an optional
equal which itself can be preceded by optional spaces. When there is no equal
sign the last seen token is pushed back into the scanned which effectively means
that a temporary token is allocated, and a one token list is pushed on the input
stack. Then scanning goes on. The same can happen with the open brace in case of
a token list assignment: it gets pushed back and the content scanned checks it
again. In the case of keywords something similar takes place, because here
\LUATEX\ checks explicitly for e.g.\ type {width}, and when it is not found again
it pushes back consumed tokens and checks for the \type {width}. In the case of
the specifiers of the box we don't need to check at all when we have an opening
brace. In the follow up, when the \type {orientation} keyword was added, and the
\type {dir} and \type {bdir} were replaced by \type {direction} a little bit more
was optimized.

In \LUATEX\ this code comes from \PDFTEX\ which takes if from \TEX, but in both
cases some code side effects occur from the transition from \PASCAL\ to \CCODE.
But, in \LUATEX\ we stick to the \CCODE, so we can try to get rid of these
artifacts. During the last years, especially when additional keywords were
introduces (for instance for attributes) already some optimization took place. In
the follow up again some optimizations were applied, for instance quite often we
can combine the check for an equal sign with skipping the spaces.

The gain is not spectacular but as all small bits add up eventually it is
measurable in a complex run. What definitely is true, is that we avoid some
memory access which in turn might pay back when multiple runs happen in parallel.

Of course one can argue that such optimizations are to be avoided but as long as
they don't obscure the code, it's okay. After all, just as one optimizes for
instance a compression algorithm or search routine, there is no reason not to
mildly optimize some of the critical code in \LUATEX. And in \CONTEXT\ we have
plenty of opportunities to check if that works out well. At some point some might
be retrofit into \LUATEX\ 1.2 (or later). \footnote {But it makes less sense now
that there are variants popping up that might depend on the stable base.}

\stopsection

\stopchapter

\stopcomponent
