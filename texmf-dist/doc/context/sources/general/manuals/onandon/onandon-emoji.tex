% language=uk

\usemodule[fonts-emoji]

\environment onandon-environment

% \definefont[MyEmoji][emojionecolor-svginot*default,svg]
% \definefont[MyEmoji][seguiemj*seguiemj-bw]
% \definefont[MyEmoji][emojionemozilla*default,overlay]
% \definefont[MyEmoji][applecoloremoji*default,bitmap]

\definefontfeature[seguiemj-cl][default][colr=yes,ccmp=yes,dist=yes]
\definefontfeature[seguiemj-bw][default][ccmp=yes,dist=yes]

\definefont[MyEmoji]     [seguiemj*seguiemj-cl]
\definefont[MyEmojiLarge][seguiemj*seguiemj-cl @ 100pt]

\definefontsynonym[emoji][seguiemj*seguiemj-cl]

\startcomponent onandon-emoji

\startchapter[title=Emoji again]

Because at the \CONTEXT\ 2016 meeting color fonts \footnote {For that occasion
the cowfont, a practical joke concerning Dutch \quote {koeieletters}, were turned
into a color font and presented at the meeting.} were on the agenda, some time
was spent on emoji (these colorful small picture glyphs). When possible I bring
kids to the Bacho\TEX\ conference so for the 2017 BachoTUG I decided to do
something with emoji that, after all, are mostly used by those younger than I am.
So, I had to take a look at the current state. Here are some observations.

The \UNICODE\ standard defines a whole lot of emoji and if mankind manages to
survive for a while one can assume that a lot more will be added. After all,
icons as well as variants keep evolving. There are several ways to organize these
symbols in groups but I will not give grouping a try. Just visit \type
{emojipedia.org} and you get served well. For this story I only mention that:

\startitemize
    \startitem
        There are quite some shapes and nearly all of them are in color. The
        yellow ones, smilies and such, are quite prominently present but there
        are many more.
    \stopitem
    \startitem
        A special subset is fulled by persons: man, woman, girl, boy and recently
        a baby.
    \stopitem
    \startitem
        The grown ups can be combined in loving couples (either or not kissing)
        and then can form families, but only upto 2 young kids or gender neutral
        babies.
    \stopitem
    \startitem
        All persons can be flagged with one of five skin tones so that not all
        persons (or heads) look bright yellow.
    \stopitem
    \startitem
        Interesting is that girls and boys are still fond of magenta (pinkish)
        and cyan (blueish) cloths and ornaments. Also haircuts are rather
        specific to the gender.
    \stopitem
\stopitemize

For rendering color emojis we have a few color related \OPENTYPE\ font properties
available: bitmaps, \SVG\, and stacked glyphs. Now, if you think of the
combinations that can be made with skin tones, you realize that fonts can become
pretty large if each combination results in a glyph. In the first half of 2017
\MICROSOFT\ released an update for its emoji font and the company took the
challenge to provide not only mixed skin tone couples, but also supported skin
tones for the kids, including a baby.

This recent addition already adds over 25.000 additional glyphs \footnote {That
is the amount I counted when I added all combinations runtime but the emojipedia
mentions twice that amount. Currently in \CONTEXT\ we resolve such combinations
when requested.} so imagine what will happen in the future. But, instead of
making a picture for each variant, a different solution has been chosen. For
coloring this seguiemj font uses the (very flexible) stacking technology: a color
shape is an overlay of colored symbols. The colors are organized in pallets and
it's no big deal to add additional pallets if needed. Instead of adding
pre|-|composed shapes (as is needed with bitmaps and \SVG) snippets are used to
build alternative glyphs and these can be combined into new shapes by
substitution and positioning (for that kerns, mark anchoring and distance
compensation is used).

So, a family can be constructed of composed shapes (man, woman, etc) that each
are composed of snippets (skull, hair, mouth, eyes). So, effectively a family of
four is a bunch of maybe 25 small glyphs overlayed and colored. In \in {figure}
[fig:emojisnippets] we see how a shape is constructed out of separate glyphs. \in
{Figure} [fig:emojisnippetsoverlay] shows how they can be overlayed with colors
(we use a dedicated color set).

\startplacefigure[title={Emoji snippets.},reference=fig:emojisnippets]
    \scale[width=\textwidth]{\framed[frame=off,offset=overlay,align=normal]\bgroup
        \forgetall
        \MyEmoji
        \ShowEmojiSnippets
          [family man light skin tone
           woman dark skin tone
           girl medium skin tone
           boy medium skin tone]
    \egroup}
\stopplacefigure

\startplacefigure[title={Emoji snippets overlayed.},reference=fig:emojisnippetsoverlay]
    \scale[width=\textwidth]{\framed[frame=off,offset=overlay,align=normal]\bgroup
        \forgetall
        \MyEmoji
        \ShowEmojiSnippetsOverlay
          [family man light skin tone
           woman dark skin tone
           girl medium skin tone
           boy medium skin tone]
    \egroup}
\stopplacefigure

When a font supports it, a sequence of emoji can be turned into a more compact
representation. In \in {figure} [fig:emojiskintones] we see how skin tones are
applied in such combinations. \in {Figure} [fig:emojiglyphs] shows the small
snippets.

\startplacefigure[title={Emoji families and such with skin tones.},reference=fig:emojiskintones,location=page]
    \startcombination[2*4]
        {\ruledhbox{\MyEmojiLarge\resolvedemoji
            {family man woman girl boy}}}
            {family man woman girl boy}
        {\ruledhbox{\MyEmojiLarge\resolvedemoji
            {family woman girl boy}}}
            {family woman girl boy}
        {\ruledhbox{\MyEmojiLarge\resolvedemoji
            {family woman woman girl boy}}}
            {family woman woman girl boy}
        {\ruledhbox{\MyEmojiLarge\resolvedemoji
            {family man girl boy}}}
            {family man girl boy}
        {\ruledhbox{\MyEmojiLarge\resolvedemoji
            {family
             man dark skin tone
             woman girl baby}}}
            {family
             man dark skin tone
             woman girl baby}
        {\ruledhbox{\MyEmojiLarge\resolvedemoji
            {family
             man light skin tone
             woman light skin tone
             girl dark skin tone}}}
            {family
             man light skin tone
             woman light skin tone
             girl dark skin tone}
        {\ruledhbox{\MyEmojiLarge\resolvedemoji
            {family woman girl boy}}}
            {family woman girl boy}
        {\ruledhbox{\MyEmojiLarge\resolvedemoji
            {family
             man light skin tone
             woman dark skin tone
             girl medium skin tone
             boy medium skin tone}}}
            {family
             man light skin tone
             woman dark skin tone
             girl medium skin tone
             boy medium skin tone}
    \stopcombination
\stopplacefigure

\startplacefigure[title={Emoji glyphs.},reference=fig:emojiglyphs]
    \scale[width=\textwidth]{\framed[frame=off,offset=overlay,align=normal]\bgroup
        \forgetall
        \MyEmoji
        \ShowEmojiGlyphs
          [family man light skin tone
           woman dark skin tone
           girl medium skin tone
           boy medium skin tone]
    \egroup}
\stopplacefigure

When we have to choose a font we need to take the following criteria into
account:

\startitemize
\startitem
    What is the quality of the shapes? For sure, outlines are best if you want to
    scale too.
\stopitem
\startitem
    How efficient is a shape constructed. In that respect a bitmap or \SVG\ image
    is just one entity.
\stopitem
\startitem
    How well can (semi) arbitrary combinations of emoji be provided. Here the
    glyph approach wins.
\stopitem
\startitem
    Are all skin colors for all human relates shapes supported? Actually it opens
    the possibility for racist fonts.
\stopitem
\startitem
    Are all reasonable combinations of persons supported? It looks like (depending
    on time and version) kissing men or women can be missing, maybe because of
    social political reasons.
\stopitem
\startitem
    Are black and white shapes provided alongside color shapes.
\stopitem
\stopitemize

Maybe an \SVG\ or bitmap image can have a lot of detail compared to a stacked
glyph but, when we're just using pictographic representations, the later is the
best choice.

When I was playing a bit with the skin tone variants and other combinations that
should result in some composed shape, I used the \UNICODE\ test files but I got
the impression that there are some errors in the test suite, for instance with
respect to modifiers. Maybe the fonts are just doing the wrong thing or maybe
some implement these sequences a bit inconsistent. This will probably improve
over time but the question is if we should intercept issues. I'm not in favour of
this because it adds more and more fuzzy code that not only wastes cycles
(energy) but is also a conceptual horror. So, when testing, imperfection has to
be accepted for now. This is no big deal as until now no one ever asked for emoji
support in \CONTEXT.

When no combined shape is provided, the original sequence shows up. A side effect
can be that zero|-|width|-| joiners and modifiers become visible. This depends on
the fonts. Users probably don't care that much about it. Now how do we suppose
that users enter these emoji (sequences) in a document source? One can imagine a
pop up in the editor but \TEX ies are often using commands for special cases.

We already showed some combined shapes. The reader might appreciate the outcome
but getting there from the input takes a bit of work. For instance a two person
\typ {man light skin tone woman medium skin tone girl medium-light skin tone baby
medium-light skin tone} involves this:

\blank
\start
    \setupalign[verytolerant]
    \showotfcomposition
      {seguiemj*seguiemj-cl}
      {1}
      {\resolvedemoji{family
            man light skin tone
            woman medium skin tone
            girl medium-light skin tone
            baby medium-light skin tone}}
\stop
\blank

A black and white example is the following \type {family woman girl}:

\blank
\start
    \setupalign[verytolerant]
    \showotfcomposition
      {seguiemj*seguiemj-bw}
      {1}
      {\resolvedemoji{family woman girl}}
\stop
\blank

I will not show all emoji, just the subset that contains the word \type {woman}
in the description. As you can see the persons in the sequences are separated by
a zero|-|width|-|joiner. There are some curious ones, for instance a \typ {woman
wearing turban} which in terms of \UNICODE\ input is a female combine with a
turban wearing man becomes a beardless woman wearing a turban. Woman vampires and
zombies are not supported so these are male properties.

\startpacked
    \MyEmoji
    \ShowEmoji[woman]
\stoppacked

So what if you don't like these colors? Because we're dealing with \TEX\ you can
assume that if there is some way around the fixed color sets, then it will be
provided. So, when you use \CONTEXT, here is away to overload them:

\startbuffer
\definecolor[emoji-red]   [r=.4]
\definecolor[emoji-green] [g=.4]
\definecolor[emoji-blue]  [b=.4]
\definecolor[emoji-yellow][r=.4,g=.4]
\definecolor[emoji-gray]  [s=1,t=.5,a=1]

\definefontcolorpalette
  [emoji-s]
  [black,emoji-gray]

\definefontcolorpalette
  [emoji-r]
  [emoji-red,emoji-gray]

\definefontcolorpalette
  [emoji-g]
  [emoji-green,emoji-gray]

\definefontcolorpalette
  [emoji-b]
  [emoji-blue,emoji-gray]

\definefontcolorpalette
  [emoji-y]
  [emoji-yellow,emoji-gray]

\definefontfeature[seguiemj-s][ccmp=yes,dist=yes,colr=emoji-s]
\definefontfeature[seguiemj-r][ccmp=yes,dist=yes,colr=emoji-r]
\definefontfeature[seguiemj-g][ccmp=yes,dist=yes,colr=emoji-g]
\definefontfeature[seguiemj-b][ccmp=yes,dist=yes,colr=emoji-b]
\definefontfeature[seguiemj-y][ccmp=yes,dist=yes,colr=emoji-y]

\definefont[MyEmojiS][seguiemj*seguiemj-s]
\definefont[MyEmojiR][seguiemj*seguiemj-r]
\definefont[MyEmojiG][seguiemj*seguiemj-g]
\definefont[MyEmojiB][seguiemj*seguiemj-b]
\definefont[MyEmojiY][seguiemj*seguiemj-y]
\stopbuffer

\typebuffer \getbuffer

\startplacefigure[title={Overloading colors by plugging in a sequence of alternate colors.},reference=fig:emojioverload,location=page]
\startcombination[4*6]
    {\scale[height=.1\textheight]{\MyEmoji \resolvedemoji{triangular ruler}}} {}
    {\scale[height=.1\textheight]{\MyEmoji \resolvedemoji{rabbit face}}}      {}
    {\scale[height=.1\textheight]{\MyEmoji \resolvedemoji{man}}}              {}
    {\scale[height=.1\textheight]{\MyEmoji \resolvedemoji{woman}}}            {}
    {\scale[height=.1\textheight]{\MyEmojiR\resolvedemoji{triangular ruler}}} {}
    {\scale[height=.1\textheight]{\MyEmojiR\resolvedemoji{rabbit face}}}      {}
    {\scale[height=.1\textheight]{\MyEmojiR\resolvedemoji{man}}}              {}
    {\scale[height=.1\textheight]{\MyEmojiR\resolvedemoji{woman}}}            {}
    {\scale[height=.1\textheight]{\MyEmojiG\resolvedemoji{triangular ruler}}} {}
    {\scale[height=.1\textheight]{\MyEmojiG\resolvedemoji{rabbit face}}}      {}
    {\scale[height=.1\textheight]{\MyEmojiG\resolvedemoji{man}}}              {}
    {\scale[height=.1\textheight]{\MyEmojiG\resolvedemoji{woman}}}            {}
    {\scale[height=.1\textheight]{\MyEmojiB\resolvedemoji{triangular ruler}}} {}
    {\scale[height=.1\textheight]{\MyEmojiB\resolvedemoji{rabbit face}}}      {}
    {\scale[height=.1\textheight]{\MyEmojiB\resolvedemoji{man}}}              {}
    {\scale[height=.1\textheight]{\MyEmojiB\resolvedemoji{woman}}}            {}
    {\scale[height=.1\textheight]{\MyEmojiY\resolvedemoji{triangular ruler}}} {}
    {\scale[height=.1\textheight]{\MyEmojiY\resolvedemoji{rabbit face}}}      {}
    {\scale[height=.1\textheight]{\MyEmojiY\resolvedemoji{man}}}              {}
    {\scale[height=.1\textheight]{\MyEmojiY\resolvedemoji{woman}}}            {}
    {\scale[height=.1\textheight]{\MyEmojiS\resolvedemoji{triangular ruler}}} {}
    {\scale[height=.1\textheight]{\MyEmojiS\resolvedemoji{rabbit face}}}      {}
    {\scale[height=.1\textheight]{\MyEmojiS\resolvedemoji{man}}}              {}
    {\scale[height=.1\textheight]{\MyEmojiS\resolvedemoji{woman}}}            {}
\stopcombination
\vskip2ex % looks nicer
\stopplacefigure

In \in {figure} [fig:emojioverload] we see how this is applied. You can provide
as many colors as needed but when you don't provide enough the last one is used.
This way we get the overlayed transparent colors in the examples. By using
transparency we don't obscure shapes.

The emojipedia mentions \quotation {Asked about the design, \MICROSOFT\ told
emojipedia that one of the reasons for the thick stroke was to allow each emoji
to be easily read on any background color.} The first glyph in the stack seems to
do the trick, so just make sure that it doesn't become white. And, before I read
that remark, while preparing a presentation with a colored background, I had
already noticed that using a background was no problem. This font definitely sets
the standard.

How do we know what colors are used? The next table shows the first color palette
of seguiemj. There are quite some colors so defining your own definitely involved
some studying.

{\MyEmoji\ShowEmojiPalettes[1]}

Normally special symbols are accessed in \CONTEXT\ with the \type {symbol}
command where symbols are organized in symbol sets. This is a rather old
mechanism and dates from the time that fonts were limited in coverage and symbols
were collected in special fonts. The emoji are accessed by their own command:
\type {\emoji}. The font used has the font synonym \type {emoji} so you need to
set that one first:

\starttyping
\definefontsynonym[emoji][seguiemj*seguiemj-cl]
\stoptyping

Here is an example:

\startbuffer
\emoji{woman light skin tone}\quad
\emoji{woman scientist}\quad
{\bfd bigger \emoji{man health worker}}
\stopbuffer

\typebuffer

or typeset: \getbuffer

The emoji symbol scales with the normal running font. When you ask for a family
with skin toned members the lookup can result in another match (or no match)
because one never knows to what extend a font supports it.

\starttabulate[|l|p|]
\NC \type{\expandedemoji} \NC the sequence constructed from the given string \NC \NR
\NC \type{\resolvedemoji} \NC a protected sequence constructed from the given string \NC \NR
\NC \type{\checkedemoji } \NC an typeset sequence with unresolved modifiers and joiners removed \NC \NR
\NC \type{\emoji}         \NC a typeset resolved sequence using the \type {emoji} font synonym \NC \NR
\NC \type{\robustemoji}   \NC a typeset checked sequence using the \type {emoji} font synonym \NC \NR
\stoptabulate

In case you wonder how some of the details above were typeset, there is a module
\type {fonts-emoji} that provides some helpers for introspection.

\starttabulate[|l|p|]
\NC \type {\ShowEmoji}                \NC show all the emoji in the current font \NC \NR
\NC \type {\ShowEmojiSnippets}        \NC show the snippets of a given emoji \NC \NR
\NC \type {\ShowEmojiSnippetsOverlay} \NC show the overlayed snippets of a given emoji \NC \NR
\NC \type {\ShowEmojiGlyphs}          \NC show the snippets of a typeset emoji \NC \NR
\NC \type {\ShowEmojiPalettes}        \NC show the color pallets in the current font \NC \NR
\stoptabulate

Examples of usage are:

\starttyping
\ShowEmojiSnippets[family man woman girl boy]
\ShowEmojiGlyphs  [family man woman baby girl]
\ShowEmoji        [^man]
\ShowEmoji
\ShowEmojiPalettes
\ShowEmojiPalettes[1]
\stoptyping

A good source of information about emoji is the mentioned \type {emojipedia.org}
website. There you find not only details about all these symbols but also has
some history. It compares updates in fonts too. It mentions for instance that in
the creative update of Windows 10, some persons grew beards in the seguiemj font
and others lost an eye. Now, if you look at the snippets shown before, you can
wonder if that eye is really gone. Maybe the color is wrong or the order of
stacking is not right. I decided not to waste time looking into that.

Another quote: \quotation {Support for color emoji presentation on \MSWINDOWS\ is
limited. Many applications on \MSWINDOWS\ display emojis with a black and white
text presentation instead of their color version.} Well, we can do better with
\TEX, but as usual not that many people really cares about that. But it's fun
anyway.

We end with a warning. When you use \quote {ligatures} like this, you really need
to check the outcome. For instance, when \MICROSOFT\ updated the font end 2017,
same gender couples got different hair style for the individuals so that one can
still distinguish them. However, kissing couples and couples in love (indicated
by a heart) seem to be removed. Who knows how and when politics creep into fonts:
is public mixed couple kissing permitted, do we support families with any mix of
gender, is associating pink with girls okay or not, how do we distinguish male
and female anyway? In \in {figure} [fig:old-and-new] we see the same combination
twice, the early 2017 rendering versus the late 2017 rendering. Can you notice
the differences?

\startplacefigure[title={Incompatible updates.},reference=fig:old-and-new]
    \externalfigure[onandon-seguiemj.pdf][width=\textwidth]
\stopplacefigure

\stopchapter

\stopcomponent
