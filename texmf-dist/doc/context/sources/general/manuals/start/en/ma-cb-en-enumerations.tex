\startcomponent ma-cb-en-enumerations

\enablemode[**en-us]

\project ma-cb

\startchapter[title=Numbered definitions]

\index{numbered definition}

\Command{\tex{defineenumeration}}
\Command{\tex{setupenumerations}}

With \type{\defineenumeration} you can number text elements like remarks or
questions. If you want to make numbered remarks in your document you use:

\shortsetup{defineenumeration}

For example:

\startbuffer[a]
\defineenumeration
  [remark]
  [alternative=top,
   text=Remark,
   inbetween={\blank[none]},
   after=\blank]
\stopbuffer

\typebuffer[a]

Now the new commands \type{\remark}, \type{\subremark}, \type{\resetremark} and
\type{\nextremark} are available and you can type remarks like this:

\startbuffer[b]
\remark In the early medieval times Hasselt was a place of
pilgrimage. The {\em Heilige Stede} (Holy Place) was torn down during
the Reformation. In 1930, after 300 years, the {\em Heilige Stede} was
reopened.

\subremark Nowadays the {\em Heilige Stede} is closed again but once
a year an open air service is held on the same spot. \par
\stopbuffer

\typebuffer[b]

This becomes:

\start
\getbuffer[a]\getbuffer[b]
\stop

You can reset numbering with \type{\resetremark} or \type{\resetsubremark} or
increment a number with \type{\nextremark} or \type{\nextsubremark}. This is
normally done automatically per chapter, section or whatever.

You can set up the layout of \type{\defineenumeration} with:

\shortsetup{setupenumerations}

You can also vary the layout of \type{remark} and \type{subremark} in the example
above with:

\starttyping
\setupenumeration[remark][headstyle=bold]
\setupenumeration[subremark][headstyle=slanted]
\stoptyping

If a number becomes obsolete you can type:

\starttyping
\remark[-]
\stoptyping

If the \type{remark} contains several paragraphs you should use the command
pair \type{\start ... \stopremark} that becomes available after defining
\type{remark} with \type{\defineenumeration[remark]}.

\startbuffer[c]
\setupenumeration
   [remark]
   [alternative=hanging,
    width=broad]

\startremark
In the early medieval times Hasselt was a place of pilgrimage. The
{\em Heilige Stede} (Holy Place) was torn down during the
Reformation.

After 300 years in 1930 the {\em Heilige Stede} was reopened.
Nowadays the {\em Heilige Stede} is closed again but once a year an
open air service is held on the same spot.
\stopremark
\stopbuffer

\typebuffer[c]

So the example above would look like this:

\start
\getbuffer[a]\getbuffer[c] \par
\stop

\stopchapter

\stopcomponent
