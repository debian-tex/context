\startcomponent ma-cb-en-sortedlists

\enablemode[**en-us]

\project ma-cb

\startchapter[reference=synonyms,title=Sorted lists]

\index{sorted lists}

\Command{\tex{definesorting}}
\Command{\tex{setupsorting}}
\Command{\tex{sort}}
\Command{\tex{placelistofsorts}}
\Command{\tex{completelistofsorts}}

If you want to create a sorted list you can use:

\shortsetup{definesorting}

For example:

\startbuffer
\define[1]\street{#1\Street{#1}}
\definesorting[Street][Streets]
\setupsorting[Street][criterium=all]

When you walk in the \street{Eikenlaan} you will cross the
\street{Vechtlaan} and \street{Gasthuisstraat}. Go left into the
\street{Gasthuisstraat} and take another left on the
\street{Heerengracht}. You walk along the canal to the
\street{Ridderstraat}, there you turn right. Cross the canal and
turn left to the \street{Julianakade}. There you can enjoy the
view over the Zwartewater.

So the streets you visited are:

\placelistofStreets
\stopbuffer

\typebuffer

This will become:

\getbuffer

Note that the Gasthuisstraat appears only once in the list.

The predefined \type{\logo} command is used for the consistent use of text logos.

When you define:
\startbuffer
\logo [HSTEX]    {Hassel\TeX}
\stopbuffer

\getbuffer\typebuffer

You can use that logo througout your text.

\startbuffer
How would you call a \TEX\ based macropackage when you work
in Hasselt? \HSTEX?
\stopbuffer

\typebuffer

\getbuffer

\stopchapter

\stopcomponent
