\startcomponent ma-cb-en-specialcharacters

\enablemode[**en-us]

\project ma-cb

\startchapter[reference=special chars,title=Special characters]

\index{special characters}

You have seen that \CONTEXT\ commands are preceded by a \tex{} (backslash). This
means that \tex{} has a special meaning to \CONTEXT. Aside from \tex{} there are
other characters that need special attention when you want them to appear in
verbatim mode or in text mode. \in{Table}[tab:specchars] gives an overview of
these special characters and what you have to type to produce them.

\placetable[here,force][tab:specchars]
  {Special characters (1).}
  {\starttable[|c|l|c|c|c|c|]
  \HL
  \NC \use2 \JustCenter{\bf Special character}           \NC \use2 \bf Verbatim  \NC \use2 \bf Text \NC\FR
  \NC \bf Character \NC \bf Name          \NC \bf Type \NC \bf Generates \NC \bf Type\NC \bf Generates \NC\LR
  \HL
  \NC \type{#}      \NC hashtag           \NC \type{\type{#}} \NC \type{#} \VL \type{\#}           \NC \#           \NC\FR
  \NC \type{$}      \NC dollar            \NC \type{\type{$}} \NC \type{$} \VL \type{\$}           \NC \$           \NC\MR
  \NC \type{&}      \NC ampersand         \NC \type{\type{&}} \NC \type{&} \VL \type{\&}           \NC \&           \NC\MR
  \NC \type} \NC \type           \NC \%           \NC\MR
  \NC \type{\}      \NC backslash         \NC \type{\type{\}} \NC \type{\} \VL \type{\backslash}   \NC \backslash   \NC\MR
  \NC \type+{+      \NC right curly brace \NC \type-\type+{+- \NC \type+{+ \VL \type+\{+           \NC \{           \NC\MR
  \NC \type+}+      \NC left curly brace  \NC \type-\type+}+- \NC \type+}+ \VL \type+\}+           \NC \}           \NC\MR
  \NC \type{|}      \NC vertical bar      \NC \type{\type{|}} \NC \type{|} \VL \type{\|}           \NC \|           \NC\MR
  \NC \type{_}      \NC underscore        \NC \type{\type{_}} \NC \type{_} \VL \type{\_}           \NC \_           \NC\MR
  \NC \type{~}      \NC tilde             \NC \type{\type{~}} \NC \type{~} \VL \type{\lettertilde} \NC \lettertilde \NC\MR
  \NC \type{^}      \NC caret             \NC \type{\type{^}} \NC \type{^} \VL \type{\letterhat}   \NC \letterhat   \NC\LR
  \HL
  \stoptable}

Other special characters have a meaning in typesetting mathematical expressions
and some can be used in math mode only (see \in{chapter}[formulas]).

\placetable
  [here,force]
  [tab:special chars]
  {Special characters (2).}
  {\starttable[|c|c|c|c|c|]
  \HL
  \NC \bf \LOW{Special character} \NC \use2 \bf Verbatim  \NC \use2 \bf Text \NC\FR
  \NC                         \NC \bf Type \NC \bf Generates \NC \bf Type \NC \bf Generates \NC\LR
  \HL
  \NC \type{+} \NC \type{\type{+}} \NC \type{+} \VL \type{$+$} \NC $+$ \NC\FR
  \NC \type{-} \NC \type{\type{-}} \NC \type{-} \VL \type{$-$} \NC $-$ \NC\MR
  \NC \type{=} \NC \type{\type{=}} \NC \type{=} \VL \type{$=$} \NC $=$ \NC\MR
  \NC \type{<} \NC \type{\type{<}} \NC \type{<} \VL \type{$<$} \NC $<$ \NC\MR
  \NC \type{>} \NC \type{\type{>}} \NC \type{>} \VL \type{$>$} \NC $>$ \NC\LR
  \HL
  \stoptable}

\stopchapter

\stopcomponent
