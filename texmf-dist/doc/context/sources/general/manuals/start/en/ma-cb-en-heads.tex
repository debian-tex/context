\startcomponent ma-cb-en-heads

\enablemode[**en-us]

\project ma-cb

\startchapter[reference=heads,title=Heads]

\index{headers}

\Command{\tex{chapter}}
\Command{\tex{paragraph}}
\Command{\tex{subparagraph}}
\Command{\tex{title}}
\Command{\tex{subject}}
\Command{\tex{subsubject}}
\Command{\tex{setuphead}}
\Command{\tex{setupheads}}

The structure of a document is determined by its chapter and section titles.
These titles are created with the commands shown in \in{table}[tab:headers]:

\placetable[here][tab:headers]{Headers.}
  {\starttable[|l|l|]
  \HL
  \NC \bf Numbered header   \NC \bf Unnumbered header   \NC\SR
  \HL
  \NC \type{\chapter}       \NC \type{\title}           \NC\FR
  \NC \type{\section}       \NC \type{\subject}         \NC\MR
  \NC \type{\subsection}    \NC \type{\subsubject}      \NC\MR
  \NC \type{\subsubsection} \NC \type{\subsubsubject}   \NC\MR
  \NC \unknown              \NC \unknown                \NC\LR
  \HL
  \stoptable}

\shortsetup{chapter}
\shortsetup{section}
\shortsetup{subsection}
\shortsetup{title}
\shortsetup{subject}
\shortsetup{subsubject}

These commands will produce a numbered or unnumbered title in a predefined
fontsize and fonttype with some vertical spacing before and after the header.

The title commands can take several arguments, like in:

\starttyping
\title[hasselt by night]{Hasselt by night}
\stoptyping

and

\starttyping
\title{Hasselt by night}
\stoptyping

The bracket pair is optional and used for internal references. If you want to
refer to this chapter you type for example \type{\at{page}[hasselt by night]}.

For a more structured way to define chapters and sections you can use the more
preferred \type{\start ... \stop} construction.

\placetable[here][tab:headers]{Structured headers.}
  {\starttable[|l|l|]
  \HL
  \NC \bf Numbered header                  \NC \bf Un-numbered header   \NC\SR
  \HL
  \NC \type{\start ... \stopchapter}       \NC \type{\start ... \stoptitle}           \NC\FR
  \NC \type{\start ... \stopsection}       \NC \type{\start ... \stopsubject}         \NC\MR
  \NC \type{\start ... \stopsubsection}    \NC \type{\start ... \stopsubsubject}      \NC\MR
  \NC \type{\start ... \stopsubsubsection} \NC \type{\start ... \stopsubsubsubject}   \NC\MR
  \NC \unknown                             \NC \unknown                               \NC\LR
  \HL
  \stoptable}

In that case the definition looks like this:

\starttyping
\starttitle[reference="hasselt by night",title="Hasselt by night"}
   ...
\stoptitle
\stoptyping

Of course the chapter and section titles can be set to your own preferences and you can even
define your own sections. This is done with the \type{\setuphead} and
\type{\definehead} command.

\shortsetup{definehead}

\shortsetup{setuphead}

\startbuffer
\definehead
  [myhead]
  [section]

\setuphead
  [myhead]
  [numberstyle=bold,
   textstyle=bold,
   before=\hairline\blank,
   after=\nowhitespace\hairline]

\myhead[headlines]{Hasselt makes headlines}
\stopbuffer

\typebuffer

A new header \type{\myhead} is defined and it inherits the properties of
\type{\section}. It would look something like this:

\getbuffer

There is one other command you should know now, and that is \type{\setupheads}.
You can use this command to set up the numbering of the numbered chapters and
sections. If you type:

\startbuffer
\setupheads
  [alternative=inmargin,
   separator=--]
\stopbuffer

\typebuffer

all numbers will appear in the margin. Section 1.1 would look like 1--1.

Commands like \type{\setupheads} are typed in the set up area of your input file.

\shortsetup{setupheads}

\stopchapter

\stopcomponent
