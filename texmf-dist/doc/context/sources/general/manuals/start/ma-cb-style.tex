% macros=mkvi

\startenvironment ma-cb-style

\usemodule[chart]

\usemodule[s][abr-03]
\usemodule[x][set-11]

\unprotect

% Setups are kind of special.

% \loadsetups[cont-\currentmainlanguage.xml]

\loadsetups[i-context]

\setupsetup
  [\c!criterium=\v!used]

\setupframedtexts
  [setuptext]
  [\c!before={\blank[\v!big]},
   \c!after={\blank[\v!big]},
   \c!background=setup-shape,
   \c!backgroundoffset=10pt,
   \c!rulethickness=5pt,
   \c!offset=15pt,
   \c!frame=\v!off]

\setupexternalfigures
  [\c!directory={../graphics}]

% The layout dimensions are based on the A4 paper dimensions because that way users
% can print this manual themselves. Let's be economical with paper. We also assume a
% decent doublesided A4 printer. We use equal margins so that a single sided run or
% print also comes out all right.

\setuplayout
  [\c!backspace=22.5mm,
   \c!width=\v!fit,
   \c!cutspace=22.5mm,
   \c!margin=20mm,
   \c!margindistance=5mm,
   \c!topspace=15mm,
   \c!header=10mm,
   \c!headerdistance=5mm,
   \c!height=\v!fit,
   \c!footerdistance=5mm,
   \c!footer=15mm,
   \c!bottomspace=15mm]

\setuppagenumbering
  [\c!alternative=\v!doublesided]

% The lucida fonts look a bit more informal.

\doifmodeelse {atpragma} {
    \setupbodyfont[lucidaot,10pt]
} {
    \setupbodyfont[palatino,10pt]
}

% All colors will go here.

% todo

% Let's keep the text compact.

\setupwhitespace
  [\v!medium]

\setupblank
  [\v!medium]

% We indent verbatim with the default indenting value.

\setuptyping
  [\c!margin=\v!standard,
   \c!blank=\v!medium]

% Manuals as usual need a bit more tolerance, because a lot of in||line verbatim is
% used.

\setuptolerance
  [\v!verytolerant,\v!stretch]

% This manual makes heavy use of backgrounds. During a run about many metaclips are
% generated.

\defineoverlay [chapter-state]     [\uniqueMPpagegraphic{chapter-state}]
\defineoverlay [pagenumber-state]  [\uniqueMPgraphic{pagenumber-state}]

\defineoverlay [basic-shape-light] [\useMPgraphic{basic-shape-light}]
\defineoverlay [basic-shape-dark]  [\useMPgraphic{basic-shape-dark}]

\defineoverlay [setup-shape]       [\useMPgraphic{setup-shape}]
\defineoverlay [note-rule]         [\uniqueMPgraphic{note-rule}]
\defineoverlay [column-rule]       [\uniqueMPgraphic{column-rule}]

% \defineoverlay
%   [MenuAchtergrond]
%   [\MPclipTwoA{\overlaywidth}{\overlayheight}{3pt}{3pt}{red}{white}]

\starttexdefinition unexpanded FootnoteRule
    \blank[2*\v!big]
    \framed
        [\c!background=note-rule,
         \c!width=.4\makeupwidth,
         \c!height=2pt,
         \c!offset=\v!overlay,
         \c!rulethickness=2pt,
         \c!frame=\v!off]
        {}
    \blank[\v!small]
\stoptexdefinition

\setupfootnotes
  [\c!rule=off,
   \c!before=\FootnoteRule]

% Chapter titles have a fancy shape around them. Because we have a lot of small
% chapters, we don't go to a new page. Titles look the same, but there we go to
% a new page.

\setuphead
  [\v!chapter]
  [\c!command=\HeadCommand,
   \c!page=,
   \c!before={\blank[3*\v!big]},
   \c!after={\blank[2*\v!big,\v!samepage]}]

\setuphead
  [\v!title]
  [\c!page=\v!right]

\starttexdefinition unexpanded HeadCommand #number #title
    \alignedline {\v!outer} {\v!left} {
        \framed [
            \c!background=basic-shape-dark,
            \c!rulethickness=10pt,
            \c!frame=\v!off,
            \c!strut=\v!no,
            \c!offset=24pt,
            \c!align=\v!middle
        ] {
            \doifmode {*\v!sectionnumber} {
                #number
                \kern.5em
                \blackrule [
                    color=green,
                    width=1pt,
                    height=1.5\ht\strutbox,
                    depth=1.25\dp\strutbox
                ]
                \kern.5em
            }
            #title
        }
    }
\stoptexdefinition

% The current chapter number is typeset in the (outer) margin and slowly moves
% down. We could have directly put it in the margin but using the footermargin as
% starting point works better. This is an old command and there is no reason to
% change the definition to more fabce MkIV version.

\setupfootertexts
  [\v!margin]
  [][\fastsetup{chapterindicator}]

\startsetups chapterindicator
    \determineheadnumber[\v!chapter]
    \ifcase\currentheadnumber\else
        \vbox to \makeupheight {
            \scratchcounter=\numexpr\lastpage-\realpageno\relax
            \vskip2cm
            \vskip0pt plus \realpageno cm
            \framed [
                \c!background=chapter-state,
                \c!width=36pt,
                \c!height=72pt,
                \c!backgroundoffset=5pt,
               %\c!align={\v!lohi,\v!middle},
                \c!frame=\v!off
            ] {
                \lower.5\dp\strutbox\hbox {
                    \bfb
                    \getmarking[\v!chapter\v!number]
                }
            }
            \vskip0pt plus \scratchcounter cm
            \vskip2cm
        }
   \fi
\stopsetups

% The index is put on a double collumned grid. The numbers is surrounded by a
% shape.

\setupregister
  [\v!index]
  [\c!command=\IndexCommand,
   \c!before={\blank[\v!line]},
   \c!after=]

\starttexdefinition unexpanded IndexCommand #text
    \framed [
        \c!background=basic-shape-dark,
        \c!width=36pt,
        \c!frame=\v!off,
        \c!offset=4pt,
        \c!align=\v!middle,
        \c!rulethickness=4pt
    ] {
        #text
    }
\stoptexdefinition

% When bound, we use a double sided layout and put the pagenumber in the margin,
% enhanced by a fancy background.

\setuppagenumbering
  [\c!location={\v!footer,\v!middle},
   \c!command=\PageNumberCommand]

\starttexdefinition unexpanded PageNumberCommand #pagenumber
    \framed [
        \c!background=pagenumber-state,
        \c!backgroundoffset=5pt,
        \c!frame=\v!off,
        \c!offset=6pt
    ] {
        \lower.5\dp\strutbox\hbox spread 60pt {
            \hss
            #pagenumber
            \hss
        }
    }
\stoptexdefinition

% We put the chapter title in the head. If we wouldn't have to center, the more
% simple setting would be:

\setupheadertexts
  [{\getmarking[\v!chapter]}]

% Guess what the next one does.

\setupitemgroup
  [\v!itemize]
  [1]
  [\v!autointro]

% The coverpage looks more complex than it is. We can reuse it.

\newbox\CoverBackgroundBox % reuse saves .8 sec, could be an object!

\definebodyfontenvironment[1.9pt]

\defineframed
  [CoverFramed]
  [%\c!foregroundstyle=\CoverFont,
   \c!background=basic-shape-dark,
   \c!backgroundoffset=1pt,
   \c!rulethickness=2pt,
   \c!frame=\v!off]

\starttexdefinition unexpanded ShowSetupOnCover #n #tag #xmlroot
    \dontleavehmode
    \CoverFramed {
        \tttf\showsetupnameonly{#1}{#2}{#3}
    }
    \kern\zeropoint
    \hskip1em plus 1em minus .25em\relax
\stoptexdefinition

\definecolor[CoverTransparency][a=1,t=.5]

\startsetups coverbackground

    \ifvoid\CoverBackgroundBox

        \global\setbox\CoverBackgroundBox
            \startnicelyfilledbox
                [\c!width=\paperwidth,
                 \c!height=\paperheight,
%                  \c!offset=\exheight,
%                  \c!offset=\emwidth,
                 \c!offset=\zeropoint,
                 \c!strut=\v!no]
                \switchtobodyfont
                    [1.9pt]
                \starttransparent[CoverTransparency]
                \placelistofsorts
                    [texcommand]
                    [\c!command=\ShowSetupOnCover,
                     \c!criterium=\v!all]% used
                \stoptransparent
            \stopnicelyfilledbox

   \fi

   \copy\CoverBackgroundBox

\stopsetups

\defineoverlay[coverbackground][\setups{coverbackground}]

\setupdocument
  [author={Ton Otten\crlf PRAGMA ADE},
   translator=,
   contributer=,
   before=\setups{coverpage},
   after=\setups{backpage}]

\defineoverlay[gotocontents][\overlaybutton{contents}]

\startsetups coverpage

    \setupbackgrounds
        [\v!rightpage]
        [\c!background=coverbackground]

    \setupbackgrounds
        [\v!text]
        [\v!text]
        [\c!background=gotocontents]

    \startmakeup
        [\v!standard]
        [\c!doublesided=\v!empty,
         \c!headerstate=\v!none,
         \c!footerstate=\v!none]

        \hbox to \hsize \bgroup
            \hss
           %\definedfont[SansBold*default at 40pt]
            \definedfont[Bold*default at 40pt]
            \framed
                [\c!background=basic-shape-dark,
                 \c!frame=\v!off,
                 \c!rulethickness=30pt,
                 \c!align=\v!middle,
                 \c!offset=40pt]
                {\dontleavehmode\hbox{\documentvariable{title}}\par
                 \dontleavehmode\hbox{\documentvariable{subtitle}}}
        \egroup

        \vfill

        \doifsomething {\documentvariable{subtitle}} {

            \hbox to \hsize \bgroup
               %\definedfont[SansBold*default at 20pt]
                \definedfont[Bold*default at 20pt]
                \framed
                    [\c!background=basic-shape-dark,
                     \c!frame=\v!off,
                     \c!rulethickness=15pt,
                     \c!align=\v!middle,
                     \c!offset=20pt]
                    {\documentvariable{version}}
                \hss
            \egroup

        }

        \hbox to \hsize \bgroup
            \hss
           %\definedfont[SansBold*default at 24pt]
            \definedfont[Bold*default at 24pt]
            \framed
                [\c!background=basic-shape-dark,
                 \c!frame=\v!off,
                 \c!rulethickness=18pt,
                 \c!align=\v!middle,
                 \c!offset=35pt]
                {\documentvariable{author}}
        \egroup

    \stopmakeup

    \setupbackgrounds
        [\v!text]
        [\v!text]
        [\c!background=]

    \setupbackgrounds
        [\v!rightpage]
        [\c!background=]

    \doifmode {screen} {

        \setupbackgrounds
            [\v!page]
            [\c!background=\v!screen,
             \c!backgroundscreen=.95]

        \setupbackgrounds
            [\v!text]
            [\v!text]
            [\c!backgroundoffset=.25cm,
             \c!depth=.125cm,
             \c!background=\v!color,
             \c!backgroundcolor=white]

    }

    \component[ma-cb-copyright]

\stopsetups

% The backpage uses the same background and overlays a piece of text.

\startsetups backpage

    \page
      [\v!yes,\v!blank,\v!right]

    \component[ma-cb-colofon]

    \page
        [\v!yes,\v!blank,\v!left]

    \setupbackgrounds
        [\v!leftpage]
        [\c!background=coverbackground]

    \startmakeup
        [\v!standard]
        [\c!page=,
         \c!doublesided=\v!no,
         \c!headerstate=\v!none,
         \c!footerstate=\v!none]

        \setuptolerance
            [\v!verytolerant]

        \vfill

        \hbox to \hsize \bgroup

            \framed
                [\c!background=\v!color,
                 \c!backgroundcolor=white,
                 \c!frame=\v!off,
                 \c!offset=10pt,
                 \c!corner=\v!round,
                 \c!width=.4\makeupwidth,
                 \c!height=\textheight,
                 \c!align=\v!middle,
                 \c!strut=\v!no]
            {
                \vfil
                \component[ma-cb-en-backpage]
                \vfil
            }

            \hss

        \egroup

        \vfill

    \stopmakeup

\stopsetups

% To save space we don't start chapters on a new page, except in appendices and the
% introduction. These settings happen in dedicated setups sections (see later). We
% also add some white space between table of content entries.

\setupsectionblock [\v!frontpart] [\c!page=\v!right,\c!before=\setups{frontpart}]
\setupsectionblock [\v!bodypart]  [\c!page=\v!right,\c!before=\setups{bodypart}]
\setupsectionblock [\v!appendix]  [\c!page=\v!right,\c!before=\setups{appendix}]
\setupsectionblock [\v!backpart]  [\c!page=\v!right,\c!before=\setups{backpart}]

\setuplist
  [\v!chapter]
  [\c!criterium=\v!all,
   \c!before=,
   \c!after=]

\startsetups frontpart

    \setuphead[\v!chapter][\c!page=\v!right]

    \writebetweenlist[\v!chapter]{\blank}

    \startnamedsection[\v!chapter][\c!title=\labeltext{document:contents}]

        \startmixedcolumns[documentcolumns]
            \placelist[\v!chapter]
        \stopmixedcolumns

    \stopnamedsection

    \page[\v!right]

\stopsetups

\startsetups bodypart

    \setuphead[\v!chapter][\c!page=]

    \writebetweenlist[\v!chapter]{\blank}

\stopsetups

\startsetups appendix

    \setuphead[\v!chapter][\c!page=\v!right]

    \writebetweenlist[\v!chapter]{\blank}

    \startnamedsection[\v!chapter][\c!title=\labeltext{document:commanddefinitions},\c!reference=commandsetups]

        \component[ma-cb-\currentmainlanguage-commandlist]

        \blank[2*\v!big]

        \start

            \switchtobodyfont[8pt]

            % somehow \blank doesn't work here

            \setupframedtexts
              [setuptext]
              [\c!before={\vskip6pt},
               \c!after={\vskip6pt}]

          % \startmixedcolumns[documentcolumns]
                \placelistofsorts[texcommand]
          % \stopmixedcolumns

        \stop

    \stopnamedsection

    \startnamedsection[\v!chapter][\c!title=\labeltext{document:commandindex}]

        \startmixedcolumns[documentcolumns]
            \placeregister[Command]
        \stopmixedcolumns

    \stopnamedsection

    \startnamedsection[\v!chapter][\c!title=\labeltext{document:subjectindex}]

        \startmixedcolumns[documentcolumns]
            \placeregister[\v!index]
        \stopmixedcolumns

    \stopnamedsection

    \startnamedsection[\v!chapter][\c!title=\labeltext{document:supportandreading}]

        \index{support}

        \startnamedsection[\v!section][\c!title=\labeltext{document:support}]

            \component[ma-cb-\currentmainlanguage-support]

        \stopnamedsection

        \startnamedsection[\v!section][\c!title=\labeltext{document:manuals}]

            % this will be done from a bib file

            \startlines
               \goto {Chemical Formulas in \CONTEXT} [ url (manual:chemic-ex)   ]
               \goto {Color Separation}              [ url (manual:color)       ]
               \goto {Columns}                       [ url (manual:columns)     ]
               \goto {\CONTEXT, the manual}          [ url (manual:context)     ]
               \goto {Dealing with \XML}             [ url (manual:xml)         ]
               \goto {Extreme Tables}                [ url (manual:extab)       ]
               \goto {Figures}                       [ url (manual:figures)     ]
               \goto {Fonts in \CONTEXT}             [ url (manual:fonts)       ]
               \goto {luatools, mtxrun, context}     [ url (manual:tools)       ]
               \goto {\METAFUN\ manual}              [ url (manual:metafun)     ]
               \goto {Natural Tables}                [ url (manual:nattab)      ]
               \goto {\PPCHTEX\ Manual}              [ url (manual:chemic)      ]
               \goto {Quick Reference (dutch)}       [ url (manual:qr-nl)       ]
               \goto {Quick Reference (english)}     [ url (manual:qr-en)       ]
               \goto {\SCITE\ in \CONTEXT}           [ url (manual:scite)       ]
               \goto {Units}                         [ url (manual:units)       ]
               \goto {Widgets}                       [ url (manual:widgets)     ]
            \stoplines

        \stopnamedsection

        \startnamedsection[\v!section][\c!title=\labeltext{document:magazines}]

            % this will be done from a bib file

            \startlines
                \goto {\CONTEXT\ Magazine 1103}      [ url (thisway:crossrefs)  ]
                \goto {Project structure}            [ url (thisway:proj-struc) ]
            \stoplines

        \stopnamedsection

    \stopnamedsection

\stopsetups

\startsetups backpart

    \setuphead[\v!chapter][\c!page=\v!right]

    \writebetweenlist[\v!chapter]{\blank}

\stopsetups

% In normal documents one will never find awful things like below. Because we want
% an international setup, we just call the chapters in an indirect way.

% We draw a nice line between columns. The next command does the job. Of course a
% normal line can be set more easily, but here we hook in a command.

\installmixedcolumnseparator{ColumnRule}%
  {\framed
     [\c!background=column-rule,
      \c!height=\mixedcolumnseparatorheight,
      \c!depth=\mixedcolumnseparatordepth,
      \c!width=\mixedcolumnseparatorwidth,
      \c!offset=\v!overlay,
      \c!rulethickness=2pt,
      \c!frame=\v!off]
     {}}

% alternative implemenation
%
% \installmixedcolumnseparator{ColumnRule}%
%   {\lower\mixedcolumnseparatordepth\hbox{\uniqueMPgraphic
%      {column-rule}%
%      {height=\the\dimexpr\mixedcolumnseparatorheight+\mixedcolumnseparatordepth,linewidth=2pt}}}

\setupmixedcolumns
  [\c!n=2,
   \c!distance=36pt,
   \c!separator=ColumnRule]

\definemixedcolumns
  [documentcolumns]
  [\c!n=2,
   \c!distance=36pt,
   \c!separator=ColumnRule]

\defineregister
  [Command]

\setupregister
  [Command]
  [\c!indicator=\v!off,
   \c!before={\blank[\v!line]}]

\protect

\stopenvironment
