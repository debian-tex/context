% interface=en modes=icon,screen language=uk

\usemodule[art-01]
\usemodule[abr-02]

\unexpanded\def\METAPOST{MetaPost}
\unexpanded\def\METAFUN {MetaFun}

\setuphead
  [section]
  [color=darkblue]

\setuptype
  [color=darkblue]

\setuptyping
  [color=darkblue]

\setuptyping
  [margin=yes]

\setupwhitespace
  [big]

\definecolor[gray][s=.2,t=.5,a=1]

\startuseMPgraphic{TitlePage}{darkness}
    StartPage ;

        numeric factor   ; factor   := 1/3 ;
        numeric multiple ; multiple := PaperHeight/PaperWidth ; % 1.6 ;
        numeric stages   ; stages   := multiple/16 ; % .1 ;
        numeric darkness ; darkness := \MPvar{darkness} ;

        def Scaled(expr s, m) =
            if m = 1 :
                scaled (2*s*PaperWidth)
            else :
                xscaled (2*s*PaperWidth) yscaled (2*s*PaperHeight)
            fi
        enddef ;

        fill Page withcolor (factor*white) ;

        fill fullcircle scaled (multiple*PaperWidth) shifted llcorner Page withcolor (factor*red) ;
        fill fullcircle scaled (multiple*PaperWidth) shifted ulcorner Page withcolor (factor*green) ;
        fill fullcircle scaled (multiple*PaperWidth) shifted urcorner Page withcolor (factor*blue) ;
        fill fullcircle scaled (multiple*PaperWidth) shifted lrcorner Page withcolor (factor*yellow) ;

        for i = llcorner Page, ulcorner Page, urcorner Page, lrcorner Page :
            for j = 0 step stages until (10*stages-eps) : % or .8
                fill fullcircle Scaled(j,1) shifted i withcolor transparent(1,\MPvar{darkness}*(1-j),white) ;
            endfor ;
        endfor ;

        draw Page withpen pencircle scaled .1PaperWidth withcolor transparent(1,.5,.5white) ;

    StopPage
\stopuseMPgraphic

\startmode[icon,screen]

  \setuppapersize[S66][S66]

  \setupbodyfont[10pt]

\stopmode

\startmode[icon]

  \starttext

  \startTEXpage
     \useMPgraphic{TitlePage}{darkness=0.4}
  \stopTEXpage

  \stoptext

\stopmode

\starttext

% title page

\definelayer
  [TitlePage]
  [width=\paperwidth,
   height=\paperheight]

\setlayer
  [TitlePage]
  {\useMPgraphic{TitlePage}{darkness=1}}

\setlayerframed
  [TitlePage]
  [preset=rightbottom,
   hoffset=.1\paperwidth,
   voffset=.1\paperwidth]
  [align=left,
   width=\hsize,
   frame=off,
   foregroundcolor=gray]
  {\definedfont[SerifBold sa 10]SciTE\endgraf
   \definedfont[SerifBold sa 2.48]IN CONTEXT MkIV\kern.25\bodyfontsize}

\startTEXpage
  \tightlayer[TitlePage]
\stopTEXpage

% main text

\subject{Warning}

\SCITE\ version 3.61 works ok but 3.62 crashes. It'a a real pity that \SCITE\
doesn't have the scintillua lexer built in, which would also make integration a
bit nicer by sharing the \LUA\ instance. The \CONTEXT\ lexing discussed here is
the lexing I assume when using \CONTEXT\ \MKIV, but alas it's not easy to get it
running on \UNIX\ and on \MACOSX\ there is no \LUA\ lexing available.

\subject{About \SCITE}

For a long time at \PRAGMA\ we used \TEXEDIT, an editor we'd written in \MODULA.
It had some project management features and recognized the project structure in
\CONTEXT\ documents. Later we rewrote this to a platform independent
reimplementation called \TEXWORK\ written in \PERLTK\ (not to be confused with
the editor with the plural name).

In the beginning of the century I can into \SCITE, written by Neil Hodgson.
Although the mentioned editors provide some functionality not present in \SCITE\
we decided to use that editor because it frees us from maintaining our own. I
ported our \TEX\ and \METAPOST\ (line based) syntax highlighting to \SCITE\ and
got a lot of others for free.

After a while I found out that there was an extension interface written in \LUA.
I played with it and wrote a few extensions too. This pleasant experience later
triggered the \LUATEX\ project.

A decade into the century \SCITE\ got another new feature: you can write dynamic
external lexers in \LUA\ using \LPEG. As in the meantime \CONTEXT\ has evolved in
a \TEX/\LUA\ hybrid, it made sense to look into this. The result is a couple of
lexers that suit \TEX, \METAPOST\ and \LUA\ usage in \CONTEXT\ \MKIV. As we also
use \XML\ as input and output format a lexer for \XML\ is also provided. And
because \PDF\ is one of the backend formats lexing of \PDF\ is also implemented.
\footnote {In the process some of the general lexing framework was adapted to
suit our demands for speed. We ship these files as well.}

In the \CONTEXT\ (standalone) distribution you will find the relevant files
under:

\starttyping
<texroot>/tex/texmf-context/context/data/scite
\stoptyping

Normally a user will not have to dive into the implementation details but in
principle you can tweak the properties files to suit your purpose.

\subject{The look and feel}

The color scheme that we use is consistent over the lexers but we use more colors
that in the traditional lexing. For instance, \TEX\ primitives, low level \TEX\
commands, \TEX\ constants, basic file structure related commands, and user
commands all get a different treatment. When spell checking is turned on, we
indicate unknown words, but also words that are known but might need checking,
for instance because they have an uppercase character. In \in {figure}
[fig:colors] we some of that in practice.

\placefigure
  [page]
  [fig:colors]
  {Nested lexers in action.}
  {\rotate
     [rotation=90]
     {\externalfigure
        [scite-context-visual.png]
        [maxheight=1.2\textwidth,
         maxwidth=.9\textheight]}}

\subject{Installing \SCITE}

Installing \SCITE\ is straightforward. We are most familiar with \MSWINDOWS\ but
for other operating systems installation is not much different. First you need to
fetch the archive from:

\starttyping
www.scintilla.org
\stoptyping

The \MSWINDOWS\ binaries are zipped in \type {wscite.zip}, and you can unzip this
in any directory you want as long as you make sure that the binary ends up in
your path or as shortcut on your desktop. So, say that you install \SCITE\ in:

\starttyping
c:\data\system\scite\wscite
\stoptyping

You need to add this path to your local path definition. Installing \SCITE\ to
some known place has the advantage that you can move it around. There are no
special dependencies on the operating system.

On \MSWINDOWS\ you can for instance install \SCITE\ in:

\starttyping
c:\data\system\scite
\stoptyping

and then end up with:

\starttyping
c:\data\system\scite\wscite
\stoptyping

and that is the path you need to add to your environment \type {PATH} variable.

On \LINUX\ the files end up in:

\starttyping
/usr/bin
/usr/share/scite
\stoptyping

Where the second path is the path we will put more files.

\subject{Installing \type {scintillua}}

Next you need to install the lpeg lexers. \footnote {Versions later than 2.11
will not run on \MSWINDOWS\ 2K. In that case you need to comment the external
lexer import.} The library is part of the \type {textadept} editor by Mitchell
(\hyphenatedurl {mitchell.att.foicica.com}) which is also based on scintilla:
The archive can be fetched from:

\starttyping
http://foicica.com/scintillua/
\stoptyping

On \MSWINDOWS\ you need to copy the files to the \type {wscite} folder (so we end
up with a \type {lexers} subfolder there). For \LINUX\ the place depends on the
distribution, for instance \type {/usr/share/scite}; this is the place where the
regular properties files live. \footnote {If you update, don't do so without
testing first. Sometimes there are changes in \SCITE\ that influence the lexers
in which case you have to wait till we have update them to suit those changes.}

So, you end up, on \MSWINDOWS\ with:

\starttyping
c:\data\system\scite\wscite\lexers
\stoptyping

And on \LINUX:

\starttyping
/usr/share/scite/lexers
\stoptyping

Beware: if you're on a 64 bit system, you need to rename the 64 bit \type {so}
library into one without a number. Unfortunately the 64 bit library is now always
available which can give surprises when the operating system gets updates. In such
a case you should downgrade or use \type {wine} with the \MSWINDOWS\ binaries
instead. After installation you need to restart \SCITE\ in order to see if things
work out as expected.

\subject{Installing the \CONTEXT\ lexers}

When we started using this nice extension, we ran into issues and as a
consequence shipped a patched \LUA\ code. We also needed some more control as we
wanted to provide more features and complex nested lexers. Because the library
\API\ changed a couple of times, we now have our own variant which will be
cleaned up over time to be more consistent with our other \LUA\ code (so that we
can also use it in \CONTEXT\ as variant verbatim lexer). We hope to be able to
use the \type {scintillua} library as it does the job.

Anyway, if you want to use \CONTEXT, you need to copy the relevant files from

\starttyping
<texroot>/tex/texmf-context/context/data/scite
\stoptyping

to the path were \SCITE\ keeps its property files (\type {*.properties}). This is
the path we already mentioned. There should be a file there called \type
{SciteGlobal.properties}.

So,in the end you get on \MSWINDOWS\ new files in:

\starttyping
c:\data\system\scite\wscite
c:\data\system\scite\wscite\context
c:\data\system\scite\wscite\context\lexer
c:\data\system\scite\wscite\context\lexer\themes
c:\data\system\scite\wscite\context\lexer\data
c:\data\system\scite\wscite\context\documents
\stoptyping

while on \LINUX\ you get:

\starttyping
/usr/bin/share/
/usr/bin/share/context
/usr/bin/share/context/lexer
/usr/bin/share/context/lexer/themes
/usr/bin/share/context/lexer/data
/usr/bin/share/context/documents
\stoptyping

At the end of the \type {SciteGlobal.properties} you need to add the following
line:

\starttyping
import context/scite-context-user
\stoptyping

After this, things should run as expected (given that \TEX\ runs at the console
as well).

% In order to run the commands needed, we assume that the following programs
% are installed:
%
% \startitemize[packed]
% \item tidy (for quick and dirty checking of \XML\ files)
% \item xsltproc (for converting \XML\ files into other formats)
% \item acrobat (for viewing files)
% \item ghostview (for viewing files, use gv on \UNIX)
% \item rxvt (a console, only needed on \UNIX)
% \stopitemize

\subject{Fonts}

The configuration file defaults to the Dejavu fonts. These free fonts are part of
the \CONTEXT\ suite (also known as the standalone distribution). Of course you
can fetch them from \type {http://dejavu-fonts.org} as well. You have to copy
them to where your operating system expects them. In the suite they are available
in:

\starttyping
<contextroot>/tex/texmf/fonts/truetype/public/dejavu
\stoptyping

\subject{Extensions}

Just a quick note to some extensions. If you select a part of the text (normally
you do this with the shift key pressed) and you hit \type {Shift-F11}, you get a
menu with some options. More (robust) ones will be provided at some point.

\subject{Spell checking}

If you want to have spell checking, you need have files with correct words on
each line. The first line of a file determines the language:

\starttyping
% language=uk
\stoptyping

When you use the external lexers, you need to provide some files. Given that you
have a text file with valid words only, you can run the following script:

\starttyping
mtxrun --script scite --words nl uk
\stoptyping

This will convert files with names like \type {spell-nl.txt} into \LUA\ files
that you need to copy to the \type {lexers/data} path. Spell checking happens
realtime when you have the language directive (just add a bogus character to
disable it). Wrong words are colored red, and words that might have a case
problem are colored orange. Recognized words are greyed and words with less than
three characters are ignored.

A spell checking file has to be put in the \type {lexers/data} directory and
looks as follows (e.g. \type {spell-uk.lua}):

\starttyping
return {
 ["max"]=40,
 ["min"]=3,
 ["n"]=151493,
 ["words"]={
  ["aardvark"]="aardvark",
  ["aardvarks"]="aardvarks",
  ["aardwolf"]="aardwolf",
  ["aardwolves"]="aardwolves",
  ...
 }
}
\stoptyping

The keys are words that get checked for the given value (which can have uppercase
characters). The word files are not distributed (but they might be at some point).

In the case of internal lexers, the following file is needed:

\starttyping
spell-uk.txt
\stoptyping

If you use the traditional lexer, this file is taken from the path determined by
the environment variable:

\starttyping
CTXSPELLPATH
\stoptyping

As already mentioned, the lpeg lexer expects them in the data path. This is
because the \LUA\ instance that does the lexing is rather minimalistic and lacks
some libraries as well as cannot access the main \SCITE\ state.

Spell checking in \type {txt} files is enabled by adding a first line:

\starttyping
[#!-%] language=uk
\stoptyping

The first character on that line is one of the four mentioned between square
brackets. So,

\starttyping
# language=uk
\stoptyping

should work. For \XML\ files there are two methods. You can use the following (at
the start of the file):

\starttyping
<?xml ... language="uk" ?>
\stoptyping

But probably better is to use the next directive just below the
usual \XML\ marker line:

\starttyping
<?context-directive editor language uk ?>
\stoptyping

\subject{Interface selection}

In a similar fashion you can drive the interface checking:

\starttyping
% interface=nl
\stoptyping

\subject{Property files}

The internal lexers are controlled by the property files while the external ones
are steered with themes. Unfortunately there is hardly any access to properties
from the external lexer code nor can we consult the file system and/or run
programs like \type {mtxrun}. This means that we cannot use configuration files
in the \CONTEXT\ distribution directly. Hopefully this changes with future
releases.

\subject{The external lexers}

These are the more advanced lexers. They provide more detail and the \CONTEXT\
lexer also supports nested \METAPOST\ and \LUA. Currently there is no detailed
configuration but this might change once they are stable.

The external lexers operate on documents while the internal ones operate on
lines. This can make the external lexers slow on large documents. We've optimized
the code somewhat for speed and memory consumption but there's only so much one
can do. While lexing each change in style needs a small table but allocating and
garbage collecting many small tables comes at a price. Of course in practice this
probably gets unnoticed. \footnote {I wrote the code in 2011 on a more than 5
years old Dell M90 laptop, so I suppose that speed is less an issue now.}

The external lpeg lexers work okay with the \MSWINDOWS\ and \LINUX\ versions of
\SCITE, but unfortunately at the time of writing this, the \LUA\ library that is
needed is not available for the \MACOSX\ version of \SCITE. Also, due to the fact
that the lexing framework is rather isolated, there are some issues that cannot
be addressed in the properly, at least not currently.

In addition to \CONTEXT\ and \METAFUN\ lexing a \LUA\ lexer is also provided so
that we can handle \CONTEXT\ \LUA\ Document (\CLD) files too. There is also an
\XML\ lexer. This one also provides spell checking. The \PDF\ lexer tries to do a
good job on \PDF\ files, but it has some limitations. There is also a simple text
file lexer that does spell checking. Finally there is a lexer for \CWEB\ files.

Don't worry if you see an orange rectangle in your \TEX\ or \XML\ document. This
indicates that there is a special space character there, for instance \type
{0xA0}, the nonbreakable space. Of course we assume that you use \UTF8 as input
encoding.

\subject{The internal lexers}

\SCITE\ has quite some built in lexers. A lexer is responsible for highlighting
the syntax of your document. The way a \TEX\ file is treated is configured in the
file:

\starttyping
tex.properties
\stoptyping

You can edit this file to your needs using the menu entry under \type {options}
in the top bar. In this file, the following settings apply to the \TEX\ lexer:

\starttyping
lexer.tex.interface.default=0
lexer.tex.use.keywords=1
lexer.tex.comment.process=0
lexer.tex.auto.if=1
\stoptyping

The option \type {lexer.tex.interface.default} determines the way keywords are
highlighted. You can control the interface from your document as well, which
makes more sense that editing the configuration file each time.

\starttyping
% interface=all|tex|nl|en|de|cz|it|ro|latex
\stoptyping

The values in the properties file and the keywords in the preamble line have the
following meaning:

\starttabulate[|lT|lT|p|]
\NC 0 \NC all   \NC all commands (preceded by a backslash)                \NC \NR
\NC 1 \NC tex   \NC \TEX, \ETEX, \PDFTEX, \OMEGA\ primitives (and macros) \NC \NR
\NC 2 \NC nl    \NC the dutch \CONTEXT\ interface                         \NC \NR
\NC 3 \NC en    \NC the english \CONTEXT\ interface                       \NC \NR
\NC 4 \NC de    \NC the german \CONTEXT\ interface                        \NC \NR
\NC 5 \NC cz    \NC the czech \CONTEXT\ interface                         \NC \NR
\NC 6 \NC it    \NC the italian \CONTEXT\ interface                       \NC \NR
\NC 7 \NC ro    \NC the romanian \CONTEXT\ interface                      \NC \NR
\NC 8 \NC latex \NC \LATEX\ (apart from packages)                         \NC \NR
\stoptabulate

The configuration file is set up in such a way that you can easily add more
keywords to the lists. The keywords for the second and higher interfaces are
defined in their own properties files. If you're curious about the way this is
configures, you can peek into the property files that start with \type
{scite-context}. When you have \CONTEXT\ installed you can generate configuration
files with

\starttyping
mtxrun --script interface --scite
\stoptyping

You need to make sure that you move the result to the right place so best not
mess around with this command and use the files from the distribution.

Back to the properties in \type {tex.properties}. You can disable keyword
coloring alltogether with:

\starttyping
lexer.tex.use.keywords=0
\stoptyping

but this is only handy for testing purposes. More interesting is that you can
influence the way comment is treated:

\starttyping
lexer.tex.comment.process=0
\stoptyping

When set to zero, comment is not interpreted as \TEX\ code and it will come out
in a uniform color. But, when set to one, you will get as much colors as a \TEX\
source. It's a matter of taste what you choose.

The lexer tries to cope with the \TEX\ syntax as good as possible and takes for
instance care of the funny \type {^^} notation. A special treatment is applied to
so called \type {\if}'s:

\starttyping
lexer.tex.auto.if=1
\stoptyping

This is the default setting. When set to one, all \type {\ifwhatever}'s will be
seen as a command. When set to zero, only the primitive \type {\if}'s will be
treated. In order not to confuse you, when this property is set to one, the lexer
will not color an \type {\ifwhatever} that follows an \type {\newif}.

\subject{The \METAPOST\ lexer}

The \METAPOST\ lexer is set up slightly different from its \TEX\ counterpart,
first of all because \METAPOST\ is more a language that \TEX. As with the \TEX\
lexer, we can control the interpretation of identifiers. The \METAPOST\ specific
configuration file is:

\starttyping
metapost.properties
\stoptyping

Here you can find properties like:

\starttyping
lexer.metapost.interface.default=1
\stoptyping

Instead of editing the configuration file you can control the lexer with the
first line in your document:

\starttyping
% interface=none|metapost|mp|metafun
\stoptyping

The numbers and keywords have the following meaning:

\starttabulate[|lT|lT|p|]
\NC 0 \NC none           \NC no highlighting of identifiers   \NC \NR
\NC 1 \NC metapost or mp \NC \METAPOST\ primitives and macros \NC \NR
\NC 2 \NC metafun        \NC \METAFUN\ macros                 \NC \NR
\stoptabulate

Similar to the \TEX\ lexer, you can influence the way comments are handled:

\starttyping
lexer.metapost.comment.process=1
\stoptyping

This will interpret comment as \METAPOST\ code, which is not that useful
(opposite to \TEX, where documentation is often coded in \TEX).

The lexer will color the \METAPOST\ keywords, and, when enabled also additional
keywords (like those of \METAFUN). The additional keywords are colored and shown
in a slanted font.

The \METAFUN\ keywords are defined in a separate file:

\starttyping
metafun-scite.properties
\stoptyping

You can either copy this file to the path where you global properties files lives,
or put a copy in the path of your user properties file. In that case you need to
add an entry to the file \type {SciTEUser.properties}:

\starttyping
import metafun-scite
\stoptyping

The lexer is able to recognize \type {btex}||\type {etex} and will treat anything
in between as just text. The same happens with strings (between \type {"}). Both
act on a per line basis.

\subject{Using \ConTeXt}

When \type {mtxrun} is in your path, \CONTEXT\ should run out of the box. You can
find \type {mtxrun} in:

\starttyping
<contextroot>/tex/texmf-mswin/bin
\stoptyping

or in a similar path that suits the operating system that you use.

When you hit \type{CTRL-12} your document will be processed. Take a look at the
\type {Tools} menu to see what more is provided.

\subject{Extensions (using \LUA)}

When the \LUA\ extensions are loaded, you will see a message in the log pane that
looks like:

\starttyping
-  see scite-ctx.properties for configuring info

-  ctx.spellcheck.wordpath set to ENV(CTXSPELLPATH)
-  ctxspellpath set to c:\data\develop\context\spell
-  ctx.spellcheck.wordpath expands to c:\data\develop\context\spell

-  ctx.wraptext.length is set to 65
-  key bindings:

Shift + F11   pop up menu with ctx options

Ctrl  + B     check spelling
Ctrl  + M     wrap text (auto indent)
Ctrl  + R     reset spelling results
Ctrl  + I     insert template
Ctrl  + E     open log file
Ctrl  + +     show language character strip (key might change)

-  recognized first lines:

xml   <?xml version='1.0' language='nl'
tex   % language=nl
\stoptyping

This message tells you what extras are available. The language character strip feature
is relatively new and displays buttons at the bottom of the screen for the characters
in a (chosen) language. This is handy when you occasionally have to key in (snippets) of
a language you're not familiar with. More alphabets will be added (we take data from some
\CONTEXT\ language relates files).

\subject{Templates}

There is an experimental template mechanism. One option is to define templates in
a properties file. The property file \type {scite-ctx-context} contains
definitions like:

\starttyping
command.25.$(file.patterns.context)=insert_template \
$(ctx.template.list.context)

ctx.template.list.context=\
    itemize=structure.itemize.context|\
    tabulate=structure.tabulate.context|\
    natural TABLE=structure.TABLE.context|\
    use MP graphic=graphics.usemp.context|\
    reuse MP graphic=graphics.reusemp.context|\
    typeface definition=fonts.typeface.context

ctx.template.structure.itemize.context=\
\startitemize\n\
\item ?\n\
\item ?\n\
\item ?\n\
\stopitemize\n
\stoptyping

The file \type {scite-ctx-example} defines \XML\ variants:

\starttyping
command.25.$(file.patterns.example)=insert_template \
$(ctx.template.list.example)

ctx.template.list.example=\
    bold=font.bold.example|\
    emphasized=font.emphasized.example|\
    |\
    inline math=math.inline.example|\
    display math=math.display.example|\
    |\
    itemize=structure.itemize.example

ctx.template.structure.itemize.example=\
<itemize>\n\
<item>?</item>\n\
<item>?</item>\n\
<item>?</item>\n\
</itemize>\n
\stoptyping

For larger projects it makes sense to keep templates with the project. In one of
our projects we have a directory in the path where the project files are kept
which holds template files:

\starttyping
..../ctx-templates/achtergronden.xml
..../ctx-templates/bewijs.xml
\stoptyping

One could define a template menu like we did previously:

\starttyping
ctx.templatelist.example=\
    achtergronden=mathadore.achtergronden|\
    bewijs=mathadore.bewijs|\

ctx.template.mathadore.achtergronden.file=smt-achtergronden.xml
ctx.template.mathadore.bewijs.file=smt-bewijs.xml
\stoptyping

However, when no such menu is defined, we will automatically scan the directory
and build the menu without user intervention.

\subject{Using \SCITE}

The following keybindings are available in \SCITE. Most of this list is taken
from the on|-|line help pages.

\startbuffer[keybindings]
\starttabulate[|l|p|]
\FL
\NC \rm \bf keybinding   \NC \bf meaning (taken from the \SCITE\ help file)                                \NC \NR
\ML
\NC \type{Ctrl+Keypad+}         \NC magnify text size                                                      \NC \NR
\NC \type{Ctrl+Keypad-}         \NC reduce text size                                                       \NC \NR
\NC \type{Ctrl+Keypad/}         \NC restore text size to normal                                            \NC \NR
\ML
\NC \type{Ctrl+Keypad*}         \NC expand or contract a fold point                                        \NC \NR
\ML
\NC \type{Ctrl+Tab}             \NC cycle through recent files                                             \NC \NR
\ML
\NC \type{Tab}                  \NC indent block                                                           \NC \NR
\NC \type{Shift+Tab}            \NC dedent block                                                           \NC \NR
\ML
\NC \type{Ctrl+BackSpace}       \NC delete to start of word                                                \NC \NR
\NC \type{Ctrl+Delete}          \NC delete to end of word                                                  \NC \NR
\NC \type{Ctrl+Shift+BackSpace} \NC delete to start of line                                                \NC \NR
\NC \type{Ctrl+Shift+Delete}    \NC delete to end of line                                                  \NC \NR
\ML
\NC \type{Ctrl+Home}            \NC go to start of document; \type{Shift} extends selection                \NC \NR
\NC \type{Ctrl+End}             \NC go to end of document; \type{Shift} extends selection                  \NC \NR
\NC \type{Alt+Home}             \NC go to start of display line; \type{Shift} extends selection            \NC \NR
\NC \type{Alt+End}              \NC go to end of display line; \type{Shift} extends selection              \NC \NR
\ML
\NC \type{Ctrl+F2}              \NC create or delete a bookmark                                            \NC \NR
\NC \type{F2}                   \NC go to next bookmark                                                    \NC \NR
\ML
\NC \type{Ctrl+F3}              \NC find selection                                                         \NC \NR
\NC \type{Ctrl+Shift+F3}        \NC find selection backwards                                               \NC \NR
\ML
\NC \type{Ctrl+Up}              \NC scroll up                                                              \NC \NR
\NC \type{Ctrl+Down}            \NC scroll down                                                            \NC \NR
\ML
\NC \type{Ctrl+C}               \NC copy selection to buffer                                               \NC \NR
\NC \type{Ctrl+V}               \NC insert content of buffer                                               \NC \NR
\NC \type{Ctrl+X}               \NC copy selection to buffer and delete selection                          \NC \NR
\ML
\NC \type{Ctrl+L}               \NC line cut                                                               \NC \NR
\NC \type{Ctrl+Shift+T}         \NC line copy                                                              \NC \NR
\NC \type{Ctrl+Shift+L}         \NC line delete                                                            \NC \NR
\NC \type{Ctrl+T}               \NC line transpose with previous                                           \NC \NR
\NC \type{Ctrl+D}               \NC line duplicate                                                         \NC \NR
\ML
\NC \type{Ctrl+K}               \NC find matching preprocessor conditional, skipping nested ones           \NC \NR
\NC \type{Ctrl+Shift+K}         \NC select to matching preprocessor conditional                            \NC \NR
\NC \type{Ctrl+J}               \NC find matching preprocessor conditional backwards, skipping nested ones \NC \NR
\NC \type{Ctrl+Shift+J}         \NC select to matching preprocessor conditional backwards                  \NC \NR
\ML
\NC \type{Ctrl+[}               \NC previous paragraph; \type{Shift} extends selection                     \NC \NR
\NC \type{Ctrl+]}               \NC next paragraph; \type{Shift} extends selection                         \NC \NR
\NC \type{Ctrl+Left}            \NC previous word; \type{Shift} extends selection                          \NC \NR
\NC \type{Ctrl+Right}           \NC next word; \type{Shift} extends selection                              \NC \NR
\NC \type{Ctrl+/}               \NC previous word part; \type{Shift} extends selection                     \NC \NR
\NC \type{Ctrl+\ }              \NC next word part; \type{Shift} extends selection                         \NC \NR
\ML
\NC \type{F12 / Ctrl+F7}        \NC check (or process)                                                     \NC \NR
\NC \type{Ctrl+F12 / Ctrl+F7}   \NC process (run)                                                          \NC \NR
\NC \type{Alt+F12 / Ctrl+F7}    \NC process (run) using the luajit vm (if applicable)                      \NC \NR
\LL
\stoptabulate
\stopbuffer

\getbuffer[keybindings]

\page

\subject{Affiliation}

\starttabulate[|l|l|]
\NC author    \NC Hans Hagen                    \NC \NR
\NC copyright \NC PRAGMA ADE, Hasselt NL        \NC \NR
\NC more info \NC \type {www.pragma-ade.com}    \NC \NR
\NC           \NC \type {www.contextgarden.net} \NC \NR
\NC version   \NC \currentdate                  \NC \NR
\stoptabulate

\stoptext
